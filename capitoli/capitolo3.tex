Avendo stabilito i principi del cloud computing e le ragioni della scelta di AWS, questo capitolo si addentra negli aspetti pratici dell'implementazione di un'infrastruttura sicura e scalabile su AWS per una startup fintech. Verranno presentati esempi concreti di configurazioni e utilizzi dei servizi AWS, focalizzandosi sulle best practice di sicurezza applicabili in un contesto con risorse limitate ma requisiti elevati, tipico di una startup nel settore finanziario.

\section{Implementazione attuale dell'infrastruttura AWS} 
L'infrastruttura AWS attualmente in uso è composta da una serie di servizi fondamentali per il funzionamento della piattaforma fintech. Tra i servizi abilitati figurano AWS Glue per l'integrazione e la trasformazione dei dati, AWS Key Management Service per la gestione delle chiavi di cifratura, e AWS Secrets Manager per la conservazione sicura delle credenziali applicative. L’elaborazione computazionale è affidata a istanze EC2, sia tramite il servizio “EC2 – Other” che “Amazon Elastic Compute Cloud – Compute”, mentre la distribuzione del traffico e l'alta disponibilità sono garantite da Amazon Elastic Load Balancing.

Per la gestione della localizzazione e dei dati geografici viene utilizzato Amazon Location Service. I dati applicativi sono gestiti tramite Amazon Relational Database Service (RDS), mentre la comunicazione asincrona tra componenti avviene tramite Amazon Simple Notification Service (SNS) e Amazon Simple Queue Service (SQS). Lo storage oggetti è affidato ad Amazon Simple Storage Service (S3), e la rete privata virtuale è gestita tramite Amazon Virtual Private Cloud (VPC). Il monitoraggio e la raccolta delle metriche sono implementati con Amazon CloudWatch. Infine, sono abilitati anche servizi accessori come “Tax” per la gestione della fatturazione e degli aspetti fiscali.

Questa configurazione riflette una tipica architettura cloud moderna, orientata alla scalabilità, alla sicurezza e alla separazione dei compiti tra i vari servizi AWS.




\section{Implementazione del Modello Zero Trust e del Principio del Minimo Privilegio}
\label{sec:zero-trust-implementation}

Come introdotto nella sezione \ref{sec:principi-cybersecurity}, il modello \textbf{Zero Trust} rappresenta un cambiamento paradigmatico rispetto alla sicurezza tradizionale basata sul perimetro. Anziché assumere fiducia implicita per le entità all'interno della rete aziendale, il principio cardine è "non fidarsi mai, verificare sempre" (\textit{never trust, always verify}). Ogni richiesta di accesso a una risorsa, indipendentemente dalla sua origine, deve essere esplicitamente autenticata, autorizzata e monitorata. Questo approccio mira a minimizzare la superficie d'attacco e a contenere l'impatto di eventuali compromissioni, risultando particolarmente critico per proteggere la \textit{business continuity} aziendale. Ritengo che l'adozione di questo principio sia particolarmente rilevante nel contesto delle startup, caratterizzate da ambienti operativi dinamici e altamente flessibili. Le startup presentano peculiarità che amplificano l'esigenza di un solido framework di sicurezza:

\begin{itemize}
    \item \textbf{Instabilità relazionale:} Le relazioni professionali nelle startup possono deteriorarsi rapidamente, sia a livello dirigenziale che operativo. Secondo un'analisi di CB Insights, i conflitti interni tra fondatori rappresentano una delle principali cause di fallimento delle startup, incidendo per circa il 13\% dei casi esaminati \cite{CBInsights2023}. 
    \item \textbf{Rischio di attacchi interni:} La fragilità dei rapporti aumenta la probabilità di attacchi da parte di ex-collaboratori con intenti vendicativi. Secondo il "2023 Data Breach Investigations Report" di Verizon, circa il 20\% delle violazioni di dati coinvolge insider con accessi privilegiati \cite{Verizon2023}.
    \item \textbf{Infrastrutture di sicurezza inadeguate:} Le startup, per limitazioni di risorse e focus prevalente sullo sviluppo del prodotto, spesso non dispongono di infrastrutture di sicurezza robuste. Un rapporto di Ponemon Institute evidenzia che le piccole organizzazioni hanno una probabilità tre volte maggiore di subire attacchi informatici rispetto alle grandi imprese, proprio a causa di investimenti insufficienti in sicurezza \cite{Ponemon2023}.
\end{itemize}
Questa sezione illustra come i principi Zero Trust possano essere tradotti in misure di sicurezza concrete all'interno dell'infrastruttura cloud di una startup, con specifico riferimento all'ambiente AWS. Ci concentreremo in particolare sulla gestione delle identità e degli accessi, un pilastro fondamentale per qualsiasi architettura Zero Trust, e sulla sua stretta interconnessione con il \textbf{Principio del Minimo Privilegio (Principle of Least Privilege - PoLP)}.

\subsection{Gestione delle Identità e degli Accessi (IAM) come Pilastro di Zero Trust in AWS}
\label{subsec:iam-zero-trust}

L'infrastruttura ospitata su un Cloud Service Provider (CSP) come AWS è un asset critico per una startup fintech. Essa contiene dati sensibili degli utenti e ospita i servizi essenziali (endpoint API, istanze EC2 per server applicativi, networking VPC, ecc.) che ne garantiscono l'operatività. La protezione di queste risorse inizia dalla gestione rigorosa di chi può accedervi e cosa può fare. \textbf{AWS Identity and Access Management (IAM)} è il servizio centrale per implementare questi controlli e costituisce una base imprescindibile per un modello Zero Trust.

Una delle prime e più critiche aree di intervento riguarda l' \textbf{account root di AWS}. Questo account possiede privilegi illimitati sull'intero ambiente AWS e rappresenta, di conseguenza, un obiettivo di altissimo valore per gli attaccanti e una fonte significativa di rischio operativo se usato impropriamente. Un'implementazione Zero Trust richiede misure stringenti per l'account root:
\begin{itemize}
    \item \textbf{Limitazione Estrema dell'Uso:} L'accesso come utente root deve essere evitato per le operazioni quotidiane e riservato esclusivamente a quelle poche attività che lo richiedono obbligatoriamente (es. modifica delle informazioni di fatturazione, chiusura dell'account, modifica dei piani di supporto).
    \item \textbf{Protezione Robusta delle Credenziali:} La password deve essere estremamente complessa e, soprattutto, l'\textbf{Autenticazione a Più Fattori (MFA)} deve essere \textit{sempre} abilitata e richiesta per l'accesso root.
    \item \textbf{Monitoraggio Continuo:} Ogni azione eseguita tramite l'account root deve essere tracciata e monitorata tramite servizi come AWS CloudTrail, generando allarmi per qualsiasi utilizzo.
\end{itemize}

Per le attività amministrative e operative ordinarie, il modello Zero Trust impone l'utilizzo di \textbf{utenti e ruoli IAM} configurati secondo il \textbf{Principio del Minimo Privilegio (PoLP)}. Come descritto nella sezione \ref{sec:principi-cybersecurity}, questo principio stabilisce che a un'entità (utente, servizio, applicazione) debbano essere concesse \textit{esclusivamente} le autorizzazioni minime indispensabili per svolgere le proprie funzioni legittime, e non un permesso di più. Ad esempio, un'applicazione che necessita solo di leggere oggetti da un bucket S3 dovrebbe avere un ruolo IAM con solo il permesso `s3:GetObject` su quel bucket specifico, invece di permessi generici su S3 o, peggio, permessi amministrativi.

\subsubsection{Sinergia tra Principio del Minimo Privilegio (PoLP) e Zero Trust}
\label{subsubsec:polp-zerotrust-correlation}

Il Principio del Minimo Privilegio non è solo una buona pratica di sicurezza a sé stante, ma è intrinsecamente legato e \textbf{fondamentale per il successo di un'architettura Zero Trust}. La loro sinergia si manifesta in diversi modi:

\begin{itemize}
    \item \textbf{Riduzione della Superficie d'Attacco:} Limitando strettamente le azioni consentite a ciascuna identità, PoLP riduce l'insieme delle operazioni che un attaccante potrebbe eseguire anche riuscendo a compromettere le credenziali di quell'identità. La verifica dell'identità (Zero Trust) è necessaria ma non sufficiente; i privilegi limitati (PoLP) ne circoscrivono le capacità.
    \item \textbf{Limitazione del Raggio d'Esplosione (\textit{Blast Radius})}:** In caso di compromissione o errore, i danni potenziali sono confinati. Un utente o servizio con privilegi minimi non può accedere o modificare risorse al di fuori del suo ambito operativo ristretto, limitando il movimento laterale dell'attaccante e l'impatto dell'incidente.
    \item \textbf{Applicazione della Verifica Esplicita:} Implementare PoLP costringe a definire policy di accesso granulari e intenzionali, basate sulle reali necessità operative. Questo si allinea perfettamente con la richiesta di Zero Trust di basare ogni decisione di accesso su policy esplicite e dinamiche, piuttosto che su autorizzazioni ampie o ereditate implicitamente.
    \item \textbf{Miglioramento del Controllo e dell'Auditabilità:} Policy di accesso minimali e specifiche sono più facili da comprendere, gestire e verificare. Ciò semplifica l'audit della postura di sicurezza e la dimostrazione della conformità, permettendo di attestare che gli accessi sono effettivamente limitati come richiesto dal modello Zero Trust.
\end{itemize}
\section{Analisi dell'attuale implementazione di IAM}
\subsection{Configurazione degli Utenti e Ruoli}

L'analisi della struttura IAM esistente rivela la presenza di tre utenti principali: \textbf{Andrea Pasini} (CTO), \textbf{Andrea Ferraboli}, e \textbf{Matteo Giuntoni}. Entrambi gli utenti Ferraboli e Giuntoni dispongono della policy `AdministratorAccess`, concedendo privilegi equivalenti a quelli dell'account root. L'utente Pasini, invece, opera direttamente come root, con la capacità di modificare o eliminare qualsiasi risorsa AWS senza restrizioni. Questa configurazione viola il \textbf{principio del minimo privilegio (PoLP)} e il principio di \textbf{zero trust}, esponendo l'infrastruttura a rischi di errore umano o attacchi interni\cite{ref5}.

Un esame dettagliato delle policy associate mostra l'assenza di \textbf{condizioni contestuali} (es. limitazioni geografiche o orarie) e l'utilizzo esclusivo di policy gestite da AWS, senza personalizzazioni per ridurre i permessi alle effettive necessità operative\cite{ref6}. Ad esempio, l'utente `finanz-backend` possiede `AmazonS3FullAccess`, sebbene le sue funzioni richiedano solo operazioni di lettura su bucket specifici.

\subsection{Criticità Identificate}

1. \textbf{Account Root Non Protetto}: L'account root non utilizza MFA hardware, affidandosi esclusivamente a credenziali statiche\cite{ref3}. Ciò espone a rischi di compromissione tramite phishing o credential stuffing.
2. \textbf{Privilegi Eccessivi per Utenti IAM}: L'assegnazione indiscriminata di `AdministratorAccess` a utenti non root crea superfici di attacco ridondanti. L'utente Pasini, in qualità di root, può eludere qualsiasi restrizione applicata tramite policy IAM\cite{ref2}.
3. \textbf{Mancanza di Meccanismi di Emergenza}: Non sono presenti account "break glass" per il ripristino dell'accesso in scenari di compromissione dell'IdP o lockout accidentale\cite{ref4}.
4. \textbf{Assenza di Monitoring Granulare}: Le policy non integrano logiche di auditing in tempo reale per azioni critiche (es. terminazione di istanze EC2 o modifiche alle regole di sicurezza)\cite{ref7}.

\subsection{Violazioni delle Best Practice AWS}

L'implementazione corrente confligge con multiple raccomandazioni del framework \textbf{AWS Foundational Security Best Practices}:

- \textbf{FSBP IAM-1}: Mancanza di MFA hardware per il root\cite{ref3}.
- \textbf{FSBP IAM-7}: Policy con privilegi non limitati al minimo necessario\cite{ref5}.
- \textbf{FSBP IAM-8}: Assenza di allineamento tra ruoli IAM e responsabilità organizzative\cite{ref2}.
\section{Implementazione delle Migliorie Proposte alla Gestione IAM}
\label{sec:implementazione_migliorie}

In questa sezione vengono dettagliate le strategie operative per rafforzare la sicurezza dell'ambiente AWS, basate sulle proposte di miglioramento precedentemente delineate. L'obiettivo è implementare controlli robusti seguendo il principio del minimo privilegio (\emph{least privilege}) e le migliori pratiche di settore.

\subsection{Ristrutturazione della Gerarchia degli Accessi}

Una gestione sicura parte dalla protezione dell'account root e dalla segmentazione granulare dei permessi.

\subsubsection{Revisione e Limitazione dell'Account Root}

L'account root possiede privilegi illimitati e il suo utilizzo deve essere strettamente confinato ad operazioni specifiche che lo richiedono esplicitamente \cite{aws:iam:bestpractices}.
\begin{enumerate}
    \item \textbf{Creazione di un Utente Amministrativo Dedicato}: L'utente Andrea Pasini verrà rimosso dall'accesso diretto come utente root. Verrà creato un utente IAM dedicato (es. `andrea.pasini`) associato a un ruolo amministrativo con permessi circoscritti (es. `CTO-AdminRole`). Questo ruolo dovrebbe garantire visibilità sull'infrastruttura ma limitare modifiche critiche, specialmente in produzione.
    \item \textbf{Policy di Restrizione per il Ruolo Amministrativo}: Al ruolo `CTO-AdminRole` verrà associata una policy IAM che neghi esplicitamente azioni distruttive su risorse critiche taggate come \enquote{produzione}. Un esempio di statement di negazione (\texttt{Deny}) è il seguente:
    \begin{lstlisting}[style=json, caption={Policy IAM per negare eliminazioni in produzione}, label=lst:deny-prod-delete]
{
  "Version": "2012-10-17",
  "Statement": [
    {
      "Sid": "DenyProdResourceDeletion",
      "Effect": "Deny",
      "Action": [
        "ec2:TerminateInstances",
        "rds:DeleteDBInstance",
        "s3:DeleteBucket",
        "vpc:DeleteVpc"
      ],
      "Resource": "*",
      "Condition": {
        "StringEquals": {
          "aws:ResourceTag/Environment": "prod"
        }
      }
    }
  ]
}
    \end{lstlisting}
    Questo approccio implementa un controllo preventivo fondamentale \cite{aws:iam:boundaries}.
    \item \textbf{Abilitazione MFA Hardware per l'Account Root}: L'account root deve essere protetto con un dispositivo Multi-Factor Authentication (MFA) hardware (es. YubiKey), come raccomandato dalle best practice di sicurezza AWS \cite{clouddefense:mfa}. Tale dispositivo sarà custodito fisicamente dal CEO o in una cassetta di sicurezza designata. Qualsiasi accesso all'account root richiederà l'uso fisico del token \cite{saraswat:breakglass}.
\end{enumerate}

\subsubsection{Segmentazione dei Ruoli tramite Permission Boundaries}

Per prevenire l'escalation involontaria o malevola dei privilegi, verranno implementate le \emph{permission boundaries} su tutti i ruoli IAM, inclusi quelli amministrativi. Un boundary definisce il perimetro massimo delle azioni consentite, indipendentemente dalle policy di autorizzazione associate all'entità \cite{aws:iam:boundaries}.
\begin{itemize}
    \item \textbf{Definizione del Boundary}: Un esempio di boundary potrebbe limitare le azioni a specifici servizi o a sole operazioni di lettura, garantendo che anche ruoli con policy ampie (come `AdministratorAccess`, sebbene sconsigliato) non possano eccedere i limiti imposti.
    \begin{lstlisting}[style=json, caption={Esempio di Permission Boundary restrittiva}, label=lst:permission-boundary]
{
  "Version": "2012-10-17",
  "Statement": [
    {
      "Sid": "AllowOnlySpecificServices",
      "Effect": "Allow",
      "Action": [
        "ec2:*",
        "rds:*",
        "s3:List*",
        "iam:List*",
        "cloudwatch:Describe*",
        "lambda:*"
      ],
      "Resource": "*"
    },
    {
       "Sid": "DenyIAMModificationOutsideBoundary",
       "Effect": "Deny",
       "Action": [
          "iam:AttachUserPolicy",
          "iam:AttachRolePolicy",
          "iam:PutUserPolicy",
          "iam:PutRolePolicy",
          "iam:CreatePolicy",
          "iam:CreatePolicyVersion",
          "iam:SetDefaultPolicyVersion",
          "iam:DeletePolicy",
          "iam:DeletePolicyVersion",
          "iam:DetachUserPolicy",
          "iam:DetachRolePolicy"
        ],
        "Resource": "*",
        "Condition": {
           "StringNotLike": {
              "iam:PermissionsBoundary": "arn:aws:iam::123456789012:policy/YourBoundaryPolicyName"
           }
        }
    }
  ]
}
    \end{lstlisting}
    \item \textbf{Applicazione Sistematica}: Ogni nuovo ruolo IAM creato dovrà avere un boundary associato come prerequisito.
\end{itemize}

\subsection{Modello Ibrido Aggiornato}
\label{subsec:modello_ibrido_aggiornato}

Il modello di \emph{Identity \& Access Management} (IAM) proposto per la startup fintech prevede \emph{tre gruppi baseline}—\texttt{dev}, \texttt{backend‑dev} e \texttt{admin}—ai quali vengono
assegnati i permessi necessari per le attività ordinarie, e
\emph{quattro ruoli operativi circoscritti} da assumere \emph{on‑demand} via AWS STS con MFA.
L’architettura riduce la \emph{blast‑radius} delle credenziali
e facilita gli audit di conformità (PCI DSS, SOC‑2) in linea con i
principi di \emph{least privilege} e \emph{zero‑trust} \cite{NIST_ZTA,NIST_SP80063,PCI_DSS,DatadogLeastPrivilege}.

%-----------------------------------------------------------------
\subsubsection{Gruppi Baseline}
\label{subsubsec:gruppi_base}

\paragraph{\texttt{dev}}%
Sviluppatori front‑end e full‑stack.  
\begin{itemize}
  \item \textbf{EC2}: avvia, interrompe e termina \emph{solo} le istanze taggate \texttt{Environment=dev};
        nessun diritto sulle istanze di produzione \cite{AWSEC2IAM}.  
  \item \textbf{Elastic Beanstalk}: deploy e \verb|eb deploy| negli ambienti \texttt{dev},
        tramite policy gestita \texttt{AWSElasticBeanstalkFullAccess} limitata con
        \texttt{Condition\{aws:ResourceTag/Environment=dev\}} \cite{AWSEBRole}.  
  \item \textbf{S3}: lettura/scrittura nei bucket \texttt{*-dev}; accesso negato ai bucket \texttt{*-prod} \cite{AWSS3Security}.  
  \item \textbf{Load Balancer}: descrizione (API \texttt{Describe*}) dei load balancer di
        sviluppo; nessuna modifica \cite{AWSELBIAM}.  
  \item \textbf{RDS}: \emph{data‑reader} su cluster Aurora \texttt{dev}; vietate operazioni \texttt{ModifyDBInstance} e \texttt{DeleteDBInstance} \cite{AWSRDSIAM}.  
\end{itemize}

\paragraph{\texttt{backend‑dev}}%
Sviluppatori back‑end con responsabilità di integrazione dati.  
\begin{itemize}
  \item Tutti i permessi del gruppo \texttt{dev}.  
  \item \textbf{RDS}: \emph{data‑writer} su \texttt{dev}; \texttt{QueryEditor} in aurora‑prod tramite
        policy \texttt{rds-db:connect} con tag‑condition che richiede
        approvazione esplicita (\texttt{aws:RequestTag/ChangeId}).  
  \item \textbf{SQS/SNS}: gestione code e topic non‑prod per pipeline event‑driven.  
  \item \textbf{Secrets Manager}: lettura di segreti \texttt{scope=dev} \cite{AWSIAMBestPractices}.  
\end{itemize}

\paragraph{\texttt{admin}}%
Cloud Engineers con controllo continuo dell’infrastruttura.  
\begin{itemize}
  \item \textbf{EC2 e Auto Scaling}: piena gestione, esclusa l’eliminazione di VPC prod.  
  \item \textbf{S3}: modifica dei lifecycle rules e delle policy di replica cross-region.
  
  \item \textbf{Elastic Load Balancing}: creazione, aggiornamento listener e target groups in tutti gli ambienti.
  
  \item \textbf{RDS}: patching, snapshot e \texttt{failover}.
  
  \item \textbf{IAM}: può creare o aggiornare policy \emph{entro} il \texttt{permissions-boundary} globale che impedisce azioni estreme (\texttt{iam:DeleteRolePolicy}, \texttt{organizations:DeleteOrganization}) \cite{AWSPermBoundaries}.
\end{itemize}

%-----------------------------------------------------------------
\subsubsection{Ruoli Operativi Specifici}
\label{subsubsec:ruoli_specifici}

I ruoli sono configurati con durata massima di 1 h e MFA obbligatoria;
i log CloudTrail vengono inviati a un bucket immutabile con
replica cross-region.

\begin{itemize}
  \item \textbf{\texttt{dev‑privileged}} – estende \texttt{dev} per operazioni
        di manutenzione \texttt{non‑prod} (migrate DB, tunning CPU credit);
        azioni limitate a risorse con tag \texttt{Environment=dev}.  
  \item \textbf{\texttt{db‑migration}} – accesso a AWS DMS e permessi
        \texttt{rds:ModifyDBInstance} in produzione durante le finestre di
        maintenance; richiede approvazione Change‑Manager.  
  \item \textbf{\texttt{incident‑responder}} – abilita scaling immediato,
        modifica security‑group, attiva \texttt{ShieldAdvanced} e
        \texttt{WAFv2} sulla WebACL corrente; assumento consentito al gruppo
        \texttt{admin}.  
  \item \textbf{\texttt{breakglass‑admin}} – superset critico conservato in
        account separato, utilizzato solo per \emph{disaster‑recovery}; il
        processo di assunzione è sigillato e monitorato da AWS Config Rules \cite{AWSSTS}.  
\end{itemize}

%-----------------------------------------------------------------
\subsubsection{Mappatura dei Permessi per Servizio}
\label{subsubsec:mappa_servizi}

\begin{description}
  \item[EC2] \texttt{dev}: \texttt{Start/Stop} istanze dev; \texttt{backend‑dev}: idem + \texttt{DescribeImages}; \texttt{admin}: pieno controllo, esclusa
        \texttt{DeleteVpc}.  
  \item[Elastic Beanstalk] \texttt{dev}: deploy su env dev; \texttt{backend‑dev}: deploy + \texttt{eb config save}; \texttt{admin}: gestione template, gestione
        application‑versions prod \cite{AWSEBRole}.  
  \item[S3] \texttt{dev}: R/W bucket *-dev; \texttt{backend‑dev}: aggiunge permessi
        \texttt{PutObjectAcl} su \emph{log bucket}; \texttt{admin}:
        \texttt{PutBucketPolicy}, \texttt{PutReplicationConfiguration} \cite{AWSS3Security}.  
  \item[Load Balancer] \texttt{dev}: \texttt{Describe*}; \texttt{backend‑dev}: \texttt{RegisterTargets} nei target‑group dev; \texttt{admin}: \texttt{CreateLoadBalancer}, \texttt{ModifyLoadBalancerAttributes} su tutti gli ambienti \cite{AWSELBIAM}.  
  \item[RDS] \texttt{dev}: \texttt{rds-db:connect} read‑only dev; \texttt{backend‑dev}:
        \texttt{ExecuteStatement} via Data API; \texttt{admin}:
        \texttt{CreateDBSnapshot}, \texttt{StartExportTask}, \texttt{FailoverDBCluster} \cite{AWSRDSIAM}.  
\end{description}

L’approccio \emph{tag‑based ABAC} riduce la necessità di policy
puntuali e consente un’espansione lineare degli ambienti (dev, staging,
prod) \cite{AWSEC2IAM,AWSELBIAM}.

%-----------------------------------------------------------------
\subsubsection{Motivazione per il Contesto Fintech}
\label{subsubsec:motivazione_fintech}
Le startup fintech devono coniugare rapidità di rilascio e requisiti di
security-compliance (PCI DSS, PSD2, ISO 27001).  
Separare i privilegi comuni (gruppi) da quelli elevati (ruoli)
garantisce che le \emph{pipeline CI/CD} non abbiano necessità di
credenziali amministrative permanenti—costante
\emph{pain-point} nei data-breach recenti \cite{MediumIAMGuide}.  
La validità temporale delle credenziali STS e la \emph{least-privilege
right-sizing} automatizzata con Access Analyzer riducono il rischio di
accessi persistenti compromessi \cite{DatadogLeastPrivilege,AWSIAMBestPractices}.

%-----------------------------------------------------------------
\subsubsection{Procedimento di Implementazione}
\label{subsubsec:procedura}

\begin{enumerate}
  \item Definire il \texttt{permissions‑boundary} globale che vieta azioni
        ad alto impatto (\texttt{organizations:*}, \texttt{iam:SetDefaultPolicyVersion}) \cite{AWSPermBoundaries}.  
  \item Versionare in Git le policy dei gruppi (\verb|iam/groups/|) e dei
        ruoli (\verb|iam/roles/|) come JSON o
        moduli Terraform; abilitare \verb|terraform plan| in CI.  
  \item Abilitare AWS Identity Center (SSO) collegato ad Okta/Azure AD e
        mappare gli \emph{entitlement} sugli ARNs dei gruppi.  
  \item Automatizzare la \emph{workflow approval} per i ruoli con AWS Step
        Functions + EventBridge + Slack.  
  \item Inviare i log CloudTrail a un bucket S3 con
        \texttt{ObjectLock = GOVERNANCE} e replica in un account
        differente (\textit{security‑hub}).  
  \item Eseguire un \emph{access‑review} trimestrale utilizzando i report
        di Access Analyzer per ridurre i permessi non utilizzati \cite{DatadogLeastPrivilege}.  
\end{enumerate}


\subsection{Introduzione di un Break Glass Account}

Per scenari di emergenza in cui gli accessi amministrativi standard non fossero disponibili o sufficienti, verrà istituito un account \emph{Break Glass} dedicato, seguendo le linee guida di architetture sicure \cite{saraswat:breakglass}.
\begin{enumerate}
    \item \textbf{Configurazione Account}: Creare un nuovo account AWS all'interno dell'Organization esistente, isolato operativamente.
    \item \textbf{Utente e Ruolo di Emergenza}: All'interno di questo account, creare un utente IAM (es. `BreakGlassUser`) protetto da MFA hardware e un ruolo IAM (es. `BreakGlassAdminRole`) con la policy gestita `AdministratorAccess`. L'accesso a questo utente/ruolo sarà strettamente controllato.
    \item \textbf{Procedura di Attivazione}: L'utilizzo del Break Glass Account richiederà un'approvazione formale e documentata da parte di almeno due figure chiave (es. CEO e CTO). Le credenziali (password e MFA) saranno conservate in luoghi sicuri e separati.
    \item \textbf{Monitoraggio e Lockdown Automatico}: Implementare un meccanismo di notifica immediata (es. via CloudWatch Events e SNS) all'attivazione dell'account Break Glass. Un processo automatizzato (es. AWS Lambda triggerata da CloudWatch Event) potrebbe limitare la validità della sessione o restringere i permessi dopo un periodo predefinito (es. 8 ore), ad esempio applicando una policy restrittiva come boundary temporaneo.
    \begin{lstlisting}[style=python, caption={Esempio Lambda per limitare utente Break Glass (concettuale)}, label=lst:breakglass-lambda]
import boto3
import os

IAM_CLIENT = boto3.client('iam')
BREAK_GLASS_USERNAME = os.environ.get('BREAK_GLASS_USER')
RESTRICTIVE_POLICY_ARN = os.environ.get('RESTRICTIVE_POLICY_ARN') # Es: AWSCloudTrailReadOnlyAccess

def lambda_handler(event, context):
    if not BREAK_GLASS_USERNAME or not RESTRICTIVE_POLICY_ARN:
        print("Error: Environment variables not set.")
        return

    try:
        print(f"Applying restrictive boundary {RESTRICTIVE_POLICY_ARN} to user {BREAK_GLASS_USERNAME}")
        IAM_CLIENT.put_user_permissions_boundary(
            UserName=BREAK_GLASS_USERNAME,
            PermissionsBoundary=RESTRICTIVE_POLICY_ARN
        )
        print(f"Successfully applied boundary.")
        # Aggiungere notifiche (es. SNS)
    except Exception as e:
        print(f"Error applying boundary: {e}")
        # Gestire l'errore / inviare notifica di fallimento
\end{lstlisting}
\end{enumerate}

\subsection{Implementazione di Politiche di Sicurezza Avanzate}

Verranno utilizzate policy a livello di Organization e credenziali temporanee per rafforzare ulteriormente la postura di sicurezza.

\subsubsection{Service Control Policies (SCPs) a Livello Organizzativo}

Le SCPs verranno applicate all'intera AWS Organization (o a specifiche Organizational Units - OUs) per imporre vincoli di sicurezza non aggirabili, nemmeno dall'amministratore locale dell'account.
\begin{itemize}
    \item \textbf{Impedire la Disattivazione di Controlli Chiave}: Applicare una SCP per negare azioni come l'eliminazione dei trail di CloudTrail o la disabilitazione di AWS Config.
    \begin{lstlisting}[style=json, caption={SCP per prevenire l'eliminazione di CloudTrail}, label=lst:scp-deny-cloudtrail-delete]
{
  "Version": "2012-10-17",
  "Statement": [
    {
      "Sid": "DenyDeleteCloudTrail",
      "Effect": "Deny",
      "Action": [
        "cloudtrail:DeleteTrail",
        "cloudtrail:StopLogging"
       ],
      "Resource": "*"
    }
  ]
}
    \end{lstlisting}
    \item \textbf{Restrizione Geografica}: Limitare l'utilizzo delle regioni AWS a quelle approvate (es. `eu-central-1`, `eu-south-1`, `eu-west-1`) per motivi di compliance (es. GDPR) e per ridurre la superficie di attacco \cite{awsbuilders:scps}.
    \begin{lstlisting}[style=json, caption={SCP per limitare le regioni utilizzabili}, label=lst:scp-region-restriction]
{
  "Version": "2012-10-17",
  "Statement": [
    {
      "Sid": "DenyNonApprovedRegions",
      "Effect": "Deny",
      "NotAction": [
          "iam:*",
          "organizations:*",
          "route53:*",
          "budgets:*",
          "waf:*",
          "cloudfront:*",
          "globalaccelerator:*",
          "support:*"
       ],
      "Resource": "*",
      "Condition": {
        "StringNotEquals": {
          "aws:RequestedRegion": [
             "eu-central-1",
             "eu-south-1",
             "eu-west-1",
             "us-east-1" %*{*) %*{*) necessario per alcuni servizi globali
          ]
        },
        "ArnNotLike": {
            "aws:PrincipalARN": "arn:aws:iam::*:role/OrganizationAccountAccessRole" %*{*) %*{*) Esempio ruolo escluso
         }
       }
    }
  ]
}
    \end{lstlisting}
\end{itemize}

\subsubsection{Utilizzo Sistematico di Credenziali Temporanee (STS)}

Le access key statiche a lunga durata rappresentano un rischio significativo se compromesse \cite{kazi:leastprivilege}. Verrà promossa e, ove possibile, imposta la sostituzione delle chiavi statiche con credenziali temporanee ottenute tramite il servizio AWS Security Token Service (STS).
\begin{itemize}
    \item \textbf{Accesso Umano}: Gli utenti IAM accederanno alla console AWS o alla CLI assumendo ruoli predefiniti, ottenendo credenziali temporanee valide per la durata della sessione.
    \item \textbf{Accesso Applicativo}: Le applicazioni (es. `finanz-backend`) in esecuzione su EC2, ECS, EKS o Lambda utilizzeranno i ruoli IAM associati alle risorse di calcolo per ottenere automaticamente credenziali temporanee, eliminando la necessità di gestire chiavi statiche nel codice o nelle configurazioni.
    \item \textbf{Script e Automazioni}: Gli script che necessitano di interagire con le API AWS dovranno utilizzare comandi come `aws sts assume-role` per ottenere credenziali temporanee legate a un ruolo specifico, limitato al principio del minimo privilegio.
    \begin{lstlisting}[style=bash, caption={Ottenere credenziali temporanee tramite STS AssumeRole}, label=lst:sts-assume-role]
# L'utente/servizio assume un ruolo con permessi specifici (es. S3 ReadOnly)
aws sts assume-role \
    --role-arn arn:aws:iam::123456789012:role/S3ReadOnlyForBackend \
    --role-session-name FinanzBackendReadSession
    
# Le credenziali restituite (AccessKeyId, SecretAccessKey, SessionToken)
# vengono usate per le chiamate API successive.
    \end{lstlisting}
\end{itemize}

\subsection{Implementazione di un Sistema di Approvazione a Due Fasi (Opzionale)}

Per operazioni ad alto impatto (es. eliminazione di bucket S3 contenenti dati critici, modifiche a gruppi di sicurezza di produzione), si può valutare l'introduzione di un workflow di approvazione multi-persona tramite AWS Step Functions.
\begin{enumerate}
    \item \textbf{Avvio del Workflow}: Un utente avvia l'operazione tramite un'interfaccia dedicata (es. Lambda function, API Gateway) che attiva la Step Function.
    \item \textbf{Richiesta di Approvazione}: La Step Function invia notifiche (es. via Amazon SNS a email o SMS) ai responsabili designati.
    \item \textbf{Approvazione Multipla}: Il workflow attende l'approvazione da parte di due (o più) amministratori distinti. L'approvazione può avvenire tramite un link in email, un'API o la console Step Functions.
    \item \textbf{Esecuzione Condizionata}: Solo a seguito delle approvazioni richieste, la Step Function esegue l'azione critica (es. invocando una Lambda function con i permessi necessari).
    \item \textbf{Auditing}: Ogni fase del processo (richiesta, approvazioni, esito) viene registrata su un database di auditing (es. DynamoDB) e/o CloudTrail per tracciabilità completa.
\end{enumerate}
Questa misura aggiunge un livello di controllo deliberato su azioni irreversibili o ad alto rischio.


\section{Progettazione di una Rete Sicura con Amazon VPC}
\label{sec:vpc-design}

La base di qualsiasi infrastruttura su AWS è la rete virtuale definita tramite \textbf{Amazon Virtual Private Cloud (VPC)}. Il VPC permette di creare un ambiente di rete logicamente isolato all'interno del cloud AWS, su cui si ha pieno controllo (range di indirizzi IP, creazione di subnet, configurazione di route table e network gateway). Una progettazione VPC sicura è il primo livello di difesa.

\subsection{Subnet Pubbliche e Private}
\label{subsec:subnets}
Una pratica fondamentale è la suddivisione del VPC in \textbf{subnet pubbliche} e \textbf{subnet private}, distribuite su diverse Availability Zones per alta disponibilità.
\begin{itemize}
    \item Le \textbf{subnet pubbliche} hanno una rotta diretta verso l'Internet Gateway (IGW) del VPC e sono tipicamente utilizzate per risorse che devono essere direttamente accessibili da Internet, come i web server o i load balancer pubblici.
    \item Le \textbf{subnet private} non hanno una rotta diretta verso l'IGW. Le risorse in queste subnet (es. application server, database, code build server) non sono direttamente raggiungibili da Internet, migliorando significativamente la sicurezza. Possono accedere a Internet in uscita (es. per scaricare patch o chiamare API esterne) tramite un \textit{NAT Gateway} o \textit{NAT Instance} posizionato nella subnet pubblica.
\end{itemize}
Per una startup fintech, i server applicativi che elaborano transazioni e i database contenenti dati sensibili dei clienti dovrebbero \textbf{sempre risiedere in subnet private}.

\subsection{Gruppi di Sicurezza e Network ACL}
\label{subsec:sg-nacl}
Il controllo del traffico all'interno del VPC è affidato a due meccanismi principali:
\begin{itemize}
    \item \textbf{Gruppi di Sicurezza (Security Groups - SG)}:** Agiscono come un firewall a livello di istanza (EC2, RDS, etc.). Sono \textit{stateful}, il che significa che se il traffico in uscita è permesso, il traffico di ritorno corrispondente è automaticamente permesso, indipendentemente dalle regole in ingresso. Si configurano specificando le porte e i protocolli permessi in ingresso e in uscita, tipicamente referenziando altri SG o specifici indirizzi IP/range. La best practice è applicare il principio del minimo privilegio: permettere solo il traffico strettamente necessario (es. permettere solo la porta 443 da un Application Load Balancer al SG dei web server).
    \item \textbf{Network Access Control Lists (Network ACLs)}:** Agiscono come un firewall a livello di subnet. Sono \textit{stateless}, quindi è necessario definire regole esplicite sia per il traffico in ingresso che per quello in uscita. Hanno regole numerate (da 1 a 32766) che vengono valutate in ordine, e la prima regola che corrisponde determina l'azione (ALLOW o DENY). Offrono un livello di difesa aggiuntivo, utile per bloccare specifici IP malevoli a livello di subnet o per applicare regole di rete più ampie. Di default, permettono tutto il traffico.
\end{itemize}
È buona norma usare entrambi: SG per controlli granulari a livello di istanza e NACL per regole più ampie a livello di subnet.

\subsection{NAT Gateway e Accesso a Internet}
\label{subsec:nat-gateway}
Come accennato, le istanze in subnet private necessitano di un meccanismo per accedere a Internet per aggiornamenti o chiamate API. AWS offre il servizio gestito \textbf{NAT Gateway}, che è altamente disponibile e scalabile. Creando un NAT Gateway in una subnet pubblica e configurando le route table delle subnet private affinché instradino il traffico destinato a Internet (0.0.0.0/0) verso il NAT Gateway, le istanze private possono comunicare con l'esterno senza avere un IP pubblico direttamente esposto.

\subsection{Connessioni Sicure (Opzionale: VPN/Direct Connect)}
\label{subsec:vpn-directconnect}
Se la startup necessita di connettere in modo sicuro la propria infrastruttura AWS a data center on-premises (raro per startup native cloud, ma possibile) o a reti di partner, AWS offre servizi come \textbf{AWS Site-to-Site VPN} (per creare tunnel IPsec crittografati su Internet) o \textbf{AWS Direct Connect} (per una connessione fisica dedicata e privata tra la rete on-premises e AWS).

\section{Gestione Sicura delle Istanze EC2}
\label{sec:ec2-security}
Le istanze \textbf{Amazon EC2} sono le macchine virtuali su cui spesso girano le applicazioni. La loro sicurezza è cruciale.

\subsection{Scelta delle AMI e Hardening}
\label{subsec:ami-hardening}
\begin{itemize}
    \item \textbf{Utilizzare AMI affidabili:} Partire da Amazon Machine Images (AMI) fornite da AWS o da venditori fidati sul Marketplace. Evitare AMI pubbliche non verificate.
    \item \textbf{Hardening del Sistema Operativo:} Applicare pratiche di hardening: disabilitare servizi non necessari, configurare correttamente il firewall locale (es. iptables/firewalld su Linux, Windows Firewall), applicare regolarmente le patch di sicurezza. AWS Systems Manager Patch Manager può automatizzare il patching.
    \item \textbf{Minimizzare il software installato:} Installare solo il software strettamente necessario per la funzione dell'istanza, riducendo la superficie d'attacco.
\end{itemize}

\subsection{Utilizzo di IAM Roles per EC2}
\label{subsec:iam-roles-ec2}
Questa è una delle pratiche di sicurezza più importanti. \textbf{Mai salvare credenziali AWS statiche (Access Key ID e Secret Access Key) direttamente su un'istanza EC2}. Invece, associare un \textbf{IAM Role} all'istanza al momento del lancio. L'applicazione in esecuzione sull'istanza può quindi ottenere credenziali temporanee tramite il servizio metadati dell'istanza, assumendo i permessi definiti nel ruolo associato. Questo elimina il rischio di esposizione di credenziali a lungo termine. Il ruolo deve seguire il principio del minimo privilegio (es. un'istanza che deve solo leggere da un bucket S3 dovrebbe avere un ruolo con solo permessi `s3:GetObject` su quel bucket).

\subsection{Scalabilità Automatica (Auto Scaling Groups)}
\label{subsec:auto-scaling}
Per garantire disponibilità e gestire picchi di carico, è fondamentale utilizzare \textbf{Auto Scaling Groups (ASG)}. Un ASG gestisce un gruppo di istanze EC2 identiche, assicurando che il numero desiderato di istanze sia sempre in esecuzione. Può aumentare (scale out) o diminuire (scale in) automaticamente il numero di istanze in base a metriche (es. utilizzo CPU, numero di richieste) o a una pianificazione. Gli ASG lavorano tipicamente in congiunzione con un \textbf{Elastic Load Balancer (ELB)} che distribuisce il traffico tra le istanze attive nell'ASG. Questo non solo migliora la disponibilità e la performance, ma anche la resilienza: se un'istanza fallisce, l'ASG la rimpiazza automaticamente.

\section{Protezione dei Dati Sensibili}
\label{sec:data-protection}
In una fintech, la protezione dei dati dei clienti e delle transazioni è di massima priorità. AWS offre diversi strumenti per questo.

\subsection{Crittografia a Riposo e in Transito}
\label{subsec:encryption}
\begin{itemize}
    \item \textbf{Crittografia a Riposo (At Rest)}:** È fondamentale crittografare i dati sensibili quando sono memorizzati. AWS facilita questo:
        \begin{itemize}
            \item \textbf{Amazon S3:} Abilitare la Server-Side Encryption (SSE-S3, SSE-KMS, SSE-C) sui bucket che contengono dati sensibili. SSE-KMS offre maggiore controllo tramite AWS Key Management Service.
            \item \textbf{Amazon EBS:} Abilitare la crittografia sui volumi EBS associati alle istanze EC2.
            \item \textbf{Amazon RDS:} Abilitare la crittografia per i database gestiti.
        \end{itemize}
    \item \textbf{Crittografia in Transito (In Transit)}:** Tutto il traffico contenente dati sensibili (es. API calls, connessioni al database, traffico tra servizi) deve usare protocolli crittografati come TLS/SSL. Configurare i Load Balancer per terminare HTTPS, usare connessioni sicure ai database RDS, e assicurarsi che le chiamate API interne usino HTTPS.
\end{itemize}

\subsection{Gestione delle Chiavi con AWS KMS}
\label{subsec:kms}
\textbf{AWS Key Management Service (KMS)} è un servizio gestito che facilita la creazione e il controllo delle chiavi di crittografia utilizzate per proteggere i dati. Permette di:
\begin{itemize}
    \item Creare e gestire Customer Master Keys (CMKs).
    \item Definire policy di accesso granulari per controllare chi (utenti o ruoli IAM) può usare quali chiavi e per quali operazioni (encrypt, decrypt).
    \item Auditare l'utilizzo delle chiavi tramite AWS CloudTrail.
\end{itemize}
Usare KMS per la crittografia lato server (SSE-KMS) su S3, EBS, RDS, ecc., offre un controllo centralizzato e sicuro sulle chiavi. Per requisiti di sicurezza ancora più elevati, si può considerare \textbf{AWS CloudHSM}.

\subsection{Backup e Disaster Recovery}
\label{subsec:backup-dr}
Avere backup regolari e testati è essenziale per il recupero da errori o attacchi (es. ransomware).
\begin{itemize}
    \item \textbf{AWS Backup:} Un servizio centralizzato per gestire e automatizzare i backup di vari servizi AWS (EBS, RDS, DynamoDB, EFS, etc.). Permette di definire policy di backup (frequenza, retention) e di copiarli in altre regioni per disaster recovery.
    \item \textbf{Snapshot RDS/EBS:} I servizi come RDS e EBS offrono funzionalità di snapshot automatici e manuali.
    \item \textbf{Versioning S3:} Abilitare il versioning sui bucket S3 critici permette di recuperare oggetti cancellati o sovrascritti accidentalmente.
    \item \textbf{Piano di Disaster Recovery (DR)}:** Definire e testare un piano di DR. Questo potrebbe includere strategie come backup cross-region, pilot light, o warm standby, a seconda dell'RTO (Recovery Time Objective) e RPO (Recovery Point Objective) richiesti.
\end{itemize}

\subsection{Sicurezza dei Bucket S3}
\label{subsec:s3-security}
Amazon S3 è ampiamente utilizzato, ma le configurazioni errate sono una causa comune di data breach.
\begin{itemize}
    \item \textbf{Block Public Access:} Abilitare sempre l'impostazione "Block Public Access" a livello di account e/o di bucket, a meno che non ci sia una ragione specifica e valida per l'accesso pubblico.
    \item \textbf{Bucket Policies e IAM Policies:} Usare policy granulari per limitare l'accesso ai bucket solo agli utenti, ruoli o servizi specifici che ne hanno bisogno.
    \item \textbf{S3 Access Points:} Creare punti di accesso specifici per diverse applicazioni o team, ognuno con la propria policy, semplificando la gestione degli accessi su larga scala.
    \item \textbf{Amazon Macie:} Usare questo servizio per scoprire e proteggere dati sensibili (es. PII, credenziali) archiviati in S3.
\end{itemize}

\section{Implementazione di Controlli IAM Efficaci}
\label{sec:iam-implementation}
Come già sottolineato, \textbf{AWS Identity and Access Management (IAM)} è fondamentale per la sicurezza.

\subsection{Principio del Minimo Privilegio}
\label{subsec:least-privilege-impl}
Applicare rigorosamente il principio del minimo privilegio a utenti, gruppi e ruoli IAM. Concedere solo i permessi strettamente necessari per svolgere un compito specifico. Ad esempio, un ruolo per un'applicazione che deve solo scrivere log in CloudWatch Logs necessita solo dei permessi `logs:CreateLogStream` e `logs:PutLogEvents`, non permessi amministrativi generici. Usare le policy condition per restringere ulteriormente l'accesso (es. permettere azioni solo da specifici IP o solo se è attiva l'MFA).

\subsection{Autenticazione a Più Fattori (MFA)}
\label{subsec:mfa-impl}
Richiedere l'uso dell'Autenticazione a Più Fattori (MFA) per \textbf{tutti} gli utenti IAM umani, specialmente per l'utente root dell'account (che dovrebbe essere usato il meno possibile) e per gli utenti con privilegi amministrativi. Questo aggiunge un livello critico di protezione contro il furto di credenziali.

\subsection{Revisione Periodica dei Permessi}
\label{subsec:iam-review}
I permessi tendono ad accumularsi ("privilege creep"). È essenziale rivedere periodicamente (es. trimestralmente) le policy IAM per rimuovere permessi non più necessari. Strumenti come \textbf{AWS IAM Access Analyzer} possono aiutare a identificare permessi eccessivi o risorse condivise esternamente.

\section{Monitoraggio Continuo e Logging}
\label{sec:monitoring-logging}
Non si può proteggere ciò che non si vede. Un monitoraggio e un logging robusti sono essenziali per rilevare attività sospette e rispondere agli incidenti.

\subsection{Abilitazione di CloudTrail e CloudWatch}
\label{subsec:cloudtrail-cloudwatch-enable}
\begin{itemize}
    \item \textbf{AWS CloudTrail:} Abilitare CloudTrail in \textbf{tutte} le regioni. CloudTrail registra quasi tutte le chiamate API effettuate nel tuo account AWS, fornendo una traccia di audit fondamentale ("chi ha fatto cosa, quando e da dove"). Assicurarsi che i log di CloudTrail siano protetti (es. inviati a un bucket S3 dedicato con logging e crittografia abilitati, e opzionalmente integrità dei file di log abilitata).
    \item \textbf{Amazon CloudWatch:} Usare CloudWatch per raccogliere metriche (es. utilizzo CPU, I/O disco, latenza del Load Balancer), log dalle applicazioni e dai sistemi operativi (tramite l'agente CloudWatch), ed eventi.
\end{itemize}

\subsection{Configurazione di Allarmi CloudWatch}
\label{subsec:cloudwatch-alarms}
Non basta raccogliere log e metriche, bisogna agire su di essi. Configurare allarmi CloudWatch per notifiche proattive su condizioni anomale o eventi critici, ad esempio:
\begin{itemize}
    \item Utilizzo elevato di CPU/Memoria/Rete su istanze critiche.
    \item Errori HTTP 5xx sul Load Balancer.
    \item Tentativi di login falliti (filtrando i log).
    \item Modifiche a risorse di sicurezza critiche (es. modifiche a Security Group, NACL, policy IAM) rilevate tramite eventi CloudTrail.
    \item Chiamate API specifiche indicative di potenziale abuso (es. `TerminateInstances` non autorizzate).
\end{itemize}
Gli allarmi possono inviare notifiche a un topic SNS (Simple Notification Service), che può poi inoltrarle via email, SMS, o triggerare funzioni Lambda per azioni automatiche.

\subsection{Utilizzo di AWS Security Hub e GuardDuty}
\label{subsec:security-hub-guardduty}
\begin{itemize}
    \item \textbf{Amazon GuardDuty:} È un servizio di rilevamento delle minacce gestito che monitora continuamente attività malevole o non autorizzate analizzando log VPC Flow Logs, CloudTrail e DNS. Rileva minacce come istanze compromesse usate per mining di criptovalute, accessi anomali da IP malevoli noti, scansioni di porte, ecc. È fondamentale abilitarlo in tutte le regioni pertinenti.
    \item \textbf{AWS Security Hub:} Fornisce una vista centralizzata degli avvisi di sicurezza (findings) provenienti da diversi servizi AWS (GuardDuty, Inspector, Macie, IAM Access Analyzer, Firewall Manager) e da prodotti di partner. Aiuta a prioritizzare e gestire i risultati della sicurezza e a verificare la conformità rispetto a standard come CIS AWS Foundations Benchmark.
\end{itemize}

\section{Automazione con Infrastructure as Code (IaC)}
\label{sec:iac}
Per garantire coerenza, ridurre errori manuali e facilitare la revisione della sicurezza, è fortemente raccomandato gestire l'infrastruttura AWS tramite \textbf{Infrastructure as Code (IaC)}.
\begin{itemize}
    \item \textbf{Strumenti:} Utilizzare strumenti come \textbf{AWS CloudFormation} (nativo AWS) o \textbf{Terraform} (agnostico rispetto al cloud) per definire l'infrastruttura (VPC, istanze, database, policy IAM, etc.) in file di testo (YAML o JSON).
    \item \textbf{Benefici:}
        \begin{itemize}
            \item \textbf{Ripetibilità e Coerenza:} L'infrastruttura può essere deployata in modo identico in diversi ambienti (dev, staging, prod) o regioni.
            \item \textbf{Versionamento:} I file IaC possono essere messi sotto controllo di versione (es. Git), tracciando le modifiche e permettendo rollback.
            \item \textbf{Automazione:} Il deployment e gli aggiornamenti sono automatizzati, riducendo il rischio di errori umani.
            \item \textbf{Audit e Revisione:} È più facile revisionare la configurazione dell'infrastruttura (e quindi la sua postura di sicurezza) analizzando i file di codice piuttosto che navigando nella console AWS.
            \item \textbf{Integrazione con CI/CD:} L'IaC si integra bene nelle pipeline di Continuous Integration/Continuous Deployment per automatizzare anche il provisioning dell'infrastruttura necessaria per le applicazioni.
        \end{itemize}
\end{itemize}
Adottare IaC sin dalle prime fasi aiuta a costruire un'infrastruttura robusta e gestibile nel tempo.

Questo capitolo ha fornito una panoramica delle implementazioni pratiche e delle best practice per costruire e proteggere un'infrastruttura AWS per una startup fintech. Naturalmente, ogni implementazione specifica richiederà ulteriori dettagli e adattamenti in base ai requisiti unici dell'applicazione e del business. I capitoli successivi potrebbero approfondire ulteriormente specifici aspetti come la gestione degli incidenti, i test di penetrazione o l'integrazione di strumenti di terze parti.