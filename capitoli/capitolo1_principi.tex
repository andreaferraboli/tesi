\subsection{Introduzione}
Nell'analisi e nello studio dell'infrastruttura di una startup fintech, per comprendere le possibili implementazioni a livello di sicurezza dobbiamo prima delineare quali siano i principi di cybersecurity a cui ogni startup, fintech o meno, deve attenersi. In questo capitolo verranno analizzati i principi di cybersecurity più importanti e quali sono le principali sfide che un'azienda di piccole dimensioni può affrontare all'inizio del proprio percorso nell'adozione di tali pratiche.

Questo capitolo esplora i principi fondamentali di sicurezza informatica che ogni organizzazione dovrebbe implementare, con particolare attenzione alle sfide uniche che le startup fintech affrontano nell'adozione di tali pratiche. Il settore fintech, caratterizzato da rapida innovazione e gestione di dati finanziari sensibili, presenta un contesto particolarmente critico dove le best practice di sicurezza si scontrano spesso con le esigenze di velocità di sviluppo, risorse limitate e necessità di time-to-market accelerato.

\section{Triade CIA}
La Triade CIA rappresenta i tre pilastri fondamentali dell'information security: confidentiality (riservatezza), integrity (integrità) e availability (disponibilità) \cite{NIST_SP_1800_26}. Questi principi costituiscono la base su cui costruire qualsiasi strategia di sicurezza informatica robusta.

\begin{itemize}
\item \textbf{Confidentiality}: La riservatezza si concentra sul preservare le restrizioni autorizzate sull'accesso e la divulgazione delle informazioni, inclusi i mezzi per proteggere la privacy personale e le informazioni proprietarie \cite{NIST_SP_1800_26}. Questo principio viene generalmente rispettato tramite la crittografia dei dati, sia a riposo (stored data) che in transito (data in transit), controlli di accesso rigorosi, come liste di controllo degli accessi (ACL), autenticazione a più fattori (MFA) e Role-Based Access Control (RBAC).

text
Nel contesto di una startup fintech, l'implementazione della riservatezza presenta sfide significative. L'accesso ai dati dei clienti e alle informazioni finanziarie deve essere rigorosamente controllato, ma i team piccoli e multifunzionali tipici delle startup spesso portano a una condivisione delle credenziali o all'assegnazione di privilegi eccessivi per "far funzionare le cose rapidamente".

La gestione delle chiavi di cifratura rappresenta un'ulteriore complessità: nelle startup dove i ruoli non sono chiaramente definiti, la responsabilità della gestione delle chiavi può essere ambigua, portando potenzialmente a compromissioni della sicurezza.

\item \textbf{Integrity}: L'integrità dei dati comporta la protezione contro modifiche o distruzioni improprie delle informazioni e garantisce la non ripudiabilità e l'autenticità delle informazioni \cite{NIST_SP_1800_26}. Mantenere l'integrità dei dati è essenziale per prevenire la diffusione di informazioni corrotte o ingannevoli, che potrebbero avere gravi ripercussioni in settori critici come quello sanitario o finanziario. Le tecniche utilizzate per preservare l'integrità includono:
\begin{itemize}
    \item Funzioni di hash crittografiche (es. SHA-256) per verificare che i dati non siano stati alterati.
    \item Firme digitali per autenticare l'origine dei dati e garantirne la non modifica.
    \item Controllo delle versioni per tracciare le modifiche e ripristinare versioni precedenti.
    \item Checksum e meccanismi di rilevamento degli errori.
\end{itemize}

Nel contesto di una startup fintech, l'integrità dei dati è uno di quegli aspetti che va ad inficiare la brand reputation della startup stessa, in quanto la fiducia degli stakeholders si basa anche sulla capacità della startup di conservare i dati dei clienti senza distorsioni e di garantire l'accuratezza nelle transazioni di dati nella maniera più professionale possibile.

\item \textbf{Availability}: Questo principio assicura l'accesso affidabile e tempestivo alle informazioni \cite{NIST_SP_1800_26}. Mira a prevenire interruzioni del servizio, sia dovute a guasti tecnici che ad attacchi malevoli come i Denial-of-Service (DoS) o Distributed Denial-of-Service (DDoS). L'indisponibilità può causare interruzioni operative, perdite economiche e danni alla reputazione. Le strategie per garantire un'elevata disponibilità comprendono:
\begin{itemize}
    \item Sistemi ridondanti (hardware, software, reti) per eliminare singoli punti di fallimento (Single Points of Failure - SPOF).
    \item Backup regolari e piani di disaster recovery (DR) e business continuity (BCP).
    \item Tecniche di bilanciamento del carico (load balancing) per distribuire il traffico di rete.
    \item Misure di protezione contro attacchi DoS/DDoS.
\end{itemize}

Nella maggior parte delle startup, l'infrastruttura di base viene sviluppata considerando una capacità di carico massimo limitato, in quanto nei primi periodi di vita dell'azienda non ci si aspetta un elevato numero di utenti. Proprio per questo motivo, l'infrastruttura presenta un punto vulnerabile che può essere sfruttato dagli attaccanti per mettere a repentaglio l'intero sistema, ad esempio con attacchi DoS/DDoS mirati al perimetro aziendale.
\end{itemize}

\section{Difesa in Profondità (Defense in Depth)}
Il principio di difesa in profondità prevede l'implementazione di una stratificazione delle risorse informatiche di protezione \cite{Cyberment}. Questo approccio permette di rallentare la penetrazione di un eventuale attacco esterno, al fine di avere poi il tempo necessario per una efficace reazione protettiva. La strategia di difesa in profondità fornisce la fondazione per una protezione multidimensionale che include tre componenti mutualmente supportive e rinforzanti: (1) architettura resistente alla penetrazione, (2) operazioni di limitazione dei danni, e (3) progettazione per la cyber resilienza e la sopravvivenza \cite{NIST_SP_800_172}.

Nelle startup fintech, l'implementazione della difesa in profondità è spesso compromessa da vincoli di risorse e pressioni temporali. Ad esempio, mentre una soluzione di autenticazione a più fattori (MFA) è essenziale per proteggere l'accesso a dati finanziari sensibili, una startup potrebbe inizialmente implementare solo l'autenticazione basata su password per accelerare l'onboarding degli utenti, pianificando di aggiungere MFA "in un secondo momento" – un momento che potrebbe non arrivare prima che si verifichi un incidente di sicurezza.

La segmentazione della rete, fondamentale per contenere eventuali violazioni, richiede una progettazione accurata dell'infrastruttura. Tuttavia, nelle fasi iniziali, molte startup fintech operano con architetture di rete piatte per semplificare lo sviluppo e ridurre il sovraccarico operativo, oltre a non disporre del capitale umano competente per gestire una tale complessità.

\section{Principio del Minimo Privilegio}
Il principio del minimo privilegio stabilisce che un sistema dovrebbe limitare i privilegi di accesso degli utenti (o dei processi che agiscono per conto degli utenti) al minimo necessario per svolgere le attività assegnate \cite{NIST_Glossary}. Questo principio dichiara che un'architettura di sicurezza è progettata in modo che a ciascuna entità siano concesse le minime autorizzazioni e risorse di sistema necessarie per svolgere la propria funzione \cite{NIST_Glossary}.

Nelle startup fintech, applicare il principio del minimo privilegio presenta sfide uniche. La cultura focalizzata sulla velocità d'esecuzione spinge spesso a trascurare la sicurezza granulare degli accessi. È forte la tentazione di assegnare privilegi amministrativi ampi e generici per accelerare lo sviluppo, piuttosto che investire tempo nella configurazione di permessi specifici per ogni compito.

Un esempio comune è concedere a tutti gli sviluppatori accesso completo al database di produzione durante la creazione di una nuova dashboard, invece di limitare ciascuno alle sole tabelle o operazioni strettamente necessarie. Sebbene sembri una scorciatoia efficiente, questa pratica crea vulnerabilità critiche: la compromissione di un singolo account può esporre una quantità sproporzionata di dati sensibili, amplificando enormemente i danni di una violazione.

\section{Separazione dei Compiti (Separation of Duties)}
La separazione dei compiti include la divisione delle funzioni di missione o business e le funzioni di supporto tra diverse persone o ruoli, conducendo funzioni di supporto al sistema con individui diversi, e assicurando che il personale di sicurezza che amministra le funzioni di controllo degli accessi non amministri anche le funzioni di audit \cite{OSCAL_Content}. Poiché le violazioni della separazione dei compiti possono estendersi a sistemi e domini di applicazioni, le organizzazioni considerano l'interezza dei sistemi e dei componenti del sistema quando sviluppano politiche sulla separazione dei compiti \cite{OSCAL_Content}.

Nelle startup fintech, dove i team sono piccoli e i ruoli spesso sovrapposti, questo principio è particolarmente difficile da attuare. Ad esempio, in una startup che sviluppa una piattaforma di prestiti P2P, potrebbe esserci un solo ingegnere responsabile sia dell'implementazione del sistema di scoring del credito sia della configurazione dei controlli di sicurezza sullo stesso sistema. Questa concentrazione di responsabilità crea un rischio intrinseco: errori o azioni malevole potrebbero passare inosservati senza un secondo paio di occhi che verifichi il lavoro.

\section{Zero Trust}
Il modello Zero Trust si basa sul concetto che un'organizzazione non dovrebbe fidarsi automaticamente di nulla sia all'interno che all'esterno dei suoi perimetri e deve verificare tutto ciò che tenta di connettersi ai suoi sistemi prima di concedere l'accesso \cite{NIST_SP_800_207}. Zero Trust è una risposta evoluta alle tendenze che includono la migrazione delle risorse di lavoro verso ambienti cloud, lavoratori che operano da dispositivi mobili ovunque si trovino e una crescente collaborazione tra organizzazioni \cite{NIST_SP_800_207}.

Per una startup fintech, l'adozione rigorosa del principio di Zero Trust può rivelarsi particolarmente gravosa. Nelle fasi iniziali, è frequente che l'intera infrastruttura sia gestita da una sola persona, con responsabilità sia di sviluppo sia di amministrazione di rete: questo crea un unico punto di falla, amplificando il rischio di errori di configurazione o di accesso non autorizzato.

Inoltre, le limitate risorse economiche e umane possono rendere difficoltoso implementare soluzioni avanzate di micro-segmentazione, sistemi di Identity and Access Management (IAM) complessi e piattaforme di monitoraggio continuo. Infine, la mancanza di separazione dei compiti e di revisioni periodiche rende più probabile la persistenza di permessi eccessivi o non aggiornati, esponendo i sistemi a potenziali attacchi laterali e perdite di dati sensibili.

\section{Economia del Meccanismo (Economy of Mechanism)}
Il principio di Economia del Meccanismo, formulato da Saltzer e Schroeder, prescrive la progettazione di meccanismi di sicurezza caratterizzati dalla massima semplicità e da dimensioni contenute \cite{Saltzer_Schroeder_1975}. Una complessità ridotta si traduce direttamente in una minore superficie d'attacco potenziale e in una significativa semplificazione delle procedure di audit, verifica formale e manutenzione del codice. Una base di codice concisa e ben strutturata minimizza l'introduzione di dipendenze potenzialmente vulnerabili e attenua la probabilità di errori di configurazione o dell'accumulo di technical debt, che possono compromettere la sicurezza nel lungo termine \cite{Smith_2012_SaltzerReview}.
Strategie implementative includono:
\begin{itemize}
\item Adozione di architetture a microservizi, ove ciascun servizio è caratterizzato da un perimetro di responsabilità chiaramente definito e isolato, limitando l'impatto di una eventuale compromissione.
\item Utilizzo di paradigmi di Infrastructure-as-Code (IaC) per la descrizione dichiarativa, la gestione versionata e l'audit automatizzato dei controlli di sicurezza infrastrutturali.
\end{itemize}
Nel contesto delle startup fintech, la pressione competitiva verso un rapido time-to-market può incentivare l'adozione di soluzioni palliative temporanee ("patching") che incrementano la complessità. Un investimento precoce nella semplicità architetturale e nella modularità si traduce in una significativa riduzione dei costi associati a future attività di refactoring del codice e di penetration testing.
\section{Impostazioni Sicure per Difetto (Fail-Safe Defaults)}
Il principio delle Impostazioni Sicure per Difetto, anch'esso introdotto da Saltzer e Schroeder, stabilisce che le decisioni relative all'accesso debbano fondarsi sulla concessione esplicita di privilegi (modello allow-list), piuttosto che sull'esclusione da un insieme di permessi negati (modello deny-list) \cite{Saltzer_Schroeder_1975}. In assenza di un'autorizzazione esplicita, l'accesso deve essere negato. Il National Institute of Standards and Technology (NIST) riprende concetti affini nel documento SP 800-27, in particolare con il "Principle 16: implement layered security (ensure no single point of vulnerability)" \cite{NIST_SP_800_27rA}, sebbene il focus di Fail-Safe Defaults sia primariamente sulla negazione implicita come comportamento predefinito.
Esempi applicativi comprendono:
\begin{itemize}
\item Configurazione di policy di tipo “Deny All” (o default deny) a livello di firewall di rete e Web Application Firewall (WAF), con l'apertura selettiva delle sole porte, protocolli e regole strettamente necessari per le funzionalità legittime.
\item In contesti cloud, implementazione di policy di Identity and Access Management (IAM) che adottano un approccio di negazione predefinita (e.g., partendo da uno stato di \texttt{NoAction} o \texttt{DenyAll}) e aggiungendo permessi granulari solo per le azioni richieste, aderendo al principio del minimo privilegio.
\end{itemize}
\section{Mediazione Completa (Complete Mediation)}
Il principio di Mediazione Completa impone che ogni tentativo di accesso a qualsiasi oggetto protetto (e.g., file, record di database, risorse API) sia soggetto a una verifica di autorizzazione completa e non aggirabile da parte di un meccanismo di riferimento fidato \cite{Saltzer_Schroeder_1975}. È cruciale evitare di fare affidamento su decisioni di accesso precedentemente memorizzate (caching) che potrebbero essere divenute obsolete, invalidate o manipolate.
Le implementazioni pratiche includono:
\begin{itemize}
\item Implementazione di API Gateway o proxy che validano l'autenticità e l'autorizzazione di ogni singola richiesta API, ad esempio mediante la verifica di token firmati digitalmente (e.g., JSON Web Tokens - JWT).
\item Applicazione di meccanismi di Row-Level Security (RLS) e column-level security, unitamente a security labels a livello di database, per garantire che le policy di accesso siano applicate direttamente ai dati, prevenendo bypass tramite query non autorizzate o accessi diretti.
\end{itemize}
In una piattaforma di gestione dei pagamenti, l'utilizzo di token di sessione a breve scadenza (short-lived tokens) e la richiesta di ri-autenticazione forte per operazioni ad alto rischio (e.g., modifica dei dati beneficiario, trasferimenti di importo elevato) sono misure essenziali per mitigare il rischio di abuso di sessioni compromesse e prevenire l'escalation di privilegi.
\section{Resilienza Cibernetica (Cyber Resiliency)}
La resilienza cibernetica, come definita dal NIST SP 800-160 Vol. 2, è la capacità di un sistema di anticipare, resistere, recuperare e adattarsi a condizioni avverse, stress, attacchi o compromissioni, mantenendo le funzionalità critiche \cite{NIST_SP_800_160v2_2019}. Le strategie fondamentali per conseguire la resilienza includono l'implementazione di ridondanza dinamica dei componenti critici, la capacità di degradazione controllata dei servizi (mantenendo le funzionalità essenziali operative anche in condizioni di parziale fallimento), e l'archiviazione sicura e immutabile dei log di sistema e di sicurezza per supportare le analisi forensi e il ripristino. Per le startup, che frequentemente operano in ambienti mono-cloud e dispongono di team DevOps con risorse limitate, è imperativo integrare fin dalle prime fasi di sviluppo piani di Business Continuity e Disaster Recovery (BC/DR), che includano test regolari dei backup e l'adozione, ove possibile, di pratiche di chaos engineering per validare proattivamente la robustezza del sistema.
\section{Responsabilizzazione e Non-Ripudio (Accountability and Non-Repudiation)}
Il principio di accountability (responsabilizzazione) esige che ogni azione significativa eseguita all'interno del sistema sia univocamente attribuibile a un'identità specifica (utente o processo) e che siano mantenute evidenze probatorie verificabili di tali azioni \cite{Feigenbaum_2020_Accountability}. La non-ripudiabilità (non-repudiation) assicura che le parti coinvolte in una transazione o comunicazione non possano negare la propria partecipazione (e.g., l'invio o la ricezione di dati). Ciò è comunemente ottenuto mediante l'uso di firme digitali, marche temporali qualificate e registri di audit immutabili \cite{NIST_Glossary_NonRepudiation}.
Esempi di meccanismi per garantire accountability e non-repudiation:
\begin{itemize}
\item Generazione di log di audit dettagliati, firmati digitalmente e marcati temporalmente (secondo lo standard RFC 3161), con policy di conservazione conformi ai requisiti normativi e di business (e.g., >= 7 anni per dati finanziari).
\item Utilizzo di sistemi basati su distributed ledger technology (DLT) o, in alternativa, di meccanismi di storage append-only con garanzie di immutabilità (e.g., Amazon S3 Object Lock in modalità compliance o WORM - Write Once Read Many) per la registrazione delle transazioni finanziarie critiche.
\end{itemize}
La conformità a normative stringenti come la PSD2 (Payment Services Directive 2) impone alle startup fintech l'obbligo di garantire e dimostrare la tracciabilità completa (end-to-end) delle operazioni critiche, come i flussi di pagamento e l'accesso ai dati dei clienti, rendendo questi principi particolarmente rilevanti \cite{NIST_SP_800_160v2_2019}.
\section{Privacy by Design (PbD)}
Il framework della Privacy by Design (PbD), introdotto da Ann Cavoukian, propugna l'integrazione della protezione dei dati personali sin dalle prime fasi del ciclo di vita della progettazione di sistemi, processi e infrastrutture tecnologiche. Si articola in sette principi fondanti, tra cui la proattività (non reattività), la privacy come impostazione predefinita (privacy by default), e l'integrazione della privacy nel design (privacy embedded into design) \cite{Cavoukian_PbD_2009}.
Misure concrete per l'attuazione della PbD includono:
\begin{itemize}
\item \textbf{Minimizzazione dei dati (Data Minimization)}: Raccolta, trattamento e conservazione dei soli dati personali strettamente necessari e pertinenti per le finalità dichiarate, legittime e specifiche del servizio.
\item \textbf{Tecniche di Anonimizzazione e Pseudonimizzazione}: Applicazione di tecniche come la differential privacy, k-anonymity, o la pseudonimizzazione per dataset analitici o quando i dati vengono condivisi con terze parti, al fine di ridurre il rischio di re-identificazione.
\item \textbf{Mappatura e revisione continua dei flussi di dati (Data Flow Mapping and Analysis)}: Mantenimento di una documentazione accurata e aggiornata dei flussi di dati personali (Data Processing Records) all'interno dell'organizzazione e revisione periodica, ad esempio ad ogni ciclo di sviluppo (sprint) o modifica significativa del sistema, per identificare e mitigare i rischi per la privacy.
\end{itemize}
Nel settore fintech, l'adozione della PbD è cruciale per assicurare la conformità al Regolamento Generale sulla Protezione dei Dati (GDPR) e alle linee guida emanate dall'European Data Protection Board (EDPB). Questo approccio non solo riduce il rischio di sanzioni amministrative e danni reputazionali, ma contribuisce anche a rafforzare la fiducia degli utenti nella gestione responsabile dei loro dati personali \cite{Cavoukian_PbD_2009}.
