A differenza delle istituzioni bancarie tradizionali, le fintech emergono spesso in un contesto con minori vincoli normativi iniziali e con una strategia di time-to-market particolarmente aggressiva. Tale approccio può condurre, talvolta, a una sottovalutazione degli aspetti di sicurezza nelle fasi primordiali di sviluppo \cite{netguru2023}. La frequenza e la rapidità dei cicli di rilascio possono indurre queste aziende a \textbf{omettere o posticipare l'implementazione di misure di sicurezza} non percepite come immediatamente essenziali per il core business \cite{netguru2023}. Ne consegue che molte soluzioni fintech, sebbene innovative sotto il profilo tecnologico, possono presentare controlli di sicurezza parziali o deboli, incrementando la probabilità di incidenti di sicurezza rispetto a istituzioni finanziarie più consolidate e regolamentate.

In questo scenario complesso, diviene cruciale l'adozione di \textbf{principi di cybersecurity strutturati} e fondati su framework riconosciuti a livello internazionale. Tali framework offrono un approccio sistematico per l'identificazione dei rischi, l'implementazione di controlli adeguati e la garanzia della resilienza dei sistemi. Nel prosieguo del capitolo, verranno analizzati i principali framework e standard di sicurezza informatica – dal \textbf{NIST Cybersecurity Framework (CSF)} all'ISO/IEC 27001, passando per linee guida NIST specifiche (SP 800-53, SP 800-82 per l'OT e SP 800-63B per le password) fino ai modelli di \textbf{Zero Trust}. Verrà illustrato come tali approcci possano essere applicati concretamente alla protezione dell'infrastruttura cloud di una startup fintech, con un focus particolare sui server e sui dati ospitati su \textbf{Amazon Web Services (AWS)}. Saranno inoltre discusse \textbf{best practice e strategie di mitigazione} delle minacce più comuni, considerando le peculiarità degli ambienti cloud-native come AWS e l'importanza di un approccio di \enquote{difesa in profondità} (defense in depth) integrato con i requisiti normativi di settore.

Prima di addentrarci nell'analisi dei framework, è fondamentale richiamare il modello di responsabilità condivisa nel cloud (Shared Responsibility Model): \textbf{AWS è responsabile della sicurezza \enquote{of the cloud}}, ovvero della protezione dell'infrastruttura fisica e dei servizi di base (data center, hardware, rete, virtualizzazione), mentre \textbf{al cliente spetta la sicurezza \enquote{in the cloud}}, cioè la configurazione sicura dei propri ambienti virtuali, la gestione degli accessi, della rete, dei dati e delle applicazioni \cite{awsResponsibility}. In altri termini, una fintech che opera su AWS deve implementare controlli di rete adeguati, cifratura, identity management, monitoring e altre misure, costruendo su una fondazione sicura fornita dal cloud provider ma senza delegare integralmente la propria responsabilità. Tenendo presente questo principio, esaminiamo ora i framework di sicurezza e come essi guidano l'implementazione di misure difensive su AWS.

\section{NIST Cybersecurity Framework (CSF)}
\label{sec:nist_csf}

Il \textbf{NIST Cybersecurity Framework (CSF)}, sviluppato dal National Institute of Standards and Technology statunitense, è un framework di riferimento ampiamente adottato a livello globale come base per la gestione del rischio cyber in organizzazioni di qualsiasi settore o dimensione \cite{awsNist}. Originariamente concepito per la protezione delle infrastrutture critiche, il CSF è strutturato in cinque funzioni fondamentali – \textbf{Identify, Protect, Detect, Respond, Recover} – che rappresentano il ciclo continuo della gestione della sicurezza. Recentemente, con la versione 2.0 del 2024, è stata introdotta una sesta funzione, \textbf{\enquote{Govern}}, a sottolineare l'importanza cruciale delle attività organizzative e di governance nella gestione del rischio cyber \cite{awsWhitepaperCSF2}. Ciascuna funzione si articola in categorie e sottocategorie di controlli di sicurezza, fornendo così una tassonomia delle capacità di cybersecurity che un'organizzazione dovrebbe sviluppare. Ad esempio, il CSF include categorie che coprono l'identificazione degli asset critici, la protezione tramite controlli di accesso e cifratura, il monitoraggio continuo degli eventi di sicurezza, la gestione degli incidenti e la resilienza operativa post-attacco.

Per una startup fintech che opera su AWS, il NIST CSF fornisce una \textbf{mappa concettuale} per implementare misure di sicurezza cloud in modo coerente e completo. AWS stessa riconosce il CSF come framework di riferimento e mette a disposizione linee guida su come allineare i propri servizi alle diverse funzioni del CSF \cite{awsWhitepaperCSF2}. In pratica:

\subsection{Identify (Identifica)}
\label{subsec:nist_csf_identify}
Questa funzione riguarda l'inventario e la classificazione di risorse, dati, software e flussi di lavoro critici. Su AWS, ciò implica mappare tutti i servizi in uso (ad esempio, istanze EC2, database RDS, bucket S3), identificare i dati sensibili (come i dati finanziari dei clienti) e valutarne l'impatto potenziale in caso di compromissione. Strumenti AWS come AWS Config e AWS Resource Explorer sono utili per mantenere la visibilità sugli asset cloud. È altresì importante identificare le dipendenze da terze parti (es. API bancarie, servizi di pagamento esterni) e i rischi associati alla supply chain, in linea con l'enfasi posta dal CSF 2.0 sulla sicurezza della catena di fornitura \cite{awsWhitepaperCSF2}.

\subsection{Protect (Proteggi)}
\label{subsec:nist_csf_protect}
Comprende tutte le misure volte a salvaguardare servizi e dati. In un'infrastruttura AWS, ciò include:
\begin{itemize}
    \item \textbf{Protezione della rete cloud}: tramite Virtual Private Cloud (VPC) ben progettati e segmentati (suddividendo ambienti di produzione, staging e test in subnet isolate), l'uso di \textbf{Security Group} e \textbf{Network ACL} per filtrare il traffico, e l'adozione di Web Application Firewall (WAF) come AWS WAF per difendersi da attacchi a livello applicativo. Possono essere integrate soluzioni di terze parti per rafforzare il perimetro, come analizzato successivamente con Check Point Quantum (Sezione \ref{sec:checkpoint_quantum}).
    \item \textbf{Sicurezza dei dati}: su AWS è fondamentale cifrare i dati sia \textbf{a riposo} (at-rest), ad esempio tramite AWS Key Management Service (KMS) per la gestione delle chiavi di cifratura e abilitando la crittografia su servizi come EBS, S3, RDS, sia \textbf{in transito} (in-transit), utilizzando protocolli TLS per API ed endpoint, e VPN/IPSec per connessioni private. Il controllo degli accessi ai dati va implementato con rigidi permessi IAM e policy di bucket S3 che limitino l'accesso solo ai ruoli o servizi autorizzati.
    \item \textbf{Gestione delle identità e degli accessi (IAM)}: il CSF prescrive di implementare il principio del minimo privilegio (Principle of Least Privilege - PoLP) e misure di autenticazione robusta. AWS IAM consente di definire ruoli e policy granulari, abilitare l'Autenticazione Multi-Fattore (MFA) sugli account (incluso l'account root, che dovrebbe essere particolarmente protetto) e centralizzare la gestione identitaria (ad esempio, integrando provider SAML/SSO per gli utenti). L'uso di IAM Roles con credenziali temporanee per servizi e applicazioni riduce il rischio di esposizione di credenziali statiche. Queste misure rispecchiano i \textbf{principi Zero Trust} (si veda la Sezione \ref{sec:zero_trust}).
    \item \textbf{Protezione dei sistemi e delle applicazioni}: ciò si traduce in hardening delle istanze (applicazione sistematica di patch a sistemi operativi e middleware, disabilitazione di servizi non necessari), utilizzo di servizi gestiti AWS (es. RDS, Lambda) che sollevano dall'onere di gestire direttamente i server e riducono la superficie d'attacco, e impostazione di backup regolari e meccanismi di disaster recovery (snapshots, replicazione tra region) per garantire resilienza (quest'ultimo aspetto sconfina nella funzione Recover).
\end{itemize}

\subsection{Detect (Individua)}
\label{subsec:nist_csf_detect}
Il framework enfatizza la capacità di rilevare tempestivamente eventi anomali e possibili incidenti di sicurezza. Su AWS, le attività di \textbf{logging e monitoring} sono fondamentali. Ogni risorsa cloud dovrebbe generare log appropriati:
\begin{itemize}
    \item AWS CloudTrail per tracciare tutte le chiamate API e le attività nell'account.
    \item AWS CloudWatch per metriche di sistema e applicative, con la possibilità di configurare allarmi in caso di valori anomali.
    \item AWS Config per registrare i cambiamenti di configurazione delle risorse.
\end{itemize}
Servizi avanzati come Amazon GuardDuty forniscono un monitoraggio continuo delle minacce analizzando pattern di traffico e log (ad esempio, identificando comportamenti anomali indicativi di credenziali compromesse o istanze malevole). Analogamente, Amazon Macie può essere utilizzato per rilevare eventuali esposizioni di dati sensibili su S3. L'aggregazione centralizzata dei log (ad esempio, in un servizio come Amazon S3 o CloudWatch Logs) e la loro correlazione tramite un sistema SIEM (Security Information and Event Management) – AWS offre AWS Security Hub per correlare avvisi da vari servizi – consente di \textbf{abilitare alerting in tempo reale} verso il team di sicurezza. Queste capacità rispondono all'esigenza di \textit{traceability}: ogni azione o modifica nell'ambiente cloud deve essere tracciata e monitorata, come raccomandato anche dall'AWS Well-Architected Framework \cite{awsWellArchitected}.

\subsection{Respond (Rispondi)}
\label{subsec:nist_csf_respond}
Questa funzione definisce le attività di \textbf{gestione degli incidenti} (incident response) nel momento in cui si verifica un problema di sicurezza. Una startup fintech, anche di piccole dimensioni, dovrebbe disporre di un piano di incident response che includa procedure per analizzare gli eventi, contenere l'incidente (ad esempio, isolando istanze compromesse, ruotando chiavi API esposte), eradicare la minaccia e ripristinare i servizi. AWS mette a disposizione strumenti che coadiuvano la risposta: AWS CloudTrail facilita le indagini forensi permettendo di ricostruire le azioni compiute da un aggressore; servizi come AWS IAM Access Analyzer possono essere usati per verificare e revocare accessi non intenzionali; AWS Systems Manager Incident Manager aiuta a orchestrare la risoluzione degli incidenti coordinando notifiche e runbook automatici. È buona prassi condurre simulazioni di incidenti (es. tabletop exercises, game-days) per addestrare il team a rispondere efficacemente, come suggerito anche dal Well-Architected Framework \cite{awsWellArchitected}. Inoltre, è necessario considerare gli adempimenti di notifica: in caso di violazione di dati personali (data breach), ad esempio, il GDPR impone la comunicazione all'autorità di controllo competente (es. Garante per la Protezione dei Dati Personali) entro 72 ore dalla scoperta, quindi il processo di incident response deve includere meccanismi di escalation manageriale e legale.

\subsection{Recover (Recupera)}
\label{subsec:nist_csf_recover}
La funzione Recover riguarda la \textbf{resilienza operativa} e la capacità di ripristinare rapidamente i servizi a seguito di un incidente o di un guasto, minimizzando l'impatto sugli utenti e sui partner commerciali. In ambito AWS, questo si traduce nel disporre di backup (preferibilmente offline o immutabili) e piani di \textbf{disaster recovery (DR)} testati regolarmente. Una fintech potrebbe, ad esempio, mantenere backup crittografati dei database finanziari (utilizzando AWS Backup per centralizzare e automatizzare i backup di RDS, EBS, DynamoDB, ecc.) e predisporre infrastrutture di ripristino in una regione AWS secondaria per far fronte a eventi catastrofici che potrebbero colpire la regione primaria. Servizi come Amazon S3 garantiscono una durabilità estremamente elevata per i dati (undici 9, ovvero 99.999999999\%) e offrono funzionalità di versioning degli oggetti, permettendo il recupero di dati alterati o cancellati accidentalmente. La fase di Recover include anche le comunicazioni post-incidente e il miglioramento continuo: dopo il ripristino, è importante condurre un'analisi post-mortem (lessons learned), comprendere le cause alla radice dell'incidente e aggiornare i controlli di sicurezza per prevenire il ripetersi di eventi simili \cite{awsWhitepaperCSF2}.

\section{ISO/IEC 27001 e Sistemi di Gestione della Sicurezza (ISMS)}
\label{sec:iso_27001}
Lo standard \textbf{ISO/IEC 27001} è il riferimento internazionale per stabilire, implementare, mantenere e migliorare continuamente un \textit{Information Security Management System} (\textbf{ISMS}), ovvero un sistema di gestione della sicurezza delle informazioni omnicomprensivo. Si tratta di un framework gestionale che adotta un approccio basato sul rischio (risk-based approach) per garantire la \textbf{riservatezza, integrità e disponibilità (CIA triad)} delle informazioni aziendali, attraverso un insieme strutturato di controlli di sicurezza organizzativi, fisici e tecnici. ISO/IEC 27001 è riconosciuto globalmente ed è applicato da organizzazioni in tutti i settori come \textbf{benchmark} di best practice per la sicurezza delle informazioni.

Il nucleo della norma è l'applicazione del ciclo PDCA (Plan-Do-Check-Act) alla sicurezza:
\begin{description}
    \item[Plan:] L'organizzazione deve condurre una valutazione dei rischi (identificando asset informativi, minacce, vulnerabilità e impatti), definire l'ambito dell'ISMS e selezionare gli obiettivi di controllo e i controlli.
    \item[Do:] Implementare e operare i controlli e i processi dell'ISMS.
    \item[Check:] Monitorare e riesaminare periodicamente l'efficacia dell'ISMS, confrontando le performance con gli obiettivi e i requisiti.
    \item[Act:] Mantenere e migliorare continuamente l'ISMS intraprendendo azioni correttive e preventive basate sui risultati del riesame.
\end{description}
L'Annex A dello standard (edizione 2022) fornisce un elenco di riferimento di 93 controlli, organizzati in 4 domini tematici (Organizzativi, Persone, Fisici, Tecnologici). La certificazione ISO/IEC 27001, rilasciata da un ente terzo accreditato, attesta che l'organizzazione aderisce a questo processo e rispetta tutti i requisiti dello standard.

Per una startup fintech, ottenere la certificazione ISO/IEC 27001 può rappresentare un importante fattore abilitante di fiducia sul mercato – specialmente se si rivolge a clientela enterprise o ad altre istituzioni finanziarie – ma può anche costituire una sfida, data la mole di processi e misure da implementare. L'adozione di servizi cloud AWS può, tuttavia, facilitare il percorso verso la conformità ISO/IEC 27001. AWS stessa è certificata ISO/IEC 27001 per la propria infrastruttura globale di servizi cloud (oltre ad altre certificazioni rilevanti come ISO/IEC 27017 per i controlli specifici per il cloud e ISO/IEC 27018 per la protezione della privacy nel cloud). Ciò significa che i data center e i servizi AWS sono gestiti secondo controlli di sicurezza riconosciuti, permettendo alla fintech di concentrarsi sui controlli applicativi e organizzativi, sapendo che molti requisiti infrastrutturali di base – ad esempio sulla protezione fisica dei server, il controllo degli accessi ai locali, la continuità elettrica e di rete – sono già coperti e attestati dalla piattaforma AWS. Ciononostante, la \textbf{responsabilità dell'implementazione} dei controlli relativi ai propri dati e configurazioni nel cloud rimane in capo al cliente, in linea con il modello di responsabilità condivisa. Ad esempio, ISO/IEC 27001 richiede di controllare gli accessi logici: la fintech dovrà definire policy IAM, regole di password e uso di MFA in AWS per soddisfare tale controllo. Richiede di tenere registri degli eventi (logging): la fintech dovrà configurare servizi come CloudTrail e CloudWatch. Richiede di cifrare informazioni sensibili: la fintech dovrà abilitare la crittografia nei servizi AWS dove risiedono dati critici.

\section{NIST SP 800-53 – Catalogo di Controlli di Sicurezza}
\label{sec:nist_sp_800_53}
La pubblicazione speciale \textbf{NIST SP 800-53} fornisce un \textbf{catalogo completo di controlli di sicurezza e privacy} per sistemi informativi federali, ma è ampiamente adottato come riferimento anche da numerose organizzazioni nel settore privato, inclusa l'industria finanziaria. Si tratta di uno standard più \textbf{prescrittivo e tecnico} rispetto al CSF, che copre aspetti di sicurezza logica, fisica, procedurale e del personale, organizzati in diverse famiglie di controlli. L'ultima revisione (Revision 5) del NIST SP 800-53, pubblicata nel 2020, contiene \textbf{20 famiglie di controlli} principali , tra cui:
\begin{enumerate}
    \item \textbf{AC – Access Control} (Controllo degli Accessi)
    \item \textbf{IA – Identification and Authentication} (Identificazione e Autenticazione)
    \item \textbf{SC – System and Communications Protection} (Protezione dei Sistemi e delle Comunicazioni, es. cifratura, segregazione di rete)
    \item \textbf{SI – System and Information Integrity} (Integrità dei Sistemi e delle Informazioni, es. anti-malware, gestione delle vulnerabilità, monitoraggio)
    \item \textbf{AU – Audit and Accountability} (Audit e Tracciabilità, es. logging)
    \item \textbf{IR – Incident Response} (Risposta agli Incidenti)
    \item \textbf{CP – Contingency Planning} (Pianificazione della Continuità Operativa e Disaster Recovery)
    \item \textbf{PE – Physical and Environmental Protection} (Protezione Fisica e Ambientale)
    \item \textbf{PS – Personnel Security} (Sicurezza del Personale, es. background check, training)
    \item Altre famiglie includono: \textbf{Risk Assessment (RA)}, \textbf{Security Assessment and Authorization (CA)}, \textbf{Configuration Management (CM)}, \textbf{Awareness and Training (AT)}, \textbf{Maintenance (MA)}, \textbf{Supply Chain Risk Management (SR)}, etc.
\end{enumerate}

Complessivamente, SP 800-53 Rev. 5 cataloga oltre 1000 controlli di sicurezza e privacy, da cui vengono derivate delle \textbf{baseline di controlli} (Low, Moderate, High) in base al livello di impatto del sistema informativo. Ad esempio, un sistema classificato come \textit{Moderate Impact} (come potrebbe essere un sistema fintech che gestisce dati finanziari sensibili ma non classificati a livello governativo) dovrebbe implementare tutti i controlli previsti dalla baseline Moderate. Le organizzazioni possono poi \textit{personalizzare} (tailor) la baseline aggiungendo, modificando o escludendo controlli in base alle esigenze specifiche, all'analisi del rischio e ai requisiti normativi applicabili .

Dal punto di vista di AWS, è importante notare che \textbf{l'infrastruttura AWS è stata validata rispetto a numerosi controlli NIST SP 800-53} nell'ambito delle certificazioni FedRAMP (Federal Risk and Authorization Management Program) Moderate e High per i servizi AWS \cite{awsNistCompliance}. Ciò significa che AWS ha superato audit di terza parte che attestano l'implementazione di controlli di sicurezza allineati a SP 800-53 per quanto riguarda la piattaforma cloud sottostante. Per la fintech cliente, questo non implica automaticamente la conformità a SP 800-53 per i propri sistemi, ma fornisce una solida base: ad esempio, i controlli di sicurezza fisica (PE) e ambientale, molti controlli di rete (SC) e parte di quelli di monitoraggio (SI) a livello infrastrutturale sono già soddisfatti dall'ambiente AWS. Rimane responsabilità del cliente implementare i controlli a livello di applicazione e configurazione cloud (ad es., definire ruoli IAM – controllo AC-2, o abilitare il versioning su S3 – parte dei controlli SC e SI). AWS fornisce anche strumenti come \textbf{AWS Audit Manager}, che include framework predefiniti per NIST SP 800-53, consentendo di valutare l'account AWS rispetto ai controlli di tale standard e di collezionare evidenze in caso di audit \cite{awsAuditManager}.

\section{Architettura Zero Trust (ZTA)}
\label{sec:zero_trust}
Tradizionalmente, la sicurezza informatica aziendale si fondava su un modello di \textbf{difesa perimetrale}: si creava una rete aziendale considerata “fidata” all’interno, separata dall’esterno “non fidato” tramite firewall e altre barriere (il cosiddetto modello “castello e fossato” o \enquote{castle and moat}). Tuttavia, con l'evoluzione delle minacce (es. insider threat, attacchi laterali), la crescente adozione del cloud computing, la mobilità degli utenti, il telelavoro e l'uso di dispositivi personali (BYOD), questo paradigma si è dimostrato progressivamente inefficace. Emerge così il concetto di \textbf{Zero Trust}, formalizzato, tra gli altri, dal NIST nella pubblicazione speciale SP 800-207 \enquote{Zero Trust Architecture} \cite{nistZeroTrust}. Questo approccio rivoluziona la strategia di sicurezza: \textit{non si deve mai implicitamente fidare di alcuna entità, sia essa interna o esterna al perimetro tradizionale, ma verificare esplicitamente ogni richiesta di accesso a risorse aziendali.} In un'Architettura Zero Trust (\textbf{ZTA}), le difese non sono più incentrate su una rete interna considerata intrinsecamente sicura, ma \textbf{sull’identità degli utenti e dei dispositivi, e sul contesto delle richieste di accesso}, indipendentemente dalla loro provenienza fisica o logica.

I principi cardine della Zero Trust, come delineati dal NIST \cite{nistZeroTrust}, includono:
\begin{itemize}
    \item \textbf{Tutte le fonti di dati e i servizi di calcolo sono considerati risorse.}
    \item \textbf{Tutte le comunicazioni sono protette indipendentemente dalla posizione di rete.} Essere su una rete interna non concede privilegi impliciti.
    \item \textbf{L'accesso alle singole risorse aziendali è concesso per sessione.} La fiducia non è persistente.
    \item \textbf{L'accesso alle risorse è determinato da policy dinamiche,} che includono lo stato osservabile dell'identità del client, dell'applicazione/servizio e dell'asset richiedente, e possono includere attributi comportamentali e ambientali (es. orario, geolocalizzazione, postura di sicurezza del dispositivo).
    \item \textbf{L'organizzazione monitora e misura l'integrità e la postura di sicurezza di tutti gli asset posseduti e associati.}
    \item \textbf{Tutte le autenticazioni e autorizzazioni sono dinamiche e applicate rigorosamente prima che l'accesso sia consentito.} Questo è un ciclo continuo di accesso, scansione e valutazione delle minacce, adattamento e ri-autenticazione.
    \item \textbf{L'organizzazione raccoglie quanti più dati possibile su asset, infrastruttura di rete e comunicazioni e li utilizza per migliorare la propria postura di sicurezza.}
\end{itemize}
In sintesi, Zero Trust “sposta” il confine di fiducia dalla rete all’entità che richiede l’accesso, in un modello in cui \textbf{ogni transazione è autenticata, autorizzata, crittografata e validata} in modo robusto, come se provenisse da un ambiente non fidato, anche se in realtà avviene all’interno del sistema. Tecnologie come la micro-segmentazione, l'autenticazione multi-fattore (MFA), l'Identity and Access Management (IAM) avanzato e il monitoraggio continuo sono fondamentali per implementare una ZTA.

\section{Sicurezza OT (Tecnologie Operative) – NIST SP 800-82}
\label{sec:sicurezza_ot}
Nel dominio della cybersecurity aziendale, oltre all’Information Technology (IT) tradizionale (server, applicazioni, dati), acquista crescente importanza la protezione delle \textbf{Tecnologie Operative (OT)}, ossia quei sistemi hardware e software che rilevano o causano un cambiamento attraverso il monitoraggio e/o il controllo diretto di dispositivi, processi ed eventi fisici. Le aziende fintech, essendo primariamente attive nel settore finanziario digitale, generalmente non operano impianti industriali o infrastrutture OT su larga scala come farebbe un'azienda manifatturiera o una utility energetica. Tuttavia, è possibile che alcune componenti con interfacce fisiche rientrino nel perimetro di una fintech: si pensi, ad esempio, agli \textbf{sportelli automatici (ATM/Bancomat)}, ai dispositivi Point of Sale (POS) smart, ai data center on-premises con sistemi di building automation (HVAC, controllo accessi fisici), o a eventuali sensori IoT impiegati per servizi finanziari innovativi (es. assicurazioni basate sull'uso) \cite{nistSP80082}.

La pubblicazione speciale \textbf{NIST SP 800-82}, \enquote{Guide to Industrial Control Systems (ICS) Security}, fornisce linee guida specifiche per migliorare la sicurezza dei sistemi OT/ICS, tenendo conto dei loro requisiti unici di prestazioni (spesso real-time), affidabilità, disponibilità e sicurezza fisica (safety) \cite{nistSP80082}. Tali sistemi presentano sfide peculiari: operano frequentemente in tempo reale, non possono tollerare interruzioni (la disponibilità e la safety spesso prevalgono sulla confidenzialità), utilizzano protocolli di comunicazione specializzati o legacy, e possono avere cicli di vita molto lunghi con componenti hardware e software non facilmente aggiornabili o sostituibili.

Per mettere in sicurezza ambienti OT, il framework NIST SP 800-82 raccomanda, tra le altre cose, di:
\begin{itemize}
    \item \textbf{Segmentare rigorosamente le reti OT dalle reti IT,} utilizzando architetture a zone e condotti (zones and conduits), e implementando gateway e firewall industriali (Industrial DMZ) per controllare e limitare il traffico.
    \item \textbf{Implementare controlli di accesso e monitoraggio specifici} per i protocolli OT, ove possibile.
    \item Assicurare l’\textbf{integrità e l’affidabilità} dei comandi inviati ai dispositivi fisici.
    \item Gestire patch e vulnerabilità OT in modo pianificato e controllato, considerando l'impatto sulla continuità operativa e sulla safety, e utilizzando misure compensative quando il patching diretto non è fattibile.
    \item Predisporre piani di incident response specifici per OT che considerino anche scenari di sicurezza fisica e safety \cite{nistSP80082}.
\end{itemize}

\section{Sicurezza delle Password e Gestione delle Identità – NIST SP 800-63B}
\label{sec:sicurezza_password}
Le \textbf{password} (o più correttamente, i \enquote{memorized secrets}) rimangono tutt’oggi uno dei principali meccanismi di autenticazione, ma rappresentano anche un punto debole frequentemente sfruttato dagli attaccanti (tramite tecniche come phishing, attacchi a dizionario, credential stuffing, password spraying, ecc.). Per questo motivo, il NIST ha pubblicato la serie di documenti \textbf{NIST SP 800-63 \enquote{Digital Identity Guidelines}}, di cui la \textbf{SP 800-63B} è specificamente dedicata all'\textit{Authentication and Lifecycle Management} \cite{nistSP80063B}. Queste linee guida forniscono raccomandazioni aggiornate su come gestire in modo sicuro le password e altri fattori di autenticazione.

Le indicazioni più recenti del NIST hanno parzialmente ribaltato alcuni concetti tradizionali sulla complessità e sulla scadenza delle password, a favore di un approccio più orientato all'usabilità e alla robustezza effettiva. In sintesi, NIST SP 800-63B suggerisce di:
\begin{itemize}
    \item \textbf{Privilegiare la lunghezza delle password} (minimo 8 caratteri se generata dall'utente, minimo 6 se generata dal sistema e casuale) e permettere l'uso di frasi (passphrases) e spazi.
    \item \textbf{Non imporre requisiti di composizione eccessivamente complessi} (es. obbligo di caratteri speciali, maiuscole, minuscole, numeri) che portano gli utenti a creare password prevedibili o a scriverle. È sufficiente richiedere una varietà di caratteri se la password è breve.
    \item \textbf{Non forzare la scadenza periodica delle password} (password aging) a meno che non vi sia evidenza di compromissione. Cambi frequenti spingono a password deboli.
    \item \textbf{Verificare le nuove password (e quelle esistenti) contro dizionari di password compromesse} note (es. quelle presenti in data breach pubblici) e parole comuni.
    \item \textbf{Limitare il numero di tentativi di autenticazione falliti} (rate limiting) per prevenire attacchi di forza bruta.
    \item \textbf{Incoraggiare fortemente e, ove possibile, imporre l'uso dell'Autenticazione Multi-Fattore (MFA)}, specialmente per accessi privilegiati o a dati sensibili.
    \item Fornire meccanismi sicuri per il recupero delle password.
\end{itemize}
Queste linee guida sono fondamentali per le startup fintech nella definizione delle policy di autenticazione per i propri utenti e dipendenti, bilanciando sicurezza e usabilità \cite{jumpcloudNistPassword}.

\section{Difesa Perimetrale Avanzata e Soluzioni di Next-Generation Firewall – Check Point Quantum}
\label{sec:checkpoint_quantum}
Oltre ai framework e alle linee guida generali, è utile considerare l’adozione di \textbf{tecnologie specifiche} per potenziare la sicurezza dell’infrastruttura cloud. Nel panorama attuale, i firewall di nuova generazione (Next-Generation Firewalls - NGFW) e le piattaforme integrate di threat prevention giocano un ruolo chiave nel proteggere reti e workload, specialmente in scenari ibridi o multi-cloud. \textbf{Check Point Quantum} è un esempio di una famiglia di prodotti di sicurezza di rete che una fintech potrebbe considerare per migliorare la propria postura difensiva, sia on-premises che nel cloud \cite{checkpointQuantum}.

In particolare, le soluzioni come Check Point Quantum Network Security offrono una protezione scalabile e multi-livello contro minacce informatiche evolute. Queste piattaforme tipicamente integrano funzionalità quali:
\begin{itemize}
    \item Firewalling stateful avanzato.
    \item Intrusion Prevention System (IPS).
    \item Application Control e URL Filtering.
    \item Antivirus e Anti-malware.
    \item Sandboxing per l'analisi di minacce zero-day (come la tecnologia SandBlast di Check Point).
    \item VPN e connettività sicura.
    \item Funzionalità di ispezione del traffico SSL/TLS.
    \item Integrazione con feed di threat intelligence.
\end{itemize}
Una console di gestione unificata permette di orchestrare le policy di sicurezza su diversi ambienti. Per le fintech su AWS, soluzioni come Check Point CloudGuard Network Security possono essere deployate come virtual appliance all'interno del VPC per fornire protezione avanzata al traffico in entrata, in uscita e laterale (est-ovest) tra le subnet, integrandosi con i servizi nativi di AWS (come Security Groups, AWS WAF, Gateway Load Balancer) per una difesa in profondità \cite{awsCheckPoint}.

\section{Best Practice e Strategie di Mitigazione Complessive}
\label{sec:best_practices}
Attraverso l’analisi dei framework e delle soluzioni sopra esposte, emergono alcuni \textbf{principi trasversali di sicurezza} che dovrebbero guidare ogni startup fintech nella protezione della propria infrastruttura, specialmente se basata su AWS. Di seguito, si riassumono le migliori pratiche e strategie di mitigazione più efficaci:
\begin{itemize}
    \item \textbf{Identità solida e minimo privilegio (IAM Robusto)}: Utilizzare account individuali, applicare rigorosamente il principio del minimo privilegio (PoLP) per utenti e servizi, e adottare l'Autenticazione Multi-Fattore (MFA) ovunque possibile, specialmente per l'account root AWS e gli utenti con privilegi elevati. Utilizzare ruoli IAM per delegare permessi temporanei alle applicazioni e ai servizi AWS, evitando l'uso di credenziali statiche hardcoded \cite{awsWellArchitected}.
    \item \textbf{Segmentazione e difesa in profondità (Defense in Depth)}: Implementare controlli di sicurezza multipli a vari livelli (rete, host, applicazione, dati). Segmentare le reti utilizzando VPC, subnet, Security Groups e Network ACLs per isolare ambienti e applicazioni. Limitare il raggio d'azione (blast radius) di eventuali compromissioni \cite{awsWellArchitected}.
    \item \textbf{Protezione dei dati critica (Data Protection)}: Classificare i dati in base alla loro sensibilità. Cifrare i dati sensibili sia a riposo (es. con AWS KMS, S3 Server-Side Encryption, EBS Encryption) sia in transito (es. TLS/SSL, VPN). Implementare meccanismi di Data Loss Prevention (DLP) se necessario e gestire attentamente i backup e la loro sicurezza \cite{awsWellArchitected}.
    \item \textbf{Monitoraggio continuo e traceability (Logging and Monitoring)}: Abilitare il logging dettagliato per tutti i servizi (AWS CloudTrail, CloudWatch Logs, VPC Flow Logs, log applicativi). Centralizzare e analizzare i log, possibilmente tramite un SIEM o strumenti come Amazon GuardDuty e AWS Security Hub, per rilevare attività sospette e abilitare alerting tempestivo. Garantire la tracciabilità delle azioni eseguite sull'infrastruttura \cite{awsWellArchitected}.
    \item \textbf{Automatizzare la sicurezza e l'infrastruttura come codice (Automation and IaC)}: Utilizzare strumenti di Infrastructure as Code (IaC) come AWS CloudFormation o Terraform per definire e gestire l'infrastruttura in modo riproducibile e versionabile. Automatizzare i controlli di sicurezza, il patching, le verifiche di conformità e le risposte agli incidenti (Security Orchestration, Automation and Response - SOAR) per ridurre l'errore umano e migliorare la reattività \cite{awsWellArchitected}.
    \item \textbf{Preparazione agli incidenti e resilienza (Incident Response and Resilience)}: Sviluppare e mantenere un piano di incident response. Predisporre piani di disaster recovery e business continuity, testandoli regolarmente attraverso simulazioni. Progettare l'architettura per l'alta disponibilità e la fault tolerance, sfruttando le Availability Zones e le Regioni AWS \cite{awsWellArchitected}.
    \item \textbf{Formazione e cultura della sicurezza (Security Awareness and Culture)}: Investire nella formazione continua del personale su tematiche di cybersecurity. Promuovere una cultura della sicurezza all'interno dell'organizzazione, adottando approcci come \enquote{Secure by Design} e \enquote{Shift Left Security} (integrare la sicurezza fin dalle prime fasi del ciclo di vita dello sviluppo software) \cite{netguru2023}.
    \item \textbf{Compliance proattiva (Proactive Compliance)}: Integrare i controlli richiesti dalle normative applicabili (es. PCI DSS, GDPR, normative finanziarie specifiche) nel ciclo di vita della sicurezza, piuttosto che trattarli come un adempimento a posteriori. Utilizzare strumenti di verifica della conformità e condurre audit periodici \cite{netguru2023}.
\end{itemize}

\chapter{Conformità Normativa per Startup Fintech: Implementazione di Standard Chiave}
\label{chap:compliance}

L'ecosistema fintech italiano, e più ampiamente europeo, richiede un approccio sistematico alla conformità normativa che integri molteplici framework regolamentari e standard di sicurezza. La convergenza di normative come la Direttiva NIS2, lo standard ISO/IEC 27001, il GDPR, le direttive sui servizi di pagamento PSD2/3, le direttive antiriciclaggio (AMLD) e il Digital Operational Resilience Act (DORA) crea un panorama complesso che le startup fintech devono navigare per operare legalmente, in sicurezza e per costruire fiducia con clienti e partner. L'implementazione efficace di questi standard non solo garantisce la conformità legale, ma stabilisce anche le fondamenta per un'architettura tecnologica robusta, resiliente e scalabile. Le startup fintech che adottano tecnologie moderne come Flutter per lo sviluppo di interfacce mobile, piattaforme cloud come AWS o Azure per l'infrastruttura, e sistemi di controllo versione come GitHub per la gestione del codice sorgente, devono integrare i requisiti di sicurezza e conformità sin dalle prime fasi di progettazione (\textit{security and compliance by design}). Questa integrazione precoce riduce significativamente i costi di adeguamento retroattivo e minimizza i rischi operativi. Il presente capitolo fornisce una disamina di questi standard, con un focus sulle implicazioni tecniche per lo sviluppo di soluzioni fintech, quali portafogli digitali e servizi di investimento.

\section{Direttiva NIS2: Sicurezza delle Reti e dei Sistemi Informativi}
\label{sec:nis2}

\subsection{Quadro Normativo e Applicabilità}
La Direttiva (UE) 2022/2555, nota come NIS2 (Network and Information Systems Directive 2), rappresenta l'evoluzione del framework europeo per la cybersecurity, con un impatto significativo sul settore fintech italiano \cite{cybersecurity360NIS2}. Questa direttiva, che abroga la precedente Direttiva NIS (UE) 2016/1148, amplia considerevolmente il perimetro di applicazione, includendo un numero maggiore di settori e entità, comprese quelle di medie dimensioni, e stabilendo una distinzione tra \enquote{soggetti essenziali} e \enquote{soggetti importanti} in base alle dimensioni dell'organizzazione e alla criticità dei servizi offerti. Per le startup fintech, la classificazione dipenderà dalla loro dimensione e dal tipo di servizi finanziari erogati (ad esempio, infrastrutture del mercato finanziario, fornitori di servizi di pagamento potrebbero rientrare tra i soggetti essenziali o importanti). La direttiva impone requisiti minimi specifici per la gestione dei rischi di cybersecurity, obblighi di notifica degli incidenti significativi e misure tecniche, operative e organizzative appropriate e proporzionate.

Le startup fintech che rientrano nell'ambito di applicazione della NIS2 devono implementare un sistema di gestione della sicurezza informatica che includa, come minimo:
\begin{itemize}
    \item Politiche di analisi dei rischi e di sicurezza dei sistemi informatici.
    \item Gestione degli incidenti.
    \item Continuità operativa, come la gestione dei backup e il ripristino in caso di disastro, e gestione delle crisi.
    \item Sicurezza della catena di approvvigionamento (supply chain security), comprese le relazioni con fornitori diretti e prestatori di servizi.
    \item Sicurezza nell'acquisizione, nello sviluppo e nella manutenzione dei sistemi informatici e di rete, inclusa la gestione e la divulgazione delle vulnerabilità.
    \item Politiche e procedure per valutare l'efficacia delle misure di gestione dei rischi di cybersecurity.
    \item Pratiche di igiene informatica di base e formazione in materia di cybersecurity.
    \item Politiche e procedure relative all'uso della crittografia e, se del caso, della cifratura.
    \item Sicurezza delle risorse umane, strategie di controllo dell'accesso e gestione degli asset.
    \item Uso di soluzioni di autenticazione a più fattori o di autenticazione continua.
\end{itemize}
L'approccio richiesto è basato sul rischio (\textit{risk-based approach}), imponendo una valutazione continua delle minacce e delle vulnerabilità, con particolare attenzione alle specificità del settore finanziario \cite{cybersecurity360NIS2}. La direttiva stabilisce inoltre obblighi di reportistica verso le autorità competenti (CSIRT nazionali e autorità di vigilanza) con tempistiche stringenti per la notifica degli incidenti significativi.

\subsection{Implementazione Tecnica per Startup Fintech}
L'implementazione della NIS2 in una startup fintech richiede un approccio strutturato che integri la sicurezza \textit{by design} nell'intera architettura tecnologica. Inizialmente, è necessario condurre un assessment completo dell'infrastruttura esistente, identificando tutti gli asset critici (sistemi informativi, dati, processi), le dipendenze tecnologiche e i potenziali punti di vulnerabilità. Questo assessment deve includere l'analisi del codice delle applicazioni (es. sviluppate in Flutter) per identificare potenziali vulnerabilità (es. OWASP Mobile Top 10), la revisione delle configurazioni dei repository di codice (es. GitHub) per garantire la sicurezza del ciclo di vita dello sviluppo (DevSecOps), e l'audit delle configurazioni dell'infrastruttura cloud (AWS/Azure) rispetto a benchmark di sicurezza riconosciuti.

La fase successiva prevede l'implementazione di controlli di sicurezza specifici per ciascun livello dell'architettura.
\begin{itemize}
    \item \textbf{A livello applicativo (es. Flutter):} Implementare pratiche di codifica sicura (\textit{secure coding practices}), validazione robusta degli input/output, crittografia end-to-end per le comunicazioni di dati sensibili, meccanismi di autenticazione forte e gestione sicura delle sessioni.
    \item \textbf{A livello di sviluppo e repository (es. GitHub):} Configurare branch protection rules, richiedere signed commits, integrare strumenti di analisi statica (SAST) e dinamica (DAST) della sicurezza delle applicazioni nel pipeline CI/CD, e utilizzare funzionalità come GitHub Advanced Security per la scansione di segreti e vulnerabilità nel codice.
    \item \textbf{A livello di infrastruttura cloud (es. AWS/Azure):} Implementare segmentazione di rete (VPC/VNet, subnet, security groups/NSGs), cifratura dei dati a riposo e in transito, Identity and Access Management (IAM) granulare con principio del minimo privilegio, logging e monitoring completi, e sistemi di rilevamento delle intrusioni (IDS/IPS).
\end{itemize}
Il sistema di monitoraggio e risposta agli incidenti deve essere progettato per garantire il rilevamento e la risposta tempestiva. Ciò include l'implementazione di soluzioni SIEM (Security Information and Event Management), l'automazione del rilevamento delle minacce e procedure standardizzate per l'escalation e la gestione degli incidenti. La documentazione di tutti i processi, delle policy e delle procedure è essenziale per dimostrare la conformità durante gli audit regolatori.

\section{Standard ISO/IEC 27001 per la Sicurezza delle Informazioni}
\label{sec:iso27001_compliance}

\subsection{Framework e Principi Fondamentali}
Come già introdotto nella Sezione \ref{sec:iso_27001}, ISO/IEC 27001 è lo standard internazionale di riferimento per i sistemi di gestione della sicurezza delle informazioni (ISMS), particolarmente critico per il settore fintech data la sensibilità dei dati finanziari gestiti. Lo standard fornisce un approccio sistematico per stabilire, implementare, mantenere e migliorare continuamente un ISMS. Per le startup fintech, l'adozione di ISO/IEC 27001 non solo migliora significativamente la postura di sicurezza, ma facilita anche la conformità con altre normative settoriali (come NIS2 e DORA) e aumenta la fiducia dei clienti e degli investitori.

Il framework ISO/IEC 27001 si basa sul ciclo Plan-Do-Check-Act (PDCA) e richiede un approccio basato sul rischio per la gestione della sicurezza delle informazioni. Le clausole 4 (Contesto dell'organizzazione), 5 (Leadership), 6 (Pianificazione), 7 (Supporto), 8 (Attività operative), 9 (Valutazione delle prestazioni) e 10 (Miglioramento) definiscono i requisiti per l'ISMS. In particolare, la clausola 6.1 (\textit{Azioni per affrontare rischi e opportunità}) è fondamentale, richiedendo l'identificazione e la valutazione sistematica dei rischi e delle opportunità legati alla gestione dei dati sensibili. L'implementazione efficace richiede un forte commitment del management, la definizione di obiettivi chiari di sicurezza (allineati agli obiettivi di business) e l'allocazione di risorse adeguate per il mantenimento dell'ISMS.

\subsection{Implementazione Operativa in Ambiente Fintech}
L'implementazione di ISO/IEC 27001 in una startup fintech inizia con la definizione del perimetro (\textit{scope}) dell'ISMS e la conduzione di un'analisi e valutazione dei rischi (\textit{risk assessment}) completa, come richiesto dalla clausola 6.1.2. Questo processo deve identificare tutti gli asset informativi (hardware, software, dati, documentazione, persone), valutare le minacce e le vulnerabilità associate, e determinare il livello di rischio. Successivamente, si procede al trattamento dei rischi (clausola 6.1.3), che può includere la mitigazione (applicando controlli), il trasferimento (es. tramite assicurazioni o outsourcing), l'accettazione o l'evitamento del rischio. L'Annex A dello standard fornisce un set di riferimento di 93 controlli, raggruppati in 4 temi (controlli organizzativi, sulle persone, fisici e tecnologici), da cui selezionare quelli applicabili in base ai risultati del trattamento dei rischi. Il requisito 6.1.3d) richiede la produzione di una \textit{Dichiarazione di Applicabilità} (Statement of Applicability - SoA) che documenti quali controlli sono stati scelti e perché.

La fase di implementazione (Do) richiede l'adozione dei controlli scelti, personalizzati per l'ambiente tecnologico della startup.
\begin{itemize}
    \item \textbf{Per applicazioni Flutter:} Implementare standard di codifica sicura (controllo A.8.25 \textit{Secure development lifecycle}), processi di revisione del codice (A.8.28 \textit{System security testing}), e test di sicurezza automatizzati nel pipeline di sviluppo (A.8.29 \textit{Security testing in development and acceptance}).
    \item \textbf{Per l'ambiente GitHub:} Configurare controlli di accesso appropriati (A.5.15 \textit{Access control}, A.5.16 \textit{Identity management}, A.5.17 \textit{Authentication information}), protezione dei branch, e logging degli audit (A.8.15 \textit{Logging}).
    \item \textbf{Per l'infrastruttura cloud AWS/Azure:} Implementare controlli come A.5.23 (\textit{Information security for use of cloud services}), A.8.9 (\textit{Configuration management}), A.8.16 (\textit{Monitoring activities}), A.8.2 (\textit{Protection against malware}), A.8.3 (\textit{Information backup}), A.8.23 (\textit{Web filtering}), e A.8.24 (\textit{Use of cryptography}) .
\end{itemize}
La fase di Check include il monitoraggio, la misurazione, l'analisi e la valutazione dell'ISMS (clausola 9.1), gli audit interni (clausola 9.2) e il riesame della direzione (clausola 9.3). La fase di Act (clausola 10) si concentra sul miglioramento continuo, affrontando le non conformità e attuando azioni correttive. La documentazione dell'ISMS (clausola 7.5) deve essere mantenuta aggiornata e deve includere politiche di sicurezza, procedure operative standard e registri.

\section{Regolamento GDPR per la Protezione dei Dati}
\label{sec:gdpr}

\subsection{Applicazione nel Contesto Fintech}
Il Regolamento Generale sulla Protezione dei Dati (GDPR - Regolamento UE 2016/679) stabilisce il framework normativo per il trattamento dei dati personali nell'Unione Europea, con implicazioni particolarmente significative per le startup fintech che gestiscono grandi volumi di dati finanziari e personali sensibili dei clienti. Il settore fintech, caratterizzato dall'innovazione tecnologica e dall'utilizzo intensivo di dati per profilazione, credit scoring e personalizzazione dei servizi, deve navigare le complessità del GDPR mantenendo al contempo la capacità di innovare.

Il Codice di condotta per i sistemi informativi gestiti da soggetti privati in tema di crediti al consumo, affidabilità e puntualità nei pagamenti, approvato dal Garante per la Protezione dei Dati Personali, estende e dettaglia l'applicazione delle regole privacy anche ai servizi fintech, inclusi i prestiti peer-to-peer (P2P lending) erogati tramite piattaforme tecnologiche. Questo aggiornamento è stato necessario per l'avanzamento della \textit{digital economy} e per l'avvio dei servizi fintech, estendendo la regolamentazione oltre i tradizionali settori del credito al consumo, mutui, leasing e noleggio a lungo termine.

\subsection{Implementazione Tecnica e Organizzativa}
L'implementazione del GDPR in una startup fintech richiede un approccio multidisciplinare che integri aspetti legali, tecnici e organizzativi. Il trattamento dei dati degli interessati nei sistemi informativi creditizi (SIC) si basa, tipicamente, sulla base giuridica del legittimo interesse del titolare del trattamento (art. 6, par. 1, lett. f GDPR), ma richiede il pieno rispetto di tutti i principi e diritti garantiti dal GDPR (trasparenza, liceità, correttezza, minimizzazione dei dati, limitazione della finalità, limitazione della conservazione, integrità e riservatezza). Particolare attenzione deve essere prestata alla trasparenza algoritmica (art. 13, 14, 22 GDPR): in caso di decisioni basate unicamente sul trattamento automatizzato, come il diniego all'accesso al credito basato su scoring, l'interessato deve essere informato sulla logica di funzionamento dell'algoritmo e ha il diritto di ottenere l'intervento umano, esprimere la propria opinione e contestare la decisione.

Dal punto di vista tecnico, l'implementazione richiede l'adozione dei principi di \textit{privacy by design} e \textit{privacy by default} (art. 25 GDPR) sin dalle prime fasi di sviluppo dell'applicazione (es. Flutter) e dell'infrastruttura.
\begin{itemize}
    \item \textbf{Minimizzazione dei dati:} Raccogliere solo i dati strettamente necessari per la finalità dichiarata.
    \item \textbf{Limitazione della finalità:} Utilizzare i dati solo per le finalità per cui sono stati raccolti e per cui è stato fornito il consenso (se applicabile) o sussiste altra base giuridica.
    \item \textbf{Limitazione della conservazione:} Conservare i dati solo per il tempo necessario al raggiungimento delle finalità. Per i dati creditizi, il Codice di condotta stabilisce tempi specifici (es. dati positivi storici fino a 60 mesi).
    \item \textbf{Misure di sicurezza adeguate (art. 32 GDPR):} Implementare pseudonimizzazione, cifratura, controlli di accesso granulari, resilienza dei sistemi.
    \item \textbf{Gestione dei diritti degli interessati (art. 15-22 GDPR):} Progettare sistemi che facilitino l'esercizio dei diritti di accesso, rettifica, cancellazione (diritto all'oblio), limitazione del trattamento, portabilità dei dati e opposizione.
\end{itemize}
L'architettura cloud (AWS/Azure) deve implementare misure tecniche e organizzative appropriate, inclusa la cifratura dei dati a riposo e in transito, la gestione sicura delle chiavi di cifratura, e controlli di accesso IAM granulari. È fondamentale redigere e mantenere un Registro delle Attività di Trattamento (art. 30 GDPR) e, se necessario (trattamenti a rischio elevato), condurre una Valutazione d'Impatto sulla Protezione dei Dati (DPIA - art. 35 GDPR). La notifica di eventuali data breach all'autorità di controllo e agli interessati (art. 33-34 GDPR) deve essere gestita tempestivamente.

Per quanto riguarda la conservazione dei dati, i dati storici positivi dei soggetti analizzati possono essere conservati per un massimo di 60 mesi dalla data di scadenza del rapporto o dalla data dell'ultimo aggiornamento, a tutela del credito e per rispondere alle richieste degli organismi di vigilanza. Tuttavia, questa conservazione deve essere bilanciata con i principi di minimizzazione. È inoltre possibile implementare sistemi di preavviso (es. via SMS o email tracciabile) per annunciare l'iscrizione di record negativi nei SIC, previa acquisizione del consenso informato degli interessati per tale modalità di comunicazione.

\section{Direttive PSD2 e PSD3 per i Servizi di Pagamento}
\label{sec:psd}

\subsection{Evoluzione del Framework Normativo}
La Seconda Direttiva sui Servizi di Pagamento (PSD2 - Direttiva UE 2015/2366) ha rivoluzionato il panorama dei pagamenti digitali in Europa, introducendo il concetto di \textit{open banking} e rafforzando la sicurezza delle transazioni online. La PSD2, entrata in vigore nel 2016 come aggiornamento della direttiva PSD originaria del 2007, ha imposto requisiti di Autenticazione Forte del Cliente (Strong Customer Authentication - SCA) per la maggior parte delle transazioni elettroniche al fine di tutelare dalle frodi, e ha richiesto alle banche (Account Servicing Payment Service Providers - ASPSP) di rendere disponibili i propri servizi di pagamento e i dati dei conti dei clienti, con il loro consenso, a fornitori terzi autorizzati (Third Party Providers - TPP), favorendo la creazione di nuovi prodotti e servizi finanziari.

La recente proposta della Commissione Europea per una nuova Direttiva sui Servizi di Pagamento (PSD3) e un Regolamento sui Servizi di Pagamento (PSR) rappresenta un'ulteriore evoluzione del framework, mirata a migliorare ulteriormente la PSD2, promuovere la concorrenza equa, migliorare la sicurezza dei pagamenti, rafforzare i diritti dei consumatori e facilitare l'accesso ai dati nell'ambito dell'\textit{open finance}. La PSD3 e il PSR mirano a superare le criticità emerse dall'attuazione della PSD2 (es. frammentazione delle API, ostacoli all'accesso per i TPP), aumentando la fiducia dei consumatori nei confronti dei pagamenti elettronici e rendendo più rapide ed efficienti le transazioni finanziarie. Basandosi sui progressi realizzati dalla PSD2, questa nuova normativa affronta questioni cruciali come il miglioramento dell'SCA, l'accesso ai sistemi di pagamento per i PSP non bancari e il potenziamento delle funzionalità dell'open banking verso l'open finance.

\subsection{Componenti Tecnici e Implementazione}
I componenti principali della PSD2 (e che verranno ulteriormente raffinati dalla PSD3/PSR) richiedono implementazioni tecniche specifiche che le startup fintech devono integrare nella loro architettura .
\begin{itemize}
    \item \textbf{Strong Customer Authentication (SCA):} Richiede l'autenticazione a più fattori per la maggior parte delle transazioni online avviate dal pagatore. L'autenticazione deve utilizzare almeno due elementi appartenenti a categorie diverse: conoscenza (qualcosa che solo l'utente conosce, es. password, PIN), possesso (qualcosa che solo l'utente possiede, es. token, smartphone su cui riceve un OTP) e inerenza (qualcosa che l'utente è, es. impronta digitale, riconoscimento facciale). Sono previste esenzioni per transazioni a basso rischio, pagamenti di basso valore, beneficiari di fiducia, ecc.
    \item \textbf{Accesso ai Conti (XS2A) e Open Banking:} Impone agli ASPSP di fornire ai TPP autorizzati l'accesso ai conti di pagamento dei clienti tramite interfacce dedicate sicure (tipicamente API), previo consenso esplicito del cliente. I TPP si dividono principalmente in:
    \begin{itemize}
        \item \textbf{Payment Initiation Service Providers (PISP):} Possono avviare ordini di pagamento per conto dell'utente dal suo conto bancario.
        \item \textbf{Account Information Service Providers (AISP):} Possono accedere alle informazioni dei conti di pagamento dell'utente per fornire servizi di aggregazione e analisi (es. personal financial management).
    \end{itemize}
\end{itemize}
L'implementazione tecnica in una startup fintech richiede lo sviluppo di interfacce sicure per l'integrazione con le API degli ASPSP (se la fintech agisce come TPP) o l'esposizione di proprie API sicure (se la fintech è un ASPSP o offre servizi assimilabili). Nell'applicazione (es. Flutter), questo richiede:
\begin{itemize}
    \item Implementazione di flussi di autenticazione SCA conformi, possibilmente delegando parte del processo all'ASPSP dell'utente.
    \item Gestione sicura dei consensi degli utenti per l'accesso ai dati e l'iniziazione dei pagamenti.
    \item Integrazione con API standardizzate (es. basate su standard come Open Banking UK, Berlin Group NextGenPSD2) o specifiche delle banche, con gestione robusta degli errori e della riconciliazione delle transazioni.
    \item Protezione delle comunicazioni tramite TLS con mutua autenticazione (mTLS) e utilizzo di certificati qualificati (eIDAS QWAC e QSealC) per l'identificazione dei TPP.
\end{itemize}
La PSD2 ha introdotto maggiore trasparenza nelle tariffe e il divieto di surcharge per i pagamenti con carta più comuni. L'architettura cloud (AWS/Azure) deve supportare la scalabilità necessaria per gestire picchi di transazioni, garantire bassa latenza e implementare un monitoraggio completo per assicurare disponibilità e performance, oltre a log di audit dettagliati per la conformità.

La transizione verso PSD3/PSR richiederà alle fintech di prepararsi per ulteriori modifiche, come una maggiore condivisione dei dati sui pagamenti per alimentare l'innovazione (verso l'open finance), un rafforzamento delle misure antifrode, e un miglioramento dell'esperienza utente nell'applicazione dell'SCA. Le startup fintech devono quindi progettare architetture flessibili e modulari.

\section{Anti-Money Laundering Directive (AMLD)}
\label{sec:amld}

\subsection{Framework Normativo per il Contrasto al Riciclaggio}
Le Direttive Antiriciclaggio (AMLD), giunte alla Quinta Direttiva (AMLD5 - Direttiva UE 2018/843) con un nuovo pacchetto legislativo (che include la Sesta Direttiva AMLD6 e un Regolamento AMLR) in fase di finalizzazione per rafforzare ulteriormente il quadro UE, stabiliscono il framework europeo per il contrasto al riciclaggio di denaro (AML - Anti-Money Laundering) e al finanziamento del terrorismo (CFT - Countering the Financing of Terrorism). Per le startup fintech, l'applicazione delle AMLD è particolarmente critica data la natura digitale dei servizi offerti e la potenziale esposizione a rischi di riciclaggio attraverso transazioni elettroniche transfrontaliere e l'uso di nuove tecnologie. La normativa richiede ai soggetti obbligati (tra cui molte fintech) l'implementazione di sistemi robusti di Adeguata Verifica della Clientela (\textit{Customer Due Diligence - CDD}), inclusa l'identificazione e verifica dell'identità del cliente e del titolare effettivo (\textit{Ultimate Beneficial Owner - UBO}), una Verifica Rafforzata (\textit{Enhanced Due Diligence - EDD}) per clienti o situazioni ad alto rischio, e procedure di monitoraggio continuo delle transazioni e di segnalazione delle operazioni sospette (\textit{Suspicious Transaction Reporting - STR}) alle Unità di Informazione Finanziaria (UIF) nazionali.

L'AMLD5 ha esteso significativamente il perimetro dei soggetti obbligati, includendo i fornitori di servizi di cambio tra valute virtuali e valute legali (fiat), i fornitori di servizi di portafoglio digitale (custodial wallet providers) e, in alcune giurisdizioni, le piattaforme di crowdfunding. Questo ampliamento è particolarmente rilevante per le startup fintech che operano nel settore dei pagamenti digitali, delle criptovalute, o del lending P2P. La direttiva richiede l'implementazione di sistemi di identificazione e verifica dell'identità dei clienti (KYC - Know Your Customer), il monitoraggio continuo delle transazioni e il mantenimento di registri dettagliati per periodi specificati.

\subsection{Implementazione Tecnica dei Controlli AML}
L'implementazione dei controlli AML in una startup fintech richiede l'integrazione di sistemi, spesso automatizzati, di monitoraggio delle transazioni e di screening dei clienti nell'architettura tecnologica esistente.
\begin{itemize}
    \item \textbf{Onboarding e KYC:} L'applicazione (es. Flutter) deve implementare processi di onboarding sicuri e conformi che includano la raccolta dei dati identificativi del cliente, la verifica dei documenti (es. tramite scansione e riconoscimento ottico, video-identificazione), l'uso di autenticazione biometrica (se appropriato e conforme al GDPR), e la verifica dell'identità in tempo reale attraverso l'integrazione con database affidabili e indipendenti o provider specializzati.
    \item \textbf{Screening:} I sistemi devono effettuare lo screening dei clienti (e dei titolari effettivi) rispetto a liste di sanzioni internazionali (es. OFAC, ONU, UE), liste di Persone Politicamente Esposte (PEP) e liste di notizie avverse (adverse media), sia in fase di onboarding che su base continuativa.
    \item \textbf{Monitoraggio delle Transazioni:} L'infrastruttura cloud (AWS/Azure) deve supportare l'implementazione di sistemi di monitoraggio delle transazioni in tempo reale o quasi reale, capaci di identificare pattern sospetti e generare alert automatici basati su regole predefinite, scenari di rischio e, sempre più, algoritmi di machine learning. Questi sistemi devono analizzare variabili come frequenza, importo, origine/destinazione geografica, controparti e tipologia di transazione per identificare potenziali attività di riciclaggio.
    \item \textbf{Gestione dei Rischi:} Implementare un approccio basato sul rischio per classificare i clienti e applicare misure di CDD o EDD appropriate.
    \item \textbf{Reporting e Record-Keeping:} I sistemi devono facilitare la generazione di report per le segnalazioni di operazioni sospette (STR) e mantenere registri completi e auditabili di tutte le attività di CDD e delle transazioni per il periodo richiesto dalla normativa (generalmente almeno 5 anni).
\end{itemize}
Il repository GitHub deve seguire pratiche di codifica sicura specifiche per i sistemi AML, garantendo l'integrità e la riservatezza dei dati sensibili dei clienti (tramite cifratura, controlli di accesso granulari) e la tracciabilità delle modifiche al software che gestisce tali controlli. L'architettura deve essere resiliente e scalabile per gestire il volume di dati e transazioni.

\section{Digital Operational Resilience Act (DORA)}
\label{sec:dora}

\subsection{Framework per la Resilienza Operativa Digitale}
Il Digital Operational Resilience Act (DORA - Regolamento UE 2022/2554) rappresenta un'importante evoluzione del framework normativo europeo per la gestione dei rischi operativi digitali nel settore finanziario. Questa normativa, applicabile a partire dal 17 gennaio 2025, stabilisce requisiti uniformi e completi per la resilienza operativa digitale di quasi tutte le entità finanziarie regolamentate nell'UE, incluse banche, imprese di investimento, gestori di fondi, compagnie di assicurazione, fornitori di servizi di cripto-asset (MiCA), e anche molte startup fintech a seconda dei servizi offerti e delle licenze possedute. DORA introduce un approccio olistico alla gestione dei rischi connessi alle Tecnologie dell'Informazione e della Comunicazione (TIC), richiedendo l'implementazione di framework completi per:
\begin{itemize}
    \item \textbf{Gestione dei rischi TIC:} Inclusa l'identificazione, protezione e prevenzione, rilevamento, risposta e ripristino.
    \item \textbf{Gestione, classificazione e segnalazione degli incidenti TIC} significativi alle autorità competenti.
    \item \textbf{Test di resilienza operativa digitale:} Inclusi test di vulnerabilità, test di penetrazione basati sulle minacce (TLPT) per le entità più critiche.
    \item \textbf{Gestione dei rischi derivanti da terze parti TIC:} Con un focus particolare sui fornitori di servizi TIC critici (CTPP), che saranno soggetti a un quadro di sorveglianza diretta a livello UE.
    \item \textbf{Condivisione di informazioni e intelligence} relative a minacce e vulnerabilità informatiche.
\end{itemize}
Per le startup fintech, DORA rappresenta una sfida significativa, ma anche un'opportunità per rafforzare la propria postura di sicurezza, data la loro dipendenza critica da infrastrutture cloud, software e servizi digitali per l'erogazione dei servizi finanziari. La normativa richiede l'implementazione di sistemi robusti di continuità operativa, piani di ripristino in caso di disastro e capacità di risposta agli incidenti. L'approccio basato sul rischio di DORA richiede una valutazione continua dell'esposizione ai rischi TIC e l'implementazione di misure di mitigazione proporzionate alla dimensione, al profilo di rischio e alla criticità delle operazioni dell'entità.

\subsection{Implementazione Operativa dei Requisiti DORA}
L'implementazione di DORA in una startup fintech richiede un approccio strutturato che integri i requisiti di resilienza operativa in tutti gli aspetti dell'architettura tecnologica e dell'organizzazione.
\begin{itemize}
    \item \textbf{Governance e Framework di Gestione dei Rischi TIC:} Il framework di governance TIC deve essere integrato nella struttura organizzativa, con responsabilità chiare definite a livello di organo di gestione. Deve essere istituito un framework di gestione dei rischi TIC che copra l'intero ciclo di vita (identificazione, valutazione, trattamento, monitoraggio e reporting).
    \item \textbf{Protezione e Prevenzione:} Implementare sistemi e controlli di sicurezza aggiornati e resilienti. L'applicazione (es. Flutter) deve essere progettata con pattern di resilienza (es. circuit breakers, retry mechanisms, graceful degradation) per mantenere le funzionalità critiche anche in caso di disruption parziali. L'architettura cloud (AWS/Azure) deve essere configurata per l'alta disponibilità (es. multi-AZ/multi-region deployment), con procedure automatizzate di backup e ripristino, load balancing e auto-scaling.
    \item \textbf{Rilevamento e Gestione degli Incidenti:} Implementare meccanismi per il rilevamento tempestivo degli incidenti TIC e procedure chiare per la loro gestione, classificazione (in base a criteri che saranno definiti dagli standard tecnici RTS/ITS) e segnalazione alle autorità.
    \item \textbf{Test di Resilienza Operativa Digitale:} Stabilire un programma di test completo e basato sul rischio, che includa valutazioni delle vulnerabilità, test di sicurezza delle applicazioni, test di penetrazione e, per le entità più significative, test di penetrazione avanzati (TLPT). I repository GitHub devono supportare procedure di test automatizzate e mantenere una documentazione completa dei risultati dei test e delle azioni di remediation.
    \item \textbf{Gestione dei Rischi da Terze Parti TIC:} Mappare le dipendenze da fornitori di servizi TIC, valutare i rischi associati (inclusi i rischi di concentrazione), definire strategie di uscita e includere clausole contrattuali specifiche che garantiscano i diritti di accesso, ispezione e audit. Particolare attenzione va posta ai fornitori cloud come AWS/Azure, che rientrano pienamente in questa categoria.
    \item \textbf{Continuità Operativa e Ripristino:} Sviluppare e testare piani di continuità operativa e di ripristino in caso di disastro per garantire il ripristino delle funzioni critiche entro obiettivi di tempo (RTO/RPO) definiti.
\end{itemize}
La documentazione è un elemento chiave di DORA, richiedendo di mantenere un inventario aggiornato dei sistemi TIC, delle dipendenze e dei processi supportati.

\section{Implementazione di Portafoglio Digitale e Servizi di Investimento}
\label{sec:digital_wallet_investment}

\subsection{Architettura per Portafoglio Digitale P2P}
L'implementazione di un portafoglio digitale (digital wallet) con funzionalità di pagamento Peer-to-Peer (P2P) e pagamenti in negozio (tramite NFC, QR code) richiede un'architettura tecnologica complessa che integri tutti i requisiti normativi discussi precedentemente (PSD2/3 per i pagamenti, AMLD per il KYC/AML, GDPR per la privacy, DORA per la resilienza, NIS2 e ISO 27001 per la sicurezza generale).

L'applicazione client (es. sviluppata in Flutter) deve:
\begin{itemize}
    \item Implementare processi di onboarding sicuri e conformi AMLD (KYC).
    \item Gestire in modo sicuro le credenziali di pagamento (es. tokenizzazione delle carte secondo PCI DSS, se applicabile).
    \item Supportare l'Autenticazione Forte del Cliente (SCA) per l'accesso al portafoglio e l'autorizzazione delle transazioni, come richiesto da PSD2/3.
    \item Integrare funzionalità di pagamento come scansione/generazione di QR code per trasferimenti P2P e pagamenti a esercenti, e interfacce NFC per pagamenti contactless.
    \item Garantire la privacy dei dati degli utenti in conformità con il GDPR.
    \item Essere sviluppata seguendo pratiche di codifica sicura e testata per vulnerabilità.
\end{itemize}
Il backend, ospitato su cloud (AWS/Azure), deve:
\begin{itemize}
    \item Gestire i profili utente, i saldi dei portafogli e lo storico delle transazioni.
    \item Processare le transazioni di pagamento in modo sicuro e resiliente, integrandosi con circuiti di pagamento o sistemi bancari (es. tramite API PSD2).
    \item Implementare sistemi di monitoraggio delle transazioni per AML e prevenzione frodi.
    \item Esporre API sicure per l'app client e, potenzialmente, per partner terzi.
    \item Garantire alta disponibilità, scalabilità e capacità di ripristino in caso di disastro (DORA).
    \item Registrare log di audit completi per tutte le operazioni.
    \item Proteggere i dati con cifratura a riposo e in transito.
\end{itemize}
La sicurezza dell'infrastruttura deve essere gestita secondo i principi di ISO 27001 e NIS2, con segmentazione di rete, controlli di accesso IAM, monitoraggio continuo e un piano di risposta agli incidenti.

\subsection{Integrazione Investment-as-a-Service}
L'implementazione di servizi di investimento, ad esempio attraverso l'integrazione di API da provider di \textit{Investment-as-a-Service} (IaaS) o operando come impresa di investimento, introduce complessità normative e tecniche aggiuntive, principalmente legate alla direttiva MiFID II / MiFIR (Markets in Financial Instruments Directive/Regulation) e alle normative nazionali di attuazione.

L'applicazione client (Flutter) e il backend devono:
\begin{itemize}
    \item Gestire l'onboarding dei clienti per i servizi di investimento, che include la raccolta di informazioni per la valutazione di adeguatezza e appropriatezza (MiFID II).
    \item Presentare in modo chiaro e trasparente le informazioni sugli strumenti finanziari, i costi, gli oneri e i rischi associati.
    \item Permettere agli utenti di visualizzare il proprio portafoglio di investimenti, le performance e lo storico delle transazioni.
    \item Implementare flussi sicuri per l'inoltro di ordini di acquisto/vendita, con adeguata autenticazione e conferma.
    \item Integrare in modo sicuro le API del provider IaaS, gestendo autenticazione (es. OAuth 2.0), autorizzazione, rate limiting e gestione degli errori.
    \item Garantire la protezione dei dati degli investitori (GDPR) e la sicurezza delle comunicazioni.
\end{itemize}
Il backend deve inoltre:
\begin{itemize}
    \item Mantenere registrazioni dettagliate di tutte le comunicazioni con i clienti e le transazioni (record-keeping MiFID II).
    \item Implementare i requisiti di best execution se la fintech è responsabile dell'esecuzione degli ordini.
    \item Gestire la reportistica normativa (es. transaction reporting a CONSOB/ESMA).
    \item Assicurare la resilienza operativa dell'infrastruttura che supporta i servizi di investimento (DORA).
\end{itemize}
L'integrazione con provider IaaS richiede una rigorosa due diligence sul fornitore (gestione dei rischi da terze parti secondo DORA e ISO 27001) e la definizione chiara delle responsabilità contrattuali.

\chapter{Conclusioni e Prospettive Future}
\label{chap:conclusioni}

L'analisi condotta in questa tesi ha evidenziato la criticità della cybersecurity e della conformità normativa per le startup fintech che intendono operare con successo e sostenibilità nel panorama finanziario moderno. L'adozione di framework di sicurezza riconosciuti come il NIST CSF e l'ISO/IEC 27001, unitamente a standard tecnici come NIST SP 800-53 e principi emergenti quali Zero Trust, fornisce una solida base per la costruzione di infrastrutture resilienti, specialmente in ambienti cloud come AWS.

Tuttavia, la sola implementazione tecnica non è sufficiente. Il complesso intreccio di normative europee e nazionali – tra cui NIS2, GDPR, PSD2/3, AMLD e il nuovo DORA – impone alle startup fintech un approccio olistico che integri la sicurezza e la conformità \textit{by design} in ogni aspetto del loro modello di business e della loro architettura tecnologica. Questo richiede non solo competenze tecniche specializzate, ma anche una profonda comprensione del quadro legale e una cultura aziendale orientata alla gestione proattiva del rischio.

Le startup che utilizzano tecnologie moderne come Flutter per lo sviluppo mobile, GitHub per la gestione del codice e piattaforme cloud come AWS/Azure, possono beneficiare della flessibilità e scalabilità di tali strumenti, ma devono essere consapevoli delle responsabilità che ne derivano in termini di configurazione sicura e monitoraggio continuo. L'implementazione di portafogli digitali e servizi di investimento, in particolare, accentua queste sfide, richiedendo un'attenzione meticolosa ai dettagli normativi specifici di tali servizi.

In prospettiva futura, è prevedibile un'ulteriore evoluzione del panorama normativo e delle minacce informatiche. Le startup fintech dovranno quindi investire costantemente in:
\begin{itemize}
    \item \textbf{Aggiornamento continuo delle competenze:} Per rimanere al passo con le nuove tecnologie, le nuove minacce e le evoluzioni normative.
    \item \textbf{Automazione della sicurezza e della compliance:} Per gestire la complessità in modo efficiente e ridurre il rischio di errore umano.
    \item \textbf{Collaborazione e condivisione di informazioni:} Sia internamente che con altre entità del settore e con le autorità, per migliorare la capacità collettiva di prevenzione e risposta.
    \item \textbf{Resilienza operativa:} Non solo per conformarsi a DORA, ma come imperativo strategico per garantire la continuità del business e la fiducia dei clienti.
\end{itemize}
In conclusione, sebbene le sfide siano significative, le startup fintech che abbracciano un approccio maturo e proattivo alla cybersecurity e alla conformità normativa non solo mitigano i rischi, ma possono anche differenziarsi sul mercato, costruendo un vantaggio competitivo basato sulla fiducia, sulla sicurezza e sulla resilienza.