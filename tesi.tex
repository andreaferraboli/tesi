
%
\documentclass[a4paper,12pt]{report}
%    \renewcommand{\baselinestretch}{1.6}      % interline spacing
%
% \includeonly{}
%
%			PREAMBOLO
%
\usepackage[a4paper]{geometry}
\usepackage{amssymb,amsmath,amsthm}
\usepackage{graphicx}
\usepackage{url}
\usepackage{hyperref}
\usepackage{epsfig}
\usepackage[italian]{babel}
\usepackage{setspace}
\usepackage{tesi}

% per le accentate
\usepackage[utf8]{inputenc}
%
\newtheorem{myteor}{Teorema}[section]
%
\newenvironment{teor}{\begin{myteor}\slshape}{\end{myteor}}
%
%
%			TITOLO
%
\begin{document}

% Aggiunta immagine sopra al titolo
\begin{center}
        \includegraphics[width=0.2\textwidth]{images/Unimi-logo.png}
        \vspace{1cm}
\end{center}

\title{Sicurezza dell'Infrastruttura AWS in una Startup Fintech: Sfide, Best Practices e Implementazione di un Modello di Sicurezza Resiliente Low Cost}
\author{Andrea Ferraboli}
\dept{Corso di Laurea in Sicurezza dei Sistemi e delle Reti Informatiche } 
\anno{2024-2025}
\matricola{09985A}
\relatore{Prof. Claudio Agostino Ardagna}
\correlatore{Lorenzo Perotta, Andrea Pasini, Simone Cortese}
%
%        \submitdate{month year in which submitted to GPO}
%		- date LaTeX'd if omitted
%	\copyrightyear{year degree conferred (next year if submitted in Dec.)}
%		- year LaTeX'd (or next year, in December) if omitted
%	\copyrighttrue or \copyrightfalse
%		- produce or don't produce a copyright page (false by default)
%	\figurespagetrue or \figurespagefalse
%		- produce or don't produce a List of Figures page
%		  (false by default)
%	\tablespagetrue or \tablespagefalse
%		- produce or don't produce a List of Tables page
%		  (false by default)
% 
%			DEDICA
%
\beforepreface
\prefacesection{}
        {\hfill \Large {\sl dedicato a \dots}}
% 
%			PREFAZIONE
%
\prefacesection{Prefazione}
%			ORGANIZZAZIONE

\tableofcontents

\chapter*{Introduzione}
\addcontentsline{toc}{chapter}{Introduzione}
\section{Contesto: Crescita delle startup fintech e importanza della sicurezza}
\section{Obiettivi della tesi e domande di ricerca}
\subsection{Quali sono le principali sfide di cybersecurity per una startup fintech che utilizza AWS?}
\subsection{Quali sono le best practice di sicurezza di AWS più rilevanti per una startup fintech?}
\subsection{Come si può implementare un'infrastruttura AWS sicura e resiliente per una startup fintech?}
\section{Struttura della tesi}

\chapter{Principi}
\section{Il concetto di "Security by Design"}
\section{Best practice e principi di sicurezza}

\chapter{Fondamenti di Sicurezza su AWS}
\section{Modello di responsabilità condivisa di AWS}
\section{Best practice di sicurezza specifiche per startup fintech}
\section{Panoramica dei principali servizi di sicurezza di AWS}
\subsection{AWS Identity and Access Management (IAM)}
\subsection{AWS Key Management Service (KMS)}
\subsection{AWS CloudTrail e Amazon CloudWatch (servizi concorrenti)}
\subsection{AWS GuardDuty, Amazon Inspector e Amazon Macie (alternativa valida)}
\subsection{Altri servizi rilevanti (es. AWS WAF, VPC)}

\chapter{Gestione delle Identità e degli Accessi (IAM)}
\section{Implementazione del principio del minimo privilegio}
\section{Creazione e gestione di utenti, gruppi e ruoli IAM}
\section{Utilizzo di policy IAM per concedere permessi granulari}
\section{Configurazione dell'autenticazione a più fattori (MFA)}
\section{Gestione delle credenziali (es. utilizzo di IAM Roles per EC2)}
\section{Audit e monitoraggio degli accessi IAM}

\chapter{Monitoraggio e Logging della Sicurezza}
\section{Configurazione di AWS CloudTrail per tracciare le attività degli utenti}
\section{Implementazione di Amazon CloudWatch per monitorare le metriche di sistema}
\section{Creazione di allarmi per eventi specifici e attività sospette}
\section{Utilizzo di AWS GuardDuty per il rilevamento automatico di minacce}
\section{Integrazione con sistemi SIEM (Security Information and Event Management)}
\section{Gestione e analisi dei log}

\chapter{Protezione dei Dati Sensibili e Conformità Normativa}
\section{Crittografia dei dati a riposo e in transito}
\section{Gestione delle chiavi di crittografia con AWS KMS o CloudHSM}
\section{Implementazione di meccanismi per la protezione dei dati in S3}
\section{Misure per la conformità a PCI DSS (se rilevante)}
\section{Misure per la conformità al GDPR e protezione dei dati personali}
\section{Valutazione e gestione del rischio di perdita di dati}

\chapter{Casi Studio e Implementazione Pratica}
\section{Esempio di architettura di sicurezza AWS per una startup fintech (proposta)}
\section{Implementazione delle best practice descritte nei capitoli precedenti}
\section{Test di penetrazione e valutazione della sicurezza dell'ambiente}
\section{Analisi dei risultati e confronto con le best practice}
\section{Integrazione di strumenti di sicurezza terzi (es. SentinelOne)}
\section{Infrastruttura di base AWS, integrazione codice e infrastruttura di un sistema honeypot}
\section{Deadcode per confondere malware (Capitolo 7)}
\section{Autenticazione con chiavi pubbliche (Capitolo 8)}
\section{Progettazione e implementazione di una Virtual Private Cloud (VPC) isolata}
\subsection{Creazione di subnet pubbliche e private}
\subsection{Utilizzo di gruppi di sicurezza e ACL per controllare il traffico di rete}
\subsection{Implementazione di Network Address Translation (NAT) e VPN/Direct Connect}
\subsection{Configurazione di load balancer per alta disponibilità e scalabilità}
\subsection{Utilizzo di container (es. ECS o EKS) per una maggiore sicurezza e scalabilità}

\chapter{Discussione e Conclusioni}
\section{Rielaborazione delle domande di ricerca iniziali e discussione dei risultati}
\section{Riflessioni sulle sfide e opportunità per la sicurezza di AWS nelle startup fintech}
\section{Prospettive future per la ricerca e l'innovazione}
\section{Raccomandazioni per la creazione di un modello di cybersecurity resiliente per startup fintech}

\chapter*{Bibliografia}
\addcontentsline{toc}{chapter}{Bibliografia}
Elenco di tutti i materiali consultati durante la stesura della tesi.


%
%			RINGRAZIAMENTI
%
\prefacesection{Ringraziamenti}
asdjhgftry.
\afterpreface
% 
% 
%			CAPITOLO 1: dshjkfg
\chapter{Introduzione}
\label{cap1}
%
%

%
%			BIBLIOGRAFIA
%
\begin{thebibliography}{00}
%
\bibitem{gotti91}
M. Gotti, I linguaggi specialistici, Firenze, La Nuova Italia, 1991.
%
\bibitem{wellek62}
R. Wellek, A. Warren, Theory of Literature , 3rd edition, New York, Harcourt, 1962.
%
\bibitem{canziani78}
A. Canziani et al., Come comunica il teatro: dal testo alla scena. Milano, Il Formichiere, 1978.
%
\bibitem{MoD67}
Ministry of Defence, Great Britain, Author and Subject Catalogues of the Naval Library, London, Ministry of Defence, HMSO, 1967.
%
\bibitem{heine23}
H. Heine, Pensieri e ghiribizzi. A cura di A. Meozzi. Lanciano, Carabba, 1923.
%
\bibitem{basso62}
L. Basso, ``Capitalismo monopolistico e strategia operaia'', Problemi del socialismo, vol. 8, n. 5, pp. 585-612, 1962.
%
\bibitem{avirovic93}
L. Avirovic, J. Dodds (a cura di), Atti del Convegno internazionale "Umberto Eco, Claudio Magris. Autori e traduttori a confronto" ( Trieste, 27-28 novembre 1989), Udine, Campanotto, 1993.
%
\bibitem{gans67}
E.L. Gans, "The Discovery of Illusion: Flaubert's Early Works, 1835-1837", unpublished Ph.D. Dissertation, Johns Hopkins University, 1967.
%
\bibitem{harrison92}
R. Harrison, Bibliography of planned languages (excluding Esperanto).  \url{http://www.vor.nu/langlab/bibliog.html}, 1992, agg. 1997.
%
\end{thebibliography}
% 
\end{document}


 
s