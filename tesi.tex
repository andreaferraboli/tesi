\documentclass[a4paper,12pt]{book}
\usepackage[utf8]{inputenc}
\usepackage[italian]{babel}
\usepackage{graphicx}
\usepackage{amsmath, amssymb}
\usepackage{hyperref}
\usepackage{geometry}
\geometry{a4paper, margin=1in}
\usepackage{fancyhdr}
\pagestyle{fancy}
\fancyhead{}
\fancyfoot{}
\fancyhead[LE,RO]{\thepage}
\fancyhead[RE]{\leftmark}
\fancyhead[LO]{\rightmark}

% Metadata
\title{Implementazione della Cybersecurity in una Startup Fintech: Il Caso Finanz}
\author{Andrea Ferraboli}
\date{Anno Accademico 2024-2025}

\begin{document}

% Frontespizio
\begin{titlepage}
    \begin{center}
        \includegraphics[width=0.2\textwidth]{images/Unimi-logo.png} \\[1cm]
        \textsc{\LARGE Università degli Studi di Milano}
        \textsc{\Large Facoltà di Scienze e Tecnologie}
        \vspace{0.5cm}
        \textsc{\Large Dipartimento di Informatica} \\
        \textbf{\huge Implementazione della Cybersecurity in una Startup Fintech: Il Caso Finanz} \\
        \vspace{1.5cm}
        \emph{Relatore:} Prof. Giovanni Degli Antoni \\
        \emph{Correlatori:} Prof. Brian W. Kernighan, Prof. Dennis M. Ritchie \\
        \vspace{2cm}
        \textbf{Elaborato Finale di:} \\
        Andrea Ferraboli \\
        Matr. Nr. 09985a \\
        \vfill
        Anno Accademico 2024-2025
    \end{center}
\end{titlepage}

% Dedica
\cleardoublepage
\thispagestyle{empty}
\begin{flushright}
    \emph{Questo lavoro è dedicato ai miei genitori}\\[1cm]
    \textit{\guillemotleft What I cannot create, I do not understand \guillemotright} -- Richard Feynman\\
    \textit{\guillemotleft It’s not only powerful, but it’s also inadequate \guillemotright} -- Miller Puckette
\end{flushright}

% Ringraziamenti
\chapter*{Ringraziamenti}
Questa sezione, facoltativa, contiene i ringraziamenti.
\addcontentsline{toc}{chapter}{Ringraziamenti}

% Indice
\tableofcontents

\chapter{Introduzione}
\section{Contesto e motivazioni}
\section{Obiettivi della tesi}
\section{Struttura del lavoro}

\chapter{Panoramica sulla Cybersecurity nel Settore Fintech}
\section{Definizione di Fintech e sue peculiarità}
\section{Minacce e vulnerabilità nel settore Fintech}
\section{Regolamentazioni e standard di sicurezza}

\chapter{Sicurezza a Livello di Codice}
\section{Introduzione alla Secure Coding}
\section{Principi di Secure Coding}
\subsection{Validazione degli input}
\subsection{Gestione sicura delle sessioni}
\subsection{Prevenzione delle vulnerabilità comuni (es. SQL Injection, XSS)}
\section{Approccio SecDevOps}
\subsection{Integrazione della sicurezza nel ciclo di sviluppo}
\subsection{Strumenti per l'analisi statica e dinamica del codice}
\subsection{Automazione dei test di sicurezza}

\chapter{Sicurezza a Livello di Infrastruttura}
\section{Architettura sicura per una startup Fintech}
\subsection{Design dell'infrastruttura cloud}
\subsection{Segmentazione della rete e controllo degli accessi}
\section{Gestione sicura dei dati}
\subsection{Crittografia dei dati a riposo e in transito}
\subsection{Gestione delle chiavi crittografiche}
\subsection{Backup e ripristino dei dati}
\section{Monitoraggio e risposta agli incidenti}
\subsection{Sistemi di rilevamento delle intrusioni (IDS)}
\subsection{Piani di risposta agli incidenti}

\chapter{Ingegneria Sociale e Sensibilizzazione}
\section{Definizione e tipologie di attacchi di ingegneria sociale}
\subsection{Phishing}
\subsection{Pretexting}
\subsection{Baiting}
\section{Simulazione di attacchi ai dipendenti}
\subsection{Metodologie di simulazione}
\subsection{Analisi dei risultati e miglioramenti}
\section{Diffusione della cultura della sicurezza}
\subsection{Formazione e awareness}
\subsection{Politiche aziendali e best practices}

\chapter{Aspetti Legali e di Compliance}
\section{Normative di riferimento (es. GDPR, PSD2)}
\section{Valutazione del rischio e audit di sicurezza}
\section{Conformità e certificazioni}

\chapter{Conclusioni}
\section{Risultati raggiunti}
\section{Prospettive future}


% Bibliografia
\begin{thebibliography}{9}
\bibitem{feynman} Richard Feynman. \textit{What I cannot create, I do not understand.}
\bibitem{puckette} Miller Puckette. \textit{It’s not only powerful, but it’s also inadequate.}
\end{thebibliography}

\end{document}
