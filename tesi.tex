\documentclass[a4paper,12pt]{book}
\usepackage[utf8]{inputenc}
\usepackage[italian]{babel}
\usepackage{graphicx}
\usepackage{amsmath, amssymb}
\usepackage{hyperref}
\usepackage{geometry}
\geometry{a4paper, margin=1in}
\usepackage{fancyhdr}
\pagestyle{fancy}
\fancyhead{}
\fancyfoot{}
\fancyhead[LE,RO]{\thepage}
\fancyhead[RE]{\leftmark}
\fancyhead[LO]{\rightmark}

% Metadata
\title{Implementazione della Cybersecurity in una Startup Fintech: Il Caso Finanz}
\author{Andrea Ferraboli}
\date{Anno Accademico 2024-2025}

\begin{document}

% Frontespizio
\begin{titlepage}
    \begin{center}
        \includegraphics[width=0.2\textwidth]{images/Unimi-logo.png} \\[1cm]
        \textsc{\LARGE Università degli Studi di Milano}
        \textsc{\Large Facoltà di Scienze e Tecnologie}
        \vspace{0.5cm}
        \textsc{\Large Dipartimento di Informatica} \\
        \textbf{\huge Implementazione della Cybersecurity in una Startup Fintech: Il Caso Finanz} \\
        \vspace{1.5cm}
        \emph{Relatore:} Prof. Giovanni Degli Antoni \\
        \emph{Correlatori:} Prof. Brian W. Kernighan, Prof. Dennis M. Ritchie \\
        \vspace{2cm}
        \textbf{Elaborato Finale di:} \\
        Andrea Ferraboli \\
        Matr. Nr. 09985a \\
        \vfill
        Anno Accademico 2024-2025
    \end{center}
\end{titlepage}

% Dedica
\cleardoublepage
\thispagestyle{empty}
\begin{flushright}
    \emph{Questo lavoro è dedicato ai miei genitori}\\[1cm]
    \textit{\guillemotleft What I cannot create, I do not understand \guillemotright} -- Richard Feynman\\
    \textit{\guillemotleft It’s not only powerful, but it’s also inadequate \guillemotright} -- Miller Puckette
\end{flushright}

% Ringraziamenti
\chapter*{Ringraziamenti}
Questa sezione, facoltativa, contiene i ringraziamenti.
\addcontentsline{toc}{chapter}{Ringraziamenti}

% Indice
\tableofcontents
\newpage

% Capitoli principali
\chapter{Introduzione}
Questo elaborato tratta l'implementazione della cybersecurity in una startup fintech, focalizzandosi sul caso specifico di "Finanz". L'obiettivo principale è evidenziare le sfide, le soluzioni adottate e i risultati ottenuti nel contesto della sicurezza informatica in ambito finanziario.

\section{Obiettivi e struttura}
Descrizione degli obiettivi della tesi e della struttura del documento.

\chapter{Stato dell'arte}
Descrizione dello stato dell'arte riguardo la cybersecurity nel settore fintech.

\section{Risorse}
Discussione sulle risorse disponibili e best practice nel settore.

\chapter{Tecnologie utilizzate}
In questo capitolo vengono analizzate le tecnologie e le metodologie utilizzate per migliorare la sicurezza informatica in "Finanz". 

\section{Sicurezza del codice}
L'implementazione della sicurezza a livello di codice, con un focus su come rendere il codice Flutter più sicuro e resistente agli attacchi.

\section{Sicurezza infrastrutturale e database}
Verifica delle politiche di sicurezza nei database e analisi delle modalità di gestione dei dati degli utenti.

\chapter{Ingegneria sociale}
Questo capitolo affronta gli aspetti legati agli attacchi di ingegneria sociale testati sui componenti dell'azienda. Viene analizzata anche la diffusione di una politica aziendale volta a sensibilizzare dipendenti e collaboratori sull'importanza della sicurezza informatica.

\chapter{Test}
Descrizione del protocollo di test adottato, risultati e osservazioni.

\section{Attacchi simulati}
Analisi degli attacchi simulati per valutare la resilienza dell'infrastruttura e delle pratiche aziendali.

\chapter{Conclusioni}
Sintesi dei risultati ottenuti, criticità e prospettive future.

\appendix
\chapter{Informazioni generali sull'attività di tirocinio}
Dettagli amministrativi e gestionali per il tirocinio.

\chapter{Documenti da produrre}
Riassunto, presentazione e materiali richiesti.

% Bibliografia
\begin{thebibliography}{9}
\bibitem{feynman} Richard Feynman. \textit{What I cannot create, I do not understand.}
\bibitem{puckette} Miller Puckette. \textit{It’s not only powerful, but it’s also inadequate.}
\end{thebibliography}

\end{document}
