\documentclass[a4paper,12pt]{report}
\usepackage[backend=biber,style=ieee]{biblatex}
\addbibresource{references.bib}
%
% \includeonly{}
%
%			PREAMBOLO
%
\usepackage[a4paper]{geometry}
\usepackage{amssymb,amsmath,amsthm}
\usepackage{graphicx}
\usepackage{url}
\usepackage{hyperref}
\usepackage[italian]{babel}
\usepackage{setspace}
\usepackage{tesi}

% per le accentate
\usepackage[utf8]{inputenc}

%
%			TITOLO
\begin{document}

% Aggiunta immagine sopra al titolo
\begin{center}
        \includegraphics[width=0.2\textwidth]{images/Unimi-logo.png}
        \vspace{1cm}
\end{center}

\title{Sicurezza dell'Infrastruttura AWS in una Startup Fintech}
\author{Andrea Ferraboli}
\dept{Corso di Laurea in Sicurezza dei Sistemi e delle Reti Informatiche } 
\anno{2024-2025}
\matricola{09985A}
\relatore{Prof. Claudio Agostino Ardagna}
\correlatore{Lorenzo Perotta, Andrea Pasini, Simone Cortese}

\beforepreface

\prefacesection{Dedica}
{\hfill \Large {\sl dedicato a chi mi vuole bene, a chi mi stima e ai miei compagni di viaggio, vi voglio bene}}

\tableofcontents
\chapter{Introduzione}
\pagenumbering{arabic}

\section{La Cybersecurity nelle Startup Fintech: Sfide, Vulnerabilità e Strategie di Protezione in un Ecosistema in Rapida Evoluzione}

Il settore fintech rappresenta oggi una delle aree più dinamiche e innovative dell'ecosistema startup, con investimenti globali che hanno raggiunto i 115 miliardi di dollari, in crescita esponenziale rispetto ai 53.2 miliardi del 2018 \cite{gartnerFintech}. Questo rapido sviluppo, caratterizzato dall'implementazione di tecnologie emergenti per i servizi finanziari, porta con sé non solo opportunità senza precedenti ma anche significative sfide in termini di sicurezza informatica. Le startup fintech, che si trovano all'intersezione tra finanza tradizionale e innovazione tecnologica, gestiscono dati estremamente sensibili diventando bersagli privilegiati per i cybercriminali. Questa tesi esplora le vulnerabilità specifiche di queste realtà, analizza le principali minacce che affrontano e propone strategie di sicurezza efficaci anche in contesti di risorse limitate, evidenziando come un approccio proattivo alla cybersecurity non rappresenti un costo ma un investimento strategico fondamentale per il successo a lungo termine di una startup fintech.
\subsection{Definizione di Fintech}

Nell'ambito economico-finanziario, con il termine \textbf{fintech} (contrazione di ``financial technology'') si indica l'\textbf{innovazione nei servizi finanziari} resa possibile dalle moderne tecnologie digitali \cite{tecnofinanza}. Una \textbf{startup fintech} è quindi una \textbf{nuova impresa} che opera nel settore della tecnologia finanziaria, basando il proprio modello di business sulle tecnologie ICT più avanzate e contrapponendosi agli approcci tradizionali degli operatori finanziari consolidati \cite{fintech_numeri}. 

Queste giovani aziende ad alta componente tecnologica mirano a migliorare l'accessibilità, l'efficienza e la qualità dei servizi finanziari, e stanno svolgendo un ruolo cruciale nella \textbf{digitalizzazione del mercato finanziario italiano} \cite{tecnofinanza}. 

Tra i servizi e le soluzioni tipicamente offerti dalle startup fintech vi sono:
\begin{itemize}
    \item \textbf{Pagamenti digitali} (ad esempio tramite app mobili)
    \item Trasferimenti di denaro \textbf{peer-to-peer}
    \item \textbf{Prestiti diretti tra privati} (social lending)
    \item \textbf{Finanziamento partecipativo} (crowdfunding)
    \item Servizi assicurativi innovativi legati all'\textbf{insurtech}
    \item Impiego di tecnologie come la \textbf{blockchain} e le \textbf{criptovalute} per abilitare nuovi servizi finanziari
\end{itemize}

In linea con la crescita globale del fenomeno, in Italia si contavano oltre \textbf{600 startup fintech e insurtech} attive nel 2023 \cite{fintech_numeri}, a testimonianza di un ecosistema in rapido sviluppo.
\subsection{Il Contesto delle Startup Fintech: Un Ecosistema Dinamico e Sfidante}

Le startup fintech operano in un ambiente caratterizzato da elevata incertezza, risorse limitate e necessità di crescita rapida, fattori che influenzano profondamente le decisioni in ambito IT e sicurezza informatica \cite{fintechChallenges}. A differenza delle istituzioni finanziarie tradizionali, queste realtà innovative non dispongono generalmente di strutture gerarchiche complesse o budget consistenti dedicati alla sicurezza, dovendo invece adottare approcci agili e flessibili.

Il contesto finanziario in cui operano le startup fintech impone pressioni significative sulle decisioni di spesa. Ogni investimento, compreso quello per l'infrastruttura IT e la sicurezza, deve essere attentamente valutato in termini di ritorno immediato e benefici a lungo termine \cite{fintechChallenges}. Questa ottimizzazione dei costi rappresenta una sfida continua, poiché la sicurezza informatica richiede investimenti costanti, spesso non producendo risultati immediatamente visibili, la cui assenza può comportare conseguenze catastrofiche. In questo equilibrio delicato, le startup fintech devono trovare il giusto compromesso tra la necessità di scalare rapidamente e l'implementazione di solide misure di protezione.

\subsection{La Distinzione tra Cybersecurity Bancaria e Fintech}

Un aspetto fondamentale da considerare è la sostanziale differenza tra l'approccio alla cybersecurity nel settore bancario tradizionale e nelle startup fintech. Mentre le banche operano in un contesto fortemente regolamentato, con obblighi legali precisi in materia di sicurezza e protezione dei dati, le fintech hanno tradizionalmente goduto di una maggiore flessibilità normativa \cite{bankingVsFintech}. Le grandi istituzioni bancarie investono ingenti risorse nel testare costantemente le proprie misure di sicurezza, consapevoli che anche il minimo incidente può comportare la perdita di migliaia di clienti e sanzioni finanziarie significative.

Le fintech, spesso costituite da piccole startup in rapida espansione, possono fungere da “overlay” per le banche, facilitando la fornitura di servizi finanziari innovativi ma operando inizialmente con regolamentazioni meno stringenti \cite{bankingVsFintech}. Questa differenza normativa sta tuttavia diminuendo, soprattutto per quelle fintech che si trasformano gradualmente in vere e proprie banche, sottoponendosi così a un maggiore scrutinio regolamentare. La sfida per le startup fintech consiste quindi nel bilanciare l'agilità operativa con l'adozione di standard di sicurezza elevati, anticipando l'evoluzione normativa del settore.

\subsection{Sfide Principali per le Startup Fintech in Ambito Cybersecurity}

Le startup fintech affrontano sfide specifiche nel campo della sicurezza informatica, che derivano dalla loro natura innovativa e dalle caratteristiche distintive del loro modello di business \cite{fintechChallenges}. La prima e più evidente sfida è rappresentata dal budget limitato per la sicurezza, che spesso costringe a difficili compromessi tra lo sviluppo di nuove funzionalità e l'implementazione di adeguate misure protettive. Questa limitazione finanziaria si riflette anche nella difficoltà di attrarre e mantenere personale specializzato in cybersecurity, un ambito in cui la domanda supera ampiamente l'offerta e le grandi aziende possono offrire compensi difficilmente pareggiabili da una startup.

La pressione per un rapido accesso al mercato rappresenta un'ulteriore sfida significativa. Nel settore fintech, essere i primi a offrire un servizio innovativo può fare la differenza tra il successo e il fallimento, ma questa corsa contro il tempo spesso porta a sottovalutare gli aspetti legati alla sicurezza \cite{fintechChallenges}. Inoltre, la scalabilità dell'infrastruttura IT rappresenta una sfida tecnica considerevole: progettare sistemi che siano non solo sicuri ma anche in grado di crescere rapidamente al crescere dell'azienda richiede competenze specifiche e una pianificazione accurata.

L'adozione di tecnologie emergenti, caratteristica distintiva delle fintech, introduce nuove superfici di attacco e vulnerabilità potenziali \cite{fintechChallenges}. Cloud computing, intelligenza artificiale, blockchain e API aperte offrono opportunità straordinarie ma richiedono approcci di sicurezza specifici e aggiornati. Allo stesso tempo, la crescente interconnessione con partner, fornitori e piattaforme di terze parti amplia ulteriormente la superficie di attacco, rendendo la gestione del rischio ancora più complessa.

Non va sottovalutato, infine, il rischio rappresentato dalle minacce interne (insider threats). Nelle fasi iniziali di una startup, quando i controlli sono meno rigidi e le procedure di sicurezza meno formalizzate, il rischio legato a dipendenti negligenti o, in casi più rari, malintenzionati, aumenta considerevolmente \cite{fintechChallenges}. La cultura della condivisione e dell'apertura, tipica delle startup, deve quindi essere bilanciata con adeguate politiche di accesso e controllo.

\subsection{Minacce e Attacchi Informatici nel Settore Fintech}

Il settore fintech, per la sua natura altamente tecnologica e la gestione di dati finanziari sensibili, è diventato uno dei bersagli preferiti dei cybercriminali \cite{cyberThreatsFintech}. Tra le minacce più diffuse e pericolose figurano gli attacchi di phishing, attraverso i quali i malintenzionati cercano di ottenere credenziali di accesso, dati personali o informazioni finanziarie utilizzando email, messaggi e siti web fraudolenti che imitano comunicazioni ufficiali \cite{cyberThreatsFintech}. Queste tecniche di social engineering sfruttano la fiducia degli utenti e le loro abitudini digitali per compromettere account e sistemi aziendali.

I malware e i ransomware rappresentano un'altra categoria di minacce particolarmente grave per le startup fintech. Questi software malevoli possono infiltrarsi nei sistemi attraverso vari vettori, bloccare l'accesso ai dati e richiedere un riscatto per ripristinarlo, causando danni finanziari diretti e interruzioni operative significative \cite{cyberThreatsFintech}. Le conseguenze di un attacco ransomware possono essere devastanti per una startup con risorse limitate, in quanto il riscatto diventa percentualmente troppo oneroso per le finanze aziendali.

Gli attacchi alle API (Application Programming Interfaces), sempre più utilizzate nel settore fintech per l'integrazione con servizi terzi, costituiscono un vettore di attacco in crescita \cite{fintechChallenges}. Le API mal configurate o non adeguatamente protette possono diventare punti di ingresso privilegiati per i cybercriminali, consentendo l'accesso non autorizzato a dati sensibili e funzionalità critiche del sistema. Simile criticità presentano le configurazioni errate dei servizi cloud, che possono esporre involontariamente dati riservati o creare vulnerabilità sfruttabili.

Le startup fintech devono inoltre considerare il rischio di attacchi DDoS (Distributed Denial of Service), che mirano a rendere inaccessibili i servizi sovraccaricando i server con richieste fraudolente \cite{fintechChallenges}. Questi attacchi, relativamente semplici da orchestrare ma potenzialmente molto dannosi, possono essere utilizzati sia come attacco diretto che come diversivo per mascherare altre attività malevoli più sofisticate.

\subsection{Conseguenze degli Attacchi e Impatto sulle Startup Fintech}

L'impatto di un attacco informatico su una startup fintech può essere multidimensionale e, in molti casi, esistenziale. A livello finanziario, oltre ai costi diretti per il ripristino dei sistemi e la gestione dell'incidente, vanno considerati i potenziali risarcimenti a clienti danneggiati, le sanzioni normative e l'aumento dei premi assicurativi \cite{fintechChallenges}. Ma è forse l'impatto reputazionale a rappresentare la minaccia più grave: in un settore basato sulla fiducia come quello finanziario, una violazione dei dati può comprometterne irreparabilmente l'immagine, portando alla perdita di clienti attuali e potenziali.

L'interruzione operativa conseguente a un attacco può avere effetti a catena, influenzando non solo i clienti diretti ma anche partner commerciali e fornitori \cite{fintechChallenges}. In un ecosistema interconnesso come quello fintech, l'interdipendenza tra diverse piattaforme e servizi amplifica ulteriormente l'impatto di un incidente di sicurezza, con effetti che possono estendersi ben oltre il perimetro aziendale immediato.

\subsection{Importanza di un Approccio Proattivo alla Cybersecurity}

Implementare una strategia di cybersecurity solida sin dalle prime fasi di sviluppo di una startup fintech non configura un semplice onere, bensì un investimento strategico di primaria importanza \cite{fintechChallenges}. L'adozione del paradigma "security by design" permette infatti di integrare la sicurezza in maniera organica nei processi aziendali e nel ciclo di sviluppo del prodotto, contribuendo alla significativa riduzione dei costi a lungo termine e alla minimizzazione dei rischi potenziali. Al contrario, la mancata attenzione alla sicurezza nelle fasi iniziali comporta l'accumulo di "security debt", ovvero un debito tecnico in ambito sicurezza che, analogamente a un mutuo con tassi elevati, diventa progressivamente più oneroso da gestire e da ripagare nel tempo. Infine, la pressione derivante dalla necessità di accelerare lo sviluppo e di raggiungere rapidamente il mercato può portare a trascurare aspetti fondamentali della sicurezza, esacerbando ulteriormente tale debito tecnico.

Un approccio preventivo alla sicurezza risulta sempre più efficace ed economico rispetto a uno reattivo \cite{fintechChallenges}. I costi per implementare misure di sicurezza di base sono generalmente inferiori rispetto a quelli necessari per rispondere a un incidente, che possono includere non solo il ripristino dei sistemi ma anche sanzioni, risarcimenti e danni reputazionali. La cybersecurity deve quindi essere considerata come parte integrante della strategia aziendale, non come un elemento accessorio o un costo da minimizzare.

Le startup fintech devono inoltre considerare che adeguati livelli di sicurezza rappresentano spesso un requisito fondamentale per attrarre investitori e partner commerciali \cite{fintechChallenges}. Durante le fasi di due diligence, l'analisi delle misure di sicurezza implementate è diventata una componente standard, e lacune significative in questo ambito possono compromettere opportunità di finanziamento o collaborazioni strategiche.

\subsection{Approccio Metodologico della Tesi}

Questa tesi si propone di affrontare le sfide della cybersecurity nelle startup fintech attraverso un approccio metodologico strutturato ma flessibile \cite{fintechChallenges}. Pur concentrandosi su un caso studio pratico specifico, l'obiettivo è fornire principi e best practice di sicurezza generici e applicabili a qualsiasi startup fintech, indipendentemente dalla piattaforma tecnologica specifica utilizzata. L'approccio adottato riconosce le limitazioni di risorse tipiche delle startup e propone soluzioni scalabili che possono evolvere con la crescita dell'organizzazione.

La metodologia si basa su tre pilastri fondamentali: l'identificazione delle minacce specifiche per il modello di business fintech, la prioritizzazione degli interventi in base al rapporto rischio/beneficio e l'implementazione di controlli di sicurezza essenziali ma efficaci \cite{fintechChallenges}. Questo approccio pragmatico consente di ottenere un livello di protezione adeguato anche con risorse limitate, concentrando gli sforzi sugli aspetti più critici.

\section{Principi di sicurezza olistici per un'infrastruttura tech}
\subsection{Introduzione}
Nell'analisi e nello studio dell'infrastruttura di una startup fintech, per comprendere le possibili implementazioni a livello di sicurezza dobbiamo prima delinare quali siano i principi di cybersecurity a cui ogni startup, fintech o meno, deve attenersi. In questo capitolo verranno analizzati i principi di cybersecurity più importanti e quali sono le principali sfide che un'azienda di piccole dimensione può affrontare all'inizio del proprio percorso nell'adozione delle medesime.Questo capitolo esplora i principi fondamentali di sicurezza informatica che ogni organizzazione dovrebbe implementare, con particolare attenzione alle sfide uniche che le startup fintech affrontano nell'adozione di tali pratiche. Il settore fintech, caratterizzato da rapida innovazione e gestione di dati finanziari sensibili, presenta un contesto particolarmente critico dove le best practice di sicurezza si scontrano spesso con le esigenze di velocità di sviluppo, risorse limitate e necessità di time-to-market accelerato.

\subsection{Principi di cybersecurity}

\subsubsection{Triade CIA}
La Triade CIA (Confidentiality, Integrity, Availability) rappresenta i pilastri fondamentali sui quali costruire qualsiasi strategia di sicurezza informatica robusta.
\begin{itemize}
    \item \textbf{Confidentiality}: La riservatezza si concentra sul garantire che le informazioni siano accessibili solo a coloro che sono autorizzati a visualizzarle.\cite{ciaPaper} Questo principio viene generalmente rispettato tramite la crittografia dei dati, sia a riposo (stored data) che in transito (data in transit), controlli di accesso rigorosi, come liste di controllo degli accessi (ACL), autenticazione a più fattori (MFA) e Role-Based Access Control (RBAC).
    Nel contesto di una startup fintech, l'implementazione della riservatezza presenta sfide significative. L'accesso ai dati dei clienti e alle informazioni finanziarie deve essere rigorosamente controllato, ma i team piccoli e multifunzionali tipici delle startup spesso portano a una condivisione delle credenziali o all'assegnazione di privilegi eccessivi per "far funzionare le cose rapidamente". Ad esempio, durante lo sviluppo di un'API di pagamento, gli sviluppatori potrebbero avere accesso completo ai database di produzione contenenti dati sensibili dei clienti, invece di utilizzare dati anonimizzati o ambienti sandbox.

    La gestione delle chiavi di cifratura rappresenta un'ulteriore complessità: nelle startup dove i ruoli non sono chiaramente definiti, la responsabilità della gestione delle chiavi può essere ambigua, portando potenzialmente a compromissioni della sicurezza. L'implementazione di soluzioni robuste di Hardware Security Module (HSM) per la gestione delle chiavi può essere percepita come un costo eccessivo nella fase iniziale della startup, portando all'adozione di alternative meno sicure.
    \item \textbf{Integrity}: L'integrità garantisce che le informazioni rimangano accurate e affidabili durante il loro intero ciclo di vita. Mantenere l'integrità dei dati è essenziale per prevenire la diffusione di informazioni corrotte o ingannevoli, che potrebbero avere gravi ripercussioni in settori critici come quello sanitario o finanziario\cite{ciaPaper}Le tecniche utilizzate per preservare l'integrità includono:
    \begin{itemize}
        \item Funzioni di hash crittografiche (es. SHA-256) per verificare che i dati non siano stati alterati.
        \item Firme digitali per autenticare l'origine dei dati e garantirne la non modifica.
        \item Controllo delle versioni per tracciare le modifiche e ripristinare versioni precedenti.
        \item Checksum e meccanismi di rilevamento degli errori.
    \end{itemize}
    Nel contesto di una startup fintech, l'integrità dei dati è uno di quegli aspetti che va ad inficiare la brand reputation della startup stessa, in quanto la fiducia degli stakeholders si basa anche sulla capacità della startup di conservare i dati dei clienti senza distorsioni e di garantire l'accuratezza nelle transazioni di dati nella maniera più professionale possibile.
    \item \textbf{Availability}: assicurare che i dati siano sempre accessibili quando necessario.\cite{ciaPaper}Questo principio mira a prevenire interruzioni del servizio, sia dovute a guasti tecnici che ad attacchi malevoli come i Denial-of-Service (DoS) o Distributed Denial-of-Service (DDoS). L'indisponibilità può causare interruzioni operative, perdite economiche e danni alla reputazione. Le strategie per garantire un'elevata disponibilità comprendono:
    \begin{itemize}
        \item Sistemi ridondanti (hardware, software, reti) per eliminare singoli punti di fallimento (Single Points of Failure - SPOF).
        \item Backup regolari e piani di disaster recovery (DR) e business continuity (BCP).
        \item Tecniche di bilanciamento del carico (load balancing) per distribuire il traffico di rete.
        \item Misure di protezione contro attacchi DoS/DDoS.
    \end{itemize}
    Nella maggior parte delle startup, l'infrastruttura di base viene sviluppata considerando una capacità di carico massimo limitato, in quanto nei primi periodo di vita dell'azienda non ci si aspetta un elevato numero di utenti. Proprio per questo motivo, l'infrastruttura presenta un punto vulnerabile che può essere sfruttato dagli attaccanti per mettere a repentaglio l'intero sistema, ad esempio con attacchi DoS/DDoS mirati al perimetro aziendale.La protezione contro attacchi DDoS richiede investimenti significativi in soluzioni di mitigazione che potrebbero non essere prioritarie nelle fasi iniziali. Inoltre, la natura delle startup spesso implica team ridotti con competenze concentrate su pochi individui, creando potenziali single point of failure umani: se l'unico ingegnere DevOps responsabile dell'infrastruttura non è disponibile durante un'emergenza, l'intera operatività potrebbe essere compromessa. 
\end{itemize}

\subsubsection{Difesa in Profondità (Defense in Depth)}
Il principio di difesa in profondità prevede l'implementazione di multiple barriere protettive, così che se una viene compromessa, altre rimangono in piedi per proteggere le risorse. Questo approccio include strategie come autenticazione multi-fattore, segmentazione della rete, e rilevamento degli endpoint.

Nelle startup fintech, l'implementazione della difesa in profondità è spesso compromessa da vincoli di risorse e pressioni temporali. Ad esempio, mentre una soluzione di autenticazione a più fattori (MFA) è essenziale per proteggere l'accesso a dati finanziari sensibili, una startup potrebbe inizialmente implementare solo l'autenticazione basata su password per accelerare l'onboarding degli utenti, pianificando di aggiungere MFA "in un secondo momento" – un momento che potrebbe non arrivare prima che si verifichi un incidente di sicurezza.

La segmentazione della rete, fondamentale per contenere eventuali violazioni, richiede una progettazione accurata dell'infrastruttura. Tuttavia, nelle fasi iniziali, molte startup fintech operano con architetture di rete piatte per semplificare lo sviluppo e ridurre il sovraccarico operativo, oltre a non disporre del capitale umano competente per gestire una tale complessità. Questo approccio, sebbene comprensibile dal punto di vista dell'agilità, espone l'organizzazione a rischi significativi: un attaccante che ottiene accesso a un singolo segmento potrebbe muoversi lateralmente attraverso l'intera infrastruttura.

I test di vulnerabilità regolari, altro elemento chiave della difesa in profondità, sono spesso condotti in modo sporadico nelle startup a causa dei costi percepiti e dell'impatto sulle timeline di sviluppo. Una startup fintech potrebbe privilegiare il lancio rapido di nuove funzionalità rispetto a cicli di test di sicurezza approfonditi, esponendosi a vulnerabilità che potrebbero essere sfruttate da attaccanti motivati.
\subsubsection{Principio del Minimo Privilegio}
Questo principio stabilisce che ogni utente, processo o sistema debba operare utilizzando il set minimo di privilegi necessari per svolgere la propria funzione. Comprende pratiche come il controllo degli accessi basato sui ruoli e l'hardening dei sistemi.

Nelle startup fintech, applicare il principio del minimo privilegio (PoLP) presenta sfide uniche. La cultura focalizzata sulla velocità d'esecuzione spinge spesso a trascurare la sicurezza granulare degli accessi. È forte la tentazione di assegnare privilegi amministrativi ampi e generici per accelerare lo sviluppo, piuttosto che investire tempo nella configurazione di permessi specifici per ogni compito.
Un esempio comune è concedere a tutti gli sviluppatori accesso completo al database di produzione durante la creazione di una nuova dashboard, invece di limitare ciascuno alle sole tabelle o operazioni strettamente necessarie. Sebbene sembri una scorciatoia efficiente, questa pratica crea vulnerabilità critiche: la compromissione di un singolo account può esporre una quantità sproporzionata di dati sensibili, amplificando enormemente i danni di una violazione.
A complicare il quadro contribuisce l'alta mobilità interna tipica delle startup. I frequenti cambi di ruolo portano facilmente al "privilege creep", ovvero all'accumulo progressivo di accessi non più necessari, poiché mancano spesso processi formali di revoca. Anche la revisione periodica dei privilegi, fondamentale per mantenere l'allineamento tra accessi e responsabilità correnti, viene spesso percepita come un intralcio burocratico e trascurata.
Ulteriori difficoltà derivano dalle dimensioni ridotte dei team, dove i ruoli si sovrappongono, dalla rapida evoluzione delle responsabilità che rende obsoleti i permessi assegnati, e dalla complessa gestione degli accessi per fornitori e collaboratori esterni. Inoltre, trovare il giusto equilibrio tra sicurezza e produttività è cruciale: restrizioni eccessive possono rallentare il lavoro e spingere a cercare pericolose soluzioni alternative.
\subsubsection{Separazione dei Compiti (Separation of Duties)}

La separazione dei compiti prevede che nessun individuo possa controllare un processo critico dall'inizio alla fine, riducendo il rischio di frodi o errori. Si implementa attraverso processi di approvazione e controllo incrociato.

Nelle startup fintech, dove i team sono piccoli e i ruoli spesso sovrapposti, questo principio è particolarmente difficile da attuare. Ad esempio, in una startup che sviluppa una piattaforma di prestiti P2P, potrebbe esserci un solo ingegnere responsabile sia dell'implementazione del sistema di scoring del credito sia della configurazione dei controlli di sicurezza sullo stesso sistema. Questa concentrazione di responsabilità crea un rischio intrinseco: errori o azioni malevole potrebbero passare inosservati senza un secondo paio di occhi che verifichi il lavoro.

La pressione per l'efficienza operativa nelle startup può anche portare a scorciatoie nei processi di approvazione. Ad esempio, invece di richiedere approvazioni multiple per modifiche alla configurazione del sistema di pagamento, una startup potrebbe consentire a un singolo amministratore di implementare cambiamenti critici senza verifiche, aumentando il rischio di errori o frodi.

L'implementazione di controlli compensativi, come audit trail completi e revisioni post-implementazione, può mitigare parzialmente questi rischi quando la separazione completa dei compiti non è praticabile per vincoli di dimensione del team. Tuttavia, anche questi controlli richiedono disciplina e risorse dedicate che potrebbero non essere prioritarie nelle fasi iniziali della startup.
\subsubsection{Zero Trust}
Il modello Zero Trust si basa sul concetto di non fidarsi mai implicitamente di alcun utente, dispositivo o servizio, sia esso interno o esterno alla rete aziendale. Ogni richiesta di accesso a risorse (server, database, applicazioni) deve essere autenticata, autorizzata e verificata in base a criteri di identità, posture dei dispositivi e contesto operativo, indipendentemente dalla posizione di rete da cui proviene. L'architettura infrastrutturale tipica prevede micro-segmentazione dei carichi di lavoro, un sistema di controllo degli accessi basato su criteri dinamici e un monitoraggio continuo del comportamento e dello stato di sicurezza di utenti e dispositivi.  

Per una startup fintech, l'adozione rigorosa del principio di Zero Trust può rivelarsi particolarmente gravosa. Nelle fasi iniziali, è frequente che l'intera infrastruttura sia gestita da una sola persona, con responsabilità sia di sviluppo sia di amministrazione di rete: questo crea un unico punto di falla, amplificando il rischio di errori di configurazione o di accesso non autorizzato. Inoltre, le limitate risorse economiche e umane possono rendere difficoltoso implementare soluzioni avanzate di micro-segmentazione, sistemi di Identity and Access Management (IAM) complessi e piattaforme di monitoraggio continuo. Infine, la mancanza di separazione dei compiti e di revisioni periodiche rende più probabile la persistenza di permessi eccessivi o non aggiornati, esponendo i sistemi a potenziali attacchi laterali e perdite di dati sensibili.  


\subsubsection{Secure by Design}
Questo principio sostiene che la sicurezza debba essere integrata sin dall'inizio nel processo di sviluppo, piuttosto che aggiunta successivamente. Include pratiche come l'integrazione dei requisiti di sicurezza nelle prime fasi del progetto e lo sviluppo sicuro.

Per le startup fintech, l'adozione del principio "Secure by Design" rappresenta una tensione tra investimenti a lungo termine nella sicurezza e la necessità di velocità di sviluppo. Durante le fasi iniziali, quando l'obiettivo primario è dimostrare la validità del prodotto e acquisire i primi clienti, la tentazione di posticipare considerazioni di sicurezza è forte.

Ad esempio, una startup che sviluppa un'applicazione di gestione patrimoniale potrebbe concentrarsi inizialmente sull'esperienza utente e sulle funzionalità di investimento, lasciando per una fase successiva l'implementazione di controlli rigorosi per la protezione dei dati personali e finanziari. Questo approccio "security as an afterthought" può portare a vulnerabilità strutturali difficili da correggere successivamente, quando l'architettura del sistema è già consolidata.

L'integrazione della sicurezza nel processo di Continuous Integration/Continuous Deployment (CI/CD) rappresenta un'altra sfida: implementare scansioni di sicurezza automatizzate, analisi statica del codice e test di penetrazione come parte del pipeline di sviluppo richiede investimenti iniziali che potrebbero sembrare distrarre risorse dallo sviluppo delle funzionalità core. Tuttavia, l'automazione della sicurezza nel ciclo di vita dello sviluppo è fondamentale per identificare e correggere vulnerabilità prima che raggiungano l'ambiente di produzione.
\subsubsection{Principio K.I.S.S. (Keep It Simple, Stupid)}

Questo principio sostiene che sistemi più semplici tendono a essere più sicuri, poiché la complessità può introdurre vulnerabilità e rendere difficile per gli utenti seguire le procedure corrette.

Paradossalmente, nelle startup fintech, il principio K.I.S.S. viene spesso frainteso. Da un lato, c'è la tendenza a semplificare eccessivamente i controlli di sicurezza per ridurre l'attrito nell'esperienza utente o accelerare i processi interni. Dall'altro, la pressione per l'innovazione può portare a soluzioni tecnologiche complesse che introducono vulnerabilità non necessarie.

Un esempio tipico riguarda le politiche di password: una startup potrebbe implementare requisiti di complessità eccessiva (come cambi frequenti e regole complicate) che finiscono per incoraggiare comportamenti insicuri come l'annotazione delle password o il riutilizzo con piccole variazioni. Un approccio più equilibrato, basato su linee guida moderne come quelle del NIST, che privilegia la lunghezza rispetto alla complessità e riduce i cambi forzati, potrebbe essere più efficace e meno oneroso per gli utenti.

Analogamente, nell'architettura dei sistemi, l'adozione prematura di tecnologie emergenti senza una valutazione approfondita delle implicazioni di sicurezza può introdurre complessità e vulnerabilità. Una startup fintech che implementa una soluzione blockchain per il tracciamento delle transazioni senza una comprensione approfondita del modello di sicurezza sottostante potrebbe creare un sistema più vulnerabile rispetto a una soluzione tradizionale ben progettata e correttamente implementata.


\chapter{Infrastruttura Cloud: Principi, Vantaggi e AWS}

\section{Principi Fondamentali dell'Infrastruttura Cloud}

L'infrastruttura cloud rappresenta un elemento rivoluzionario nel panorama IT contemporaneo, offrendo alle aziende, e in particolare alle startup, un modello alternativo per la gestione delle proprie risorse tecnologiche. Definibile come una combinazione di elementi hardware e software che include potenza di elaborazione, rete, storage e risorse di virtualizzazione, l'infrastruttura cloud costituisce il fondamento essenziale su cui si basa l'intero ecosistema del cloud computing.

\subsection{Componenti dell'Infrastruttura Cloud}

Un'infrastruttura cloud si compone di quattro elementi fondamentali, ciascuno dei quali svolge un ruolo cruciale nel fornire servizi cloud efficienti e affidabili:

\begin{itemize}
\item \textbf{Hardware}: Rappresenta la base fisica dell'infrastruttura e include dispositivi come backup, firewall, bilanciatori di carico, apparecchiature di rete, router, server e array di storage.

text
\item \textbf{Virtualizzazione}: Costituisce il livello che astrae le risorse dai dispositivi hardware fisici, consentendo una gestione più flessibile e ottimizzata delle risorse disponibili[13].

\item \textbf{Storage}: Consente alle organizzazioni di archiviare grandi volumi di dati nel cloud anziché in data center fisici interni, offrendo scalabilità e accessibilità immediate[13].

\item \textbf{Rete}: Fornisce la connettività necessaria tra i vari componenti dell'infrastruttura, garantendo comunicazioni efficienti sia internamente che verso l'esterno[13].
\end{itemize}

\subsection{Modelli di Deployment Cloud}

L'infrastruttura cloud può essere implementata secondo diversi modelli, ognuno dei quali risponde a specifiche esigenze organizzative:

\begin{itemize}
\item \textbf{Public Cloud}: Infrastruttura gestita da fornitori terzi che offrono servizi attraverso Internet a molteplici clienti, condividendo risorse fisiche ma mantenendo l'isolamento logico.

text
\item \textbf{Private Cloud}: Infrastruttura dedicata esclusivamente a una singola organizzazione, offrendo maggiore controllo e personalizzazione ma richiedendo investimenti più significativi.

\item \textbf{Hybrid Cloud}: Combinazione di cloud pubblico e privato che consente alle organizzazioni di sfruttare i vantaggi di entrambi i modelli, ottimizzando costi e prestazioni.

\item \textbf{Multi-Cloud}: Approccio che prevede l'utilizzo di servizi cloud di più fornitori, riducendo la dipendenza da un singolo provider e massimizzando la flessibilità.
\end{itemize}

\section{Cloud vs On-Premise: Un Confronto Strategico}

La scelta tra un'infrastruttura cloud e una soluzione on-premise rappresenta una decisione strategica fondamentale per qualsiasi organizzazione, ma assume particolare rilevanza nel contesto delle startup, dove risorse limitate e necessità di rapida scalabilità sono fattori determinanti.

\subsection{Differenze Fondamentali}

La distinzione essenziale tra infrastruttura on-premise e cloud risiede nella localizzazione fisica di hardware, software e applicazioni. Nel modello on-premise, l'azienda mantiene tutta l'infrastruttura IT internamente e la gestisce direttamente o attraverso terze parti. Nel modello cloud, invece, l'intera infrastruttura è ospitata esternamente, con la responsabilità di monitoraggio e manutenzione affidata al provider.

Da questa fondamentale differenza derivano molteplici implicazioni che influenzano la scelta tra i due modelli:

\begin{itemize}
\item \textbf{Controllo}: Le applicazioni on-premise consentono un livello di controllo che il cloud spesso non può garantire, aspetto potenzialmente critico per determinate tipologie di dati o operazioni.

text
\item \textbf{Struttura dei costi}: L'on-premise richiede significativi investimenti iniziali (CapEx) ma costi operativi potenzialmente inferiori nel lungo periodo, mentre il cloud prevede un modello basato principalmente su costi operativi continuativi (OpEx)[4].

\item \textbf{Scalabilità}: Il cloud offre la possibilità di scalare risorse in modo quasi istantaneo, pagando solo per ciò che si utilizza, mentre le soluzioni on-premise richiedono pianificazione anticipata e investimenti per gestire picchi di domanda[4].
\end{itemize}

\subsection{Vantaggi del Cloud per le Startup}

Le startup, caratterizzate da risorse limitate e necessità di rapida crescita, trovano nel cloud computing una soluzione particolarmente vantaggiosa per numerosi motivi:

\begin{itemize}
\item \textbf{Costi Ridotti}: Il modello pay-as-you-go elimina la necessità di ingenti investimenti iniziali in hardware e software, liberando capitale per altre aree critiche dell'attività.

text
\item \textbf{Accesso Universale}: La possibilità di lavorare da qualsiasi luogo e in qualsiasi momento elimina barriere geografiche e favorisce modelli di lavoro flessibili, particolarmente preziosi per startup con team distribuiti[3].

\item \textbf{Collaborazione Efficiente}: Gli strumenti di collaborazione nel cloud facilitano la comunicazione tra i membri del team, aumentando la produttività complessiva e accelerando lo sviluppo[3].

\item \textbf{Sicurezza Avanzata}: I fornitori di servizi cloud investono massicciamente in sicurezza, offrendo livelli di protezione generalmente superiori a quelli che una startup potrebbe implementare autonomamente con risorse limitate[3].

\item \textbf{Aggiornamenti Automatici}: L'eliminazione della necessità di gestire manualmente aggiornamenti software consente al team di concentrarsi sullo sviluppo del core business anziché sulla manutenzione dell'infrastruttura[3].

\item \textbf{Time-to-Market}: La rapidità di provisioning delle risorse consente di accelerare significativamente lo sviluppo e il lancio di nuovi prodotti e servizi, fattore critico in mercati altamente competitivi come il fintech.
\end{itemize}

\subsection{Considerazioni per Startup Fintech}

Per le startup operanti nel settore fintech, la scelta dell'infrastruttura assume connotazioni particolari legate alla natura sensibile dei dati trattati e ai requisiti normativi specifici:

\begin{itemize}
\item \textbf{Compliance Normativa}: Il settore finanziario è soggetto a regolamentazioni stringenti in materia di gestione dei dati e sicurezza. Le soluzioni cloud enterprise offrono generalmente certificazioni e conformità ai principali standard normativi.

text
\item \textbf{Continuità Operativa}: Per servizi finanziari, interruzioni anche brevi possono causare perdite significative. L'infrastruttura cloud, con la sua architettura distribuita, offre generalmente maggiori garanzie di disponibilità rispetto a soluzioni on-premise di piccola scala.

\item \textbf{Scalabilità Predittiva}: La possibilità di scalare rapidamente per gestire picchi di traffico o crescita improvvisa della base utenti rappresenta un vantaggio competitivo fondamentale nel settore fintech.
\end{itemize}

\section{Amazon Web Services (AWS): Panoramica e Funzionalità}

Nel contesto del nostro caso di studio, la scelta è ricaduta su Amazon Web Services (AWS) come provider di infrastruttura cloud. AWS rappresenta uno dei principali attori nel mercato del cloud computing a livello globale, offrendo un ecosistema completo di servizi che coprono praticamente ogni aspetto delle esigenze IT di un'organizzazione moderna.

\subsection{Architettura e Principi Fondamentali di AWS}

AWS basa la propria offerta su sei principi fondamentali, noti come "AWS Well-Architected Framework", che definiscono le best practice per la progettazione e l'implementazione di infrastrutture cloud efficaci e sicure:

\begin{itemize}
\item \textbf{Eccellenza Operativa}: Focus su esecuzione e monitoraggio dei sistemi e miglioramento continuo di processi e procedure, includendo l'automazione delle modifiche, la reazione agli eventi e la definizione di standard per la gestione delle operazioni.

text
\item \textbf{Sicurezza}: Protezione di informazioni e sistemi, con particolare attenzione a riservatezza e integrità dei dati, gestione delle identità e degli accessi, e implementazione di controlli a tutti i livelli[5].

\item \textbf{Affidabilità}: Capacità di un sistema di recuperare da interruzioni dell'infrastruttura o del servizio, acquisire dinamicamente risorse per soddisfare la domanda e mitigare interruzioni come errori di configurazione o problemi temporanei di rete[5].

\item \textbf{Efficienza delle Prestazioni}: Utilizzo efficiente delle risorse informatiche per soddisfare i requisiti di sistema e mantenere tale efficienza con l'evolversi delle tecnologie e delle esigenze aziendali[5].

\item \textbf{Ottimizzazione dei Costi}: Evitare spese non necessarie attraverso la comprensione e il controllo della provenienza del denaro e l'analisi della spesa nel tempo, selezionando le risorse più appropriate e scalabili in base alle necessità[5].

\item \textbf{Sostenibilità}: Minimizzazione dell'impatto ambientale dell'utilizzo del cloud attraverso la riduzione del consumo energetico e dell'efficienza delle risorse[5].
\end{itemize}

\subsection{Struttura Organizzativa dei Servizi AWS}

AWS organizza i propri servizi in categorie funzionali, consentendo alle organizzazioni di selezionare e combinare esattamente ciò di cui hanno bisogno:

\begin{itemize}
\item \textbf{Compute}: Servizi per l'elaborazione, inclusi Amazon EC2 (virtual servers), AWS Lambda (serverless computing), e Amazon ECS/EKS (container management).

text
\item \textbf{Storage}: Soluzioni di archiviazione come Amazon S3 (object storage), Amazon EBS (block storage), e Amazon Glacier (archivio a lungo termine).

\item \textbf{Database}: Servizi di database relazionali e NoSQL, inclusi Amazon RDS, Amazon DynamoDB e Amazon DocumentDB.

\item \textbf{Networking}: Componenti per la creazione di reti virtuali, tra cui Amazon VPC, AWS Direct Connect e Amazon Route 53.

\item \textbf{Security, Identity \& Compliance}: Strumenti per la protezione dell'infrastruttura e dei dati, come AWS IAM, AWS Shield e AWS WAF.

\item \textbf{Analytics \& AI/ML}: Servizi per analisi dati e implementazione di intelligenza artificiale e machine learning.

\item \textbf{Developer Tools}: Strumenti per facilitare lo sviluppo, il test e il deployment di applicazioni.
\end{itemize}

\subsection{Regioni e Zone di Disponibilità}

Un aspetto distintivo dell'architettura AWS è la sua distribuzione geografica attraverso Regioni e Zone di Disponibilità (AZ):

\begin{itemize}
\item \textbf{Regioni AWS}: Aree geografiche distinte che ospitano cluster di data center, separate fisicamente l'una dall'altra per garantire isolamento in caso di disastri regionali.

text
\item \textbf{Zone di Disponibilità}: All'interno di ciascuna Regione, multiple zone di disponibilità rappresentano data center fisicamente separati ma connessi attraverso reti a bassa latenza. Questa architettura consente di implementare sistemi altamente disponibili e resilienti.
\end{itemize}

Questa struttura geograficamente distribuita offre vantaggi significativi, particolarmente rilevanti per startup fintech che necessitano di alta disponibilità e conformità a normative sulla residenza dei dati.

\section{La Scelta di AWS per una Startup Fintech}

Nel contesto del nostro caso di studio, la scelta di AWS come provider di infrastruttura cloud è stata motivata da diverse considerazioni strategiche:

\subsection{Fattori Determinanti}

\begin{itemize}
\item \textbf{Ampiezza dell'Ecosistema}: L'estensivo portfolio di servizi AWS consente di implementare praticamente qualsiasi architettura necessaria senza dover ricorrere a provider esterni, semplificando l'integrazione e la gestione.

text
\item \textbf{Maturità della Piattaforma}: Come pioniere del cloud computing, AWS offre servizi consolidati e ampiamente testati, riducendo i rischi associati all'adozione di tecnologie emergenti.

\item \textbf{Conformità Normativa}: AWS mantiene un ampio spettro di certificazioni di conformità (ISO 27001, PCI DSS, SOC) particolarmente rilevanti nel settore finanziario.

\item \textbf{Supporto per DevOps}: L'integrazione nativa con strumenti di CI/CD e l'approccio "Infrastructure as Code" facilita l'implementazione di metodologie DevOps, fondamentali per il rapido sviluppo e iterazione tipici di una startup.

\item \textbf{Community e Risorse}: L'ampia community AWS e la ricca documentazione disponibile rappresentano risorse preziose per team in crescita con competenze in evoluzione.
\end{itemize}

\subsection{Considerazioni Economiche}

Per una startup fintech, l'aspetto economico dell'infrastruttura rappresenta una considerazione fondamentale. AWS offre diversi vantaggi in questo senso:

\begin{itemize}
\item \textbf{Modello Pay-as-you-go}: Consente di avviare l'infrastruttura con investimenti minimi e scalare i costi proporzionalmente alla crescita dell'utilizzo.

text
\item \textbf{AWS Activate}: Programma specifico per startup che offre crediti gratuiti, supporto tecnico e formazione, riducendo ulteriormente le barriere d'ingresso.

\item \textbf{Servizi a Consumo Variabile}: La possibilità di attivare e disattivare risorse on-demand consente di ottimizzare i costi durante le fasi di sviluppo e test.
\end{itemize}

\chapter{Implementazioni Pratiche nell'Infrastruttura AWS}

\section{Architettura di Base per una Startup Fintech}

La progettazione di un'infrastruttura cloud efficace per una startup fintech richiede un attento bilanciamento tra sicurezza, scalabilità, costi e compliance normativa. Di seguito presentiamo un'architettura di riferimento implementata nel nostro caso di studio, applicabile con appropriate modifiche a scenari analoghi.

\subsection{Multi-Account Strategy}

Una delle best practice fondamentali adottate è stata l'implementazione di una strategia multi-account AWS, che fornisce isolamento logico e sicurezza migliorata:

\begin{itemize}
\item \textbf{Account di Management}: Dedicato alla gestione centralizzata di tutti gli altri account, ospita AWS Organizations e implementa policy di governance globali.

text
\item \textbf{Account di Security}: Isolato e con accessi limitati, ospita i log centralizzati, strumenti di audit e configurazioni di sicurezza cross-account.

\item \textbf{Account di Development}: Ambiente dedicato allo sviluppo e test, con politiche di sicurezza meno restrittive per facilitare l'innovazione.

\item \textbf{Account di Staging}: Replica dell'ambiente di produzione utilizzata per test finali prima del rilascio.

\item \textbf{Account di Production}: Ambiente di produzione altamente protetto, con politiche di accesso rigorose e monitoraggio avanzato.
\end{itemize}

Questa separazione fornisce "defense in depth" e limita la superficie di attacco, aspetto particolarmente critico per applicazioni che gestiscono dati finanziari sensibili.

\subsection{Networking Segregato}

L'implementazione di una rete virtuale ben segmentata rappresenta un elemento fondamentale dell'architettura:

\begin{itemize}
\item \textbf{Virtual Private Cloud (VPC)} dedicata per ogni ambiente, con separazione completa tra development, staging e production.

text
\item \textbf{Subnets} organizzate secondo un modello multi-tier:
\begin{itemize}
    \item \textit{Public subnets}: Ospitano solo i load balancer e i bastion host.
    \item \textit{Private application subnets}: Contengono i server applicativi e i servizi di elaborazione.
    \item \textit{Private data subnets}: Isolate e accessibili solo dai tier applicativi, ospitano database e storage persistente.
\end{itemize}

\item \textbf{Network Access Control Lists (NACLs)} e \textbf{Security Groups} configurati secondo il principio del privilegio minimo, permettendo solo il traffico esplicitamente autorizzato.

\item \textbf{VPC Endpoints} per comunicare con servizi AWS senza attraversare internet, aumentando sicurezza e riducendo latenza.
\end{itemize}

\section{Implementazioni per la Sicurezza}

La sicurezza rappresenta una priorità assoluta per qualsiasi startup fintech. Nel nostro caso di studio, abbiamo implementato un approccio a più livelli:

\subsection{Identity and Access Management}

\begin{itemize}
\item \textbf{AWS IAM} configurato con politiche di accesso basate sul principio del privilegio minimo:
\begin{verbatim}
{
"Version": "2012-10-17",
"Statement": [
{
"Effect": "Allow",
"Action": [
"dynamodb:GetItem",
"dynamodb:Query"
],
"Resource": "arn:aws:dynamodb:eu-west-1:123456789012:table/CustomerFinancialData",
"Condition": {
"StringEquals": {
"aws:PrincipalTag/Department": "RiskAnalysis"
}
}
}
]
}
\end{verbatim}

text
\item \textbf{Multi-Factor Authentication (MFA)} obbligatoria per tutti gli utenti con accesso alla console AWS.

\item \textbf{AWS Organizations Service Control Policies (SCPs)} per imporre vincoli a livello organizzativo, ad esempio:
\begin{verbatim}
{
  "Version": "2012-10-17",
  "Statement": [
    {
      "Effect": "Deny",
      "Action": [
        "ec2:RunInstances",
        "rds:CreateDBInstance"
      ],
      "Resource": "*",
      "Condition": {
        "StringNotEquals": {
          "aws:RequestedRegion": [
            "eu-west-1",
            "eu-central-1"
          ]
        }
      }
    }
  ]
}
\end{verbatim}

\item \textbf{Rotazione automatica delle credenziali} implementata tramite AWS Secrets Manager:
\begin{verbatim}
aws secretsmanager create-secret \
    --name production/database/credentials \
    --description "Database credentials for production" \
    --secret-string '{"username":"admin","password":"abcd1234"}' \
    --rotation-lambda-arn "arn:aws:lambda:eu-west-1:123456789012:function:RotateDBCredentials" \
    --rotation-rules '{"AutomaticallyAfterDays": 30}'
\end{verbatim}
\end{itemize}

\subsection{Crittografia e Protezione dei Dati}

La protezione dei dati finanziari richiede misure di crittografia robuste:

\begin{itemize}
\item \textbf{Crittografia at-rest} implementata per tutti i dati sensibili:
\begin{itemize}
\item Amazon S3 con encryption by default e policy che bloccano upload non crittografati:
\begin{verbatim}
{
"Version": "2012-10-17",
"Statement": [
{
"Effect": "Deny",
"Principal": "",
"Action": "s3:PutObject",
"Resource": "arn:aws:s3:::financial-documents/",
"Condition": {
"StringNotEquals": {
"s3:x-amz-server-side-encryption": "AES256"
}
}
}
]
}
\end{verbatim}

text
    \item RDS e DynamoDB con encryption enabled.
    \item EBS volumes crittografati con chiavi KMS dedicate.
\end{itemize}

\item \textbf{Crittografia in-transit} tramite TLS 1.2+ per tutte le comunicazioni:
\begin{itemize}
    \item Configurazione dei load balancer per accettare solo connessioni HTTPS:
    \begin{verbatim}
    aws elbv2 create-listener \
        --load-balancer-arn arn:aws:elasticloadbalancing:eu-west-1:123456789012:loadbalancer/app/financial-app/abcdef123456 \
        --protocol HTTPS \
        --port 443 \
        --ssl-policy ELBSecurityPolicy-FS-1-2-Res-2020-10 \
        --certificates CertificateArn=arn:aws:acm:eu-west-1:123456789012:certificate/12345678-1234-1234-1234-123456789012
    \end{verbatim}
\end{itemize}

\item \textbf{AWS KMS} per la gestione centralizzata delle chiavi di crittografia con rotazione automatica:
\begin{verbatim}
aws kms create-key \
    --description "Key for financial data encryption" \
    --policy '{
      "Version": "2012-10-17",
      "Statement": [
        {
          "Effect": "Allow",
          "Principal": {
            "AWS": "arn:aws:iam::123456789012:role/FinancialDataProcessing"
          },
          "Action": [
            "kms:Encrypt",
            "kms:Decrypt",
            "kms:GenerateDataKey"
          ],
          "Resource": "*"
        }
      ]
    }'
\end{verbatim}
\end{itemize}

\subsection{Monitoraggio e Rilevamento delle Minacce}

Un sistema robusto di monitoraggio è essenziale per identificare e rispondere rapidamente a potenziali minacce:

\begin{itemize}
\item \textbf{AWS CloudTrail} configurato in tutti gli account con log centralizzati nell'account di sicurezza:
\begin{verbatim}
aws cloudtrail create-trail
--name OrganizationTrail
--s3-bucket-name security-account-logs
--is-organization-trail
--kms-key-id arn:aws:kms:eu-west-1:123456789012:key/abcdef12-3456-7890-abcd-ef1234567890
--is-multi-region-trail
\end{verbatim}

text
\item \textbf{Amazon GuardDuty} abilitato in tutte le regioni con findings inviati a un sistema centralizzato:
\begin{verbatim}
aws guardduty create-detector \
    --enable \
    --finding-publishing-frequency FIFTEEN_MINUTES \
    --data-sources '{
      "S3Logs": {
        "Enable": true
      },
      "Kubernetes": {
        "AuditLogs": {
          "Enable": true
        }
      }
    }'
\end{verbatim}

\item \textbf{AWS Config} per valutazione continua della conformità:
\begin{verbatim}
aws configservice put-config-rule \
    --config-rule '{
      "ConfigRuleName": "encrypted-volumes",
      "Description": "Ensures all EBS volumes are encrypted",
      "Source": {
        "Owner": "AWS",
        "SourceIdentifier": "ENCRYPTED_VOLUMES"
      },
      "Scope": {
        "ComplianceResourceTypes": [
          "AWS::EC2::Volume"
        ]
      }
    }'
\end{verbatim}

\item \textbf{Amazon CloudWatch Logs} con filtri metrici per rilevare attività anomale:
\begin{verbatim}
aws logs put-metric-filter \
    --log-group-name CloudTrail/DefaultLogGroup \
    --filter-name RootAccountUsage \
    --filter-pattern '{ $.userIdentity.type = "Root" && $.userIdentity.invokedBy NOT EXISTS }' \
    --metric-transformations \
        metricName=RootAccountUsageCount,metricNamespace=CloudTrailMetrics,metricValue=1
\end{verbatim}
\end{itemize}

\section{Architetture per l'Alta Disponibilità}

Per un servizio finanziario, la disponibilità continua rappresenta un requisito fondamentale. Nel nostro caso di studio, abbiamo implementato:

\subsection{Architettura Multi-AZ}

\begin{itemize}
\item \textbf{Distribuzione delle risorse} attraverso multiple Zone di Disponibilità:
\begin{verbatim}
aws ec2 run-instances
--image-id ami-0abcdef1234567890
--count 3
--instance-type t3.medium
--key-name FinTechKeyPair
--security-group-ids sg-12345678
--subnet-id subnet-abcdef12 subnet-bcdef123 subnet-cdef1234
--tag-specifications 'ResourceType=instance,Tags=[{Key=Name,Value=FinTechApp}]'
\end{verbatim}

text
\item \textbf{Amazon RDS Multi-AZ} per database altamente disponibili:
\begin{verbatim}
aws rds create-db-instance \
    --db-instance-identifier financial-db \
    --db-instance-class db.r5.large \
    --engine postgres \
    --master-username admin \
    --master-user-password **** \
    --allocated-storage 100 \
    --storage-type gp2 \
    --multi-az \
    --storage-encrypted \
    --backup-retention-period 7
\end{verbatim}

\item \textbf{Application Load Balancer} configurato per distribuire il traffico tra istanze in diverse AZ:
\begin{verbatim}
aws elbv2 create-load-balancer \
    --name financial-app-lb \
    --subnets subnet-abcdef12 subnet-bcdef123 \
    --security-groups sg-12345678 \
    --scheme internet-facing \
    --type application
\end{verbatim}
\end{itemize}

\subsection{Disaster Recovery}

Un piano di disaster recovery robusto è essenziale per garantire la continuità operativa:

\begin{itemize}
\item \textbf{Backup regolari} attraverso diverse soluzioni:
\begin{itemize}
\item AWS Backup per un approccio centralizzato:
\begin{verbatim}
aws backup create-backup-plan
--backup-plan '{
"BackupPlanName": "FinancialDataBackup",
"Rules": [
{
"RuleName": "DailyBackups",
"TargetBackupVaultName": "FinancialVault",
"ScheduleExpression": "cron(0 0 * * ? *)",
"StartWindowMinutes": 60,
"CompletionWindowMinutes": 180,
"Lifecycle": {
"DeleteAfterDays": 30
},
"CopyActions": [
{
"DestinationBackupVaultArn": "arn:aws:backup:eu-central-1:123456789012:backup-vault:SecondaryRegionVault",
"Lifecycle": {
"DeleteAfterDays": 30
}
}
]
}
]
}'
\end{verbatim}

text
    \item RDS automated snapshots con copy cross-region:
    \begin{verbatim}
    aws rds copy-db-snapshot \
        --source-db-snapshot-identifier arn:aws:rds:eu-west-1:123456789012:snapshot:financial-db-snapshot \
        --target-db-snapshot-identifier financial-db-snapshot-copy \
        --kms-key-id arn:aws:kms:eu-central-1:123456789012:key/abcdef12-3456-7890-abcd-ef1234567890 \
        --source-region eu-west-1 \
        --region eu-central-1
    \end{verbatim}
\end{itemize}

\item \textbf{Replica cross-region} per servizi critici:
\begin{itemize}
    \item DynamoDB Global Tables:
    \begin{verbatim}
    aws dynamodb create-global-table \
        --global-table-name CustomerTransactions \
        --replication-group RegionName=eu-west-1 RegionName=eu-central-1
    \end{verbatim}
    
    \item S3 Cross-Region Replication:
    \begin{verbatim}
    aws s3api put-bucket-replication \
        --bucket primary-financial-documents \
        --replication-configuration '{
          "Role": "arn:aws:iam::123456789012:role/S3ReplicationRole",
          "Rules": [
            {
              "Status": "Enabled",
              "Priority": 1,
              "DeleteMarkerReplication": { "Status": "Enabled" },
              "Destination": {
                "Bucket": "arn:aws:s3:::dr-financial-documents",
                "EncryptionConfiguration": {
                  "ReplicaKmsKeyID": "arn:aws:kms:eu-central-1:123456789012:key/abcdef12-3456-7890-abcd-ef1234567890"
                }
              },
              "SourceSelectionCriteria": {
                "SseKmsEncryptedObjects": {
                  "Status": "Enabled"
                }
              }
            }
          ]
        }'
    \end{verbatim}
\end{itemize}
\end{itemize}

\section{Infrastruttura come Codice e Automazione}

Per garantire consistenza, riproducibilità e scalabilità dell'infrastruttura, abbiamo adottato un approccio "Infrastructure as Code" con ampio utilizzo di automazione:

\subsection{AWS CloudFormation per la Gestione dell'Infrastruttura}

\begin{itemize}
\item \textbf{Template modulari} organizzati per componenti funzionali:
\begin{verbatim}
AWSTemplateFormatVersion: '2010-09-09'
Description: 'Network infrastructure for FinTech application'

text
Resources:
  VPC:
    Type: AWS::EC2::VPC
    Properties:
      CidrBlock: 10.0.0.0/16
      EnableDnsSupport: true
      EnableDnsHostnames: true
      Tags:
        - Key: Name
          Value: FinTech-VPC

  PublicSubnet1:
    Type: AWS::EC2::Subnet
    Properties:
      VpcId: !Ref VPC
      CidrBlock: 10.0.1.0/24
      AvailabilityZone: !Select [0, !GetAZs '']
      MapPublicIpOnLaunch: true
      Tags:
        - Key: Name
          Value: Public-Subnet-1

  # Definizione di altre subnet, route tables, security groups, etc.
\end{verbatim}

\item \textbf{Nested stacks} per gestire componenti complessi:
\begin{verbatim}
AWSTemplateFormatVersion: '2010-09-09'
Description: 'Master template for FinTech application'

Resources:
  NetworkStack:
    Type: AWS::CloudFormation::Stack
    Properties:
      TemplateURL: https://s3.amazonaws.com/cf-templates/network.yaml

  DatabaseStack:
    Type: AWS::CloudFormation::Stack
    Properties:
      TemplateURL: https://s3.amazonaws.com/cf-templates/database.yaml
      Parameters:
        VpcId: !GetAtt NetworkStack.Outputs.VpcId
        DatabaseSubnet1: !GetAtt NetworkStack.Outputs.PrivateSubnet1
        DatabaseSubnet2: !GetAtt NetworkStack.Outputs.PrivateSubnet2
\end{verbatim}

\item \textbf{Change sets} per review delle modifiche prima dell'applicazione:
\begin{verbatim}
aws cloudformation create-change-set \
    --stack-name FinTechProduction \
    --change-set-name UpdateSecurityFeatures \
    --template-body file://updated-template.yaml \
    --capabilities CAPABILITY_NAMED_IAM
\end{verbatim}
\end{itemize}

\subsection{CI/CD Pipeline per Infrastruttura e Applicazioni}

\begin{itemize}
\item \textbf{AWS CodePipeline} integrato con repository Git:
\begin{verbatim}
aws codepipeline create-pipeline
--pipeline '{
"name": "FinTech-Infrastructure-Pipeline",
"roleArn": "arn:aws:iam::123456789012:role/CodePipelineServiceRole",
"artifactStore": {
"type": "S3",
"location": "codepipeline-artifacts-bucket"
},
"stages": [
{
"name": "Source",
"actions": [
{
"name": "Source",
"actionTypeId": {
"category": "Source",
"owner": "AWS",
"provider": "CodeStarSourceConnection",
"version": "1"
},
"configuration": {
"ConnectionArn": "arn:aws:codestar-connections:eu-west-1:123456789012:connection/abcdef12-3456-7890-abcd-ef1234567890",
"FullRepositoryId": "organization/infrastructure-repo",
"BranchName": "main"
},
"outputArtifacts": [
{
"name": "SourceCode"
}
]
}
]
},
{
"name": "Validate",
"actions": [
{
"name": "CFNLint",
"actionTypeId": {
"category": "Test",
"owner": "AWS",
"provider": "CodeBuild",
"version": "1"
},
"configuration": {
"ProjectName": "CFNLintValidation"
},
"inputArtifacts": [
{
"name": "SourceCode"
}
]
}
]
},
{
"name": "Deploy",
"actions": [
{
"name": "UpdateInfrastructure",
"actionTypeId": {
"category": "Deploy",
"owner": "AWS",
"provider": "CloudFormation",
"version": "1"
},
"configuration": {
"ActionMode": "CREATE_UPDATE",
"StackName": "FinTechProduction",
"Capabilities": "CAPABILITY_NAMED_IAM",
"TemplatePath": "SourceCode::templates/master.yaml",
"RoleArn": "arn:aws:iam::123456789012:role/CloudFormationExecutionRole"
},
"inputArtifacts": [
{
"name": "SourceCode"
}
]
}
]
}
]
}'
\end{verbatim}

text
\item \textbf{Validazione automatica} delle modifiche all'infrastruttura:
\begin{itemize}
    \item Test con AWS CloudFormation Linter.
    \item Security scanning con tools come cfn-nag per identificare configurazioni non sicure.
    \item Controlli di policy compliance prima del deployment.
\end{itemize}
\end{itemize}

\section{Ottimizzazione dei Costi}

L'ottimizzazione dei costi rappresenta una priorità per le startup fintech, che devono bilanciare esigenze di sicurezza e resilienza con budget limitati:

\subsection{Strategie di Ottimizzazione Implementate}

\begin{itemize}
\item \textbf{Rightsizing} delle risorse basato su analisi dei pattern di utilizzo:
\begin{verbatim}
aws compute-optimizer get-ec2-instance-recommendations
--instance-arns arn:aws:ec2:eu-west-1:123456789012:instance/i-0abcdef1234567890
\end{verbatim}

text
\item \textbf{Auto Scaling} per adattare dinamicamente la capacità al carico:
\begin{verbatim}
aws autoscaling create-auto-scaling-group \
    --auto-scaling-group-name FinTechAppASG \
    --launch-template LaunchTemplateId=lt-0123456789abcdef0 \
    --min-size 2 \
    --max-size 10 \
    --desired-capacity 2 \
    --vpc-zone-identifier "subnet-abcdef12,subnet-bcdef123" \
    --target-group-arns "arn:aws:elasticloadbalancing:eu-west-1:123456789012:targetgroup/FinTechAppTargets/abcdef123456789" \
    --health-check-type ELB \
    --health-check-grace-period 300
\end{verbatim}

\item \textbf{Scheduled scaling} per carichi prevedibili:
\begin{verbatim}
aws autoscaling put-scheduled-update-group-action \
    --auto-scaling-group-name FinTechAppASG \
    --scheduled-action-name ScaleUpForMarketOpen \
    --recurrence "0 8 * * 1-5" \
    --min-size 5 \
    --max-size 20 \
    --desired-capacity 10
\end{verbatim}

\item \textbf{Savings Plans} e \textbf{Reserved Instances} per carichi base prevedibili:
\begin{verbatim}
aws savingsplans create-savings-plan \
    --savings-plan-offering-id offering-12345678901234567 \
    --commitment 1.0 \
    --upfront-payment-amount 0 \
    --term-in-years 1 \
    --payment-option "No Upfront"
\end{verbatim}

\item \textbf{Lifecycle policies} per gestire automaticamente dati e backup meno recenti:
\begin{verbatim}
aws s3api put-bucket-lifecycle-configuration \
    --bucket financial-data-bucket \
    --lifecycle-configuration '{
      "Rules": [
        {
          "ID": "MoveToGlacier",
          "Status": "Enabled",
          "Filter": {
            "Prefix": "transaction-logs/"
          },
          "Transitions": [
            {
              "Days": 30,
              "StorageClass": "STANDARD_IA"
            },
            {
              "Days": 90,
              "StorageClass": "GLACIER"
            }
          ],
          "Expiration": {
            "Days": 365
          }
        }
      ]
    }'
\end{verbatim}
\end{itemize}

\subsection{Monitoraggio e Ottimizzazione Continua}

\begin{itemize}
\item \textbf{AWS Cost Explorer} per analisi dettagliate e trend:
\begin{verbatim}
aws ce get-cost-and-usage
--time-period Start=2025-01-01,End=2025-03-31
--granularity MONTHLY
--metrics "BlendedCost" "UnblendedCost" "UsageQuantity"
--group-by Type=DIMENSION,Key=SERVICE Type=TAG,Key=Environment
\end{verbatim}

text
\item \textbf{AWS Budgets} per monitoraggio proattivo e alerting:
\begin{verbatim}
aws budgets create-budget \
    --account-id 123456789012 \
    --budget '{
      "BudgetName": "DevelopmentEnvironmentBudget",
      "BudgetLimit": {
        "Amount": "1000",
        "Unit": "USD"
      },
      "CostFilters": {
        "TagKeyValue": [
          "user:Environment$Development"
        ]
      },
      "TimeUnit": "MONTHLY",
      "BudgetType": "COST"
    }' \
    --notifications-with-subscribers '[
      {
        "Notification": {
          "NotificationType": "ACTUAL",
          "ComparisonOperator": "GREATER_THAN",
          "Threshold": 80,
          "ThresholdType": "PERCENTAGE"
        },
        "Subscribers": [
          {
            "SubscriptionType": "EMAIL",
            "Address": "finance@fintech-startup.com"
          }
        ]
      }
    ]'
\end{verbatim}

\item \textbf{AWS Trusted Advisor} per raccomandazioni continue su ottimizzazione dei costi e sicurezza:
\begin{verbatim}
aws support describe-trusted-advisor-checks \
    --language en \
    --region us-east-1
\end{verbatim}
\end{itemize}

%			BIBLIOGRAFIA
\printbibliography

% 
\end{document}