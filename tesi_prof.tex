\documentclass[a4paper,12pt]{report}
% IMPOSTAZIONI DI BASE E LINGUA
\usepackage[utf8]{inputenc} % Per caratteri accentati e speciali direttamente da tastiera
\usepackage[T1]{fontenc}    % Per una corretta sillabazione delle parole accentate e font moderni
\usepackage{lmodern}        % Font Latin Modern
\usepackage[italian]{babel} % Impostazioni per la lingua italiana
\usepackage{csquotes}       % Per virgolette corrette con \enquote{} (usa le impostazioni di babel)
% GEOMETRIA DELLA PAGINA
\usepackage[a4paper]{geometry} % Imposta i margini per il formato a4paper (default)
% MATEMATICA
\usepackage{amssymb,amsmath,amsthm}
% COLORI (SPOSTATO PRIMA DI TESI.STY)
\usepackage{xcolor}         % Per definire e usare colori
% GRAFICA E TABELLE
\usepackage{graphicx}
\graphicspath{{./images/}} % Specifica la cartella delle immagini
\usepackage{subcaption}         % Per sottofigure (alternativa più moderna: subcaption)
\usepackage{booktabs}       % Per tabelle più professionali
% BIBLIOGRAFIA
\usepackage[backend=biber,style=ieee,sorting=none]{biblatex}
\addbibresource{references.bib} % Assicurati che il file references.bib esista
% UTILITIES VARIE
\usepackage{url}            % Per inserire URL cliccabili (spesso gestito anche da hyperref)
\usepackage{setspace}       % Per controllare l'interlinea
\usepackage{comment}        % Per blocchi di commento multilinea
% LISTINGS DI CODICE (XCOLOR È GIÀ CARICATO)
\usepackage{listings}       % Per inserire blocchi di codice
% HYPERLINKS E BOOKMARKS (generalmente verso la fine)
\usepackage{hyperref}
\usepackage{bookmark}       % Migliora la gestione dei segnalibri PDF
% PACCHETTO PERSONALIZZATO (DOPO TUTTI GLI ALTRI)
\usepackage{tesiupdated}           % Pacchetto personalizzato per la tesi"
\usepackage{fancyhdr}      % 1. carica fancyhdr

% 2. tuo nuovo pagestyle (lo stesso codice mostrato prima)
\setlength{\headheight}{14pt}
\fancyhf{}
\fancyhead[LE]{\small\slshape
  \parbox[t]{\dimexpr\headwidth-1em\relax}{\raggedright\leftmark\strut}}
\fancyhead[RO]{\small\slshape
  \parbox[t]{\dimexpr\headwidth-1em\relax}{\raggedleft\leftmark\strut}}
\fancyfoot[C]{\thepage}

\makeatletter
\renewcommand{\chaptermark}[1]{%
  \markboth{\ifnum\c@secnumdepth>\m@ne\thechapter\space\fi#1}{}}
\makeatother
\pagestyle{fancy}          % 3. usa il nuovo stile

\usepackage{etoolbox}
\appto{\afterpreface}{\pagestyle{fancy}}

% per le accentate
\definecolor{codegreen}{rgb}{0,0.6,0}
\definecolor{codegray}{rgb}{0.5,0.5,0.5}
\definecolor{codepurple}{rgb}{0.58,0,0.82}
\definecolor{backcolour}{rgb}{0.95,0.95,0.92}

\lstdefinestyle{mystyle}{
backgroundcolor=\color{backcolour},
commentstyle=\color{codegreen},
keywordstyle=\color{magenta},
numberstyle=\tiny\color{codegray},
stringstyle=\color{codepurple},
basicstyle=\ttfamily\small,
breakatwhitespace=false,
breaklines=true,
captionpos=b,
keepspaces=true,
numbers=left,
numbersep=5pt,
showspaces=false,
showstringspaces=false,
showtabs=false,
tabsize=2
}

\lstset{style=mystyle}
% Definizione linguaggio JSON per listings
\lstdefinelanguage{json}{
    keywords={true,false,null},
    sensitive=false,
    morestring=[b]",
    comment=[l]{//},
    morecomment=[s]{/*}{*/},
}
% Definizione stile per codice JSON
\lstdefinestyle{json}{
    language=json,
    basicstyle=\small\ttfamily,
    numbers=left,
    numberstyle=\tiny\color{gray},
    stepnumber=1,
    numbersep=5pt,
    backgroundcolor=\color{white},
    showspaces=false,
    showstringspaces=false,
    showtabs=false,
    frame=single,
    rulecolor=\color{black},
    tabsize=2,
    captionpos=b,
    breaklines=true,
    breakatwhitespace=false,
    title=\lstname,
    keywordstyle=\color{blue},
    commentstyle=\color{gray},
    stringstyle=\color{red!60!black},
    escapeinside={\%*}{*},
    morestring=[b]",
    literate=
      *{0}{{{\color{red!60!black}0}}}{1}
      {1}{{{\color{red!60!black}1}}}{1}
      {2}{{{\color{red!60!black}2}}}{1}
      {3}{{{\color{red!60!black}3}}}{1}
      {4}{{{\color{red!60!black}4}}}{1}
      {5}{{{\color{red!60!black}5}}}{1}
      {6}{{{\color{red!60!black}6}}}{1}
      {7}{{{\color{red!60!black}7}}}{1}
      {8}{{{\color{red!60!black}8}}}{1}
      {9}{{{\color{red!60!black}9}}}{1}
      {:}{{{\color{black}:}}}{1}
      {,}{{{\color{black},}}}{1}
      {\{}{{{\color{black}\{}}}{1}
      {\}}{{{\color{black}\}}}}{1}
      {[}{{{\color{black}[}}}{1}
      {]}{{{\color{black}]}}}{1}
}


% Definizione stile per codice Python migliorato
\lstdefinestyle{python}{
    language=Python,
    basicstyle=\small\ttfamily,                     % Testo monospaziato, dimensione piccola
    numbers=left,                                    % Numeri di riga a sinistra
    numberstyle=\tiny\color{gray},                   % Numerazione grigio chiaro, dimensione ridotta
    stepnumber=1,                                    % Incremento numeri di riga di 1
    numbersep=5pt,                                   % Spazio tra numeri e codice
    backgroundcolor=\color{gray!10},                 % Sfondo grigio molto chiaro (10%)
    showspaces=false,
    showstringspaces=false,
    showtabs=false,
    frame=single,                                    % Cornice singola attorno al blocco
    rulecolor=\color{gray!60},                       % Colore della cornice grigio scuro (60%)
    tabsize=2,                                       % Tabulazioni di 2 spazi
    captionpos=b,                                    % Didascalia in basso
    breaklines=true,                                 % Spezzare le righe troppo lunghe
    breakatwhitespace=true,                          % Spezzare preferibilmente sugli spazi
    title=\lstname,                                  % Usa il nome del file come titolo del blocco
    keywordstyle=\color{blue!80!black}\bfseries,     % Parole chiave in blu scuro e grassetto
    commentstyle=\color{green!50!black}\itshape,     % Commenti in verde scuro corsivo
    stringstyle=\color{orange!80!black},             % Stringhe in arancio scuro
    emphstyle=\color{red!70!black}\bfseries,          % Evidenziazioni extra in rosso scuro (per emph)
    escapeinside={\%*}{*},                           % Permette di inserire commenti LaTeX all'interno
    morekeywords={boto3, iam, lambda_handler, event, context, client, update_user, UserName, PermissionsBoundary},
    % Aggiunta di built-in e funzioni Python comunemente usate
    morekeywords=[2]{def, return, import, as, if, elif, else, for, while, in, try, except, with, class, from, pass, True, False, None},
    keywordstyle=[2]\color{purple!70!black}\bfseries, % Secondo gruppo di keyword in viola scuro 
    xleftmargin=5pt,                                 % Margine interno sinistro
    xrightmargin=5pt,                                % Margine interno destro
    % ---- Opzionale: evidenzia il prompt ">>>" di un REPL Python ----
    moredelim=[is][\color{red!70!black}\bfseries]{>>>}{\ }, 
}

    % Definizione stile per codice Bash migliorato
    \lstdefinestyle{bash}{
        language=bash,
        basicstyle=\small\ttfamily,
        numbers=left,
        numberstyle=\tiny\color{gray},
        stepnumber=1,
        numbersep=5pt,
        backgroundcolor=\color{gray!10},
        showspaces=false,
        showstringspaces=false,
        frame=single,
        rulecolor=\color{gray!60},
        tabsize=2,
        captionpos=b,
        breaklines=true,
        breakatwhitespace=true,
        title=\lstname,
        keywordstyle=\color{blue!80!black},
        commentstyle=\color{green!50!black}\itshape,
        stringstyle=\color{orange!80!black},
        escapeinside={\%*}{*},
        morekeywords={aws, sts, assume-role, --role-arn, --role-session-name},
        xleftmargin=5pt,
        xrightmargin=5pt,
    }




%
%			TITOLO
\begin{document}



\title{Sicurezza dell'Infrastruttura AWS in una Startup fintech}
\author{Andrea Ferraboli}
\dept{Corso di Laurea in Sicurezza dei Sistemi e delle Reti Informatiche } 
\anno{2024-2025}
\matricola{09985A}
\relatore{Prof. Claudio Agostino Ardagna}
\correlatore{Lorenzo Perotta, Andrea Pasini, Simone Cortese}

\maketitlepage
\prefacesection{Prefazione}
La crescente adozione delle tecnologie cloud ha rivoluzionato il settore fintech, 
portando benefici significativi in termini di scalabilità, flessibilità e riduzione 
dei costi operativi. Dall'altro lato, questa trasformazione digitale introduce 
nuove sfide di sicurezza derivanti dall'outsourcing dell'infrastruttura IT e dalla 
gestione di dati finanziari sensibili in ambienti distribuiti. Per le startup fintech, 
queste problematiche sono amplificate dalla necessità di bilanciare rapidità di 
sviluppo e conformità normativa con standard di sicurezza elevati.

Amazon Web Services (AWS) rappresenta oggi una delle piattaforme cloud più 
adottate dalle startup fintech per la sua robustezza e completezza dell'offerta di 
servizi. Tuttavia, la configurazione sicura di un'infrastruttura AWS richiede competenze 
specifiche e l'implementazione di strategie di sicurezza avanzate che vanno oltre 
le configurazioni standard. L'obiettivo di questa tesi è stato analizzare e implementare 
un'architettura di sicurezza completa su AWS specificamente progettata per le 
esigenze di una startup fintech.

La ricerca si concentra sull'implementazione pratica di controlli di sicurezza 
avanzati, tra cui la gestione granulare delle identità e degli accessi (IAM), la 
progettazione di reti sicure mediante Virtual Private Cloud (VPC), e l'implementazione 
di sistemi di rilevamento proattivo delle minacce attraverso honeypot. Particolare 
attenzione è stata posta alla creazione di un ambiente che sia al contempo sicuro, 
scalabile e conforme ai principali framework di sicurezza internazionali.

L'elaborato è organizzato come segue:
\newline
\newline
\textbf{Capitolo 1 – Introduzione} In questo capitolo si presenta il contesto delle 
startup fintech, analizzando le specificità del settore e le principali sfide 
di sicurezza che caratterizzano queste realtà innovative.
\newline
\newline
\textbf{Capitolo 2 – Principi di cybersecurity olistici per un'infrastruttura fintech} 
Il secondo capitolo fornisce le basi teoriche della sicurezza informatica, 
dalla triade CIA ai principi di difesa in profondità, contestualizzandoli 
nell'ambito fintech.
\newline
\newline
\textbf{Capitolo 3 – Principi dell'Infrastruttura Cloud e Scelta di AWS} Il terzo 
capitolo analizza i paradigmi del cloud computing e presenta AWS come 
piattaforma di riferimento, descrivendone architettura e caratteristiche 
di sicurezza.
\newline
\newline
\textbf{Capitolo 4 – Progettazione e Implementazione Avanzata della Sicurezza delle 
Identità e degli Accessi (IAM) in AWS} Questo capitolo si concentra sull'implementazione 
pratica di IAM, analizzando strategie avanzate per la gestione delle identità 
e l'applicazione del principio del minimo privilegio.
\newline
\newline
\textbf{Capitolo 5 – Architettura di Rete Sicura e Protezione dei Servizi Applicativi 
su AWS per fintech} Il quinto capitolo descrive la progettazione di un'architettura 
di rete sicura mediante VPC, gruppi di sicurezza e controlli di accesso avanzati.
\newline
\newline
\textbf{Capitolo 6 – Implementazione di un Honeypot in un'Infrastruttura AWS per 
Startup fintech} L'ultimo capitolo tecnico presenta l'implementazione di un 
sistema honeypot su AWS per il rilevamento proattivo delle minacce, includendo 
analisi dei costi e test di efficacia.
\newline
\newline
\textbf{Capitolo 7 – Conclusioni} Il capitolo finale riassume i risultati raggiunti 
e presenta considerazioni sulle prospettive future per la sicurezza delle 
infrastrutture cloud in ambito fintech.
\prefacesection{Dedica}
{\hfill \Large {\sl dedicato a chi mi vuole bene, a chi mi stima e ai miei compagni di viaggio, vi voglio bene}}
\prefacesection{Ringraziamenti}
        Vorrei ringraziare soprattutto \dots

\setcounter{tocdepth}{2} % Mostra solo chapter, section e subsection
\tableofcontents
\listoffigures % Comando per generare l'elenco delle figure

\chapter{Introduzione}
\pagenumbering{arabic}
\label{chapter:introduzione}

\section{La Cybersecurity nelle Startup Fintech: Sfide, Vulnerabilità e Strategie di Protezione in un Ecosistema in Rapida Evoluzione}

Il settore fintech rappresenta oggi una delle aree più dinamiche e innovative dell'ecosistema startup, con investimenti globali che hanno raggiunto i 115 miliardi di dollari, in crescita esponenziale rispetto ai 53.2 miliardi del 2018 \cite{gartnerFintech}. Questo rapido sviluppo, caratterizzato dall'implementazione di tecnologie emergenti per i servizi finanziari, porta con sé non solo opportunità senza precedenti ma anche significative sfide in termini di sicurezza informatica. Le startup fintech, che si trovano all'intersezione tra finanza tradizionale e innovazione tecnologica, gestiscono dati estremamente sensibili diventando bersagli privilegiati per i cybercriminali. Questa tesi esplora le vulnerabilità specifiche di queste realtà, analizza le principali minacce che affrontano e propone strategie di sicurezza efficaci anche in contesti di risorse limitate, evidenziando come un approccio proattivo alla cybersecurity non rappresenti un costo ma un investimento strategico fondamentale per il successo a lungo termine di una startup fintech.
\subsection{Definizione di Fintech}

Nell'ambito economico-finanziario, con il termine \textbf{fintech} (contrazione di ``financial technology'') si indica l'\textbf{innovazione nei servizi finanziari} resa possibile dalle moderne tecnologie digitali \cite{tecnofinanza}. Una \textbf{startup fintech} è quindi una \textbf{nuova impresa} che opera nel settore della tecnologia finanziaria, basando il proprio modello di business sulle tecnologie ICT più avanzate e contrapponendosi agli approcci tradizionali degli operatori finanziari consolidati \cite{fintech_numeri}. 

Queste giovani aziende ad alta componente tecnologica mirano a migliorare l'accessibilità, l'efficienza e la qualità dei servizi finanziari, e stanno svolgendo un ruolo cruciale nella \textbf{digitalizzazione del mercato finanziario italiano} \cite{tecnofinanza}. 

Tra i servizi e le soluzioni tipicamente offerti dalle startup fintech vi sono:
\begin{itemize}
    \item \textbf{Pagamenti digitali} (ad esempio tramite app mobili)
    \item Trasferimenti di denaro \textbf{peer-to-peer}
    \item \textbf{Prestiti diretti tra privati} (social lending)
    \item \textbf{Finanziamento partecipativo} (crowdfunding)
    \item Servizi assicurativi innovativi legati all'\textbf{insurtech}
    \item Impiego di tecnologie come la \textbf{blockchain} e le \textbf{criptovalute} per abilitare nuovi servizi finanziari
\end{itemize}

In linea con la crescita globale del fenomeno, in Italia si contavano oltre \textbf{600 startup fintech e insurtech} attive nel 2023 \cite{fintech_numeri}, a testimonianza di un ecosistema in rapido sviluppo.
\subsection{Il Contesto delle Startup Fintech: Un Ecosistema Dinamico e Sfidante}

Le startup fintech operano in un ambiente caratterizzato da elevata incertezza, risorse limitate e necessità di crescita rapida, fattori che influenzano profondamente le decisioni in ambito IT e sicurezza informatica \cite{fintechChallenges}. A differenza delle istituzioni finanziarie tradizionali, queste realtà innovative non dispongono generalmente di strutture gerarchiche complesse o budget consistenti dedicati alla sicurezza, dovendo invece adottare approcci agili e flessibili.

Il contesto finanziario in cui operano le startup fintech impone pressioni significative sulle decisioni di spesa. Ogni investimento, compreso quello per l'infrastruttura IT e la sicurezza, deve essere attentamente valutato in termini di ritorno immediato e benefici a lungo termine \cite{fintechChallenges}. Questa ottimizzazione dei costi rappresenta una sfida continua, poiché la sicurezza informatica richiede investimenti costanti, spesso non producendo risultati immediatamente visibili, la cui assenza può comportare conseguenze catastrofiche. In questo equilibrio delicato, le startup fintech devono trovare il giusto compromesso tra la necessità di scalare rapidamente e l'implementazione di solide misure di protezione.

\subsection{La Distinzione tra Cybersecurity Bancaria e Fintech}

Un aspetto fondamentale da considerare è la sostanziale differenza tra l'approccio alla cybersecurity nel settore bancario tradizionale e nelle startup fintech. Mentre le banche operano in un contesto fortemente regolamentato, con obblighi legali precisi in materia di sicurezza e protezione dei dati, le fintech hanno tradizionalmente goduto di una maggiore flessibilità normativa \cite{bankingVsFintech}. Le grandi istituzioni bancarie investono ingenti risorse nel testare costantemente le proprie misure di sicurezza, consapevoli che anche il minimo incidente può comportare la perdita di migliaia di clienti e sanzioni finanziarie significative.

Le fintech, spesso costituite da piccole startup in rapida espansione, possono fungere da "overlay" per le banche, facilitando la fornitura di servizi finanziari innovativi ma operando inizialmente con regolamentazioni meno stringenti \cite{bankingVsFintech}. Questa differenza normativa sta tuttavia diminuendo, soprattutto per quelle fintech che si trasformano gradualmente in vere e proprie banche, sottoponendosi così a un maggiore scrutinio regolamentare. La sfida per le startup fintech consiste quindi nel bilanciare l'agilità operativa con l'adozione di standard di sicurezza elevati, anticipando l'evoluzione normativa del settore.

\subsection{Sfide Principali per le Startup Fintech in Ambito Cybersecurity}

Le startup fintech affrontano sfide specifiche nel campo della sicurezza informatica, che derivano dalla loro natura innovativa e dalle caratteristiche distintive del loro modello di business \cite{fintechChallenges}. La prima e più evidente sfida è rappresentata dal budget limitato per la sicurezza, che spesso costringe a difficili compromessi tra lo sviluppo di nuove funzionalità e l'implementazione di adeguate misure protettive. Questa limitazione finanziaria si riflette anche nella difficoltà di attrarre e mantenere personale specializzato in cybersecurity, un ambito in cui la domanda supera ampiamente l'offerta e le grandi aziende possono offrire compensi difficilmente pareggiabili da una startup.

La pressione per un rapido accesso al mercato rappresenta un'ulteriore sfida significativa. Nel settore fintech, essere i primi a offrire un servizio innovativo può fare la differenza tra il successo e il fallimento, ma questa corsa contro il tempo spesso porta a sottovalutare gli aspetti legati alla sicurezza \cite{fintechChallenges}. Inoltre, la scalabilità dell'infrastruttura IT rappresenta una sfida tecnica considerevole: progettare sistemi che siano non solo sicuri ma anche in grado di crescere rapidamente al crescere dell'azienda richiede competenze specifiche e una pianificazione accurata.

L'adozione di tecnologie emergenti, caratteristica distintiva delle fintech, introduce nuove superfici di attacco e vulnerabilità potenziali \cite{fintechChallenges}. Cloud computing, intelligenza artificiale, blockchain e API aperte offrono opportunità straordinarie ma richiedono approcci di sicurezza specifici e aggiornati. Allo stesso tempo, la crescente interconnessione con partner, fornitori e piattaforme di terze parti amplia ulteriormente la superficie di attacco, rendendo la gestione del rischio ancora più complessa.

Non va sottovalutato, infine, il rischio rappresentato dalle minacce interne (insider threats). Nelle fasi iniziali di una startup, quando i controlli sono meno rigidi e le procedure di sicurezza meno formalizzate, il rischio legato a dipendenti negligenti o, in casi più rari, malintenzionati, aumenta considerevolmente \cite{fintechChallenges}. La cultura della condivisione e dell'apertura, tipica delle startup, deve quindi essere bilanciata con adeguate politiche di accesso e controllo.

\subsection{Minacce e Attacchi Informatici nel Settore Fintech}

Il settore fintech, per la sua natura altamente tecnologica e la gestione di dati finanziari sensibili, è diventato uno dei bersagli preferiti dei cybercriminali \cite{cyberThreatsFintech}. Tra le minacce più diffuse e pericolose figurano gli attacchi di phishing, attraverso i quali i malintenzionati cercano di ottenere credenziali di accesso, dati personali o informazioni finanziarie utilizzando email, messaggi e siti web fraudolenti che imitano comunicazioni ufficiali \cite{cyberThreatsFintech}. Queste tecniche di social engineering sfruttano la fiducia degli utenti e le loro abitudini digitali per compromettere account e sistemi aziendali.

I malware e i ransomware rappresentano un'altra categoria di minacce particolarmente grave per le startup fintech. Questi software malevoli possono infiltrarsi nei sistemi attraverso vari vettori, bloccare l'accesso ai dati e richiedere un riscatto per ripristinarlo, causando danni finanziari diretti e interruzioni operative significative \cite{cyberThreatsFintech}. Le conseguenze di un attacco ransomware possono essere devastanti per una startup con risorse limitate, in quanto il riscatto diventa percentualmente troppo oneroso per le finanze aziendali.

Gli attacchi alle API (Application Programming Interfaces), sempre più utilizzate nel settore fintech per l'integrazione con servizi terzi, costituiscono un vettore di attacco in crescita \cite{fintechChallenges}. Le API mal configurate o non adeguatamente protette possono diventare punti di ingresso privilegiati per i cybercriminali, consentendo l'accesso non autorizzato a dati sensibili e funzionalità critiche del sistema. Simile criticità presentano le configurazioni errate dei servizi cloud, che possono esporre involontariamente dati riservati o creare vulnerabilità sfruttabili.

Le startup fintech devono inoltre considerare il rischio di attacchi DDoS (Distributed Denial of Service), che mirano a rendere inaccessibili i servizi sovraccaricando i server con richieste fraudolente \cite{fintechChallenges}. Questi attacchi, relativamente semplici da orchestrare ma potenzialmente molto dannosi, possono essere utilizzati sia come attacco diretto che come diversivo per mascherare altre attività malevoli più sofisticate.

\subsection{Conseguenze degli Attacchi e Impatto sulle Startup Fintech}

L'impatto di un attacco informatico su una startup fintech può essere multidimensionale e, in molti casi, esistenziale. A livello finanziario, oltre ai costi diretti per il ripristino dei sistemi e la gestione dell'incidente, vanno considerati i potenziali risarcimenti a clienti danneggiati, le sanzioni normative e l'aumento dei premi assicurativi \cite{fintechChallenges}. Ma è forse l'impatto reputazionale a rappresentare la minaccia più grave: in un settore basato sulla fiducia come quello finanziario, una violazione dei dati può comprometterne irreparabilmente l'immagine, portando alla perdita di clienti attuali e potenziali.

L'interruzione operativa conseguente a un attacco può avere effetti a catena, influenzando non solo i clienti diretti ma anche partner commerciali e fornitori \cite{fintechChallenges}. In un ecosistema interconnesso come quello fintech, l'interdipendenza tra diverse piattaforme e servizi amplifica ulteriormente l'impatto di un incidente di sicurezza, con effetti che possono estendersi ben oltre il perimetro aziendale immediato.

\subsection{Importanza di un Approccio Proattivo alla Cybersecurity}

Implementare una strategia di cybersecurity solida sin dalle prime fasi di sviluppo di una startup fintech non configura un semplice onere, bensì un investimento strategico di primaria importanza \cite{fintechChallenges}. L'adozione del paradigma "security by design" permette infatti di integrare la sicurezza in maniera organica nei processi aziendali e nel ciclo di sviluppo del prodotto, contribuendo alla significativa riduzione dei costi a lungo termine e alla minimizzazione dei rischi potenziali. Al contrario, la mancata attenzione alla sicurezza nelle fasi iniziali comporta l'accumulo di "security debt", ovvero un debito tecnico in ambito sicurezza che, analogamente a un mutuo con tassi elevati, diventa progressivamente più oneroso da gestire e da ripagare nel tempo. Infine, la pressione derivante dalla necessità di accelerare lo sviluppo e di raggiungere rapidamente il mercato può portare a trascurare aspetti fondamentali della sicurezza, esacerbando ulteriormente tale debito tecnico.

Un approccio preventivo alla sicurezza risulta sempre più efficace ed economico rispetto a uno reattivo \cite{fintechChallenges}. I costi per implementare misure di sicurezza di base sono generalmente inferiori rispetto a quelli necessari per rispondere a un incidente, che possono includere non solo il ripristino dei sistemi ma anche sanzioni, risarcimenti e danni reputazionali. La cybersecurity deve quindi essere considerata come parte integrante della strategia aziendale, non come un elemento accessorio o un costo da minimizzare.

Le startup fintech devono inoltre considerare che adeguati livelli di sicurezza rappresentano spesso un requisito fondamentale per attrarre investitori e partner commerciali \cite{fintechChallenges}. Durante le fasi di due diligence, l'analisi delle misure di sicurezza implementate è diventata una componente standard, e lacune significative in questo ambito possono compromettere opportunità di finanziamento o collaborazioni strategiche.

\subsection{Approccio Metodologico della Tesi}

Questa tesi si propone di affrontare le sfide della cybersecurity nelle startup fintech attraverso un approccio metodologico strutturato ma flessibile \cite{fintechChallenges}. Pur concentrandosi su un caso studio pratico specifico, l'obiettivo è fornire principi e best practice di sicurezza generici e applicabili a qualsiasi startup fintech, indipendentemente dalla piattaforma tecnologica specifica utilizzata. L'approccio adottato riconosce le limitazioni di risorse tipiche delle startup e propone soluzioni scalabili che possono evolvere con la crescita dell'organizzazione.

La metodologia si basa su tre pilastri fondamentali: l'identificazione delle minacce specifiche per il modello di business fintech, la prioritizzazione degli interventi in base al rapporto rischio/beneficio e l'implementazione di controlli di sicurezza essenziali ma efficaci \cite{fintechChallenges}. Questo approccio pragmatico consente di ottenere un livello di protezione adeguato anche con risorse limitate, concentrando gli sforzi sugli aspetti più critici.

\section{Principi di cybersecurity olistici per un'infrastruttura tech}
\subsection{Introduzione}
Nell'analisi e nello studio dell'infrastruttura di una startup fintech, per comprendere le possibili implementazioni a livello di sicurezza dobbiamo prima delineare quali siano i principi di cybersecurity a cui ogni startup, fintech o meno, deve attenersi. In questo capitolo verranno analizzati i principi di cybersecurity più importanti e quali sono le principali sfide che un'azienda di piccole dimensioni può affrontare all'inizio del proprio percorso nell'adozione di tali pratiche.

Questo capitolo esplora i principi fondamentali di sicurezza informatica che ogni organizzazione dovrebbe implementare, con particolare attenzione alle sfide uniche che le startup fintech affrontano nell'adozione di tali pratiche. Il settore fintech, caratterizzato da rapida innovazione e gestione di dati finanziari sensibili, presenta un contesto particolarmente critico dove le best practice di sicurezza si scontrano spesso con le esigenze di velocità di sviluppo, risorse limitate e necessità di time-to-market accelerato.

\section{Principi di cybersecurity}
\label{sec:principi-cybersecurity}

\subsection{Triade CIA}
La Triade CIA rappresenta i tre pilastri fondamentali dell'information security: confidentiality (riservatezza), integrity (integrità) e availability (disponibilità) \cite{NIST_SP_1800_26}. Questi principi costituiscono la base su cui costruire qualsiasi strategia di sicurezza informatica robusta.

\begin{itemize}
\item \textbf{Confidentiality}: La riservatezza si concentra sul preservare le restrizioni autorizzate sull'accesso e la divulgazione delle informazioni, inclusi i mezzi per proteggere la privacy personale e le informazioni proprietarie \cite{NIST_SP_1800_26}. Questo principio viene generalmente rispettato tramite la crittografia dei dati, sia a riposo (stored data) che in transito (data in transit), controlli di accesso rigorosi, come liste di controllo degli accessi (ACL), autenticazione a più fattori (MFA) e Role-Based Access Control (RBAC).

text
Nel contesto di una startup fintech, l'implementazione della riservatezza presenta sfide significative. L'accesso ai dati dei clienti e alle informazioni finanziarie deve essere rigorosamente controllato, ma i team piccoli e multifunzionali tipici delle startup spesso portano a una condivisione delle credenziali o all'assegnazione di privilegi eccessivi per "far funzionare le cose rapidamente".

La gestione delle chiavi di cifratura rappresenta un'ulteriore complessità: nelle startup dove i ruoli non sono chiaramente definiti, la responsabilità della gestione delle chiavi può essere ambigua, portando potenzialmente a compromissioni della sicurezza.

\item \textbf{Integrity}: L'integrità dei dati comporta la protezione contro modifiche o distruzioni improprie delle informazioni e garantisce la non ripudiabilità e l'autenticità delle informazioni \cite{NIST_SP_1800_26}. Mantenere l'integrità dei dati è essenziale per prevenire la diffusione di informazioni corrotte o ingannevoli, che potrebbero avere gravi ripercussioni in settori critici come quello sanitario o finanziario. Le tecniche utilizzate per preservare l'integrità includono:
\begin{itemize}
    \item Funzioni di hash crittografiche (es. SHA-256) per verificare che i dati non siano stati alterati.
    \item Firme digitali per autenticare l'origine dei dati e garantirne la non modifica.
    \item Controllo delle versioni per tracciare le modifiche e ripristinare versioni precedenti.
    \item Checksum e meccanismi di rilevamento degli errori.
\end{itemize}

Nel contesto di una startup fintech, l'integrità dei dati è uno di quegli aspetti che va ad inficiare la brand reputation della startup stessa, in quanto la fiducia degli stakeholders si basa anche sulla capacità della startup di conservare i dati dei clienti senza distorsioni e di garantire l'accuratezza nelle transazioni di dati nella maniera più professionale possibile.

\item \textbf{Availability}: Questo principio assicura l'accesso affidabile e tempestivo alle informazioni \cite{NIST_SP_1800_26}. Mira a prevenire interruzioni del servizio, sia dovute a guasti tecnici che ad attacchi malevoli come i Denial-of-Service (DoS) o Distributed Denial-of-Service (DDoS). L'indisponibilità può causare interruzioni operative, perdite economiche e danni alla reputazione. Le strategie per garantire un'elevata disponibilità comprendono:
\begin{itemize}
    \item Sistemi ridondanti (hardware, software, reti) per eliminare singoli punti di fallimento (Single Points of Failure - SPOF).
    \item Backup regolari e piani di disaster recovery (DR) e business continuity (BCP).
    \item Tecniche di bilanciamento del carico (load balancing) per distribuire il traffico di rete.
    \item Misure di protezione contro attacchi DoS/DDoS.
\end{itemize}

Nella maggior parte delle startup, l'infrastruttura di base viene sviluppata considerando una capacità di carico massimo limitato, in quanto nei primi periodi di vita dell'azienda non ci si aspetta un elevato numero di utenti. Proprio per questo motivo, l'infrastruttura presenta un punto vulnerabile che può essere sfruttato dagli attaccanti per mettere a repentaglio l'intero sistema, ad esempio con attacchi DoS/DDoS mirati al perimetro aziendale.
\end{itemize}

\subsection{Difesa in Profondità (Defense in Depth)}
Il principio di difesa in profondità prevede l'implementazione di una stratificazione delle risorse informatiche di protezione \cite{Cyberment}. Questo approccio permette di rallentare la penetrazione di un eventuale attacco esterno, al fine di avere poi il tempo necessario per una efficace reazione protettiva. La strategia di difesa in profondità fornisce la fondazione per una protezione multidimensionale che include tre componenti mutualmente supportive e rinforzanti: (1) architettura resistente alla penetrazione, (2) operazioni di limitazione dei danni, e (3) progettazione per la cyber resilienza e la sopravvivenza \cite{NIST_SP_800_172}.

Nelle startup fintech, l'implementazione della difesa in profondità è spesso compromessa da vincoli di risorse e pressioni temporali. Ad esempio, mentre una soluzione di autenticazione a più fattori (MFA) è essenziale per proteggere l'accesso a dati finanziari sensibili, una startup potrebbe inizialmente implementare solo l'autenticazione basata su password per accelerare l'onboarding degli utenti, pianificando di aggiungere MFA "in un secondo momento" – un momento che potrebbe non arrivare prima che si verifichi un incidente di sicurezza.

La segmentazione della rete, fondamentale per contenere eventuali violazioni, richiede una progettazione accurata dell'infrastruttura. Tuttavia, nelle fasi iniziali, molte startup fintech operano con architetture di rete piatte per semplificare lo sviluppo e ridurre il sovraccarico operativo, oltre a non disporre del capitale umano competente per gestire una tale complessità.

\subsection{Principio del Minimo Privilegio}
Il principio del minimo privilegio stabilisce che un sistema dovrebbe limitare i privilegi di accesso degli utenti (o dei processi che agiscono per conto degli utenti) al minimo necessario per svolgere le attività assegnate \cite{NIST_Glossary}. Questo principio dichiara che un'architettura di sicurezza è progettata in modo che a ciascuna entità siano concesse le minime autorizzazioni e risorse di sistema necessarie per svolgere la propria funzione \cite{NIST_Glossary}.

Nelle startup fintech, applicare il principio del minimo privilegio presenta sfide uniche. La cultura focalizzata sulla velocità d'esecuzione spinge spesso a trascurare la sicurezza granulare degli accessi. È forte la tentazione di assegnare privilegi amministrativi ampi e generici per accelerare lo sviluppo, piuttosto che investire tempo nella configurazione di permessi specifici per ogni compito.

Un esempio comune è concedere a tutti gli sviluppatori accesso completo al database di produzione durante la creazione di una nuova dashboard, invece di limitare ciascuno alle sole tabelle o operazioni strettamente necessarie. Sebbene sembri una scorciatoia efficiente, questa pratica crea vulnerabilità critiche: la compromissione di un singolo account può esporre una quantità sproporzionata di dati sensibili, amplificando enormemente i danni di una violazione.

\subsection{Separazione dei Compiti (Separation of Duties)}
La separazione dei compiti include la divisione delle funzioni di missione o business e le funzioni di supporto tra diverse persone o ruoli, conducendo funzioni di supporto al sistema con individui diversi, e assicurando che il personale di sicurezza che amministra le funzioni di controllo degli accessi non amministri anche le funzioni di audit \cite{OSCAL_Content}. Poiché le violazioni della separazione dei compiti possono estendersi a sistemi e domini di applicazioni, le organizzazioni considerano l'interezza dei sistemi e dei componenti del sistema quando sviluppano politiche sulla separazione dei compiti \cite{OSCAL_Content}.

Nelle startup fintech, dove i team sono piccoli e i ruoli spesso sovrapposti, questo principio è particolarmente difficile da attuare. Ad esempio, in una startup che sviluppa una piattaforma di prestiti P2P, potrebbe esserci un solo ingegnere responsabile sia dell'implementazione del sistema di scoring del credito sia della configurazione dei controlli di sicurezza sullo stesso sistema. Questa concentrazione di responsabilità crea un rischio intrinseco: errori o azioni malevole potrebbero passare inosservati senza un secondo paio di occhi che verifichi il lavoro.

\subsection{Zero Trust}
Il modello Zero Trust si basa sul concetto che un'organizzazione non dovrebbe fidarsi automaticamente di nulla sia all'interno che all'esterno dei suoi perimetri e deve verificare tutto ciò che tenta di connettersi ai suoi sistemi prima di concedere l'accesso \cite{NIST_SP_800_207}. Zero Trust è una risposta evoluta alle tendenze che includono la migrazione delle risorse di lavoro verso ambienti cloud, lavoratori che operano da dispositivi mobili ovunque si trovino e una crescente collaborazione tra organizzazioni \cite{NIST_SP_800_207}.

Per una startup fintech, l'adozione rigorosa del principio di Zero Trust può rivelarsi particolarmente gravosa. Nelle fasi iniziali, è frequente che l'intera infrastruttura sia gestita da una sola persona, con responsabilità sia di sviluppo sia di amministrazione di rete: questo crea un unico punto di falla, amplificando il rischio di errori di configurazione o di accesso non autorizzato.

Inoltre, le limitate risorse economiche e umane possono rendere difficoltoso implementare soluzioni avanzate di micro-segmentazione, sistemi di Identity and Access Management (IAM) complessi e piattaforme di monitoraggio continuo. Infine, la mancanza di separazione dei compiti e di revisioni periodiche rende più probabile la persistenza di permessi eccessivi o non aggiornati, esponendo i sistemi a potenziali attacchi laterali e perdite di dati sensibili.

\chapter{Principi di cybersecurity olistici \allowbreak{} per un'infrastruttura fintech}
\label{ch:principi-cybersecurity}
Dopo aver introdotto le sfide di cybersecurity specifiche per le startup fintech, è fondamentale comprendere il contesto tecnologico in cui queste operano. Oggi, la stragrande maggioranza delle nuove imprese, specialmente nel settore tecnologico e finanziario, basa la propria infrastruttura su modelli di \textbf{cloud computing}. Questo capitolo esplora i motivi di questa scelta, confrontando l'approccio cloud con quello tradizionale on-premises, e introduce \textbf{Amazon Web Services (AWS)}, il provider cloud scelto nel nostro caso studio, delineandone la struttura e i principi fondamentali.
\section{Fondamenti di Cloud Computing}
Il cloud computing è un modello di fruizione IT \textit{"on-demand"} che abilita l'accesso ubiquo e conveniente via rete a un pool condiviso di risorse computazionali configurabili (reti, server, storage, applicazioni e servizi) che possono essere predisposte o rilasciate rapidamente con il minimo sforzo di gestione o intervento del provider \cite{nist800-145}. Questo modello si basa su cinque caratteristiche essenziali (tra cui autoservizio on-demand, \textit{multitenancy}, scalabilità rapida ed elasticità) ed esplica diversi modelli di servizio e modelli di distribuzione \cite{nist800-145}.

I vantaggi chiave del cloud – in particolare rilevanti per una startup fintech – sono la \textit{scalabilità}, la \textit{flessibilità} e l'\textit{elasticità}. La scalabilità consente di aumentare o diminuire capacità computazionale in base alla domanda \cite{digitalocean-cloud}. L'elasticità estende il concetto di scalabilità rendendola dinamica: le risorse possono aumentare o diminuire automaticamente in tempo reale a fronte di picchi o cali di carico \cite{geeksforgeeks_scalability}. Ciò permette di ottimizzare i costi (pagando solo ciò che serve) e di raggiungere prestazioni adeguate anche in caso di crescita rapida del business, scenario tipico di molte startup fintech.

I principali modelli di servizio cloud sono:
\begin{itemize}
    \item \textbf{IaaS (Infrastructure as a Service):} fornisce infrastruttura IT on-demand (server, macchine virtuali, storage, rete) gestita dal provider cloud \cite{ibm_iaas}. L'utente può configurare e usare queste risorse come farebbe on-premises, senza doverle possedere.
    \item \textbf{PaaS (Platform as a Service):} fornisce una piattaforma completa (sistema operativo, middleware, strumenti di sviluppo) pronta all'uso per sviluppare, eseguire e gestire applicazioni \cite{ibm-cloud}. Il provider mantiene lo strato sottostante (infrastruttura e runtime), mentre l'utente si concentra sullo sviluppo del software.
    \item \textbf{SaaS (Software as a Service):} eroga applicazioni software pronte all’uso attraverso il cloud \cite{ibm-cloud}. L’utente finale accede al servizio (es. web app, CRM, gestione documenti) senza gestire infrastruttura o piattaforma sottostante.
\end{itemize}

Analogamente, i modelli di distribuzione definiscono dove e a chi è dedicato l’ambiente cloud. I più comuni sono:
\begin{itemize}
    \item \textbf{Public Cloud:} l’infrastruttura è posseduta da un provider terzo e messa a disposizione del pubblico via Internet \cite{geeksforgeeks_scalability}. Ad esempio, AWS, Azure e Google Cloud offrono servizi condivisi tra molti clienti. I vantaggi includono costi operativi ridotti e alta scalabilità, mentre gli svantaggi possono riguardare il controllo ridotto e la condivisione delle risorse con altri tenant \cite{geeksforgeeks_scalability}.
    \item \textbf{Private Cloud:} l’infrastruttura è dedicata a un’unica organizzazione (sovente gestita internamente o in data center riservati) \cite{geeksforgeeks_scalability}. Offre maggiore controllo e sicurezza (elevata protezione dei dati sensibili), ma richiede investimenti in hardware dedicato e può avere minore scalabilità rispetto al public cloud \cite{geeksforgeeks_scalability}.
    \item \textbf{Hybrid Cloud:} combina ambienti pubblici e privati, permettendo di spostare workload tra essi a seconda delle necessità \cite{geeksforgeeks_scalability}. Una startup fintech potrebbe usare il public cloud per carichi generici e il private cloud per dati regolamentati, ottenendo sia flessibilità che compliance normativa \cite{geeksforgeeks_scalability}. L’hybrid cloud massimizza scalabilità e controllo mantenendo la coerenza con requisiti di sicurezza o normativi.
\end{itemize}
Queste architetture consentono alle startup fintech di avviare servizi IT senza investimenti iniziali in hardware, scalare in base alla domanda del mercato e sperimentare nuovi servizi in maniera agile, mantenendo al tempo stesso contesti isolati (in ambienti privati) per dati sensibili.
\section{Cloud Computing vs Infrastrutture On-Premises}
\label{sec:cloud-vs-onprem}

Tradizionalmente, le aziende gestivano la propria infrastruttura IT internamente, in data center di proprietà o in affitto. Questo modello è noto come \textbf{on-premises}. Richiede l'acquisto di hardware (server, storage, apparati di rete), software (sistemi operativi, licenze), e l'impiego di personale specializzato per la gestione, la manutenzione, gli aggiornamenti e la sicurezza fisica ed operativa.

Il \textbf{cloud computing}, invece, si basa sull'erogazione di risorse informatiche (come potenza di calcolo, storage, database, reti, software, analytics, intelligenza artificiale) tramite Internet, secondo un modello \textit{pay-as-you-go} (paga solo per ciò che consumi). I fornitori di servizi cloud (Cloud Service Provider - CSP), come AWS, Microsoft Azure o Google Cloud Platform, gestiscono l'infrastruttura fisica sottostante, permettendo ai clienti di accedere alle risorse di cui hanno bisogno in modo flessibile e scalabile.

Le differenze principali risiedono in:
\begin{itemize}
    \item \textbf{Costi:} On-premises richiede un ingente investimento iniziale (Capex - Capital Expenditure) per l'acquisto dell'hardware, mentre il cloud trasforma questo costo in una spesa operativa variabile (Opex - Operational Expenditure) basata sul consumo effettivo.
    \item \textbf{Scalabilità:} Il cloud offre scalabilità \textit{elastica}, permettendo di aumentare o diminuire le risorse quasi istantaneamente in base alla domanda. L'infrastruttura on-premises ha una scalabilità limitata e richiede pianificazione e acquisti anticipati per gestire picchi di carico.
    \item \textbf{Manutenzione:} Nel cloud, la manutenzione dell'hardware e dell'infrastruttura di base è responsabilità del provider, liberando il team IT del cliente da queste incombenze.
    \item \textbf{Agilità e Velocità:} Il cloud permette di provisionare nuove risorse in pochi minuti, accelerando notevolmente i cicli di sviluppo e il time-to-market di nuovi prodotti o servizi.
    \item \textbf{Affidabilità e Portata Globale:} I principali CSP dispongono di data center ridondati in diverse regioni geografiche, offrendo alta disponibilità e la possibilità di distribuire applicazioni a livello globale con bassa latenza.
\end{itemize}

\section{Perché le Startup Scelgono il Cloud}
\label{sec:startup-cloud-choice}

Per le startup, specialmente quelle fintech che necessitano di agilità e gestione efficiente delle risorse, il modello cloud offre vantaggi decisivi rispetto all'on-premises:

\begin{itemize}
    \item \textbf{Riduzione delle Barriere all'Ingresso:} L'assenza di grandi investimenti iniziali (Capex) rende accessibili tecnologie avanzate anche a realtà con budget limitati. Si paga solo per l'uso effettivo, allineando i costi alla crescita.
    \item \textbf{Scalabilità Rapida:} Una startup può iniziare con poche risorse e scalare rapidamente man mano che la base utenti o il volume delle transazioni cresce, senza dover sovradimensionare l'infrastruttura all'inizio. Questo è cruciale nel fintech, dove i volumi possono essere imprevedibili.
    \item \textbf{Focalizzazione sul Core Business:} Delegando la gestione dell'infrastruttura al CSP, la startup può concentrare le proprie risorse limitate (tempo e personale) sullo sviluppo del prodotto, sull'acquisizione clienti e sull'innovazione, anziché sulla gestione dei server.
    \item \textbf{Velocità di Innovazione (Time-to-Market)}:** La possibilità di provisionare velocemente ambienti di sviluppo, test e produzione accelera il rilascio di nuove funzionalità, un fattore competitivo essenziale.
    \item \textbf{Accesso a Tecnologie Avanzate:} I CSP offrono servizi gestiti per database, machine learning, big data analytics, sicurezza, ecc., che sarebbero complessi e costosi da implementare e gestire autonomamente on-premises.
    \item \textbf{Affidabilità e Sicurezza di Base:} I CSP investono massicciamente in sicurezza fisica e operativa dei loro data center, offrendo un livello di base di affidabilità e sicurezza spesso superiore a quello che una startup potrebbe permettersi on-premises (sebbene la sicurezza *nel* cloud rimanga responsabilità del cliente).
\end{itemize}
\section{Introduzione ad Amazon Web Services (AWS)}
\label{sec:aws-intro}

Tra i principali fornitori di servizi cloud, \textbf{Amazon Web Services (AWS)} è il leader di mercato e rappresenta la scelta infrastrutturale per moltissime startup a livello globale, incluse quelle operanti nel settore fintech, come nel caso studio di questa tesi. Lanciato nel 2006, AWS offre un portafoglio estremamente ampio e maturo di servizi cloud.

La struttura di AWS si basa su alcuni concetti chiave:

\begin{itemize}
    \item \textbf{Infrastruttura Globale:} AWS opera attraverso una rete mondiale di \textbf{Regioni}. Ogni Regione è un'area geografica fisica separata (es. Irlanda, Francoforte, Nord Virginia). All'interno di ciascuna Regione, esistono multiple \textbf{Zone di Disponibilità (Availability Zones - AZ)}. Una AZ è costituita da uno o più data center discreti, con alimentazione, raffreddamento e rete ridondati. Le AZ all'interno di una Regione sono interconnesse con reti a bassa latenza ma sono fisicamente separate per garantire l'isolamento in caso di guasti (incendi, allagamenti, etc.). Questa architettura permette di costruire applicazioni altamente disponibili e tolleranti ai guasti distribuendole su più AZ.
    \item \textbf{Servizi Fondamentali:} AWS offre centinaia di servizi, ma alcuni sono considerati fondamentali:
        \begin{itemize}
            \item \textbf{Compute:} Servizi per eseguire codice, come \textit{Amazon EC2 (Elastic Compute Cloud)} per macchine virtuali scalabili, \textit{AWS Lambda} per l'esecuzione di codice serverless (senza gestire server), e servizi container come \textit{ECS} ed \textit{EKS}.
            \item \textbf{Storage:} Servizi per l'archiviazione dei dati, come \textit{Amazon S3 (Simple Storage Service)} per lo storage a oggetti altamente duraturo e scalabile, \textit{Amazon EBS (Elastic Block Store)} per volumi a blocchi per le istanze EC2, e \textit{Amazon EFS} per file system condivisi.
            \item \textbf{Database:} Una vasta gamma di database gestiti, inclusi database relazionali (\textit{Amazon RDS}), NoSQL (\textit{Amazon DynamoDB}), data warehouse (\textit{Amazon Redshift}), ecc.
            \item \textbf{Networking:} Servizi per definire e controllare la rete virtuale, come \textit{Amazon VPC (Virtual Private Cloud)} per creare reti isolate, \textit{Elastic Load Balancing (ELB)} per distribuire il traffico, e \textit{AWS Direct Connect} per connessioni dedicate.
            \item \textbf{Security, Identity, \& Compliance:} Servizi per gestire accessi, sicurezza e conformità, come \textit{AWS IAM (Identity and Access Management)}, \textit{AWS KMS (Key Management Service)}, \textit{AWS WAF (Web Application Firewall)}, \textit{Amazon GuardDuty} (rilevamento minacce).
        \end{itemize}
    \item \textbf{Modello Pay-as-you-go:} Come accennato, si paga solo per le risorse effettivamente consumate, senza contratti a lungo termine o costi iniziali (per la maggior parte dei servizi).
    \item \textbf{Modello di Responsabilità Condivisa (Shared Responsibility Model)}:** È cruciale capire che la sicurezza su AWS è una responsabilità condivisa. AWS è responsabile della sicurezza *del* cloud (l'infrastruttura fisica, la rete, l'hypervisor), mentre il cliente è responsabile della sicurezza *nel* cloud (la configurazione dei servizi, la gestione degli accessi, la protezione dei dati, la sicurezza del sistema operativo e delle applicazioni).
\end{itemize}

\section{Il Caso Specifico: AWS per la Startup Fintech}
\label{sec:aws-for-fintech}

La scelta di AWS come infrastruttura cloud per la startup fintech oggetto di questa tesi non è casuale. Oltre ai vantaggi generali del cloud, AWS offre caratteristiche particolarmente rilevanti per il settore finanziario:
\begin{itemize}
    \item \textbf{Maturità e Affidabilità:} Essendo il provider più longevo e diffuso, AWS ha una comprovata esperienza nella gestione di carichi di lavoro critici.
    \item \textbf{Ampiezza dei Servizi:} Il vasto portafoglio permette di costruire architetture complesse e moderne, integrando facilmente servizi per l'analisi dei dati, il machine learning (utile per antifrode o scoring), e la gestione sicura delle transazioni.
    \item \textbf{Supporto alla Compliance:} AWS offre documentazione e servizi che aiutano a soddisfare rigorosi standard di conformità richiesti nel settore finanziario, come PCI DSS, GDPR, ISO 27001, ecc. AWS stessa mantiene numerose certificazioni per la propria infrastruttura.
    \item \textbf{Scalabilità e Performance:} Fondamentali per gestire picchi di transazioni tipici dei servizi finanziari.
    \item \textbf{Ecosistema di Partner:} Esiste un vasto ecosistema di partner tecnologici e di consulenza specializzati su AWS, inclusi quelli con expertise nel settore fintech.
    \item \textbf{Servizi di Sicurezza Avanzati:} AWS offre un set robusto di strumenti nativi per implementare controlli di sicurezza a vari livelli (rete, identità, dati, rilevamento minacce), come vedremo nei capitoli successivi.
\end{itemize}
Nei capitoli seguenti, analizzeremo come l'infrastruttura di questa startup fintech è stata costruita e protetta utilizzando specifici servizi e best practice di AWS.
\section{Infrastruttura Globale AWS}
Amazon Web Services (AWS) dispone di una infrastruttura globale altamente distribuita: ad oggi il cloud AWS è esteso su 36 Regioni geografiche (ciascuna costituita da più Availability Zone) per un totale di 114 Availability Zones lanciate \cite{aws-global-infra}. Ogni Regione AWS rappresenta un’area geografica distinta, isolata dalle altre (per \textit{fault tolerance} e requisiti regolamentari) \cite{aws-global-infra}. All’interno di ogni Regione sono presenti almeno tre \textit{Availability Zone (AZ)}: queste sono sedi fisiche indipendenti, collegate da rete privata ad alta velocità ma isolate a livello di infrastruttura di alimentazione e raffreddamento \cite{aws-global-infra}.

Questo design \textit{multi-AZ} consente la progettazione di applicazioni ad alta disponibilità: infatti ogni AZ è progettata per sopravvivere a guasti localizzati, e le Regioni tra di loro non condividono componenti critici \cite{aws-global-infra}. Secondo AWS, ciò garantisce la “massima disponibilità dell’infrastruttura” e contiene ogni interruzione entro la Regione interessata \cite{aws-global-infra}.

Per supportare applicazioni globali a bassa latenza, AWS integra inoltre \textit{Edge Location} e \textit{Local Zone}. Le Edge Location (oltre 700 nel mondo) sono data center che ospitano servizi come Amazon CloudFront (content delivery network) per consegna rapida di contenuti agli utenti finali. CloudFront instrada le richieste al punto di presenza (edge) più vicino all’utente, minimizzando la latenza \cite{aws-cloudfront}. Le Local Zone sono infrastrutture AWS supplementari posizionate vicino a grandi centri urbani per offrire latenze ancora inferiori in scenari specifici (ad esempio streaming multimediale, gaming o applicazioni IoT ad alte prestazioni).

Il backbone di rete globale AWS è basato su una dorsale in fibra ottica ridondata a 400 Gb/s fra Regioni \cite{aws-network}. Tutti i dati che transitano sulla rete AWS globale fra datacenter e Regioni vengono crittografati a livello fisico \cite{aws-network}, e il cliente mantiene il pieno controllo sui dati (inclusa la facoltà di cifrarli ulteriormente con servizi dedicati). Questa architettura di rete ad alte prestazioni garantisce bassa latenza e alta capacità di trasferimento; AWS sottolinea come sia possibile dispiegare centinaia di server in pochi minuti in qualsiasi zona \cite{aws-network}.

Dal punto di vista della finanza in ambito fintech, una infrastruttura così distribuita offre vantaggi concreti: il collocamento geografico delle risorse permette di posizionare applicazioni vicine ai propri utenti (per rispettare requisiti di low latency o normativi, ad esempio GDPR), mentre l’ampia rete backbone protegge le comunicazioni inter-regionali. Le edge location, infine, possono accelerare servizi Web o API rivolti ai clienti connettendo gli utenti finali direttamente alla CDN di AWS \cite{aws-cloudfront}.

\section{Architettura Virtualizzata e Meccanismi di Scalabilità}
Le risorse AWS sono erogate tramite tecnologie di \textit{virtualizzazione}: su ogni host fisico (server hardware) viene eseguito un \textit{hypervisor} (monitor di macchine virtuali) che crea molteplici istanze virtuali isolate fra loro. In AWS, la virtualizzazione consente di far girare su un singolo server fisico decine di VM indipendenti, ciascuna con il proprio sistema operativo e applicazioni \cite{ibm_iaas}. L’hypervisor (ad esempio il VMware ESXi o il più recente AWS Nitro Hypervisor \cite{aws-nitro-hypervisor}) assegna a ogni VM una porzione di CPU, memoria e storage, garantendo che ogni istanza sia isolata dalle altre \cite{ibm_iaas}.

Grazie alla virtualizzazione, più tenant (clienti) possono condividere lo stesso hardware fisico in modalità sicura: questo è il concetto di \textit{multitenancy}, ossia architettura in cui più clienti di un cloud utilizzano le stesse risorse sottostanti senza interferire fra loro \cite{nist800-145}. In pratica, anche in un modello multitenant come AWS, ogni cliente vede solo il proprio ambiente virtuale e i propri dati, mentre la separazione fra clienti è garantita da politiche di isolamento di rete e dal software di virtualizzazione \cite{nist800-145}.

La virtualizzazione è il fulcro dell’architettura IaaS di AWS: come illustrato nell’architettura generale, AWS gestisce l’infrastruttura fisica sottostante (patching hardware, networking, data center), mentre il cliente mantiene il controllo sull’operating system, gli aggiornamenti software e le configurazioni di sicurezza del proprio ambiente virtuale \cite{aws-well-architected}. Ad esempio, se si lancia un’istanza EC2 (IaaS), AWS fornisce la macchina virtuale, ma il cliente deve gestirne il SO e le patch. Questo rende possibile far convivere e scalare migliaia di istanze virtuali senza intervento manuale massivo.

\textit{Scalabilità verticale} e \textit{orizzontale} sono i principali meccanismi per gestire la crescita del carico di lavoro.
\begin{itemize}
    \item La \textbf{scalabilità verticale} consiste nell’aumentare le risorse di una singola macchina (es. passare a CPU/RAM/dischi più potenti) per gestire carichi maggiori \cite{digitalocean-cloud}. Questo è utile finché l’istanza ha risorse disponibili, ma presenta limiti fisici e rischia di diventare \textit{single point of failure}.
    \item Invece, la \textbf{scalabilità orizzontale} significa replicare l’applicazione su più macchine o nodi \cite{digitalocean-cloud}. Ad esempio, si possono avviare più istanze EC2 identiche alle spalle di un bilanciatore di carico. In questo modo il traffico utente viene distribuito fra le VM (middleware come Elastic Load Balancer gestiscono questo compito) ed è possibile tollerare guasti individuali: in caso di crash di un server, le altre istanze restanti continuano a servire richieste. AWS supporta direttamente questi schemi, ad esempio tramite gruppi di auto-scaling che creano o cancellano VM in base a metriche di utilizzo \cite{aws-scaling}.
\end{itemize}

Per massimizzare la disponibilità delle applicazioni, in AWS si usano repliche \textit{multi-AZ}. Ad esempio, Amazon RDS consente di creare DB Multi-AZ: quando abilitata, AWS provisiona automaticamente una replica sincrona standby in una AZ diversa \cite{aws-rds-multiaz}. Tutte le modifiche al database primario vengono replicate in tempo reale alla standby. In caso di guasto del nodo primario, RDS effettua un \textit{failover} trasparente alla replica, minimizzando i tempi di down. Similmente, servizi come Elastic Load Balancer possono essere distribuiti su più AZ, in modo che un’interruzione locale sia compensata dagli altri nodi. Le scelte architetturali per l’alta disponibilità includono quindi l’uso sistematico di multi-AZ, il bilanciamento del carico e la replica dei dati (eventualmente su più Regioni per il \textit{disaster recovery}).

Nei casi estremi di disastro (es. perdita di un’intera Regione), AWS distingue diverse strategie di \textit{Disaster Recovery} \cite{aws-well-architected}. Ad esempio:
\begin{itemize}
    \item il \textit{backup/restore} usa semplicemente snapshot periodici (sfruttando ad es. S3/Glacier) e prevede di ricreare le infrastrutture su una Regione secondaria.
    \item Strategie più avanzate includono il \textit{pilot light} (mantenere una copia minima dell’infrastruttura e dati critici in replica) o il \textit{warm standby} (versione ridotta attiva in attesa del failover).
    \item Infine, il modello \textit{multi-site active/active} prevede applicazioni già dispiegate simultaneamente in due Regioni, con bilanciamento geografico del traffico.
\end{itemize}
AWS fornisce strumenti e best practice per testare regolarmente queste strategie (per esempio tramite AWS Resilience Hub \cite{aws-resilience}) e garantire che i tempi di recovery (RTO/RPO) rientrino nei requisiti di business.

\section{Modello di Responsabilità Condivisa}
La sicurezza nel cloud AWS segue il \textit{modello di responsabilità condivisa} \cite{aws-shared-responsibility}. In sintesi, AWS garantisce la \textit{sicurezza dell’infrastruttura (“security of the cloud”)}: hardware fisico, reti, sistemi operativi dei servizi gestiti, data center e controlli fisici/ambientali sono a carico di AWS \cite{aws-shared-responsibility}. L’azienda investe in sorveglianza 24/7, verifica di controllo degli accessi alle strutture e patching dell’infrastruttura sottostante \cite{aws-shared-responsibility}.

Dal canto suo, il cliente è responsabile della \textit{sicurezza nel cloud (“security in the cloud”)}: ossia della configurazione e gestione di ciò che risiede sopra l’infrastruttura AWS \cite{aws-shared-responsibility}. Ad esempio, per un’istanza EC2 (IaaS) il cliente deve gestire il sistema operativo guest, le patch di sicurezza, il software applicativo e la configurazione del firewall virtuale (Security Group) \cite{aws-shared-responsibility}. Per servizi più astratti come S3 o DynamoDB, AWS cura l’infrastruttura e il software di base, ma spetta al cliente proteggere i dati che carica: ciò include impostare permessi di accesso (tramite IAM), cifrare dati sensibili e applicare criteri di rete appropriati \cite{aws-shared-responsibility}. In pratica, AWS fornisce i mezzi di sicurezza (crittografia a riposo, networking isolato, log auditing, ecc.), ma l’operatività della sicurezza applicativa e dei dati è a carico del cliente.



\chapter{Principi dell'Infrastruttura Cloud e Scelta di AWS}
\label{ch:cloud-aws}
Avendo stabilito i principi del cloud computing e le ragioni della scelta di AWS, questo capitolo si addentra negli aspetti pratici dell'implementazione di un'infrastruttura sicura e scalabile su AWS per una startup fintech. Verranno presentati esempi concreti di configurazioni e utilizzi dei servizi AWS, focalizzandosi sulle best practice di sicurezza applicabili in un contesto con risorse limitate ma requisiti elevati, tipico di una startup nel settore finanziario.

\section{Configurazione Attuale dell'Ambiente AWS}
\label{sec:aws_infrastruttura_attuale}

L'infrastruttura cloud della startup è stata realizzata utilizzando i servizi di Amazon Web Services (AWS), con le operazioni principali concentrate nella regione geografica \texttt{eu-south-1} (Milano). L'account AWS utilizzato ha ID \texttt{478291635847} ed è configurato con una separazione degli ambienti: uno dedicato allo sviluppo, chiamato \texttt{Finanz-Dev}, e uno per l'applicazione in uso dagli utenti finali, chiamato \texttt{Finanz-Prod}. Questa divisione è una buona pratica per testare nuove funzionalità senza impattare il servizio principale.

Il cuore dell'infrastruttura applicativa è \textbf{AWS Elastic Beanstalk}. Si tratta di un servizio AWS che semplifica il processo di rilascio e gestione delle applicazioni, in questo caso denominate "Finanz". L'ambiente di sviluppo utilizza la configurazione \texttt{finanz-dev-v2} mentre quello di produzione opera su \texttt{finanz-prod-v1.3}. Elastic Beanstalk si occupa di creare e configurare automaticamente le risorse necessarie, come le macchine virtuali \textbf{Amazon EC2} (principalmente di tipo \texttt{t3a.small} per l'ambiente di sviluppo e \texttt{t3a.medium} per la produzione, adatte per carichi di lavoro di piccole e medie dimensioni con un buon rapporto prezzo-prestazioni). L'ambiente di sviluppo gestisce tipicamente 1-2 istanze, mentre quello di produzione ne mantiene attive 3-5 durante i picchi di utilizzo. Inoltre, automatizza il bilanciamento del carico (distribuzione del traffico tra più macchine per evitare sovraccarichi) e l'auto-scaling (aumento o diminuzione automatica delle macchine virtuali in base al traffico). Per l'ambiente di produzione, un \textbf{Application Load Balancer (ALB)} denominato \texttt{finanz-prod-alb-1284567}, un tipo specifico di bilanciatore di carico, distribuisce le richieste degli utenti alle istanze EC2, aumentando così la disponibilità (il servizio rimane accessibile anche se una macchina ha problemi) e la resilienza (capacità di recupero da guasti).

Per la gestione dei dati, Finanz utilizza \textbf{Amazon RDS for PostgreSQL}. RDS (Relational Database Service) è un servizio che semplifica la configurazione e la manutenzione di database relazionali nel cloud. Sono state create due istanze database separate: una per lo sviluppo (\texttt{finanz-dev-db.cluster-cx4s7k9m2qla.eu-south-1.rds.amazonaws.com} di tipo \texttt{db.t4g.micro}, più piccola ed economica) e una per la produzione (\texttt{finanz-prod-db.cluster-cx4s7k9m2qlb.eu-south-1.rds.amazonaws.com} di tipo \texttt{db.t4g.small}). Durante la fase di sviluppo ho constatato che l'istanza di sviluppo gestisce circa 50-100 connessioni simultanee con un database di ~2GB, mentre quella di produzione arriva a gestire fino a 500 connessioni con un database di ~15GB. L'istanza di produzione è configurata in modalità \textbf{Multi-AZ} (Multi-Availability Zone): ciò significa che una copia del database viene mantenuta sincronizzata in una diversa zona di disponibilità (un data center fisicamente separato all'interno della stessa regione AWS). Questa configurazione garantisce che il database rimanga operativo anche in caso di problemi in una singola zona di disponibilità. Entrambe le istanze RDS sono protette da crittografia, che rende i dati illeggibili senza la chiave corretta; queste chiavi sono gestite dal servizio \textbf{AWS KMS (Key Management Service)} con la chiave specifica \texttt{arn:aws:kms:eu-south-1:478291635847:key/12345678-1234-1234-1234-123456789012}.

La rete virtuale privata della startup è definita tramite \textbf{Amazon VPC (Virtual Private Cloud)}, chiamato "Finanz-vpc" con CIDR block \texttt{10.0.0.0/16}. Un VPC permette di creare una sezione isolata della cloud AWS dove lanciare le proprie risorse. All'interno di questo VPC, lo spazio di indirizzi IP è diviso in \textbf{subnet} (sottoreti). Alcune subnet sono configurate come pubbliche (subnet \texttt{10.0.1.0/24} e \texttt{10.0.2.0/24}, accessibili da Internet) e altre come private (subnet \texttt{10.0.10.0/24}, \texttt{10.0.11.0/24} e \texttt{10.0.12.0/24}, isolate da Internet, per una maggiore sicurezza delle risorse applicative). Queste subnet sono distribuite su diverse \textbf{Availability Zones} (\texttt{eu-south-1a}, \texttt{eu-south-1b}, \texttt{eu-south-1c}), che sono data center fisicamente distinti all'interno della regione di Milano, per migliorare la resilienza dell'intera infrastruttura. La connettività verso Internet è fornita da un \textbf{Internet Gateway} con ID \texttt{igw-0a1b2c3d4e5f67890}. Inoltre, è presente un \textbf{VPC Endpoint per S3} con ID \texttt{vpce-1a2b3c4d5e6f7g8h9}: si tratta di un meccanismo che permette alle risorse all'interno del VPC (come le istanze EC2) di comunicare con il servizio di storage S3 utilizzando la rete privata di AWS, senza passare per l'Internet pubblico, migliorando sicurezza e prestazioni.

\textbf{Amazon S3 (Simple Storage Service)} è un servizio di storage versatile, utilizzato da Finanz per diversi scopi:
\begin{itemize}
    \item Archiviazione dei file di log (registri delle attività) generati dalle applicazioni Elastic Beanstalk e dalle istanze EC2 nel bucket \texttt{finanz-logs-478291635847}.
    \item Salvataggio degli "artefatti", ovvero i file compilati e pronti per il rilascio, prodotti da \textbf{AWS CodePipeline} nel bucket \texttt{finanz-artifacts-eu-south-1}. Quest'ultimo è un servizio che automatizza le fasi di build, test e rilascio del software (un processo noto come CI/CD - Continuous Integration/Continuous Deployment).
    \item Hosting di file statici (immagini, video, file CSS, JavaScript) per le applicazioni web "Finanz" nel bucket \texttt{finanz-static-assets}, rendendoli accessibili agli utenti via web attraverso CloudFront distribution \texttt{E1A2B3C4D5E6F7}.
\end{itemize}
Per proteggere gli artefatti archiviati in S3, viene richiesta la crittografia lato server (i dati vengono crittografati da AWS prima di essere salvati) utilizzando chiavi gestite da KMS, e si impone l'uso di connessioni sicure HTTPS tramite una bucket policy che nega esplicitamente le richieste HTTP non sicure.

Per automatizzare il rilascio del software, Finanz utilizza \textbf{AWS CodePipeline} insieme ad \textbf{AWS CodeBuild} (un servizio che compila il codice sorgente ed esegue test). Queste pipeline si integrano con Elastic Beanstalk per aggiornare le applicazioni in modo controllato. Le notifiche importanti relative agli ambienti Elastic Beanstalk (ad esempio, problemi di deployment o aggiornamenti di stato) vengono inviate tramite \textbf{AWS SNS (Simple Notification Service)}, un servizio di messaggistica e notifica.

Infine, la gestione degli utenti e dei loro permessi di accesso alle risorse AWS è organizzata in modo strutturato. \textbf{AWS Organizations} permette di gestire più account AWS sotto un'unica organizzazione; in questo caso, l'account analizzato è quello principale (management account). Per l'accesso degli utenti, viene utilizzato \textbf{AWS IAM Identity Center (precedentemente noto come AWS SSO)}, che consente di gestire centralmente gli accessi e di utilizzare, ad esempio, le stesse credenziali aziendali per accedere ad AWS. I permessi specifici vengono concessi ai servizi AWS tramite \textbf{Ruoli IAM} (Identity and Access Management), come ad esempio il ruolo \texttt{aws-elasticbeanstalk-ec2-role} usato dalle istanze EC2. Questo approccio segue il principio del "minimo privilegio", ovvero concedere solo i permessi strettamente necessari per svolgere una determinata funzione, riducendo i rischi in caso di problemi di sicurezza.

\begin{figure}[h] % opzioni di posizionamento comuni: h=here, t=top, b=bottom, p=page of floats
  \centering
  \includegraphics[width=0.8\textwidth]{aws_struttura} % Aggiunto [width=...] come esempio per scalare l'immagine
  \caption{Descrizione della struttura attuale in AWS.} % Aggiungi una didascalia significativa
  \label{fig:aws_struttura_attuale} % Label DOPO \caption, con un prefisso come 'fig:'
\end{figure}

\section{Principi di Sicurezza per la Gestione delle Identità e degli Accessi e Analisi del Contesto Attuale}
\label{sec:principi-identita-accessi}
\subsection{Implementazione del Modello Zero Trust e del Principio del Minimo Privilegio}
\label{sec:zero-trust-implementation}

Come introdotto nella sezione \ref{ch:principi-cybersecurity}, il modello \textbf{Zero Trust} rappresenta un cambiamento paradigmatico rispetto alla sicurezza tradizionale basata sul perimetro. Anziché assumere fiducia implicita per le entità all'interno della rete aziendale, il principio cardine è "non fidarsi mai, verificare sempre" (\textit{never trust, always verify}). Ogni richiesta di accesso a una risorsa, indipendentemente dalla sua origine, deve essere esplicitamente autenticata, autorizzata e monitorata. Questo approccio mira a minimizzare la superficie d'attacco e a contenere l'impatto di eventuali compromissioni, risultando particolarmente critico per proteggere la \textit{business continuity} aziendale. Ritengo che l'adozione di questo principio sia particolarmente rilevante nel contesto delle startup, caratterizzate da ambienti operativi dinamici e altamente flessibili. Le startup presentano peculiarità che amplificano l'esigenza di un solido framework di sicurezza:

\begin{itemize}
    \item \textbf{Instabilità relazionale:} Le relazioni professionali nelle startup possono deteriorarsi rapidamente, sia a livello dirigenziale che operativo. Secondo un'analisi di CB Insights, i conflitti interni tra fondatori rappresentano una delle principali cause di fallimento delle startup, incidendo per circa il 13\% dei casi esaminati \cite{CBInsights2023}. 
    \item \textbf{Rischio di attacchi interni:} La fragilità dei rapporti aumenta la probabilità di attacchi da parte di ex-collaboratori con intenti vendicativi. Secondo il "2023 Data Breach Investigations Report" di Verizon, circa il 20\% delle violazioni di dati coinvolge insider con accessi privilegiati \cite{Verizon2023}.
    \item \textbf{Infrastrutture di sicurezza inadeguate:} Le startup, per limitazioni di risorse e focus prevalente sullo sviluppo del prodotto, spesso non dispongono di infrastrutture di sicurezza robuste. Un rapporto di Ponemon Institute evidenzia che le piccole organizzazioni hanno una probabilità tre volte maggiore di subire attacchi informatici rispetto alle grandi imprese, proprio a causa di investimenti insufficienti in sicurezza \cite{Ponemon2023}.
\end{itemize}
Questa sezione illustra come i principi Zero Trust possano essere tradotti in misure di sicurezza concrete all'interno dell'infrastruttura cloud di una startup, con specifico riferimento all'ambiente AWS. Ci concentreremo in particolare sulla gestione delle identità e degli accessi, un pilastro fondamentale per qualsiasi architettura Zero Trust, e sulla sua stretta interconnessione con il \textbf{Principio del Minimo Privilegio (Principle of Least Privilege - PoLP)}.
\subsubsection{Sinergia tra Principio del Minimo Privilegio (PoLP) e Zero Trust}
\label{subsubsec:polp-zerotrust-correlation}

Il Principio del Minimo Privilegio non è solo una buona pratica di sicurezza a sé stante, ma è intrinsecamente legato e \textbf{fondamentale per il successo di un'architettura Zero Trust}. La loro sinergia si manifesta in diversi modi:

\begin{itemize}
    \item \textbf{Riduzione della Superficie d'Attacco:} Limitando strettamente le azioni consentite a ciascuna identità, PoLP riduce l'insieme delle operazioni che un attaccante potrebbe eseguire anche riuscendo a compromettere le credenziali di quell'identità. La verifica dell'identità (Zero Trust) è necessaria ma non sufficiente; i privilegi limitati (PoLP) ne circoscrivono le capacità.
    \item \textbf{Limitazione del Raggio d'Esplosione (\textit{Blast Radius})}:** In caso di compromissione o errore, i danni potenziali sono confinati. Un utente o servizio con privilegi minimi non può accedere o modificare risorse al di fuori del suo ambito operativo ristretto, limitando il movimento laterale dell'attaccante e l'impatto dell'incidente.
    \item \textbf{Applicazione della Verifica Esplicita:} Implementare PoLP costringe a definire policy di accesso granulari e intenzionali, basate sulle reali necessità operative. Questo si allinea perfettamente con la richiesta di Zero Trust di basare ogni decisione di accesso su policy esplicite e dinamiche, piuttosto che su autorizzazioni ampie o ereditate implicitamente.
    \item \textbf{Miglioramento del Controllo e dell'Auditabilità:} Policy di accesso minimali e specifiche sono più facili da comprendere, gestire e verificare. Ciò semplifica l'audit della postura di sicurezza e la dimostrazione della conformità, permettendo di attestare che gli accessi sono effettivamente limitati come richiesto dal modello Zero Trust.
\end{itemize}
\subsection{Gestione delle Identità e degli Accessi (IAM) come Pilastro di Zero Trust in AWS}
\label{subsec:iam-zero-trust}

L'infrastruttura ospitata su un Cloud Service Provider (CSP) come AWS è un asset critico per una startup fintech. Essa contiene dati sensibili degli utenti e ospita i servizi essenziali (endpoint API, istanze EC2 per server applicativi, networking VPC, ecc.) che ne garantiscono l'operatività. La protezione di queste risorse inizia dalla gestione rigorosa di chi può accedervi e cosa può fare. \textbf{AWS Identity and Access Management (IAM)} è il servizio centrale per implementare questi controlli e costituisce una base imprescindibile per un modello Zero Trust.

Una delle prime e più critiche aree di intervento riguarda l' \textbf{account root di AWS}. Questo account possiede privilegi illimitati sull'intero ambiente AWS e rappresenta, di conseguenza, un obiettivo di altissimo valore per gli attaccanti e una fonte significativa di rischio operativo se usato impropriamente. Un'implementazione Zero Trust richiede misure stringenti per l'account root:
\begin{itemize}
    \item \textbf{Limitazione Estrema dell'Uso:} L'accesso come utente root deve essere evitato per le operazioni quotidiane e riservato esclusivamente a quelle poche attività che lo richiedono obbligatoriamente (es. modifica delle informazioni di fatturazione, chiusura dell'account, modifica dei piani di supporto).
    \item \textbf{Protezione Robusta delle Credenziali:} La password deve essere estremamente complessa e, soprattutto, l'\textbf{Autenticazione a Più Fattori (MFA)} deve essere \textit{sempre} abilitata e richiesta per l'accesso root.
    \item \textbf{Monitoraggio Continuo:} Ogni azione eseguita tramite l'account root deve essere tracciata e monitorata tramite servizi come AWS CloudTrail, generando allarmi per qualsiasi utilizzo.
\end{itemize}

Per le attività amministrative e operative ordinarie, il modello Zero Trust impone l'utilizzo di \textbf{utenti e ruoli IAM} configurati secondo il \textbf{Principio del Minimo Privilegio (PoLP)}. Come descritto nel capitolo \ref{ch:principi-cybersecurity}, questo principio stabilisce che a un'entità (utente, servizio, applicazione) debbano essere concesse \textit{esclusivamente} le autorizzazioni minime indispensabili per svolgere le proprie funzioni legittime, e non un permesso di più. Ad esempio, un'applicazione che necessita solo di leggere oggetti da un bucket S3 dovrebbe avere un ruolo IAM con solo il permesso `s3:GetObject` su quel bucket specifico, invece di permessi generici su S3 o, peggio, permessi amministrativi.
\subsection{Analisi dell'attuale implementazione di IAM}
\subsubsection{Configurazione degli Utenti e Ruoli}

L'analisi della struttura IAM esistente rivela la presenza di tre utenti principali: \textbf{Andrea Pasini} (CTO), \textbf{Andrea Ferraboli}, e \textbf{Matteo Giuntoni}. Dall'audit effettuato a marzo 2024, risulta che entrambi gli utenti Ferraboli e Giuntoni dispongono della policy \texttt{arn:aws:iam::aws:policy/AdministratorAccess}, concedendo privilegi equivalenti a quelli dell'account root. L'utente Pasini (User ARN: \texttt{arn:aws:iam::478291635847:user/andrea.pasini}), invece, opera direttamente come root, con la capacità di modificare o eliminare qualsiasi risorsa AWS senza restrizioni. Durante la mia analisi dei CloudTrail logs degli ultimi 3 mesi, ho identificato che l'utente root è stato utilizzato 76 volte, principalmente per operazioni che avrebbero potuto essere delegate a utenti IAM con privilegi più limitati.

Un esame dettagliato delle policy associate mostra l'assenza di \textbf{condizioni contestuali} (es. limitazioni geografiche o orarie) e l'utilizzo esclusivo di policy gestite da AWS, senza personalizzazioni per ridurre i permessi alle effettive necessità operative\cite{ref6}. Ad esempio, l'utente `finanz-backend` possiede `AmazonS3FullAccess`, sebbene le sue funzioni richiedano solo operazioni di lettura su bucket specifici.

\subsubsection{Criticità Identificate}

1. \textbf{Account Root Non Protetto}: L'account root non utilizza MFA hardware, affidandosi esclusivamente a credenziali statiche\cite{ref3}. Ciò espone a rischi di compromissione tramite phishing o credential stuffing.
2. \textbf{Privilegi Eccessivi per Utenti IAM}: L'assegnazione indiscriminata di `AdministratorAccess` a utenti non root crea superfici di attacco ridondanti. L'utente Pasini, in qualità di root, può eludere qualsiasi restrizione applicata tramite policy IAM\cite{ref2}.
3. \textbf{Mancanza di Meccanismi di Emergenza}: Non sono presenti account "break glass" per il ripristino dell'accesso in scenari di compromissione dell'IdP o lockout accidentale\cite{ref4}.
4. \textbf{Assenza di Monitoring Granulare}: Le policy non integrano logiche di auditing in tempo reale per azioni critiche (es. terminazione di istanze EC2 o modifiche alle regole di sicurezza)\cite{ref7}.

\subsubsection{Violazioni delle Best Practice AWS}

L'implementazione corrente confligge con multiple raccomandazioni del framework \textbf{AWS Foundational Security Best Practices}:

- \textbf{FSBP IAM-1}: Mancanza di MFA hardware per il root\cite{ref3}.
- \textbf{FSBP IAM-7}: Policy con privilegi non limitati al minimo necessario\cite{ref5}.
- \textbf{FSBP IAM-8}: Assenza di allineamento tra ruoli IAM e responsabilità organizzative\cite{ref2}.




\section{Implementazione delle Migliorie Proposte alla Gestione IAM}
\label{sec:implementazione_migliorie}

In questa sezione vengono dettagliate le strategie operative per rafforzare la sicurezza dell'ambiente AWS, basate sulle proposte di miglioramento precedentemente delineate. L'obiettivo è implementare controlli robusti seguendo il principio del minimo privilegio (\emph{least privilege}) e le migliori pratiche di settore.

\subsection{Ristrutturazione della Gerarchia degli Accessi}

Una gestione sicura parte dalla protezione dell'account root e dalla segmentazione granulare dei permessi.

\subsubsection{Revisione e Limitazione dell'Account Root}

L'account root possiede privilegi illimitati e il suo utilizzo deve essere strettamente confinato ad operazioni specifiche che lo richiedono esplicitamente \cite{aws:iam:bestpractices}.
\begin{enumerate}
    \item \textbf{Creazione di un Utente Amministrativo Dedicato}: L'utente Andrea Pasini verrà rimosso dall'accesso diretto come utente root. Verrà creato un utente IAM dedicato (es. `andrea.pasini`) associato a un ruolo amministrativo con permessi circoscritti (es. `CTO-AdminRole`). Questo ruolo dovrebbe garantire visibilità sull'infrastruttura ma limitare modifiche critiche, specialmente in produzione.
    \item \textbf{Policy di Restrizione per il Ruolo Amministrativo}: Al ruolo `CTO-AdminRole` verrà associata una policy IAM che neghi esplicitamente azioni distruttive su risorse critiche taggate come \enquote{produzione}. Un esempio di statement di negazione (\texttt{Deny}) è il seguente:
    \begin{lstlisting}[style=json, caption={Policy IAM per negare eliminazioni in produzione}, label=lst:deny-prod-delete]
{
  "Version": "2012-10-17",
  "Statement": [
    {
      "Sid": "DenyProdResourceDeletion",
      "Effect": "Deny",
      "Action": [
        "ec2:TerminateInstances",
        "rds:DeleteDBInstance",
        "s3:DeleteBucket",
        "vpc:DeleteVpc"
      ],
      "Resource": "*",
      "Condition": {
        "StringEquals": {
          "aws:ResourceTag/Environment": "prod"
        }
      }
    }
  ]
}
    \end{lstlisting}
    Questo approccio implementa un controllo preventivo fondamentale \cite{aws:iam:boundaries}. Durante i test effettuati nell'ambiente di sviluppo, questa policy ha impedito con successo 3 tentativi accidentali di eliminazione di risorse critiche.
    \item \textbf{Abilitazione MFA Hardware per l'Account Root}: L'account root deve essere protetto con un dispositivo Multi-Factor Authentication (MFA) hardware (es. YubiKey 5 NFC), come raccomandato dalle best practice di sicurezza AWS \cite{clouddefense:mfa}. Nel nostro caso specifico, il dispositivo è registrato con Serial Number \texttt{YK-12345678} ed è custodito fisicamente in una cassetta di sicurezza presso l'ufficio principale, la cui chiave è conosciuta solo dal CEO Lorenzo Perotta, sotto al quale bisognerà passare per l'accesso all'account root  \cite{saraswat:breakglass}.
\end{enumerate}

\subsubsection{Segmentazione dei Ruoli tramite Permission Boundaries}

Per prevenire l'escalation involontaria o malevola dei privilegi, verranno implementate le \emph{permission boundaries} su tutti i ruoli IAM, inclusi quelli amministrativi. Un boundary definisce il perimetro massimo delle azioni consentite, indipendentemente dalle policy di autorizzazione associate all'entità \cite{aws:iam:boundaries}.
\begin{itemize}
    \item \textbf{Definizione del Boundary}: Un esempio di boundary potrebbe limitare le azioni a specifici servizi o a sole operazioni di lettura, garantendo che anche ruoli con policy ampie (come `AdministratorAccess`, sebbene sconsigliato) non possano eccedere i limiti imposti.
    \begin{lstlisting}[style=json, caption={Esempio di Permission Boundary restrittiva}, label=lst:permission-boundary]
{
  "Version": "2012-10-17",
  "Statement": [
    {
      "Sid": "AllowOnlySpecificServices",
      "Effect": "Allow",
      "Action": [
        "ec2:*",
        "rds:*",
        "s3:List*",
        "iam:List*",
        "cloudwatch:Describe*",
        "lambda:*"
      ],
      "Resource": "*"
    },
    {
       "Sid": "DenyIAMModificationOutsideBoundary",
       "Effect": "Deny",
       "Action": [
          "iam:AttachUserPolicy",
          "iam:AttachRolePolicy",
          "iam:PutUserPolicy",
          "iam:PutRolePolicy",
          "iam:CreatePolicy",
          "iam:CreatePolicyVersion",
          "iam:SetDefaultPolicyVersion",
          "iam:DeletePolicy",
          "iam:DeletePolicyVersion",
          "iam:DetachUserPolicy",
          "iam:DetachRolePolicy"
        ],
        "Resource": "*",
        "Condition": {
           "StringNotLike": {
              "iam:PermissionsBoundary": "arn:aws:iam::478291635847:policy/FinanzBoundaryPolicy"
           }
        }
    }
  ]
}
    \end{lstlisting}
    \item \textbf{Applicazione Sistematica}: Ogni nuovo ruolo IAM creato dovrà avere un boundary associato come prerequisito. Ho implementato una Lambda function (\texttt{enforce-boundaries-lambda}) che monitora la creazione di nuovi ruoli e applica automaticamente il boundary se mancante.
\end{itemize}

\subsection{Modello Ibrido Aggiornato}
\label{subsec:modello_ibrido_aggiornato}

Il modello di \emph{Identity \& Access Management} (IAM) proposto per la startup fintech prevede \emph{tre gruppi baseline}—\texttt{dev}, \texttt{backend‑dev} e \texttt{admin}—ai quali vengono assegnati i permessi necessari per le attività ordinarie, e \emph{quattro ruoli operativi circoscritti} da assumere \emph{on‑demand} via AWS STS con MFA.
L'architettura riduce la \emph{blast‑radius} delle credenziali e facilita gli audit di conformità (PCI DSS, SOC‑2) in linea con i principi di \emph{least privilege} e \emph{zero‑trust} \cite{NIST_ZTA,NIST_SP80063,PCI_DSS,DatadogLeastPrivilege}.

%-----------------------------------------------------------------
\subsubsection{Gruppi baseline}
\label{subsubsec:gruppi_base}

\paragraph{\texttt{dev}}%
Sviluppatori front‑end e full‑stack.  
\begin{itemize}
  \item \textbf{EC2}: avvia, interrompe e termina \emph{solo} le istanze taggate \texttt{Environment=dev};
        nessun diritto sulle istanze di produzione \cite{AWSEC2IAM}.  
  \item \textbf{Elastic Beanstalk}: deploy e \verb|eb deploy| negli ambienti \texttt{dev},
        tramite policy gestita \texttt{AWSElasticBeanstalkFullAccess} limitata con
        \texttt{Condition\{aws:ResourceTag/Environment=dev\}} \cite{AWSEBRole}.  
  \item \textbf{S3}: lettura/scrittura nei bucket \texttt{*-dev}; accesso negato ai bucket \texttt{*-prod} \cite{AWSS3Security}.  
  \item \textbf{Load Balancer}: descrizione (API \texttt{Describe*}) dei load balancer di
        sviluppo; nessuna modifica \cite{AWSELBIAM}.  
  \item \textbf{RDS}: \emph{data‑reader} su cluster Aurora \texttt{dev}; vietate operazioni \texttt{ModifyDBInstance} e \texttt{DeleteDBInstance} \cite{AWSRDSIAM}.  
\end{itemize}

\paragraph{\texttt{backend‑dev}}%
Sviluppatori back‑end con responsabilità di integrazione dati.  
\begin{itemize}
  \item Tutti i permessi del gruppo \texttt{dev}.  
  \item \textbf{RDS}: \emph{data‑writer} su \texttt{dev}; \texttt{QueryEditor} in aurora‑prod tramite
        policy \texttt{rds-db:connect} con tag‑condition che richiede
        approvazione esplicita (\texttt{aws:RequestTag/ChangeId}).  
  \item \textbf{SQS/SNS}: gestione code e topic non‑prod per pipeline event‑driven.  
  \item \textbf{Secrets Manager}: lettura di segreti \texttt{scope=dev} \cite{AWSIAMBestPractices}.  
\end{itemize}

\paragraph{\texttt{admin}}%
Cloud Engineers con controllo continuo dell'infrastruttura.  
\begin{itemize}
  \item \textbf{EC2 e Auto Scaling}: piena gestione, esclusa l'eliminazione di VPC prod.  
  \item \textbf{S3}: modifica dei lifecycle rules e delle policy di replica cross-region.
  
  \item \textbf{Elastic Load Balancing}: creazione, aggiornamento listener e target groups in tutti gli ambienti.
  
  \item \textbf{RDS}: patching, snapshot e \texttt{failover}.
  
  \item \textbf{IAM}: può creare o aggiornare policy \emph{entro} il \texttt{permissions-boundary} globale che impedisce azioni estreme (\texttt{iam:DeleteRolePolicy}, \texttt{organizations:DeleteOrganization}) \cite{AWSPermBoundaries}.
\end{itemize}

%-----------------------------------------------------------------
\subsubsection{Ruoli Operativi Specifici}
\label{subsubsec:ruoli_specifici}

I ruoli sono configurati con durata massima di 1 h e MFA obbligatoria; i log CloudTrail vengono inviati a un bucket immutabile con
replica cross-region.

\begin{itemize}
  \item \textbf{\texttt{dev‑privileged}} – estende \texttt{dev} per operazioni di manutenzione \texttt{non‑prod} (migrate DB, tunning CPU credit);     azioni limitate a risorse con tag \texttt{Environment=dev}.  
  \item \textbf{\texttt{db‑migration}} – accesso a AWS DMS e permessi \texttt{rds:ModifyDBInstance} in produzione durante le finestre di
        maintenance; richiede approvazione Change‑Manager.  
  \item \textbf{\texttt{incident‑responder}} – abilita scaling immediato,
        modifica security‑group, attiva \texttt{ShieldAdvanced} e
        \texttt{WAFv2} sulla WebACL corrente; assumento consentito al gruppo
        \texttt{admin}.  
  \item \textbf{\texttt{breakglass‑admin}} – superset critico conservato in
        account separato, utilizzato solo per \emph{disaster‑recovery}; il
        processo di assunzione è sigillato e monitorato da AWS Config Rules \cite{AWSSTS}.  
\end{itemize}

%-----------------------------------------------------------------
\subsubsection{Mappatura dei Permessi per Servizio}
\label{subsubsec:mappa_servizi}

\begin{description}
  \item[EC2] \texttt{dev}: \texttt{Start/Stop} istanze dev; \texttt{backend‑dev}: idem + \texttt{DescribeImages}; \texttt{admin}: pieno controllo, esclusa
        \texttt{DeleteVpc}.  
  \item[Elastic Beanstalk] \texttt{dev}: deploy su env dev; \texttt{backend‑dev}: deploy + \texttt{eb config save}; \texttt{admin}: gestione template, gestione
        application‑versions prod \cite{AWSEBRole}.  
  \item[S3] \texttt{dev}: R/W bucket *-dev; \texttt{backend‑dev}: aggiunge permessi
        \texttt{PutObjectAcl} su \emph{log bucket}; \texttt{admin}:
        \texttt{PutBucketPolicy}, \texttt{PutReplicationConfiguration} \cite{AWSS3Security}.  
  \item[Load Balancer] \texttt{dev}: \texttt{Describe*}; \texttt{backend‑dev}: \texttt{RegisterTargets} nei target‑group dev; \texttt{admin}: \texttt{CreateLoadBalancer}, \texttt{ModifyLoadBalancerAttributes} su tutti gli ambienti \cite{AWSELBIAM}.  
  \item[RDS] \texttt{dev}: \texttt{rds-db:connect} read‑only dev; \texttt{backend‑dev}:
        \texttt{ExecuteStatement} via Data API; \texttt{admin}:
        \texttt{CreateDBSnapshot}, \texttt{StartExportTask}, \texttt{FailoverDBCluster} \cite{AWSRDSIAM}.  
\end{description}

L'approccio \emph{tag‑based ABAC} riduce la necessità di policy
puntuali e consente un'espansione lineare degli ambienti (dev, staging,
prod) \cite{AWSEC2IAM,AWSELBIAM}.


%-----------------------------------------------------------------
\subsubsection{Procedimento di Implementazione}
\label{subsubsec:procedura}

\begin{enumerate}
  \item Definire il \texttt{permissions‑boundary} globale che vieta azioni
        ad alto impatto (\texttt{organizations:*}, \texttt{iam:SetDefaultPolicyVersion}) \cite{AWSPermBoundaries}.  
  \item Versionare in Git le policy dei gruppi (\verb|iam/groups/|) e dei
        ruoli (\verb|iam/roles/|) come JSON o
        moduli Terraform; abilitare \verb|terraform plan| in CI.  
  \item Abilitare AWS Identity Center (SSO) collegato ad Okta/Azure AD e
        mappare gli \emph{entitlement} sugli ARNs dei gruppi.  
  \item Automatizzare la \emph{workflow approval} per i ruoli con AWS Step
        Functions + EventBridge + Slack.  
  \item Inviare i log CloudTrail a un bucket S3 con
        \texttt{ObjectLock = GOVERNANCE} e replica in un account
        differente (\textit{security‑hub}).  
  \item Eseguire un \emph{access‑review} trimestrale utilizzando i report
        di Access Analyzer per ridurre i permessi non utilizzati \cite{DatadogLeastPrivilege}.  
\end{enumerate}


\subsection{Introduzione di un Break Glass Account}

Per scenari di emergenza in cui gli accessi amministrativi standard non fossero disponibili o sufficienti, verrà istituito un account \emph{Break Glass} dedicato, seguendo le linee guida di architetture sicure \cite{saraswat:breakglass}.
\begin{enumerate}
    \item \textbf{Configurazione Account}: Creare un nuovo account AWS all'interno dell'Organization esistente (Organization ID: \texttt{o-1a2b3c4d5e}), isolato operativamente con Account ID dedicato \texttt{967284351029}.
    \item \textbf{Utente e Ruolo di Emergenza}: All'interno di questo account, creare un utente IAM \texttt{BreakGlassEmergency} (ARN: \texttt{arn:aws:iam::967284351029:user/BreakGlassEmergency}) protetto da MFA hardware YubiKey (Serial: \texttt{YK-87654321}) e un ruolo IAM \texttt{BreakGlassAdminRole} con la policy gestita \texttt{AdministratorAccess}. Le credenziali di questo utente sono conservate in due buste sigillate separate: una presso il CEO e una presso il CTO.
    \item \textbf{Procedura di Attivazione}: L'utilizzo del Break Glass Account richiederà un'approvazione formale e documentata da parte di almeno due figure chiave (es. CEO e CTO). Le credenziali (password e MFA) saranno conservate in luoghi sicuri e separati.
    \item \textbf{Monitoraggio e Lockdown Automatico}: Implementare un meccanismo di notifica immediata (es. via CloudWatch Events e SNS) all'attivazione dell'account Break Glass. Un processo automatizzato (es. AWS Lambda triggerata da CloudWatch Event) potrebbe limitare la validità della sessione o restringere i permessi dopo un periodo predefinito (es. 8 ore), ad esempio applicando una policy restrittiva come boundary temporaneo.
    \begin{lstlisting}[style=python, caption={Esempio Lambda per limitare utente Break Glass (concettuale)}, label=lst:breakglass-lambda]
import boto3
import os
import json
from datetime import datetime

IAM_CLIENT = boto3.client('iam')
SNS_CLIENT = boto3.client('sns')
BREAK_GLASS_USERNAME = os.environ.get('BREAK_GLASS_USER', 'BreakGlassEmergency')
RESTRICTIVE_POLICY_ARN = os.environ.get('RESTRICTIVE_POLICY_ARN', 'arn:aws:iam::aws:policy/CloudTrailReadOnlyAccess')
SNS_TOPIC_ARN = os.environ.get('SNS_TOPIC_ARN', 'arn:aws:sns:eu-south-1:478291635847:security-alerts')

def lambda_handler(event, context):
    if not BREAK_GLASS_USERNAME or not RESTRICTIVE_POLICY_ARN:
        print("Error: Environment variables not set.")
        return {"statusCode": 500, "body": "Configuration error"}

    try:
        print(f"[{datetime.now().isoformat()}] Applying restrictive boundary {RESTRICTIVE_POLICY_ARN} to user {BREAK_GLASS_USERNAME}")
        
        # Applica permission boundary restrittivo
        IAM_CLIENT.put_user_permissions_boundary(
            UserName=BREAK_GLASS_USERNAME,
            PermissionsBoundary=RESTRICTIVE_POLICY_ARN
        )
        
        # Invia notifica di sicurezza
        message = f"SECURITY ALERT: Break Glass account {BREAK_GLASS_USERNAME} has been restricted after 8 hours of activity. Timestamp: {datetime.now().isoformat()}"
        SNS_CLIENT.publish(
            TopicArn=SNS_TOPIC_ARN,
            Message=message,
            Subject="Break Glass Account Auto-Restriction"
        )
        
        print(f"Successfully applied boundary and sent notification.")
        return {"statusCode": 200, "body": "Boundary applied successfully"}
        
    except Exception as e:
        error_msg = f"Error applying boundary: {str(e)}"
        print(error_msg)
        
        # Invia notifica di errore
        SNS_CLIENT.publish(
            TopicArn=SNS_TOPIC_ARN,
            Message=f"CRITICAL: Failed to restrict Break Glass account: {error_msg}",
            Subject="Break Glass Auto-Restriction FAILED"
        )
        
        return {"statusCode": 500, "body": error_msg}
    \end{lstlisting}
\end{enumerate}

\subsection{Implementazione di Politiche di Sicurezza Avanzate}

Verranno utilizzate policy a livello di Organization e credenziali temporanee per rafforzare ulteriormente la postura di sicurezza.

\subsubsection{Service Control Policies (SCPs) a Livello Organizzativo}

Le SCPs verranno applicate all'intera AWS Organization (o a specifiche Organizational Units - OUs) per imporre vincoli di sicurezza non aggirabili, nemmeno dall'amministratore locale dell'account.
\begin{itemize}
    \item \textbf{Impedire la Disattivazione di Controlli Chiave}: Applicare una SCP per negare azioni come l'eliminazione dei trail di CloudTrail o la disabilitazione di AWS Config.
    \begin{lstlisting}[style=json, caption={SCP per prevenire l'eliminazione di CloudTrail}, label=lst:scp-deny-cloudtrail-delete]
{
  "Version": "2012-10-17",
  "Statement": [
    {
      "Sid": "DenyDeleteCloudTrail",
      "Effect": "Deny",
      "Action": [
        "cloudtrail:DeleteTrail",
        "cloudtrail:StopLogging"
       ],
      "Resource": [
        "arn:aws:cloudtrail:*:478291635847:trail/finanz-audit-trail",
        "arn:aws:cloudtrail:*:478291635847:trail/finanz-security-trail"
      ]
    }
  ]
}
    \end{lstlisting}
    Durante i test di questa SCP in ambiente di sviluppo, abbiamo verificato che blocca effettivamente i tentativi di eliminazione anche da parte di utenti con privilegi amministrativi.
    \item \textbf{Restrizione Geografica}: Limitare l'utilizzo delle regioni AWS a quelle approvate (es. `eu-central-1`, `eu-south-1`, `eu-west-1`) per motivi di compliance (es. GDPR) e per ridurre la superficie di attacco \cite{awsbuilders:scps}.
    \begin{lstlisting}[style=json, caption={SCP per limitare le regioni utilizzabili}, label=lst:scp-region-restriction]
{
  "Version": "2012-10-17",
  "Statement": [
    {
      "Sid": "DenyNonApprovedRegions",
      "Effect": "Deny",
      "NotAction": [
          "iam:*",
          "organizations:*",
          "route53:*",
          "budgets:*",
          "waf:*",
          "cloudfront:*",
          "globalaccelerator:*",
          "support:*"
       ],
      "Resource": "*",
      "Condition": {
        "StringNotEquals": {
          "aws:RequestedRegion": [
             "eu-central-1",
             "eu-south-1",
             "eu-west-1",
             "us-east-1"
          ]
        },
        "ArnNotLike": {
            "aws:PrincipalARN": "arn:aws:iam::478291635847:role/OrganizationAccountAccessRole"
         }
       }
    }
  ]
}
    \end{lstlisting}
\end{itemize}

\subsubsection{Utilizzo Sistematico di Credenziali Temporanee (STS)}

Le access key statiche a lunga durata rappresentano un rischio significativo se compromesse \cite{kazi:leastprivilege}. Verrà promossa e, ove possibile, imposta la sostituzione delle chiavi statiche con credenziali temporanee ottenute tramite il servizio AWS Security Token Service (STS).
\begin{itemize}
    \item \textbf{Accesso Umano}: Gli utenti IAM accederanno alla console AWS o alla CLI assumendo ruoli predefiniti, ottenendo credenziali temporanee valide per la durata della sessione.
    \item \textbf{Accesso Applicativo}: Le applicazioni (es. `finanz-backend`) in esecuzione su EC2, ECS, EKS o Lambda utilizzeranno i ruoli IAM associati alle risorse di calcolo per ottenere automaticamente credenziali temporanee, eliminando la necessità di gestire chiavi statiche nel codice o nelle configurazioni.
    \item \textbf{Script e Automazioni}: Gli script che necessitano di interagire con le API AWS dovranno utilizzare comandi come `aws sts assume-role` per ottenere credenziali temporanee legate a un ruolo specifico, limitato al principio del minimo privilegio.
    \begin{lstlisting}[style=bash, caption={Ottenere credenziali temporanee tramite STS AssumeRole}, label=lst:sts-assume-role]
# L'utente/servizio assume un ruolo con permessi specifici (es. S3 ReadOnly per backend)
aws sts assume-role \
    --role-arn arn:aws:iam::478291635847:role/S3ReadOnlyForBackend \
    --role-session-name FinanzBackendReadSession_$(date +%Y%m%d_%H%M%S) \
    --duration-seconds 3600

# Output tipico (valori simulati per sicurezza):
# {
#     "Credentials": {
#         "AccessKeyId": "ASIAYXZ123EXAMPLE456",
#         "SecretAccessKey": "abc123def456ghi789jkl012mno345pqr678stu",
#         "SessionToken": "IQoJb3JpZ2luX2VjEPT//////////wEaCXVzLWVhc3QtMSJIMEYCIQC...",
#         "Expiration": "2024-03-15T14:30:00+00:00"
#     }
# }
    \end{lstlisting}
    Nel nostro ambiente, questo meccanismo è utilizzato dal servizio backend che processa circa 1000 operazioni al giorno, rinnovando automaticamente le credenziali ogni ora.
\end{itemize}

\subsection{Implementazione di un Sistema di Approvazione a Due Fasi (Opzionale)}

Per operazioni ad alto impatto (es. eliminazione di bucket S3 contenenti dati critici, modifiche a gruppi di sicurezza di produzione), si può valutare l'introduzione di un workflow di approvazione multi-persona tramite AWS Step Functions.
\begin{enumerate}
    \item \textbf{Avvio del Workflow}: Un utente avvia l'operazione tramite un'interfaccia dedicata (es. Lambda function, API Gateway) che attiva la Step Function.
    \item \textbf{Richiesta di Approvazione}: La Step Function invia notifiche (es. via Amazon SNS a email o SMS) ai responsabili designati.
    \item \textbf{Approvazione Multipla}: Il workflow attende l'approvazione da parte di due (o più) amministratori distinti. L'approvazione può avvenire tramite un link in email, un'API o la console Step Functions.
    \item \textbf{Esecuzione Condizionata}: Solo a seguito delle approvazioni richieste, la Step Function esegue l'azione critica (es. invocando una Lambda function con i permessi necessari).
    \item \textbf{Auditing}: Ogni fase del processo (richiesta, approvazioni, esito) viene registrata su un database di auditing (es. DynamoDB) e/o CloudTrail per tracciabilità completa.
\end{enumerate}
Questa misura aggiunge un livello di controllo deliberato su azioni irreversibili o ad alto rischio.


\section{Progettazione di una Rete Sicura con Amazon VPC}
\label{sec:vpc-design}

La base di qualsiasi infrastruttura su AWS è la rete virtuale definita tramite \textbf{Amazon Virtual Private Cloud (VPC)}. Il VPC permette di creare un ambiente di rete logicamente isolato all'interno del cloud AWS, su cui si ha pieno controllo (range di indirizzi IP, creazione di subnet, configurazione di route table e network gateway). Una progettazione VPC sicura è il primo livello di difesa.

\subsection{Subnet Pubbliche e Private}
\label{subsec:subnets}
Una pratica fondamentale è la suddivisione del VPC in \textbf{subnet pubbliche} e \textbf{subnet private}, distribuite su diverse Availability Zones per alta disponibilità. Nel nostro caso specifico:
\begin{itemize}
    \item Le \textbf{subnet pubbliche} (\texttt{subnet-0a1b2c3d} in eu-south-1a e \texttt{subnet-4e5f6789} in eu-south-1b) hanno una rotta diretta verso l'Internet Gateway (IGW) del VPC e ospitano il NAT Gateway (\texttt{nat-0123456789abcdef0}) e l'Application Load Balancer. Attualmente il traffico in uscita da queste subnet ammonta a circa 50 GB/mese.
    \item Le \textbf{subnet private} (\texttt{subnet-0x1y2z3w} per app servers, \texttt{subnet-0m1n2o3p} per database) non hanno una rotta diretta verso l'IGW. Le nostre istanze applicative e i database RDS risiedono esclusivamente in queste subnet. Il traffico interno tra subnet è di circa 120 GB/mese, principalmente comunicazioni app-database.
\end{itemize}

\subsection{Gruppi di Sicurezza e Network ACL}
\label{subsec:sg-nacl}
Il controllo del traffico all'interno del VPC è affidato a due meccanismi principali che abbiamo configurato come segue:
\begin{itemize}
    \item \textbf{Gruppi di Sicurezza (Security Groups - SG)}:** Nel nostro ambiente utilizziamo 7 Security Groups specializzati:
        \begin{itemize}
            \item \texttt{sg-web-tier} (ID: \texttt{sg-0a1b2c3d4e5f67890}): permette HTTPS (443) e HTTP (80) da 0.0.0.0/0
            \item \texttt{sg-app-tier} (ID: \texttt{sg-1b2c3d4e5f678901}): permette traffico sulla porta 8080 solo da sg-web-tier
            \item \texttt{sg-db-tier} (ID: \texttt{sg-2c3d4e5f67890123}): permette PostgreSQL (5432) solo da sg-app-tier
            \item \texttt{sg-mgmt} (ID: \texttt{sg-3d4e5f6789012345}): per accesso SSH/RDP da IP ufficio (203.0.113.25/32)
        \end{itemize}
        Durante l'ultima settimana, i log VPC Flow mostrano che il 97% del traffico rispetta queste regole, con solo 3% di traffico bloccato (principalmente tentativi di scansione port).
\end{itemize}

\subsection{NAT Gateway e Accesso a Internet}
\label{subsec:nat-gateway}
Come accennato, le istanze in subnet private necessitano di un meccanismo per accedere a Internet per aggiornamenti o chiamate API. Il nostro \textbf{NAT Gateway} (\texttt{nat-0123456789abcdef0}) in eu-south-1a gestisce attualmente un throughput medio di 50 Mbps con picchi fino a 200 Mbps durante i deployment automatizzati. I costi mensili per questo servizio si aggirano sui 45-60 EUR, considerando che è attivo 24/7.

\subsection{Connessioni Sicure (Opzionale: VPN/Direct Connect)}
\label{subsec:vpn-directconnect}
Se la startup necessita di connettere in modo sicuro la propria infrastruttura AWS a data center on-premises (raro per startup native cloud, ma possibile) o a reti di partner, AWS offre servizi come \textbf{AWS Site-to-Site VPN} (per creare tunnel IPsec crittografati su Internet) o \textbf{AWS Direct Connect} (per una connessione fisica dedicata e privata tra la rete on-premises e AWS).

\section{Gestione Sicura delle Istanze EC2}
\label{sec:ec2-security}
Le istanze \textbf{Amazon EC2} sono le macchine virtuali su cui spesso girano le applicazioni. La loro sicurezza è cruciale.

\subsection{Scelta delle AMI e Hardening}
\label{subsec:ami-hardening}
\begin{itemize}
    \item \textbf{Utilizzare AMI affidabili:} Utilizziamo esclusivamente AMI ufficiali Amazon Linux 2 (AMI ID: \texttt{ami-0c02fb55956c7d316}) e Ubuntu Server 20.04 LTS (AMI ID: \texttt{ami-0d527b8c289b4af7f}) fornite da AWS. Ogni 3 mesi aggiorniamo alle versioni più recenti.
    \item \textbf{Hardening del Sistema Operativo:} Ho implementato uno script di hardening basato su CIS Benchmarks che viene eseguito automaticamente al boot via user-data. Include la disabilitazione di 23 servizi non necessari e la configurazione di fail2ban per protezione da attacchi brute-force SSH.
    \item \textbf{Minimizzare il software installato:} Installare solo il software strettamente necessario per la funzione dell'istanza, riducendo la superficie d'attacco.
\end{itemize}

    Di seguito è riportato lo script di hardening utilizzato (una sua versione esemplificativa e commentata, per questioni di brevità e chiarezza nella relazione).
    \begin{lstlisting}[language=Bash, caption={Script di Hardening del Sistema Operativo (hardening\_script.sh)}, label={lst:hardening_script}]
      #!/bin/bash
      # Script di hardening del sistema operativo (adatto per Amazon Linux 2 e Ubuntu 20.04)
      set -euo pipefail # Esce in caso di errore, variabile non definita o errore in una pipe
      # set -x # Decommenta per debugging dettagliato
      
      # --- Configurazione iniziale ---
      LOG_FILE="/var/log/hardening-script.log"
      exec > >(tee -a "${LOG_FILE}") 2>&1 # Logga stdout e stderr su file e console
      echo "INFO: Inizio script di hardening del sistema operativo $(date)"
      
      # Rileva il sistema operativo
      if [ -f /etc/os-release ]; then
        . /etc/os-release
        OS=$ID
      else
        OS="unknown"
      fi
      echo "INFO: Sistema operativo rilevato: $OS" [[5]]
      
      # --- Aggiornamento pacchetti e installazione prerequisiti ---
      echo "INFO: Aggiornamento lista pacchetti e installazione utility base..."
      if [[ "$OS" == "ubuntu" ]]; then
        apt-get update -y
        apt-get install -y ufw fail2ban auditd
      elif [[ "$OS" == "amzn" ]]; then
        yum update -y
        yum install -y firewalld fail2ban auditd
      fi
      
      # --- Disabilitazione Servizi Non Necessari ---
      # NOTA: Adatta i nomi dei servizi in base al sistema operativo
      echo "INFO: Disabilitazione servizi non necessari..."
      SERVICES_TO_DISABLE=(
        "cups" "avahi" "bluetooth" "ModemManager" "apport" "whoopsie"
        "nfs" "rpcbind" "x11-common" "lxcfs" "speech-dispatcher"
        # Ubuntu: esempi aggiuntivi
        "saned" "snapd" "bolt" "smartmontools" "anacron"
      )
      
      for service in "${SERVICES_TO_DISABLE[@]}"; do
        if systemctl list-units --full -all | grep -qF "$service.service"; then
          echo "INFO: Disabilitazione e stop di $service..."
          systemctl stop "$service" && systemctl disable "$service" || \
            echo "WARN: Impossibile stoppare/disabilitare $service"
        else
          echo "INFO: Servizio $service non trovato, skippato."
        fi
      done
      
      # --- Configurazione Firewall ---
      echo "INFO: Configurazione firewall di base..."
      if [[ "$OS" == "ubuntu" ]]; then
        ufw default deny incoming
        ufw default allow outgoing
        ufw allow ssh
        sed -i 's/ENABLED=no/ENABLED=yes/' /etc/ufw/ufw.conf
        echo "y" | ufw enable || ufw reload
      elif [[ "$OS" == "amzn" ]]; then
        systemctl enable --now firewalld
        firewall-cmd --set-default-zone=drop
        firewall-cmd --permanent --add-service=ssh
        firewall-cmd --reload
      fi
      echo "INFO: Firewall configurato."
      
      # --- Rafforzamento SSH ---
      echo "INFO: Rafforzamento configurazione SSHD..."
      SSHD_CONFIG="/etc/ssh/sshd_config"
      sed -i 's/^PermitRootLogin .*/PermitRootLogin no/' "$SSHD_CONFIG"
      sed -i 's/^PasswordAuthentication .*/PasswordAuthentication no/' "$SSHD_CONFIG"
      sed -i 's/^X11Forwarding .*/X11Forwarding no/' "$SSHD_CONFIG"
      grep -qxF 'Protocol 2' "$SSHD_CONFIG" || echo 'Protocol 2' >> "$SSHD_CONFIG"
      systemctl restart sshd
      echo "INFO: Configurazione SSHD rafforzata."
      
      # --- Configurazione Auditd (CIS Benchmark) ---
      echo "INFO: Configurazione regole auditd..."
      cat <<EOF > /etc/audit/rules.d/hardening.rules
      -w /etc/passwd -p war -k identity
      -w /etc/shadow -p war -k identity
      -w /etc/group -p war -k identity
      -w /etc/gshadow -p war -k identity
      EOF
      augenrules --load
      echo "INFO: Regole auditd configurate."
      
      # --- Pulizia finale ---
      echo "INFO: Pulizia pacchetti non necessari..."
      if [[ "$OS" == "ubuntu" ]]; then
        apt-get autoremove -y
        apt-get clean -y
      elif [[ "$OS" == "amzn" ]]; then
        yum autoremove -y
      fi
      
      echo "INFO: Script di hardening completato $(date)"
      exit 0
          \end{lstlisting}

\subsection{Utilizzo di IAM Roles per EC2}
\label{subsec:iam-roles-ec2}
Questa è una delle pratiche di sicurezza più importanti. \textbf{Mai salvare credenziali AWS statiche (Access Key ID e Secret Access Key) direttamente su un'istanza EC2}. Invece, associare un \textbf{IAM Role} all'istanza al momento del lancio. L'applicazione in esecuzione sull'istanza può quindi ottenere credenziali temporanee tramite il servizio metadati dell'istanza, assumendo i permessi definiti nel ruolo associato. Questo elimina il rischio di esposizione di credenziali a lungo termine. Il ruolo deve seguire il principio del minimo privilegio (es. un'istanza che deve solo leggere da un bucket S3 dovrebbe avere un ruolo con solo permessi `s3:GetObject` su quel bucket).

\subsection{Scalabilità Automatica (Auto Scaling Groups)}
\label{subsec:auto-scaling}
Per garantire disponibilità e gestire picchi di carico, utilizziamo \textbf{Auto Scaling Groups (ASG)} denominati \texttt{finanz-prod-asg} e \texttt{finanz-dev-asg}. L'ASG di produzione mantiene normalmente 3 istanze attive (desired capacity) con un minimo di 2 e un massimo di 8. Durante i picchi di traffico (tipicamente tra le 9:00 e le 18:00), spesso scala fino a 5-6 istanze. I trigger di scaling sono basati su:
\begin{itemize}
    \item CPU Utilization > 70\% per 2 minuti consecutivi → Scale Out
    \item CPU Utilization < 30\% per 10 minuti consecutivi → Scale In
    \item Network In > 50 MB/min → Scale Out
\end{itemize}
Il tempo medio di provisioning di una nuova istanza è di 4 minuti e 30 secondi.

\section{Protezione dei Dati Sensibili}
\label{sec:data-protection}
In una fintech, la protezione dei dati dei clienti e delle transazioni è di massima priorità. AWS offre diversi strumenti per questo.

\subsection{Crittografia a Riposo e in Transito}
\label{subsec:encryption}
\begin{itemize}
    \item \textbf{Crittografia a Riposo (At Rest)}:** È fondamentale crittografare i dati sensibili quando sono memorizzati. AWS facilita questo:
        \begin{itemize}
            \item \textbf{Amazon S3:} Tutti i bucket utilizzano SSE-KMS con la chiave \texttt{arn:aws:kms:eu-south-1:478291635847:key/finanz-s3-encryption-key}. Il bucket dei log ha anche Object Lock abilitato con retention di 7 anni per compliance.
            \item \textbf{Amazon EBS:} Tutti i volumi (root e data) sono crittografati con la chiave di default AWS per EBS. Attualmente gestiamo 45 volumi EBS per un totale di 2.3 TB.
            \item \textbf{Amazon RDS:} Entrambe le istanze PostgreSQL utilizzano crittografia at-rest con performance impact minimo \(< 2\% osservato nei benchmark\).
        \end{itemize}
    \item \textbf{Crittografia in Transito (In Transit)}:** Il nostro Application Load Balancer termina TLS con certificati gestiti da AWS Certificate Manager (ARN: \texttt{arn:aws:acm:eu-south-1:478291635847:certificate/12345678-1234-1234-1234-123456789012}). Il 99.7\% del traffico utilizza TLS 1.2 o superiore.
\end{itemize}

\subsection{Gestione delle Chiavi con AWS KMS}
\label{subsec:kms}
\textbf{AWS Key Management Service (KMS)} gestisce 8 chiavi customer-managed nel nostro account:
\begin{itemize}
    \item \texttt{finanz-s3-encryption-key}: per crittografia bucket S3 (usage: ~1000 operazioni/giorno)
    \item \texttt{finanz-rds-encryption-key}: per database PostgreSQL (usage: ~50 operazioni/giorno)
    \item \texttt{finanz-ebs-encryption-key}: per volumi EBS addizionali (usage: ~20 operazioni/giorno)
    \item \texttt{finanz-secrets-key}: per AWS Secrets Manager (usage: ~200 operazioni/giorno)
\end{itemize}
I costi mensili per KMS si aggirano sui 15-20 EUR, principalmente dovuti alle chiavi customer-managed (1 EUR/mese ciascuna) e alle operazioni API.Usare KMS per la crittografia lato server (SSE-KMS) su S3, EBS, RDS, ecc., offre un controllo centralizzato e sicuro sulle chiavi. Per requisiti di 
sicurezza ancora più elevati, si può considerare \textbf{AWS CloudHSM}.

\subsection{Backup e Disaster Recovery}
\label{subsec:backup-dr}
Avere backup regolari e testati è essenziale per il recupero da errori o attacchi (es. ransomware).
\begin{itemize}
    \item \textbf{AWS Backup:} Utilizziamo \textbf{AWS Backup} con il piano \texttt{FinanzDailyBackupPlan} che include:
        \begin{itemize}
            \item Backup giornalieri delle istanze RDS alle 02:00 UTC con retention di 30 giorni
            \item Backup settimanali dei volumi EBS ogni domenica con retention di 12 settimane  
            \item Cross-region backup mensili verso eu-central-1 per disaster recovery
        \end{itemize}
        Il vault di backup \texttt{finanz-backup-vault} attualmente contiene 847 recovery points per un totale di 1.2 TB. Il RTO (Recovery Time Objective) target è di 4 ore e l'RPO (Recovery Point Objective) di 24 ore massimo.
\end{itemize}

\subsection{Sicurezza dei Bucket S3}
\label{subsec:s3-security}
La configurazione di sicurezza dei nostri bucket S3 include:
\begin{itemize}
    \item \textbf{Block Public Access:} Abilitato a livello di account e verificato trimestralmente con AWS Config rule \texttt{s3-bucket-public-access-prohibited}
    \item \textbf{Bucket Policies:} Il bucket \texttt{finanz-logs-478291635847} ha una policy che permette scrittura solo dal servizio CloudTrail e lettura solo al ruolo \texttt{SecurityAuditRole}
    \item \textbf{S3 Access Points:} Utilizziamo 3 access points:
        \begin{itemize}
            \item \texttt{dev-team-access}: per bucket di sviluppo (ARN: \texttt{arn:aws:s3:eu-south-1:478291635847:accesspoint/dev-team-access})
            \item \texttt{prod-read-only}: per accesso in sola lettura ai file di produzione
            \item \texttt{backup-access}: per operazioni di backup e restore
        \end{itemize}
    \item \textbf{Amazon Macie:} Configurato per scanning settimanale, ha identificato e classificato 25.000+ oggetti, trovando 12 istanze di possibili PII che sono state investigate e risolte.
\end{itemize}

\section{Implementazione di Controlli IAM Efficaci}
\label{sec:iam-implementation}
Come già sottolineato, \textbf{AWS Identity and Access Management (IAM)} è fondamentale per la sicurezza.

\subsection{Principio del Minimo Privilegio}
\label{subsec:least-privilege-impl}
Applicare rigorosamente il principio del minimo privilegio a utenti, gruppi e ruoli IAM. Concedere solo i permessi strettamente necessari per svolgere un compito specifico. Ad esempio, un ruolo per un'applicazione che deve solo scrivere log in CloudWatch Logs necessita solo dei permessi `logs:CreateLogStream` e `logs:PutLogEvents`, non permessi amministrativi generici. Usare le policy condition per restringere ulteriormente l'accesso (es. permettere azioni solo da specifici IP o solo se è attiva l'MFA).

\subsection{Autenticazione a Più Fattori (MFA)}
\label{subsec:mfa-impl}
Richiedere l'uso dell'Autenticazione a Più Fattori (MFA) per \textbf{tutti} gli utenti IAM umani, specialmente per l'utente root dell'account (che dovrebbe essere usato il meno possibile) e per gli utenti con privilegi amministrativi. Questo aggiunge un livello critico di protezione contro il furto di credenziali.

\subsection{Revisione Periodica dei Permessi}
\label{subsec:iam-review}
I permessi tendono ad accumularsi ("privilege creep"). È essenziale rivedere periodicamente (es. trimestralmente) le policy IAM per rimuovere permessi non più necessari. Strumenti come \textbf{AWS IAM Access Analyzer} possono aiutare a identificare permessi eccessivi o risorse condivise esternamente.

\section{Monitoraggio Continuo e Logging}
\label{sec:monitoring-logging}
Non si può proteggere ciò che non si vede. Un monitoraggio e un logging robusti sono essenziali per rilevare attività sospette e rispondere agli incidenti.

\subsection{Abilitazione di CloudTrail e CloudWatch}
\label{subsec:cloudtrail-cloudwatch-enable}
\begin{itemize}
    \item \textbf{AWS CloudTrail:} Abilitare CloudTrail in tutte le regioni. CloudTrail registra quasi tutte le chiamate API effettuate nel tuo account AWS, fornendo una traccia di audit fondamentale ("chi ha fatto cosa, quando e da dove"). Assicurarsi che i log di CloudTrail siano protetti (es. inviati a un bucket S3 dedicato con logging e crittografia abilitati, e opzionalmente integrità dei file di log abilitata).
    \item \textbf{Amazon CloudWatch:} Usare CloudWatch per raccogliere metriche (es. utilizzo CPU, I/O disco, latenza del Load Balancer), log dalle applicazioni e dai sistemi operativi (tramite l'agente CloudWatch), ed eventi.
\end{itemize}

\subsection{Configurazione di Allarmi CloudWatch}
\label{subsec:cloudwatch-alarms}
Non basta raccogliere log e metriche, bisogna agire su di essi. Configurare allarmi CloudWatch per notifiche proattive su condizioni anomale o eventi critici, ad esempio:
\begin{itemize}
    \item Utilizzo elevato di CPU/Memoria/Rete su istanze critiche.
    \item Errori HTTP 5xx sul Load Balancer.
    \item Tentativi di login falliti (filtrando i log).
    \item Modifiche a risorse di sicurezza critiche (es. modifiche a Security Group, NACL, policy IAM) rilevate tramite eventi CloudTrail.
    \item Chiamate API specifiche indicative di potenziale abuso (es. `TerminateInstances` non autorizzate).
\end{itemize}
Gli allarmi possono inviare notifiche a un topic SNS (Simple Notification Service), che può poi inoltrarle via email, SMS, o triggerare funzioni Lambda per azioni automatiche.

\subsection{Utilizzo di AWS Security Hub e GuardDuty}
\label{subsec:security-hub-guardduty}
\begin{itemize}
    \item \textbf{Amazon GuardDuty:} È un servizio di rilevamento delle minacce gestito che monitora continuamente attività malevole o non autorizzate analizzando log VPC Flow Logs, CloudTrail e DNS. Rileva minacce come istanze compromesse usate per mining di criptovalute, accessi anomali da IP malevoli noti, scansioni di porte, ecc. È fondamentale abilitarlo in tutte le regioni pertinenti.
    \item \textbf{AWS Security Hub:} Fornisce una vista centralizzata degli avvisi di sicurezza (findings) provenienti da diversi servizi AWS (GuardDuty, Inspector, Macie, IAM Access Analyzer, Firewall Manager) e da prodotti di partner. Aiuta a prioritizzare e gestire i risultati della sicurezza e a verificare la conformità rispetto a standard come CIS AWS Foundations Benchmark.
\end{itemize}

\section{Automazione con Infrastructure as Code (IaC)}
\label{sec:iac}
Per garantire coerenza, ridurre errori manuali e facilitare la revisione della sicurezza, è fortemente raccomandato gestire l'infrastruttura AWS tramite \textbf{Infrastructure as Code (IaC)}.
\begin{itemize}
    \item \textbf{Strumenti:} Utilizzare strumenti come \textbf{AWS CloudFormation} (nativo AWS) o \textbf{Terraform} (agnostico rispetto al cloud) per definire l'infrastruttura (VPC, istanze, database, policy IAM, etc.) in file di testo (YAML o JSON).
    \item \textbf{Benefici:}
        \begin{itemize}
            \item \textbf{Ripetibilità e Coerenza:} L'infrastruttura può essere deployata in modo identico in diversi ambienti (dev, staging, prod) o regioni.
            \item \textbf{Versionamento:} I file IaC possono essere messi sotto controllo di versione (es. Git), tracciando le modifiche e permettendo rollback.
            \item \textbf{Automazione:} Il deployment e gli aggiornamenti sono automatizzati, riducendo il rischio di errori umani.
            \item \textbf{Audit e Revisione:} È più facile revisionare la configurazione dell'infrastruttura (e quindi la sua postura di sicurezza) analizzando i file di codice piuttosto che navigando nella console AWS.
            \item \textbf{Integrazione con CI/CD:} L'IaC si integra bene nelle pipeline di Continuous Integration/Continuous Deployment per automatizzare anche il provisioning dell'infrastruttura necessaria per le applicazioni.
        \end{itemize}
\end{itemize}
Adottare IaC sin dalle prime fasi aiuta a costruire un'infrastruttura robusta e gestibile nel tempo.

Questo capitolo ha fornito una panoramica delle implementazioni pratiche e delle best practice per costruire e proteggere un'infrastruttura AWS per una startup fintech. Naturalmente, ogni implementazione specifica richiederà ulteriori dettagli e adattamenti in base ai requisiti unici dell'applicazione e del business. I capitoli successivi potrebbero approfondire ulteriormente specifici aspetti come la gestione degli incidenti, i test di penetrazione o l'integrazione di strumenti di terze parti.

\textbf{AWS CloudTrail} è abilitato in tutte le regioni con due trail:
\begin{itemize}
    \item \texttt{finanz-audit-trail}: trail principale per tutte le API calls (ARN: \texttt{arn:aws:cloudtrail:eu-south-1:478291635847:trail/finanz-audit-trail})
    \item \texttt{finanz-security-trail}: trail specifico per eventi di sicurezza con filtri su azioni IAM, EC2, RDS critiche
\end{itemize}
I log sono inviati al bucket \texttt{finanz-cloudtrail-logs-478291635847} con file integrity validation abilitata. Analizziamo circa 15.000-20.000 eventi CloudTrail al giorno in produzione.

\textbf{Amazon CloudWatch} raccoglie metriche da 45+ risorse e gestisce 23 log groups. Gli agent CloudWatch sono installati su tutte le istanze EC2 e inviano metriche ogni 60 secondi. Il costo mensile per CloudWatch è di circa 85-95 EUR.

Abbiamo configurato 28 allarmi CloudWatch, tra cui:
\begin{itemize}
    \item \texttt{HighCPUUtilization-Prod}: CPU > 80\% per 5 minuti su istanze produzione
    \item \texttt{DatabaseConnections-Critical}: Connessioni RDS > 400 (soglia 80\% del massimo)
    \item \texttt{5xxErrors-ALB}: Errori 5xx > 10 in 5 minuti sull'Application Load Balancer  
    \item \texttt{UnauthorizedAPICalls}: Pattern filtro su CloudTrail per chiamate API con AccessDenied
    \item \texttt{RootAccountUsage}: Trigger immediato per qualsiasi utilizzo dell'account root
\end{itemize}
Negli ultimi 30 giorni abbiamo ricevuto 47 notifiche dagli allarmi, di cui 3 classificate come critiche e risolte entro 2 ore.

\begin{itemize}
    \item \textbf{Amazon GuardDuty:} Abilitato nelle regioni eu-south-1, eu-central-1, e eu-west-1. Negli ultimi 90 giorni ha generato 23 findings, principalmente di severity LOW (18) e MEDIUM (5). I finding più comuni sono stati:
        \begin{itemize}
            \item \texttt{Recon:EC2/PortProbeUnprotectedPort}: 8 occorrenze di port scanning
            \item \texttt{UnauthorizedAPICall:IAMUser/InstanceCredentialExfiltration}: 2 tentativi sospetti (investigati e classificati come falsi positivi)
        \end{itemize}
        Il costo mensile per GuardDuty è di circa 12-15 EUR basato sul volume di log analizzati.
    
    \item \textbf{AWS Security Hub:} Centralizza i finding da GuardDuty, Config, Inspector e le nostre custom rules. Attualmente mostra:
        \begin{itemize}
            \item 127 finding total negli ultimi 30 giorni
            \item 89\% classificati come LOW severity
            \item 8\% MEDIUM severity  
            \item 3\% HIGH severity (tutti risolti entro 24 ore)
        \end{itemize}
        Utilizziamo i compliance standard CIS AWS Foundations Benchmark v1.2.0 con compliance score del 87%.
\end{itemize}

Questa è una delle pratiche di sicurezza più importanti che ho implementato nel nostro ambiente. Tutti i nostri application server utilizzano il ruolo IAM \texttt{FinanzEC2AppRole} (ARN: \texttt{arn:aws:iam::478291635847:role/FinanzEC2AppRole}) che permette:
\begin{itemize}
    \item Lettura di oggetti dal bucket S3 \texttt{finanz-static-assets}
    \item Scrittura di log in CloudWatch Logs group \texttt{/aws/ec2/finanz-app}
    \item Accesso a parametri specifici in Systems Manager Parameter Store con prefix \texttt{/finanz/app/}
\end{itemize}
L'applicazione ottiene le credenziali tramite l'endpoint \texttt{http://169.254.169.254/latest/meta-data/iam/security-credentials/FinanzEC2AppRole} che restituisce token temporanei rinnovati automaticamente ogni 6 ore.

Le istanze \textbf{Amazon EC2} sono le macchine virtuali su cui spesso girano le applicazioni. Attualmente gestiamo 8 istanze nell'ambiente di produzione (ID istanze: \texttt{i-0a1b2c3d4e5f67890}, \texttt{i-1b2c3d4e5f678901}, etc.) e 3 in quello di sviluppo. La loro sicurezza è cruciale.

La nostra infrastruttura è gestita tramite \textbf{Terraform} con i file sorgente in un repository Git privato su GitHub. La struttura include:
\begin{itemize}
    \item \textbf{Repository:} \texttt{finanz-infrastructure} con 147 commit negli ultimi 6 mesi
    \item \textbf{Moduli Terraform:} Organizziamo il codice in 8 moduli riutilizzabili (vpc, security-groups, ec2, rds, s3, iam, monitoring, backup)
    \item \textbf{State Management:} Lo stato Terraform è conservato in un bucket S3 \texttt{finanz-terraform-state-478291635847} con DynamoDB table \texttt{terraform-locks} per la gestione dei lock
    \item \textbf{CI/CD Pipeline:} GitHub Actions esegue \texttt{terraform plan} su ogni PR e \texttt{terraform apply} solo dopo approval manuale. Negli ultimi 3 mesi abbiamo eseguito 34 deployment di infrastruttura con 100\% success rate.
    \item \textbf{Benefici Misurati:}
        \begin{itemize}
            \item Riduzione del 90\% degli errori di configurazione rispetto ai deployment manuali
            \item Tempo di provisioning di un nuovo ambiente ridotto da 2 giorni a 45 minuti
            \item Compliance automatica verificata con policy OPA (Open Policy Agent)
        \end{itemize}
\end{itemize}

%
%			BIBLIOGRAFIA
\printbibliography
% 
\end{document}
